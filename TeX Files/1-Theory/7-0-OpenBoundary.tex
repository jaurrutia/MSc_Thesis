% !TeX root = ../tesis.tex



\subsubsection{Sommerfeld's Radiation Condition}

In 1912 Arnold Sommerfeld first stated its radiation condition in order to guarantee the uniqueness of the solution to the scalar Helmholtz equation [Eq. \eqref{eq:HelmoltzScalar}] applied to the scalar potential $\phi = \phi(\vb{r})$ \cite{schot_eighty_1992}. Its condition for uniqueness can be found by substituting the scalar potential, generated by bounded sources, and the Green's function to the scalar Helmholtz equation in a 3D domain $\Omega$, given by \cite{zangwill_modern_2013,jackson_classical_1999}
%
\index{Green!Function!to Scalar Helmholtz Equation}%
\index{Helmholtz!Equation!Scalar!Green Function}%
\index{Green!Second Identity}%
%
\begin{align}
    \qty(\nabla^2 + k^2) g(\vb{r}|\vb{r}')   = \delta(\vb{r}-\vb{r}')
        \qquad
        \iff
        \qquad
    g(\vb{r}|\vb{r}')   = \frac{\exp[i\vb{k}\cdot(\vb{r}-\vb{r}')]}{\norm{\vb{r}-\vb{r}'}},
\end{align}
%
into Green's second identity\footnote{%
        Let $\psi$ and $\phi$ be $\mathcal{C}^2$ in $\Omega$, then %
        $\int_\Omega (\psi\nabla^2\psi-\phi\nabla^2\psi)\dd[3]{r} = %
        \oint_{\partial\Omega}\vu{n}\cdot(\psi\nabla\phi-\phi\nabla\psi)\dd[2]{r}$, with $\vu{n}$ a normal vector to the boundary $\delta\Omega$  \cite{zangwill_modern_2013}.}, %
 leading to
\begin{align}
    4\pi \phi(\vb{r}) =& \oint_{\partial\Omega}  \frac{\exp[i\vb{k}\cdot(\vb{r}-\vb{r}')]}{\norm{\vb{r}-\vb{r}'}^2} \phi(\vb{r}) \dd[2]{r}+
                 \oint_{\partial\Omega} \frac{\dd[2]{r}}{\norm{\vb{r}-\vb{r}'}^2} \qty[\norm{\vb{r}-\vb{r}'}\qty(\pdv{ \phi(\vb{r})}{r}-ik\cdot\vu{e}_r \phi(\vb{r})) ].
    \label{eq:pot-Green}
\end{align}
%
The boundary condition that the potential $\phi$ must decay to zero as $r\to\infty$ sets Eq. \eqref{eq:pot-Green} equal to zero, which is  hold immediately by the left hand side of Eq. \eqref{eq:pot-Green}. The first integral in the right hand side of Eq. \eqref{eq:pot-Green} equals zero since $\phi(r\to 0 ) = 0$ while $\dd[2]{r}/\norm{\vb{r}-\vb{r}}\approx\dd[2]{r}/r^2$ remains bounded, thus the term between brackets in the second integral must equal to zero in the far-field regime \cite{schot_eighty_1992}, that is
%
\index{Boundary Conditions!Open!Sommerfeld's Radiation Condition}%
\index{Partial Differential Equation (PDE)!Boundary Conditions!Open!Sommerfeld's Radiation Condition}%
\index{Sommerfeld!Radiation Condition!Scalar}
%
\begin{align}
    \lim_{r\to \infty} r\qty(\pdv{\phi(\vb{r})}{r} - ik\phi(\vb{r})) = 0,
    \label{eq:SommScal}
\end{align}
%
where $\norm{\vb{r}-\vb{r}'} \approx r$ and $\vb{k} = k \vu{e}_r$. The Eq. \eqref{eq:SommScal} is known as the Sommerfeld's radiation condition since it states that in the far-field there can only be outgoing waves that decay uniformly in all directions \cite{schot_eighty_1992,jin_theory_2010,bondeson_computational_2005}.

A generalization of the Eq. \eqref{eq:SommScal} to the electric field $\vb{E}$ and the $\vb{H}$ field, both of which are solution to the vectorial Helmholtz equation [Eq. \eqref{eq:Helmholtz}], can be obtained by substituting into the Green's second identity for vector fields%
	%
	\footnote{%
    This can be obtained if  Eq. \eqref{eq:curlcurlE} is subtracted by itself interchanging $\boldsymbol{\eta}_j $ and $\vb{E}$ and setting $\boldsymbol{\eta}_i(\vb{r}) = \vb{n}g(\vb{r}|\vb{r}')$ as explained by \citeauthor{stratton_diffraction_1939} \cite{stratton_diffraction_1939}.}%
    %
    the electric field  $\vb{E}$ ---or magnetic field $\vb{H}$--- and the vectorial Green's function \cite{schot_eighty_1992,silver_microwave_1984,colton_inverse_2019}. Yet, an equivalent derivation is achieved by separating the electromagnetic fields into two contributions and introducing  two vectorial potentials according to the origin of their sources: magnetic and electric  charges and currents induced into a bounded volume \cite{jin_theory_2010,bondeson_computational_2005}. Under such considerations, the electromagnetic fields can be written as:
%
\begin{align}
\vb{E} = \vb{E}_\text{e} + \vb{E}_\text{m}
    \qquad
    \text{and}
    \qquad
\vb{H} = \vb{H}_\text{e} + \vb{H}_\text{m},
\label{eq:EemHem}
\end{align}
%
where the subscript `e' (`m') stands for the electric (magnetic) sources \cite{jin_theory_2010}. Substitution of Eqs. \eqref{eq:EemHem} into the time harmonic Maxwell's equations [Eqs. \eqref{eq:MaxwellsEq}] leads to
%
\index{Maxwell Equations!Harmonic Time Dependent!Electric and Magnetic Contributions}%
%
\begin{subequations}
    \label{eq:MaxwellsEq-em}
\begin{align}
    \nabla \cdot \qty(\varepsilon \vb{E}_\text{e})  &= \rho_\text{ext},
                    &\nabla \cdot \qty(\varepsilon \vb{E}_\text{m})  &= 0,\\
    \nabla \cdot  (\mu\vb{H}_\text{e})  &= 0,
                    & \nabla \cdot  (\mu\vb{H}_\text{m})  &= \rho_\text{m},\\
    \nabla \times \vb{E}_\text{e}  &= i\omega \mu \vb{H}_\text{e},
                    & \nabla \times \vb{E}_\text{m}  &= i\omega \mu \vb{H}_\text{m} + \vb{J}_\text{m},\\
    \nabla \times \vb{H}_\text{e}  &= \vb{J}_\text{ext} - i\omega \varepsilon \vb{E}_\text{e},
                & \nabla \times \vb{H}_\text{m}  &=- i\omega \varepsilon \vb{E}_\text{m},
\end{align}
\end{subequations}%
%
\index{Lorenz!Gauge}%
\index{Gauge!Lorenz}%
%
where $\rho_\text{m}$ and $\vb{J}_\text{m}$ are  induced charge and current densities due to the magnetization of the sources \cite{jin_theory_2010}. From the magnetic Gauss's law applied on $\mu\vb{H}_\text{e}$, it is defined the vector potential $\vb{A}$ and, analogously, the vector potential $\vb{F}$ arises from the electric Gauss's law on $\varepsilon\vb{E}_\text{m}$ \cite{jin_theory_2010}. Then, scalar potentials $\phi_\text{e}$ and $\phi_\text{m}$ for $\vb{E}_\text{e} $ and $\vb{H}_\text{m}$ are obtained from the Faraday-Lenz's law and the Ampère-Maxwell's law applied on them, accordingly \cite{jin_theory_2010}. If the electric scalar and vector potentials are chosen so that they follow the Lorenz gauge, and the same is imposed for the magnetic scalar and vector potential, that is $\nabla\cdot\vb{A} = -i\omega\mu\varepsilon \phi_\text{e}$ and $\nabla\cdot\vb{F} = -i\omega\mu\varepsilon \phi_\text{m}$ \cite{zangwill_modern_2013}, then the EM fields are given by
%
\begin{align}
    \vb{E} = - \frac{\nabla[\nabla\cdot\vb{A}]}{i\omega\varepsilon\mu} + i \omega\vb{A} + \frac{1}{\varepsilon}\nabla\times\vb{F}
    \qquad
    \text{and}
    \qquad
    \vb{H} =- \frac{\nabla[\nabla\cdot\vb{F}]}{i\omega\varepsilon\mu} + i \omega\vb{F} + \frac{1}{\mu}\nabla\times\vb{A},
    \label{eq:EMFields}
\end{align}
%
where the vectorial potentials $\vb{A}$ and $\vb{F}$ are also solution to Helmholtz equation on each component and thus can be expressed as \cite{jin_theory_2010}
%
\begin{align}
    \vb{A} = \frac{\mu}{4\pi} \int_\Omega \vb{J}_\text{ext}  \frac{\exp[i\vb{k}\cdot(\vb{r}-\vb{r}')]}{\norm{\vb{r}-\vb{r}'}}\dd{\Omega'}
    \qquad
    \text{and}
    \qquad
    \vb{F} = \frac{\varepsilon}{4\pi} \int_\Omega \vb{J}_\text{m}  \frac{\exp[i\vb{k}\cdot(\vb{r}-\vb{r}')]}{\norm{\vb{r}-\vb{r}'}}\dd{\Omega'}.
\end{align}
%
In the far-field regime it follows that $\norm{\vb{r}-\vb{r}'}^{-1}\approx r^{-1}$ and $\vb{k}\cdot \vb{r} = kr$ \cite{jackson_classical_1999,zangwill_modern_2013}, therefore
%
\begin{subequations}
    \label{eq:AF}
\begin{align}
    \vb{A} &= \frac{\mu\exp(ikr)}{4\pi r}\vb{N},         &\text{with} \qquad \vb{N} &= \int_\Omega \vb{J}_\text{ext}  \exp(-i\vb{k}\cdot\vb{r}')\dd{\Omega'},\\
    \vb{F} &= \frac{\varepsilon\exp(ikr)}{4\pi r}\vb{L}, &\text{with} \qquad \vb{L} &= \int_\Omega \vb{J}_\text{m}  \exp(-i\vb{k}\cdot\vb{r}')\dd{\Omega'},
\end{align}
\end{subequations}
%
and that also the operator $\nabla$ acts as $\nabla\to i\vb{k} = ik\vb{e}_r$ since the electric field can be written as a plane traveling in the $\vb{k}$ direction \cite{jin_theory_2010,jackson_classical_1999}. Substituting Eqs. \eqref{eq:AF} into Eq. \eqref{eq:EMFields} leads to the following expressions for the electromagnetic fields in the far-field regime \cite{jin_theory_2010}:
%
\begin{subequations}
    \label{eq:EHFarlim}
\begin{align}
    \lim_{r\to\infty}\vb{E} &= -i k\frac{\exp(ikr)}{4\pi r}
                \left[ \vu{e}_r\times\vb{L}-\sqrt{\frac{\mu}{\varepsilon}}  \Big(\vb{N}-(\vu{e}_r\cdot\vb{N})\vu{e}_r\Big) \right],
            \label{eq:EFarlim}\\
    \lim_{r\to\infty}\vb{H} &=i k\frac{\exp(ikr)}{4\pi r}
                \left[\sqrt{\frac{\varepsilon}{\mu}}  \Big(\vb{L}-(\vu{e}_r\cdot\vb{L})\vu{e}_r + \vu{e}_r\times\vb{N}\Big) \right],
             \label{eq:HFarlim}
\end{align}
\end{subequations}
%
where the dispersion relation $k^2 = \omega^2\mu\varepsilon$ was employed. From Eq. \eqref{eq:EHFarlim} it can be seen that the EM fields in the far-field regime have no radial components and, by calculating the cross product $\vu{e}_r\times\vb{E}$ in the far-field, and comparing with Eq. \eqref{eq:HFarlim}, one obtains
%
\begin{align}
    \lim_{r\to\infty} \qty(\vu{e}_r\times\vb{E} - \sqrt{\frac{\mu}{\varepsilon}} \vb{H}) = \vb{0},
    \label{eq:FarLimit}
\end{align}
%
\index{Wave!Plane!Dispersion Relation}%
%
which states that the electric field is perpendicular to the direction of propagation and to the $\vb{H}$ field in the far-field and that their amplitudes have a fixed ratio of $\sqrt{{\mu}/{\varepsilon}}$, known as the impedance of the medium \cite{jin_theory_2010,schot_eighty_1992}. Lastly, Eq. \eqref{eq:FarLimit} can be rewritten in terms of only $\vb{E}$ with aide of the Faraday-Lenz's law and the dispersion relation for a plane wave, which yields the generalization of the Eq. \eqref{eq:SommScal}
%
\index{Scattering!Sommerfeld!Radiation Condition!Vectorial}%
\index{Sommerfeld!Radiation Condition!Vectorial}%
%
\begin{tcolorbox}[title = Generalized Sommerfeld or Silver-Müller Radiation Condition, ams align, breakable ]
    \lim_{r\to \infty} r\qty(\nabla\times \vb{E} - i k \vu{e}_r\times\vb{E}) = \vb{0},
    \label{eq:SommVec}
\end{tcolorbox}%
 \noindent%
which is also known as the Silver-Müller radiation condition \cite{colton_inverse_2019,silver_microwave_1984}.

The implementation of the light scattering problem [Eq. \eqref{eq:Scatt-Weak-All}] into numerical methods, such as the FEM, has the disadvantage that it is a problem solved in an unbounded, or an open, domain. Nevertheless, the evaluation of Eq. \eqref{eq:SommVec} into $\partial\Omega$ guarantees that the obtained solution reproduces that of the light scattering \cite{jin_theory_2010,bondeson_computational_2005}. Since the Sommerfeld's radiation  condition is a non-homogeneous Neumann boundary condition, it  requires Eq. \eqref{eq:SommVec} to be evaluated at a surface, with a normal vector $\vu{n}$. Due to its integration on $\partial\Omega$, Sommerfeld's radiation condition is mostly used when the scatterer  is small relative to  $\Omega$ and when the scattered electric field normally illuminates the boundary \cite{jin_theory_2010,bondeson_computational_2005}.  Were any of these conditions are not met, another ABC must be implemented.

    \subsubsection{Perfectly Matching Layer}

    \index{Boundary Conditions!Open!Perfectly Matching Layer (PML)}%
    The Perfectly Matching Layer (PML) is an Absorbing Boundary Condition (ABC) described by a mathematical  domain $\Omega_\text{PML}$ \cite{jin_theory_2010} that surrounds  $\Omega$, where Eq. \eqref{eq:Scatt-Weak-All} is to be solved, which has the property that any reflection on its boundary is damped \cite{bondeson_computational_2005,jin_theory_2010,chew_complex_1997}. The PML was originally developed by \citeauthor{berenger_perfectly_1994} \cite{berenger_perfectly_1994} in 1994 as a highly absorbing media for finite differences time domain simulations and then it was proposed as a \textit{complex coordinate stretching} viewpoint\footnote{The complex coordinate stretching approach to the PML for FEM simulations employs a curvilinear system with complex evaluated scale factors to suppress any non-physical reflections.} \cite{chew_complex_1997}, thus attributing the non-reflectivity of the PML to its geometric properties. The FEM solution to the scattering problem with time harmonic dependency, such as described in Eq. \eqref{eq:Scatt-Weak-All}, exploits the later approach since the introduction of the PML does not modifies the method \cite{jin_theory_2010}.

    To determine which conditions are needed for the PML to damp all reflections in $\Omega$, let us use stretched coordinates on the PML domain $\Omega_\text{PML}$ where it is assumed that the far-field approximation of the electromagnetic fields is valid. Thus, the gradient in such coordinate system can be written as
    %
     \begin{subequations}
         \label{eq:stretched-operators}
     \begin{align}
         \nabla_\text{s} \equiv \qty(\frac{\vu{e}_x }{s_x}\pdv{x} + \frac{\vu{e}_y}{s_y}\pdv{y} + \frac{\vu{e}_z}{s_z}\pdv{z}) \to \vb{k} = \frac{k_x}{s_x}\vu{e}_x + \frac{k_y}{s_y}\vu{e}_y  +\frac{k_y}{s_z}\vu{e}_z,
     \label{eq:kstretch}
     \end{align}
     %
     \index{Coordinate System!Stretched}%
     %
     where the subscript `s' stands for stretched and $\vb{k}$ is the wave vector of the traveling electric plane wave in the far-field. The stretching factors $s_{x_i}$, with $x_i \in \{x, y, z\}$, depend only on the coordinate of its stretching direction, that is, $s_{x_i} = s_{x_i}(x_i)$ and are, in general, complex quantities \cite{chew_complex_1997,jin_theory_2010}; on the domain $\Omega$, outside the PML, the scale factors are equal to one. Additionally, the divergence and curl operators in the stretched coordinate system are \cite{arfken_mathematical_2001}:
     %
     \begin{align}
        \nabla_\text{s}\cdot \vb{v} =&
         %            \frac{1}{s_x s_y s_z}\qty[\pdv{x}\big( s_y s_z v_x\big) + \pdv{y} \big( s_x s_z v_y\big)+ \pdv{z}\big( s_x s_y v_z\big)]
                   \frac{1}{s_x s_y s_z} \nabla\cdot \qty[\text{diag}(s_y s_z,s_x s_z,s_x s_y)\vb{v}],
         \label{eq:divs}
         \\
        \nabla_\text{s}\times \vb{v} =& %\frac{1}{s_x s_y s_z}\mqty|s_x\vu{e}_x & s_y\vu{e}_y & s_z\vu{e}_z \\
                                         %                    \pdv*{x} & \pdv*{y} & \pdv*{z} \\
                                         %                    s_x v_x & s_y v_y & s_z v_z|
                                     = \text{diag}\qty(\frac{1}{s_y s_z},\frac{1}{s_x s_z},\frac{1}{s_x s_y}) \nabla\times
                                     \qty[\text{diag}(s_x,s_y,s_z) \vb{v}],
         \label{eq:curls}
     \end{align}
    \end{subequations}
    %
    where $\vb{v} = v_x\vu{e}_x+v_y\vu{e}_y+v_z\vu{e}_z$ is an arbitrary vector and $\text{diag}(a,b,c)$ is a matrix whose only non-zero elements are its arguments placed along its diagonal. From Eq. \eqref{eq:kstretch}, the dispersion relation of a plane wave  is
    %
    \index{Wave!Plane!Dispersion Relation!Stretched System}%
    \index{Plane!Wave!Dispersion Relation!Stretched System}%
    %
    \begin{align}
        \vb{k} \cdot \vb{k}  = k^2 = \mu\varepsilon\omega^2 =
            \qty(\frac{k_x}{s_x}^2 )+ \qty(\frac{k_y}{s_y})^2 + \qty(\frac{k_z}{s_z})^2,
     \label{eq:k-reldisp}
    \end{align}
    %
    whose solution is given by \cite{jin_theory_2010,chew_complex_1997}
    %
    \begin{align}
        k_x = k s_x \sin\theta\cos\varphi, \qquad
            k_y = k s_y \sin\theta\sin\varphi, \qquad \text{and}\qquad
                k_z = k s_z \cos\theta,
     \label{eq:kstretchcomp}
    \end{align}
    %
    with $\varphi$ and $\theta$ the azimuthal and polar angles, respectively. If any of the stretching factors are chosen so that $\Im[s_{x_i}]<0$, then the EM fields in the PML decay exponentially in the $x_i$-direction.

    In between the domains $\Omega$ and $\Omega_\text{PML}$ there is a boundary, which can be locally considered as a plane interface. The Fresnel's reflection amplitude coefficients ---for s- and p-polarization relative to the boundary between the domains--- can be defined as usual since the ratio of the incident and the reflected electric field in the stretched coordinate system does not depend on any stretching coefficient $s_{x_i}$ \cite{jin_theory_2010,chew_complex_1997}. From the continuity of the tangential component of the electric field, the reflection amplitude coefficients for both polarization states are \cite{jackson_classical_1999}
    %
    \begin{align}
       r_\text{s} = \frac{k^\text{(PML)}_z s^\text{(PML)}_z\mu_{{}_\text{PML}} - k^{(\Omega)}_z s^{(\Omega)}_z\mu_{{}_\Omega}}
                        {k^\text{(PML)}_z s^\text{(PML)}_z\mu_{{}_\text{PML}} + k^{(\Omega)}_z s^{(\Omega)}_z\mu_{{}_\Omega}}
           \quad
           \text{and}
           \quad
       r_\text{p} = \frac{k^\text{(PML)}_z s^\text{(PML)}_z\varepsilon_{{}_\text{PML}} - k^{(\Omega)}_z s^{(\Omega)}_z\varepsilon_{{}_\Omega}}
                        {k^\text{(PML)}_z s^\text{(PML)}_z\varepsilon_{{}_\text{PML}} + k^{(\Omega)}_z s^{(\Omega)}_z\varepsilon_{{}_\Omega}}.
       \label{eq:refl-Fresnel}
    \end{align}
   %
   where the $z$ component of the wave vector $\vb{k}$ is perpendicular to the locally plane interface, and the superscripts denote whether the functions are evaluated at $\Omega$ or $\Omega_\text{PML}$. In a similar manner, the phase matching condition of the reflected and incident wave at the interface between $\Omega$ and $\Omega_k$ states that \cite{jin_theory_2010}
   %
   \begin{subequations}
       \label{eq:PhaseMatch}
    \begin{align}
       k^\text{(PML)} s^\text{(PML)}_x \sin\theta_{{}_\text{PML}}\cos\varphi_{{}_\text{PML}} =
                        k^{(\Omega)} s^{(\Omega)}_x \sin\theta_{{}_\Omega}\cos\varphi_{{}_\Omega},\\
        k^\text{(PML)} s^\text{(PML)}_y \sin\theta_{{}_\text{PML}}\sin\varphi_{{}_\text{PML}} =
                     k^{(\Omega)} s^{(\Omega)}_y \sin\theta_{{}_\Omega}\sin\varphi_{{}_\Omega}.
    \end{align}
    \end{subequations}
    \noindent
   %
   From Eqs. \eqref{eq:k-reldisp}--\eqref{eq:PhaseMatch} it can be seen that the reflection amplitude coefficient for both polarization states vanish if the magnetic permeability and the electric permittivity, as well as the stretch coefficient $s_x$ and $s_y$, of the PML matches those of the domain $\Omega$ since such conditions leads to  $\varphi_{{}_\text{PML}} = \varphi_{{}_\Omega}$ and  $\theta_{{}_\text{PML}} = \theta_{{}_\Omega}$ \cite{jin_theory_2010}. Therefore, a prefect matching layer can be described as a thin film $\Omega_\text{PML}$, surrounding $\Omega$, with the following specifications
   %
   \begin{tcolorbox}[title = Perfectly Matching Layer Conditions, ams align, breakable ]
        \left. \mqty{\varepsilon_{{}_\Omega}=\varepsilon_{{}_\text{PML}} ,
                                                &\mu_{{}_\Omega}=\mu_{{}_\text{PML}}
                                                \\
                                                \\
                                            s^{(\Omega)}_x  = s^\text{(PML)}_x,
                                            &  s^{(\Omega)}_y  = s^\text{(PML)}_y} \right\}
                    \qquad\implies \qquad
           r_\text{s} = r_\text{p} = 0 .
        \label{eq:PMLgen}
   \end{tcolorbox}
   %
   \noindent%
   Let us note that no condition have been established to the stretching coefficient $s_z$ and that Eq. \eqref{eq:PMLgen} is valid for any frequency $\omega$, any angle of incidence, and any value of the stretched coordinates in the parallel directions to the interface \cite{jin_theory_2010,bondeson_computational_2005}. The PML conditions allow non-zero transmission coefficients, thus incoming waves can arise due to the finite size of $\Omega_\text{PML}$; to avoid such non-physical incoming waves, $s_z$ is chosen as a complex quantity with a negative imaginary part, so the waves vanish exponentially after entering the PML domain \cite{chew_complex_1997,jin_theory_2010}.

   In order to employ the PML conditions [Eq. \eqref{eq:PMLgen}] in the domain $\Omega$ where the FEM is employed to solve the scattering problem [Eq. \eqref{eq:Scatt-Weak-All}], let us write the Maxwell's equations in the PML ---with the complex stretched coordinate system [Eq. \eqref{eq:stretched-operators}]--- in the stretched coordinate system employed in $\Omega$. To do so, let us relate the EM fields in the PML, $\vb{E}^\text{(PML)}$ and $\vb{H}^\text{(PML)}$, with the EM fields in the domain $\Omega$, $\vb{E}^{(\Omega)}$ and $\vb{H}^{(\Omega)}$,  as \cite{jin_theory_2010}
   %
   \index{Maxwell Equations!Harmonic Time Dependent!Stretched Coordinate System}
   %
   \begin{subequations}
       \label{eq:EH-Omega-PML}
   \begin{align}
        \vb{E}^{(\Omega)} = \text{diag}(s_x,s_y,s_z)\vb{E}^\text{(PML)}
            \qquad
            &\iff
            \qquad
         \vb{E}^\text{(PML)} = \text{diag}\qty(\frac{1}{s_x},\frac{1}{s_y},\frac{1}{s_z})\vb{E}^{(\Omega)},
       \label{eq:E-Omega-PML}
    \\
       \vb{H}^{(\Omega)} = \text{diag}(s_x,s_y,s_z)\vb{H}^\text{(PML)}
            \qquad
             &\iff
            \qquad
       \vb{H}^\text{(PML)} = \text{diag}\qty(\frac{1}{s_x},\frac{1}{s_y},\frac{1}{s_z})\vb{H}^{(\Omega)}.
       \label{eq:H-Omega-PML}
   \end{align}
    \end{subequations}
   %
    The Maxwell's equations in the domain $\Omega_\text{PML}$ are written as in Eq. \eqref{eq:MaxwellsEq} ---with no external sources--- with the operators defined in Eq. \eqref{eq:stretched-operators} applied to $\vb{E}^\text{(PML)}$ and $\vb{H}^\text{(PML)}$ but they can be rewritten in the $\Omega$ domain by substituting  Eqs. \eqref{eq:EH-Omega-PML} into them and isolating the operators in the non-stretched coordinates applied to $\vb{E}^{(\Omega)}$ and $\vb{H}^{(\Omega)}$. This procedure yields \cite{jin_theory_2010}
    %
    \begin{subequations}
        \label{eq:MaxwellsEq-PML}
    \begin{align}
        \nabla_\text{s} \cdot \qty(\varepsilon \vb{E}^\text{(PML)})  = 0
            \qquad &\implies\qquad
            \nabla \cdot \Big[\Big(\varepsilon \mathbb{\Lambda}\Big)\vb{E}^{(\Omega)}\Big] =  0,
             \\
        \nabla_\text{s}  \cdot    \Big(\mu \vb{H}^\text{(PML)}\Big) = 0
            \qquad &\implies\qquad
            \nabla \cdot \Big[\Big(\mu \mathbb{\Lambda} \Big)\vb{H}^{(\Omega)}\Big] =  0,
            \\
        \nabla_\text{s} \times \vb{E}^\text{(PML)}  = i\omega \mu \vb{H}^\text{(PML)}
            \qquad &\implies\qquad
            \nabla \times \Big(\vb{E}^{(\Omega)}\Big) = i\omega\Big( \mu\mathbb{\Lambda}\Big) \vb{H}^{(\Omega)},
            \\
        \nabla_\text{s}  \times\Big(\mu \vb{H}^\text{(PML)} \Big) =   - i\omega \varepsilon \vb{E}^\text{(PML)}
            \qquad &\implies\qquad
            \nabla \times \Big[\Big(\mu \mathbb{\Lambda}\Big) \vb{E}^{(\Omega)}\Big] =  -i\omega\Big(\varepsilon\mathbb{\Lambda}\Big)\vb{E}^{(\Omega)},
    \end{align}
    \end{subequations}
    %
    where $\mathbb{\Lambda}$ is given by
    %
    \begin{align}
        \mathbb{\Lambda} = \text{diag}\qty(\frac{s_y s_z}{s_x},\frac{s_x s_z}{s_y},\frac{s_x s_y}{s_z}).
    \end{align}
    %
    Within the approach in Eqs. \eqref{eq:MaxwellsEq-PML}, the PML can be implemented into the weak formulation of the scattering problem  by introducing a homogeneous, but anisotropic, optical response to the material that embeds the scatterers, and thus simulate an infinite embedding media \cite{chew_complex_1997,bondeson_computational_2005,jin_theory_2010}. Comparing the Maxwell's equations representation in Eqs. \eqref{eq:MaxwellsEq} and \eqref{eq:MaxwellsEq-PML}, the anisotropy of the media is introduced into the weak formulation of the scattering problem [Eq. \eqref{eq:Scatt-Weak-All}] by replacing $\vb{E}$ by $\mathbb{\Lambda}\vb{E}$ \cite{jin_theory_2010}.

     As stated in the PML conditions [Eq. \eqref{eq:PMLgen}], the scalar values of $\varepsilon$ and $\mu$, and of $s_x$ and $s_y$, must match in the medium surrounding the sactterers and in the PML domain. The anisotropy of the dielectric function  sets $s_x = s_y = s_z = 1$  in $\Omega$ so the scattering problem is described exactly as in Eq. \eqref{eq:Scatt-Weak-All}.  In order to guarantee no incoming waves in the domain $\Omega$ due to the finite size of $\Omega_\text{PML}$, let us choose $s_z$ with a negative imaginary part in $\Omega_\text{PML}$.  Lastly, anisotropy is introduced into the finite element formulation of the scattering problem [Eqs. \eqref{eq:Vec-FEM}] by including the matrix $\mathbb{\Lambda}$ as discussed above. Therefore, the PML implementation into the FEM is summarized as follows:
     %
     \index{Boundary Conditions!Open!Perfectly Matching Layer (PML)!Implementation}%
     %
     \begin{subequations}
         \label{eq:PML-FEM}
    %
    \begin{tcolorbox}[title = Perfectly Matching Layer Implementation into Finite Element Method, ams align, breakable ]
     %
        \mathbb{A}\vb{e} = \vb{0} \qquad &\text{with}\qquad \text{ $\mathbb{A}$ and $\vb{e}$ as in Eqs. \eqref{eq:Vec-FEM},}\\
         \mathbb{\Lambda} = \text{diag}\qty(s_z, s_z ,s_z^{-1}),      \qquad&\text{with}\qquad     s_z= 1 \text{ in } \Omega \text{ and } \Im[s_z] < 0 \text{ in } \Omega_\text{PML}, \\
         \qty(\mu^{-1}\nabla\times\boldsymbol{\eta}_i)\cdot\qty(\nabla\times\boldsymbol{\eta}_j)   \qquad&\to\qquad   \qty(\mathbb{\Lambda}^{-1}\mu^{-1}\nabla\times\boldsymbol{\eta}_i)\cdot  \qty(\nabla\times\boldsymbol{\eta}_j),\\
         \boldsymbol{\eta}_i \cdot \boldsymbol{\eta}_j  \qquad&\to\qquad  \boldsymbol{\eta}_i \cdot \mathbb{\Lambda} \boldsymbol{\eta}_j,
    \end{tcolorbox}
\end{subequations}
\noindent
    %
    which is valid for any angular frequency $\omega$. As a last comment, Eq. \eqref{eq:PML-FEM} solves the scattering problem within the domain $\Omega$ but the obtained values of the scattered EM fields in $\Omega_\text{PML}$ have no physical meaning and are of no concern \cite{jin_theory_2010}.
