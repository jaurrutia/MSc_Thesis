% !TeX root = ../tesis.tex



The finite element approximation allows the use of low order interpolating functions by defining the subdomains $\Omega_k$ such that
%
\begin{align}
    \bigcup_{k=1}^M \Omega_k = \Omega,
        \qquad
        \text{and}
        \qquad
    \bigcap_{k=1}^M \Omega_k = \emptyset,
\label{eq:finite}
\end{align}
%
that is that, all $\Omega_k$ together represent the original domain $\Omega$ and that they do not overlap nevertheless, the boundaries of the finite elements are shared by neighboring elements \cite{dhatt_finite_2012}. Then, the finite element approximation restricts  $\tilde{u}_k$  ---the nodal approximation  on each subdomain--- to depend only on the nodal points on $\Omega_k$ and on its boundary $\partial\Omega_k$, while all $\tilde{u}_k$ must be continuous across $\partial \Omega_k$ and obey the  differentiability condition they are bound to, whether the strong or weak formulation is employed \cite{dhatt_finite_2012}.

A finite element $\Omega_k$ is a subdomain of $\Omega$ following Eq. \eqref{eq:finite} \cite{dhatt_finite_2012}  but its formal definition requires $\Omega_k$ to be a manifold embedded into $\Omega$, as well as to chose a polynomial function space on $\Omega_k$, and to define a collection of $N_k$ linear functionals $\mathcal{F}_{\ell_k}$ on $\Omega_k$ \cite{larson_finite_2013}. The description of $\Omega_k$ as a manifold determines its geometrical properties such as dimensionality, shape and curvature, while the polynomials function space sets the order of the interpolating functions $\{\phi_{i_k}\}_{i_k\leq N_k}$. By applying $\phi_{i_k}$ into  $\mathcal{F}_{\ell_k}$, a system of $N_k$ algebraic  equations is obtained:
%
\begin{align}
    \mathcal{F}_{\ell_k}[\phi_{i_k}] = \delta_{{\ell_k} {i_k}},
    \label{eq:linfunc}
\end{align}
%
from which the interpolating functions are determined. Since the $N_k$ linear functional imposes conditions on the interpolating functions, the $N_k$ corresponds to the number of degrees of freedom of the finite element \cite{larson_finite_2013}.

The finite element corresponds to a particular way of discretization of the domain $\Omega$ and thus it is convenient to define the manifold $\Omega_k$ by its geometrical nodes $\vb{r}_{m_k}$, which are a finite collection of  points in $\Omega_k$. In Fig. \ref{fig:FinEleEx} some examples\footnote{Even though the elements shown in Fig. \ref{fig:FinEleEx} are triangles (2D) and  triangular pyramids (3D), their shapes can also be composed by squares and prisms; see \cite{dhatt_finite_2012} for a more detailed list.} of common finite elements in one, two and three dimensions are shown: a straight line segment, a triangular surface and a tetrahedron \cite{dhatt_finite_2012} . The markers corresponds to the geometrical nodes in $\Omega_k$, from which the red markers corresponds to the geometrical nodes that define the shape of $\Omega_k$, since the lines joining two of them are the edges of the finite elements in two and three dimensions. The finite element in Fig. \ref{fig:FinEleEx:a} are known as reference elements since they have boundaries with no curvature and their geometrical nodes on its edges are equally spaced \cite{dhatt_finite_2012,larson_finite_2013}.   Even so, the finite elements are, in general, of curvilinear shape, as shown in Fig. \ref{fig:FinEleEx:b},  which may be preferred over the reference elements for $\Omega$ with no Cartesian symmetries. It is possible to perform a transformation $T:T(\boldsymbol{\xi})\to\vb{r}$ on the reference elements ($\boldsymbol{\xi}$) to reshape it into the so-called real-space element ($\vb{r}$), which corresponds to the physical space where the Eq. \eqref{eq:WRM} is to be solve \cite{dhatt_finite_2012,fletcher_computational_1984}.  Let us recall that the choice of finite elements with straight or curvilinear boundaries is related to the discretization method of the domain $\Omega$. For example, were $\Omega$ a cylinder in 3D, the use of finite elements with straight boundaries arises an error due to truncation of $\Omega$ at its boundary, which can be minimized by increasing the number of finite elements, while curvilinear finite elements may fill such space without increasing the number of finite elements.

 \begin{figure}[h!]
    \centering
    \small
        \def\svgwidth{.8\textwidth}
        \includeinkscape{FEM-Theory/1-systems-Elements-orders}
        \vspace*{-15em}\\
    \hspace*{-.8\textwidth}
        \begin{subfigure}{\textwidth}\caption{ }\label{fig:FinEleEx:a}\end{subfigure}
        \vspace*{5em}\\
    \hspace*{-.8\textwidth}
        \begin{subfigure}{\textwidth}\caption{ }\label{fig:FinEleEx:b}\end{subfigure}
    \vspace*{6em}\\
 \caption[References an Real-space Finite Elements]{ \textbf{a)} Reference and \textbf{b)} real-space finite elements for one (line segment), two (triangular surface) and three (tetrahedron) dimensional domains. The geometrical nodes on each element are signalized by the blue markers and their edges corresponds to the lines between the red markers; the number of nodes along each edge defines the order of the element. The transformation $T$ reshapes the reference finite element from its coordinate system $\boldsymbol{\xi}$ into a real-space element $\vb{r}$. }
\label{fig:FinEleEx}
\end{figure}

The transformation $T$ is a change of coordinates from the coordinate system $\boldsymbol{\xi}$  of the reference finite element into the real-space system $\vb{r}$. The use of Eq. \eqref{eq:linfunc} in the reference elements yield the different kind of interpolating functions, which can be employed to solve the weak formulation of a PDE system [Eq. \eqref{eq:WRM}] by transforming the derivatives in the real space coordinate system by means of the Jacobian matrix $\mathbbm{J}$, whose elements are $J_{ij} = \partial \xi_i/\partial r_j$, and its determinant, the Jacobian, that is \cite{dhatt_finite_2012,fletcher_computational_1984}:
%
\begin{align}
    \pdv{r_i} = \pdv{\xi_i}{r_j} \pdv{\xi_j},
    \qquad
    \text{and}
    \qquad
    \dd{\Omega_k} \to \det[\mathbbm{J}]\dd{\Omega_k}.
\label{eq:jac}
\end{align}
%
The Eq. \eqref{eq:jac} sets a constriction into the discretization of the original domain $\Omega$ and its partition into finite elements, since the Jacobian must be non singular ---different from zero--- in all points in $\Omega_k$, meaning that the transformation $T$ of the reference element into the real-space finite element is bijective \cite{dhatt_finite_2012,fletcher_computational_1984}. To avoid singular points in $\Omega_k$, the real-space finite element must not be deformed considerably when transformed into the reference element \cite{dhatt_finite_2012}.

In order to build the interpolating functions $\phi_{i_k}(\boldsymbol{\xi})$ in the reference finite element, the polynomial functions space on $\Omega_k$ must be chosen by selecting the number of geometrical nodes found along the  edges \cite{larson_finite_2013}, so that the boundary conditions are met. For example, if there are $m$ geometrical nodes on each edge, the finite element is said to be of order $m-1$ since a polynomial of order $m-1$ is guaranteed to pass through the values given to  $m$ nodes. For example, the reference finite elements in Fig. \ref{fig:FinEleEx:a} are a cubic order 1D finite element (three geometrical nodes along the edges), a quadratic order triangular finite element (2D shape with three node along the edges), and a linear tetrahedral finite element (3D volume with four triangular faces and two nodes at each edge).

Once the polynomial functions space on the manifold $\Omega_k$ is set, this can be spanned by the set of interpolating functions $\phi_{i_k}(\boldsymbol{\xi})$, that are determined by means of the $N_k$ linear functional  $\mathcal{F}_\ell$ \cite{larson_finite_2013}.   The election of the  linear functionals give rise to different sets of interpolating functions and thus different families of finite elements. For example, the linear functional given by
     %
\begin{align}
 \mathcal{F}^{L}_{\ell_k}[f(\boldsymbol{\xi})] = f(\boldsymbol{\xi}_{\ell_k}) ,
\label{eq:Lag}
\end{align}
%
with  $\boldsymbol{\xi}_{\ell_k}$ the $\ell_k$-th  geometrical nodes, from a collection of $n_k$, in the reference finite element, defines the Lagrange finite element family since the  interpolating functions obtained by employing Eqs. \eqref{eq:linfunc} and \eqref{eq:Lag} are the Lagrange polynomials \cite{larson_finite_2013,dhatt_finite_2012,fletcher_computational_1984}. The functional $\mathcal{F}^{L}_{\ell_k}$ is a point evaluation in the $N_k = n_k$  geometrical nodes, which imposes no condition on the derivatives of $\phi_{i_k}$ at the boundary of $\Omega_k$, therefore the Lagrange finite element family returns a set of $N_k = n_k$ interpolating functions with no continuous first derivative between finite elements. One linear functional which does return interpolating functions with continuous first derivatives is the following:
%
\begin{align}
     \mathcal{F}^{H}_\ell[f(\boldsymbol{\xi})] = f(\boldsymbol{\xi}_{\ell_k}) ,
     \qquad
     \text{and}
     \qquad
     \mathcal{F}^{H}_{\ell_k'}[f(\boldsymbol{\xi})] = \vu{t}\cdot\nabla  f(\boldsymbol{\xi}_{\ell_k'}) = 0,
\label{eq:Her}
\end{align}
%
with $\boldsymbol{\xi}_{\ell_k}$ on of the $n_k$ geometric nodes in $\Omega_k$, $\boldsymbol{\xi}_{\ell_k'}$ one of the $n_k'$  geometric nodes at vertices of $\Omega_k$, and $\vu{t}$ is a unit vector parallel to its edges \cite{larson_finite_2013}. The functional in Eq. \eqref{eq:Her} give rise to the Hermite finite element family since the resulting interpolating functions are the Hermite polynomials, which allows for a solution on $\Omega$ 1-differentiable due to its $N_k = n_k + \times n'_k$ degrees of freedom \cite{dhatt_finite_2012,larson_finite_2013}. The Lagrange and the Hermite finite element families are two of the most common and simple nevertheless, one can build yet another family known as the serendipity finite element family  if the geometrical nodes outside the boundary of $\Omega_k$ are not considered in Eq. \eqref{eq:Lag} \cite{fletcher_computational_1984}. In Fig. \ref{fig:ExampleElements} the interpolating functions for a one and two dimensional finite elements under the Lagrange linear functional [Eq. \eqref{eq:Lag}] are shown:  a one dimensional cubic finite reference element [Fig. \ref{fig:ExampleElements:a}] and a serendipity triangular reference finite element of quadratic order.

\begin{figure}[t!]
    \hspace*{-.5\textwidth}
    \vspace*{7.5em}%
     \begin{subfigure}{\textwidth}\caption{ }\label{fig:ExampleElements:a}\end{subfigure}\\
    \hspace*{-.7\textwidth}
     \begin{subfigure}{\textwidth}\caption{ }\label{fig:ExampleElements:b}\end{subfigure}
     \vspace*{-11.5em}\\
    \centering
    \scriptsize
    \def\svgwidth{.9\textwidth}
    \includeinkscape{FEM-Theory/2-Example-Elements-2D}
\caption[Example of interpolating functions for 1D and 2D finite element]{Interpolating functions of \textbf{a)} a Lagrange cubic 1D reference finite element and \textbf{b)} a serendipity Lagrange quadratic 2D triangular reference element. The markers corresponds to the evaluation of the linear functionals [Eqs. \eqref{eq:linfunc} and $\eqref{eq:Lag}$] on the interpolating functions on each case.}
\label{fig:ExampleElements}
\end{figure}
