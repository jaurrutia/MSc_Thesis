% !TeX root = ../tesis.tex

The optical properties of a particle, either in the near or the far field regime,  are determined by the Mie coefficients   $a_\ell$ and $b_\ell$ [Eqs. \eqref{eq:MieCoef}] since the exact solution to the scattered electric field $\vb{E}^\text{sca}$ [ [Eq. \eqref{eq:EscaLC}] is a linear combination of the vector spherical harmonics $\vb{N}^{(3)}_{\text{e}1\ell}$ and $\vb{M}^{(3)}_{\text{o}1\ell}$ modulated by  $a_\ell$ and $b_\ell$, respectively. Thus, the physical interpretation of each term $\ell$ in the linear combination, as well as of $\vb{N}^{(3)}_{\text{e}1\ell}$ and $\vb{M}^{(3)}_{\text{o}1\ell}$, can be determined by visualizing each one of them independently. By understanding the contribution of each term, the optical response of a particle illuminated by a plane wave can be studied in the near and far regime.

The Figure \ref{fig:Multipoles} shows the decomposition of the scattered electric field  $\vb{E}^\text{sca}$ of a spherical particle into its contributions proportional to $a_1$ [Fig. \ref{fig:VSH:a1}], $b_1$ [Fig. \ref{fig:VSH:b1}], $a_2$ [Fig. \ref{fig:VSH:a2}] and $b_2$ [Fig. \ref{fig:VSH:b2}], when the particle is illuminated by a $x$-polarized plane wave traveling in the $z$ direction. The vectorial behavior of the $a_\ell$ contributions to $\vb{E}^\text{sca}$ are given by the VSH $\vb{N}^{(3)}_{\text{e}\ell 1}$ and by $\vb{M}^{(3)}_{\text{o}\ell 1}$ for the $b_\ell$ contributions. The  scattered electric field  is evaluated at a spherical surface (gray sphere) larger than the scatter: the arrow stream on the spherical surface corresponds to the pointing direction of each contribution to $\vb{E}^\text{sca}$ parallel to the evaluation sphere, while the color code corresponds to the magnitude of the scattered electric field at each point; the solid shape at the center of each axis is a contour surface $\norm{\vb{E}^\text{sca}}$.
%

\begin{figure}[h!]
	\def\svgwidth{1\textwidth} \small
  \vspace*{3.0em}
  \hspace*{0.em}
    \begin{subfigure}{.24\textwidth}\caption{\centering $\vb{E}^\text{sca} \sim  i a_1 \vb{N}^{(3)}_{\text{e}11}$}\label{fig:VSH:a1}\end{subfigure}
  	\begin{subfigure}{.24\textwidth}\caption{\centering $\vb{E}^\text{sca} \sim  - b_1 \vb{M}^{(3)}_{\text{o}11}$}\label{fig:VSH:b1}\end{subfigure}
	\begin{subfigure}{.24\textwidth}\caption{\centering $\vb{E}^\text{sca} \sim  i a_2 \vb{N}^{(3)}_{\text{e}12}$}\label{fig:VSH:a2}\end{subfigure}
	\begin{subfigure}{.24\textwidth}\caption{\centering $\vb{E}^\text{sca} \sim  - b_2 \vb{M}^{(3)}_{\text{o}12}$}\label{fig:VSH:b2}\end{subfigure}
  \vspace*{-6.em}\\
  \includeinkscape{VSH/3-VSH}
  \vspace*{-2em}
  \caption[Multipolar Contributions to the Scattered Electric Field]{ Decomposition of the  scattered electric field $\vb{E}^\text{sca}$ into its contributions  proportional to \textbf{a)} $a_1$,  \textbf{b)}  $b_1$, \textbf{c)} $a_2$ and \textbf{d)} $b_2$ [see Eq. \eqref{eq:EscaLC}] when a spherical particle (not shown) is illuminated by an $x$-polarized plane wave traveling in the $z$ direction. The scattered electric field $\vb{E}^\text{sca}$ is evaluated at a sphere larger than the particle: the arrow stream corresponds to the projection parallel to the evaluation sphere of $\vb{E}^\text{sca}$  and the color code corresponds to the magnitude of  $\vb{E}^\text{sca}$. A contour surface of the magnitude of $\vb{E}^\text{sca}$ is located at the center of each axis. }
\label{fig:Multipoles}
\end{figure}

The general effect of each contribution to the scattered electric field  $\vb{E}^\text{sca}$  can be understood by analyzing their behavior around the points where the scattered electric field drops to  zero; such points are called nodes and are shown in dark bluish colors in  Fig. \ref{fig:Multipoles}. The number of nodes over the evaluation sphere (gray surface) is proportional to the chosen value of $\ell$. For example, if $\ell = 1$ [Figs. \ref{fig:VSH:a1} and \ref{fig:VSH:b1}] there is a pair of such nodes and if $\ell = 2$ [Figs. \ref{fig:VSH:a2} and \ref{fig:VSH:b2}] there are two pairs, where each pair consists of two nodes at opposite sides of the evaluation sphere. When comparing the contributions proportional to $a_1$ and $a_2$ [Figs. \ref{fig:VSH:a1} and \ref{fig:VSH:a2}] with contributions proportional to $b_1$ and $b_2$ [Figs. \ref{fig:VSH:b1} and \ref{fig:VSH:b2}], one difference is the location of the pairs of nodes for a fixed value of $\ell$, which differ spatially by a rotation around the $z$ axis of an angle $\varphi = \pi/2$.  Another difference between the contribution of $a_\ell$ and the $b_\ell$ to  $\vb{E}^\text{sca}$ are the trajectories they performed around each pair of nodes: In the $a_\ell$ contributions the scattered electric field flows from one node to its pair, thus following an open path, while the scattered electric field for the $b_\ell$ contributions circulates around the nodes forming a closed path. Taking into account such behaviors of the scattered electric field, it can be seen that the $a_\ell$ ($b_\ell$) contribution describes the electric field of an electric (magnetic) dipole when $\ell = 1$ and of an electric (magnetic) quadrupole when $\ell = 2$. Extrapolating such behavior for an arbitrary $\ell$,  it can be concluded that the $a_\ell$ contributions to the scattered electric field, described by the VSH $\vb{N}^{(3)}_{\text{e}\ell 1}$, correspond to the electric field of an electric multipole of order $\ell$, while the $b_\ell$ contribution, described by the VSH $\vb{M}^{(3)}_{\text{o}\ell 1}$, correspond to the electric field of a magnetic multipole of order $\ell$.

The scattered electric field $\vb{E}^\text{sca}$ of a spherical particle can be written, according to Eq. \eqref{eq:EscaLC}, as a linear contribution of electric fields associated to electric and magnetic multipoles, as shown in Fig. \ref{fig:Multipoles}, modulated by the Mie coefficients $a_\ell$ and $b_\ell$, respectively. Thus, the field $\vb{E}^\text{sca}$ can reproduce the pattern of a pure  electric or magnetic multipole of order $\ell$ if $a_\ell$ or $b_\ell$ are maximized, accordingly. In such cases, the scattered electric field is a standing wave on the surface of the spherical particle known as a Localized Surface Plasmon (LSP). Since the Mie coefficients [Eqs. \eqref{eq:MieCoef}]  depend on the material and size of the spherical scatter, on the wavelength and traveling media of the incident plane wave, and on  the order $\ell$, they values of $a_\ell$ and $b_\ell$ are maximized when there is a coupling between the scatterer and the plane wave for a fixed $\ell$, which yields a Localized Surface Plasmon Resonance (LSPR). The condition to obtain a LSPR is given by the limit when the denominators of the Eqs. \eqref{eq:MieCoef} tend to zero, that is
%
\begin{align}
 	\xi_\ell(x)\psi_\ell'(mx)-m\psi_\ell(mx)\xi_\ell'(x) &\to 0, \qquad\qquad \text{(Electric LSPR),}
 		\label{eq:E-LSPR}\\
 	m\xi_\ell(x)\psi_\ell'(mx)-\psi_\ell(mx)\xi_\ell'(x) &\to 0, \qquad\qquad \text{(Magnetic LSPR),}
 		\label{eq:M-LSPR}
\end{align}
%
where $\psi_\ell( \rho) = \rho j_\ell(\rho)$ and $\xi(\rho) = \rho h_\ell^{(1)}(\rho)$  are the Riccati-Bessel functions, the operator $(')$ denotes the derivative respect to their argument,  $x= 2\pi n_\text{m} (a/\lambda)$ is the size parameter, with $a$ the radius of the particle and $\lambda$ the wavelength of the incident plane wave, and $m =  n_\text{p} / n_\text{mat}$ is the contrast between the refractive indices of the particle ($n_\text{p}$) and the matrix ($n_\text{m}$), both of which are in general wavelength dependent. A more closed condition for the  LSPR can be achieved by proposing a model for the refractive index of the particle as it is done in \cite{maciel_escudero_linear_2017} where the Drude Model ---see Eq. \eqref{eq:Drude} in Appendix \ref{app:SizeCorrection}--- is employed nevertheless, the roots in Eqs. \eqref{eq:E-LSPR} and \eqref{eq:M-LSPR} can be found numerically.

The system of interest in this work consists of a spherical gold nanoparticle (AuNP) of  radius $a = 12.5$ nm, whose experimental dielectric function is reported by \citeauthor{johnson_optical_1972} \cite{johnson_optical_1972}.  This experimental data corresponds to a bulk sample, meaning that it may not reproduce the optical behavior of a NP since surface effects cannot to be neglected  due to their spatial dimensions \cite{noguez_surface_2007}. In order to study  the optical properties of AuNP, for example to determine the conditions for its LSPRs from   Eqs. \eqref{eq:E-LSPR}  and \eqref{eq:M-LSPR}, while considering  surface effects,  a size correction to the dielectric function of the AuNP was performed as described in Appendix \ref{app:SizeCorrection}. A more detailed discussion on such size effects is performed by analyzing the far field regime in the next section.

The induced electric field $\vb{E}^\text{ind}$, that is the scattered and internal electric fields, of a spherical AuNP of radius $a = 12.5$ nm were calculated at the conditions of the dipolar ($\ell = 1$) LSPR when the AuNP is embedded into an air matrix ($n_\text{mat} = 1$) and when it is illuminated by an $x$-polarized electric field $\vb{E}^\text{i}$ traveling in the $z$ direction. In Fig. \ref{fig:NearField} the norm of $\vb{E}^\text{ind}$ is evaluated at the plane $y = 0$ [Fig. \ref{fig:NearField:par}] where the incident electric field is parallel ($\parallel$) to the scattering plane, and at the plane $x = 0$ [Fig. \ref{fig:NearField:perp}]  where the incident electric field is perpendicular ($\perp$) to the scattering plane; in both figures the dashed lines corresponds to the surface of the AuNP. The excitation wavelength $\lambda$ of the LSPR for the described system was calculated by employing the size corrected dielectric function for the AuNP in Eq. \eqref{eq:E-LSPR}.

% --------------------------------------				 NearFields 	   ------------------------------
% --------------------------------------   12.5 nm AuNP @ Air     ------------------------------
% --------------------------------------         fig:NearField         -------------------------------
\begin{figure}[t!]
	\def\svgwidth{\textwidth} \small\centering
		\vspace*{4.em}
		\hspace*{-.45\textwidth}
	\begin{subfigure}{.49\textwidth}%
		\caption{ } \label{fig:NearField:par}%
		\end{subfigure}%
	\begin{subfigure}{.49\textwidth}%
		\caption{ }\label{fig:NearField:perp}%
		\end{subfigure}%
	\vspace*{-7.em}\\
	\includeinkscape{Near/4-NearField}
	\vspace*{-2em}
	\caption[Induced Electric Field of a 12.5 nm Au Spherical NP Embedded into Air at the LSPR]{Induced electric field $\vb{E}^\text{int}$ evaluated at the planes \textbf{a)} $y = 0$ and \textbf{b)} $x = 0$  of a 12.5 nm Au spherical NP (dashed lines) embedded into air ($n_\text{mat} = 1$) when illuminated by an incident plane wave with an $x$ polarized electric field $\vb{E}^\text{i}$ traveling in the direction $\vb{k}^\text{i}$ along the $z$ axis with an	 excitation wavelength $\lambda = 509$ nm of the LSPR. At the plane $x = 0$, the incident electric field is parallel to the scattering plane, while it is perpendicular to it at $x = 0$. The optical response of the 12.5 nm AuNP was modeled by a size correction to the experimental data reported by \citeauthor{johnson_optical_1972} \cite{johnson_optical_1972}.}
	\label{fig:NearField}
 \end{figure}
 % --------------------------------------         fig:NearField         -------------------------------

By comparing the magnitude of the induced electric $\vb{E}^\text{ind}$ field in Fig. \ref{fig:NearField} outside the AuNP, which was calculated up to  the multipolar contribution of $\ell = 7$ accordingly with the Wacombe criteria for convergence \cite{bohren_absorption_1983}, with the electric dipolar contribution of the scattered electric field in Fig. \ref{fig:VSH:a1}, the same contour pattern is found. The norm of $\vb{E}^\text{ind}$ evaluated at a plane parallel to the scattering plane ($y=0$) shows a contour pattern of two lobes, which is characteristic of an electric dipole. When the induced electric field is evaluated at a perpendicular plane relative to the scattering plane ($x = 0$), the pattern observed corresponds to the azimuthal symmetry of the dipolar electric field. Lastly, it can be seen that there is an enhancement of $\sim 2$ times of the induced electric field relative to the incident electric field at the surface of the AuNP in the direction parallel to the incident  electric field  [reddish zones in Fig. \ref{fig:NearField:par}]; such enhancement corresponds to the LSP.

From the analysis of the electric field scattered by the particle in the near field regime, the LSP can be visualized on the surface of the particle at the conditions imposed by Eq. \eqref{eq:E-LSPR} and \eqref{eq:M-LSPR} for the electric and magnetic multipoles, respectively. The conditions to excite the LSPR are dictated by the Mie coefficients, therefore the LSPR can also be identified in the far field regime since the amplitude scattering matrix, from which  any optical properties in the far field regime can be calculated, is written according to Eq. \eqref{eq:FscaS} in terms of $a_\ell$ and $b_\ell$. In the following section, the optical properties in the far field regime are calculated and their relation to the LSPR is established, yielding to the observation of the LSPR in the far field.
