% !TeX root = ../tesis.tex

To compare the optical response of a NP in the presence of a substrate with that of a NP in a totally homogeneous environment, let us first analyze the spectral response given by the Mie Theory when the matrix and the size of the NP varies. In Fig. \ref{fig:Mie:redshift} it is shown the wavelength
 of resonance $\lambda_\text{res}$, that is the wavelength at which the scattering (orange) and extinction (black) efficiencies are maximized,  as function of the radius $a$ of a AuNP embedded in a matrix of air [Fig. \ref{sfig:red:1}] and of glass [Fig. \ref{sfig:red:2}], with a refractive index of $n_\text{m} = 1$ and $n_\text{m} = 1.5$, respectively, and as function of the refractive index of the matrix $n_\text{m}$ for a AuNP with a radius of {$a = 12.5$ nm} [Fig. \ref{sfig:red:3}] and with a radius of $a = 50$ nm [Fig. \ref{sfig:red:4}]. For the optical response of the AuNP it was employed the experimental data as reported (filled circles) by \citeauthor{johnson_optical_1972} \cite{johnson_optical_1972}  and by considering a size correction to it (empty circles).

\begin{figure}[h!]
    \def\svgwidth{\textwidth}
    \includeinkscape[pretex = \small]{Redshift/redshift}
    \vspace*{-22em} \\
    \hspace*{-3.2em}%
        \begin{subfigure}{.24\textwidth}\caption{ }\label{sfig:red:1}\end{subfigure}%
        \begin{subfigure}{.24\textwidth}\caption{ }\label{sfig:red:2}\end{subfigure}%
        \begin{subfigure}{.25\textwidth}\caption{ }\label{sfig:red:3}\end{subfigure}%
        \begin{subfigure}{.24\textwidth}\caption{ }\label{sfig:red:4}\end{subfigure}
    \vspace*{17.5em}\\
    \caption[Spectral redshift of the scattering and extinction  efficiencies of a spherical AuNP as function of its size and the embedding media]{Resonance wavelength $\lambda_\text{res}$ of the scattering (orange) and extinction (black) efficiencies of a AuNP as function of the NP's radius when embedded \textbf{a)} into air ($n_\text{m} = 1)$ and \textbf{b)} into glass ($n_\text{m} = 1.5$), and as function of the refractive index of the matrix  $n_\text{m}$ for a AuNP of radius equal to  \textbf{c)} 12.5 nm and \textbf{d)} 50 nm, employing the dielectric function for the Au as reported by \citeauthor{johnson_optical_1972} (filled circle) and considering a size correction to it (empty circle).}
    \label{fig:Mie:redshift}
\end{figure}

From the results shown in Fig. \ref{fig:Mie:redshift} it can be seen that, in general, the wavelength of resonance  $\lambda_\text{res}$ for the extinction is smaller than for the scattering  and that the distance between them decreases as either the size of the AuNP or the refractive index of the matrix increase, meaning that the contribution of the scattering to the extinction of light becomes larger alongside these parameters, as discussed in Section \ref{ss:AuMie}. An increase in $a$ or in $m_\text{m}$ diminishes the difference between the results with and without a size correction, as shown in Figs. \ref{sfig:red:1}, \ref{sfig:red:2} and \ref{sfig:red:4},  however if these results are contrasted  it can be noted on the one hand that the resonance wavelength only redshifts as $a$ grows when no size correction is considered (filled cirlces) but that a size corrected dielectric function (empty circles) give rise to a blueshift of $\lambda_\text{res}$ for values of radius $\lessapprox 15/n_\text{m}$ as seen if Figs. \ref{sfig:red:1} and \ref{sfig:red:2}. On the other hand, an increase in $n_\text{m}$ for a fixed radius presents only redshifts either with or without a size corrected dielectric function [see Fig. \ref{sfig:red:3}].

The spectral behavior of the scattering and extinction of light due to a spherical NP summarized in Fig. \ref{fig:Mie:redshift} was calculated by assuming a homogeneous medium (the matrix) embedding the NP and thus allowing the direction of the illuminating plane wave to be arbitrary, yet yielding the same results. In the following sections, the homogeneity of the surroundings of the NP is substituted by two semiinfinite media and thus modifying the optical response of the system depending on how it is illuminated.
