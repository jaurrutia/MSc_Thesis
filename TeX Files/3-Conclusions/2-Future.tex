% !TeX root = ../tesis.tex

In this thesis the optical response of a single spherical AuNP of radius 12.5 nm partially embedded in an air matrix and a glass substrate were studied and it was de terminated the conditions at which the system is  able to interact with the its surrounding above the substrate and  at which its optical response is maximized. This features suit the 12.5~nm~AuNP as a candidate for the meta-atoms conforming a metasurface tailored for biosensing.  While the system of interest consisted of a spherical AuNP of specific size, the results in this thesis are valid as long as the AuNP size allows for the dipolar approximation to be valid. In this sense, theoretical models that describe the optical properties of bidimensional arrays of small particles supported on a substrate, like the Thin Island Theory \cite{bedeaux_optical_2004} or the Dipolar Model \cite{barrera1991optical} which are Effective Medium Theories (EMT), can be employed for partially embedded AuNPs as well. To include the partial embedding in such EMT, it is proposed to modify an homogenization theory for the medium surrounding the nanosphere ---like the Bruggeman or power law formulas \cite{sihvola_electromagnetic_2008}---, so that the LSPR of a single nanosphere is excited at the wavelength at which the LSPR of the partially embedded nanosphere is, which does not only depend on the incrustation parameter but also on the polarization and direction of the incident electric plane wave. Once a homogenization theory for the surroundings of the partially embedded nanosphere is developed, EMT can be used considering that the nanospheres are perfectly supported on a substrate and embedded in the homogenized medium, then the optical response of the real system can be described with a stratified system of three layers: substrate-homogenized EMT-matrix. It is expected that this approach can describe the optical properties of biosensing-aimed-metasurfaces with the embedding feature that can allow them to be used long-lastingly under realistic conditions.
