%%%%%%%%%%%%%%%%%%%%%%%%%%%%%%%%%%%%%%%%%%%%%%%%%%%%%%%%%%%%%%%%%%%%%%%%%%%%%%%%
%                     Bachelor thesis template 
%						 Science Faculty 
%								at 
%			National Autonomous University of Mexico (UNAM)
%%%%%%%%%%%%%%%%%%%%%%%%%%%%%%%%%%%%%%%%%%%%%%%%%%%%%%%%%%%%%%%%%%%%%%%%%%%%%%%%
% based on Harish Bhanderi's PhD/MPhil template, then Uni Cambridge
% http://www-h.eng.cam.ac.uk/help/tpl/textprocessing/ThesisStyle/
% corrected and extended in 2007 by Jakob Suckale, then MPI-iCBG PhD programme
% and made available through OpenWetWare.org - the free biology wiki
%
%                     Under GNU License v3
%
% Adapted for the Engineering School at UNAM by Jesús Velázquez y Marco Ruiz
% Then, adapter fot he Science Faculty at UNAM by Jonathan Urrutia
%
%
%
% All used packages are found in  
%
%					./Latex/Classes/PhDthesisPSnPDF.cls
%
% within this las documments there are some lines to be UnCommented for a printed or a digital version of the output file:
%
%			line 184
%			lines 231-261
%
%	Since this is a template for a thesis made at UNAM (Mexico) the titles may be in spaninish but within PhDthesisPSnPDF.cls it can be change
%

\documentclass[11pt]{Latex/Classes/PhDthesisPSnPDF}

\usepackage{blindtext}         %  For dummy text.
% The \blindtext or \Blindtext commands throughout this template generate dummy text to fill the template out.

\include{Latex/Comands}           % Special commands written by the author

\usepackage{setspace}

%%-------------------------------------------------------------------------------
%                                  Information of the Student                   
%-------------------------------------------------------------------------------
% --- Default information for Bachelor's degree
%\author{Jonathan Alexis Urrutia Anguiano} 
%\title{Optical response of partially embedded nanospheres}
%\programa{Posgrado en Ciencias Físicas} % Licenciatura en Física
%\degree{Maestro en Ciencias}% Maestro en / Doctor en
%\director{Dr. Alejandro Reyes Coronado}% Thesis director
%\facultad{Facultad de Ciencias}
%\lugar{Ciudad de México, México}% Place of the dissertation
%\degreedate{2021}% Year of the dissertation
%\portadatrue   %Uncomment for color cover


% --- Default information for Grad degree
\posgradotrue
\author{Jonathan Alexis Urrutia Anguiano} 
\title{Optical response of partially embedded nanospheres}
\programa{Posgrado en Ciencias Físicas} % Licenciatura en Física
\degree{Maestro en Ciencias}% Maestro en / Doctor en
\director{Dr. Alejandro Reyes Coronado}% Thesis director
\directordep{Facultad de Ciencias, UNAM}
\lugar{Ciudad de México, México}% Place of the dissertation
\degreedate{2021}% Year of the dissertation
\campo{Física}

\comitetrue

\ctutoruno{Dra. Cirlali Sánchez-Aké}
\ctutorunodep{Instituto de Ciencias Aplicadas y Tecnología, UNAM}
\ctutordos{Dr. Giuseppe Pirrcuccio}
\ctutordosdep{Instituto de Física, UNAM}





% ----------------------------- Datos del jurado de Licenciatura
%\student{Paternal last name\\ Maternal Last name\\ Names\\ Telephone number\\ Universidad Nacional Autónoma de México\\ Facultad de Ciencias\\ Física\\ Student Number}
%\secretario{Dr \\ Secretary (thesis director) \\ Last name \\ Last name}
%\presidente{Dr \\ President \\ Last name \\ Last name}
%\vocal{Dr \\ Vocal \\ Last name \\ Last name}
%\supuno{Dr \\ substitute 1 \\ Last name \\ Last name}
%\supdos{Dr \\ Substitute 2 \\ Last name \\ Last name}
%\pags{pages}

\keywords{tesis,autor,tutor,etc}            % For metadata 
\subject{tema_1,tema_2}                     % Subjects for metadata














%-------------------------------------------------------------------------------
%                                   COVER                                   
%-------------------------------------------------------------------------------
\begin{document}
%
\maketitle									
%-------------------------------------------------------------------------------
%                                   FRONT MATTER                                
%-------------------------------------------------------------------------------
\frontmatter

%% !TeX root = ../tesis.tex

\begin{acknowledgements}
\addcontentsline{toc}{chapter}{\protect\numberline{}Acknowledgements}


Agradezco al CONACyT  por la beca de estudios de posgrado que me otorgó por dos años,a sí como el apoyo por parte del proyecto de investigación DGAPA-UNAM PAPIIT IN107122.

\end{acknowledgements}




          
%\include{0-2-Declaration/1-Declaration}
%\begin{dedication}

``Desde el alba hasta entrada la noche\\
No cesó el funeral clamoreo:\\
¡Qué pompa! ¡Qué lujo!\\
¡Qué fausto! ¡Qué entierro!''

Todas las campanas con eco pausado

Rosalía de Castro
\end{dedication}
                  
% !TeX root = ../tesis.tex

% Thesis Abstract -----------------------------------------------------

%\begin{abstractslong}    %uncommenting this line, gives a different abstract heading
\begin{abstracts}        %this creates the heading for the abstract page
\addcontentsline{toc}{chapter}{\protect\numberline{}Abstract}
\vfill
\small
Plasmonic metasurfaces, metallic nanostructures supported on a substrate, have been used as alternatives for biosensing due to their low-cost and easy-to-use features, and due to their light enhancement and confinement capacity. In common biosensing techniques, a liquid flows over the substrate, where the nanostructure is located, so there is a  detachment risk. Therefore, a partial embedding of the nanostructure in the substrate is desirable, which modifies its optical response under ideal conditions. In this thesis, it is studied the optical response of a single spherical gold nanoparticle of radius 12.5 nm, suited for biosensing-aimed-metasurfaces, when the nanosphere is partially embedded in between an air matrix and glass substrate, both which form a flat interface, and illuminated by an electromagnetic plane wave with wavelengths in the optical range, considering two states of polarization as well as different angles of incidence. The optical properties of the partially embedded nanosphere, that is, the scattering, absorption and extinction cross sections and the induced electric field in the near and far-field regimes, are calculated by means of the Finite Element Method and compared with the analytical solutions of two limiting cases: a nanosphere embedded in an infinite matrix of air, and in an infinite matrix of glass. Based on the obtained numerical results, it was determined optimal configurations for  biosensing with a disordered metasurface of partially embedded nanosphere of radius 12.5 nm in the diluted regime.\\[2em]

Las metasuperfices plasmónicas, nanoestructuras metálicas soportadas por un sustrato, han sido utilizadas como alternativas para el biosensado por su bajo costo de fabricación y fácil uso, debido a su capacidad de realce y confinamiento de la luz. En el proceso de biosensado, es común que un líquido fluya por encima del sustrato donde se encuentra la nanoestructura, por lo que existe un riesgo de desprendimiento de la misma. Por tanto, es deseable una incrustación parcial de la nanostructura en el sustrato, lo que modifica su respuesta óptica en condiciones ideales. En esta tesis, se estudia la respuesta óptica de una sola nanopartícula esférica de oro de 12.5 nm de radio, adecuada para metasuperficies de biosensado, cuando la nanoesfera se incrusta parcialmente en una sustrato plano de vidrio con una matriz de aire, e iluminada por una onda plana electromagnética con longitudes de onda en el rango óptico, considerando los dos estados de polarización así como diferentes ángulos de incidencia. Las secciones transversales de esparcimiento, absorción y extinción, así como el campo eléctrico inducido por la nanoesfera en los regímenes de campo cercano y lejano, se calculan con el método de elementos finitos y se comparan con las soluciones analíticas en dos casos límite: una nanoesfera embebida en una matriz infinita de aire, y en una matriz infinita de vidrio. Con base en los resultados numéricos obtenidos, se encontraron configuraciones óptimas para el biosensado considerando una metasuperficie desordenada conformada por nanoesferas de oro de 12.5 nm de radio en el régimen diluido.



\end{abstracts}
%\end{abstractlongs}


% ----------------------------------------------------------------------
       

%-------------------------------------------------------------------------------
%                                INDICES                                    |
%-------------------------------------------------------------------------------
%
\setcounter{secnumdepth}{3} % organisational level that receives a numbers
\setcounter{tocdepth}{3}    % print table of contents for level 3

\tableofcontents            % Print main index

%: ----------------------- list of figures/tables ------------------------
%\listoffigures              % Genera el ínidce de figuras, comentar línea si no se usa
%\listoftables               % Genera índice de tablas, comentar línea si no se usa


%-------------------------------------------------------------------------------
%                                MAIN MATTER                                   %-------------------------------------------------------------------------------
% the main text starts here with the introduction, 1st chapter,...
\mainmatter

\def\baselinestretch{1}                   % Line spacing

% !TeX root = ../tesis.tex

\chapter*{Introduction}
\addcontentsline{toc}{chapter}{\protect\numberline{}Introduction}	  		% Comment if you don't want the introduction to appear on the table of content. It will not have a number
\label{chapter:intro}

% this file is called up by thesis.tex
% content in this file will be fed into the main document

 It is recommended to fill in this part of the document with the following information:

\begin{itemize}
	\item Your field: Context about the field your are working \\
	\textbf{Plasmonics -> Metameterials -> Biosensing}
	\item Motivation: Backgroung about your thesis work and why did you choose this project and why is it important.\\
	\textbf{Fabrication -> Partially embedded NPs -> No analytical (approximated) method physically introduces the incrustation degree. There are numerical solutions and Effective Medium Theories approaching the problem but the later only as a fitting method. }
	\item Objectives: What question are you answering with your work.\\
	\textbf{Can optical non invasive tests (IR-Vis) retrieve the average incrustation degree for monolayers of small spherical particles?}
	\item Methology: What are your secondary goals so you achieve your objective. Also, how are you answering yout question: which method or model.\\
	\textbf{Bruggeman homogenization theories on bidimensional systems?\\
	Is the dipolar approximation is enough or do we need more multipolar terms?\\
	Do we need the depolarization factors?}
	\item Structure: How is this thesis divides and what is the content of each chapter.
\end{itemize}
            

\chapter{Optical properties of single plasmonic nanoparticles}
	\section{Mie Theory: Quasi-static Approximation}
	\section{Depolarization Factors}
	\section{Substrate effects}

\chapter{Collections: Effective Medium Theories}
		\section{3D theories: summary}	
			\subsection{Bruggemann \& Maxwell Garnett}
		\section{2-D Arrays}
			\subsection{Island Theory}
			\subsection{Dipolar Model}

\chapter{Results and discussion}
	\section{Single particle: Incrustation degree (COMSOL + approximate solutions)}
	\section{Analytical extention to a monolayer}

\chapter{Conclusions}

\appendix
\chapter{Finite Element Method}
%
%\section{Template}
%% !TeX root = ../tesis.tex

\chapter{Theory}
\label{chapter:theory}

\vspace*{7em}

It is recommended to write a summary about the contents of this chapter as an introduction to them. 

\blindtext

\section{The Basics}
\label{section:basics}

If you want to frame some equations because you consider them important, use the \textbf{tcolorbox} command. Also, you may use the \textbf{subequations} command sometimes. Fot example with the Maxwell's equations \cite{griffiths2013electrodynamics}:\vspace*{-.75em}
%
	\begin{subequations} \label{eqs:Maxwell}
	\begin{tcolorbox}[title = Ecuaciones de Maxwell en el sistema internacional de unidades,
	ams align, breakable]
	\nabla \cdot\vb{E} &= \frac{\rho_{tot}}{\varepsilon_0}, &\mbox{(Ley de Gauss eléctrica)}  
	\label{seq:GE} \\
	\nabla \cdot\vb{B} &= 0,						&\mbox{(Ley de Gauss magnética)}   
	\label{seq:GM} \\
	\nabla \times\vb{E} &= -\pdv{\vb{B}}{t}, 	&\mbox{(Ley de Faraday-Lenz)}		
	\label{seq:FL}\\
	\nabla \times\vb{B} &= \mu_0 \vb{J}_{tot} +\varepsilon_0\mu_0 \pdv{\vb{E}}{t}, &
	\mbox{(Ley de Ampère-Maxwell)} \label{seq:AM}
	\end{tcolorbox}\end{subequations}\vspace*{-.75em}\noindent
%
and if you want them to appear in the analytical index just use the \textbf{index} command \textbackslash index\{ \}.\index{Maxwell!ecuaciones de}

If you want to show two equations in only one row, use the macros \textbf{\textbackslash eqhalf}, for example\cite{hecht1998optics} \index{Ecuación!de onda}, for the Fourier transform \footnote{\setstretch{1.0} $\mathcal{F}[f(\vb{r},\omega)] = \int_{-\infty}^\infty f(\vb{r},t) e^{i(\vb{k}\cdot\vb{r} -\omega t)} dt$, con $\vb{k}$ una función de $\omega$. La transformada de Fourier inversa es entonces $\mathcal{F}^{-1}[f(\vb{r},t)] =\frac{1}{2\pi} \int_{-\infty}^\infty f(\vb{r},\omega) e^{i(\vb{k}\cdot\vb{r} -\omega t)} d\omega$.\index{Fourier! Transform}} or the Helholtz equation \index{Equation! Helmholtz} for $\vb{E}$ y $\vb{B}$ \cite{griffiths2013electrodynamics}

	\begin{subequations}%
	\eqhalf{\nabla^2\vb{E} + k^2 \vb{E}=\vb{0},}%
	\eqhalf{\nabla^2\vb{B} + k^2 \vb{B}=\vb{0}.}\label{eq:Helmholtz}%
	\end{subequations}\vspace*{-1em}

\noindent leading to plane waves as follow

	\begin{subequations}%
	\eqhalf{\vb{E}(\vb{r},t) =\vb{E_0}e^{i(\vb{k}\cdot\vb{r} -\omega t)},}%
	\eqhalf{\vb{B}(\vb{r}, t) =\vb{B_0}e^{i(\vb{k}\cdot\vb{r} -\omega t),}}	
	\label{eqs:ondasPlanas}\end{subequations}\vspace*{-1em}
		
\noindent \blindtext \vspace*{-.75em}
%
	\begin{tcolorbox}[title = Índice de refracción, ams align]
	n(\omega) = \sqrt{\frac{\mu\varepsilon(\omega)}{\varepsilon_0 \mu_0}}.
		\label{eq:indice} 
	\end{tcolorbox}\vspace*{-.75em}

For the figures, you can use this format:
%
	\begin{figure}[h!]\centering
	\begin{subfigure}{.05\textwidth}%
		\caption{}\label{sfig:secondary1}\vspace*{5cm}
	\end{subfigure}
	\begin{subfigure}{.43\textwidth} 
			\includegraphics[width=\linewidth]{1-Theory/figs/plant}		
	\end{subfigure}
	\begin{subfigure}{.05\textwidth}%
		\vspace{-5cm}\caption{}\label{sfig:secondaty2}
		\end{subfigure}
	\begin{subfigure}{.43\textwidth} 
			\includegraphics[width=\linewidth]{1-Theory/figs/plant}
	\end{subfigure}%
	\vspace*{-.25cm}
	\caption[Example of Figure title]{The explanation of your figures. \blindtext}	\label{fig:Main}	
	\end{figure}	
				
\Blindtext

%	% !TeX root = ../tesis.tex


\section{Somethinf more specific}
\label{section:basics}
				
\Blindtext

%% !TeX root = ../tesis.tex


\chapter{Results}

\section{What I got}
\label{section:results}
				
\Blindtext

%% !TeX root = ../tesis.tex
\chapter*{Conclusions}\addcontentsline{toc}{chapter}{\protect\numberline{}Conclusions}
\label{chapter:concl}

Make a short summary of the content of the thesis \cite{reyes2018analytical,pena-gomar2006coherent,barrera1991optical,garcia2012multiple} and then your conclusions. Also explain your future steps on this project 

\blindtext

\blindtext
        
%
%%-------------------------------------------------------------------------------
%%                               References                                   |
%%-------------------------------------------------------------------------------
%
%\appendix
%\input{5-Apendices/1-Something_extra.tex}  

\setlength\bibitemsep{.1\itemsep}
\printbibliography

\newpage
\listoffigures

\printindex
%-------------------------------------------------------------------------------
%                              Appendix                                   |
%-------------------------------------------------------------------------------


           
\end{document}
