% !TeX root = ../tesis.tex

In the past section, it was studied the optical properties of a spherical AuNP with radius $a = 12.5$ nm when this was located at the interface between a glass substrate and an air matrix and illuminated in the normal direction to the glass-air interface. On the one hand, it was found that the AuNP can be still described by a mostly dipolar contribution, as in the Mie-limiting case (AuNP in a homogeneous medium) even though the homogeneity and symmetry of its surroundings is broken. On the other, the greatest near-field enhancement was localized on  the surface of the AuNP in contact with the substrate, which is undesired if the partially embedded AuNP is to be used as the unit cell for a bidimensional array suited for biosensing. In order to find an optimal configuration of the system suited for interactions above the substrae, in this Section the optical properties of a partially embedded AuNP are analyzed considering an oblique incidence to the system (both below and above the critical angle), meaning that the the AuNP is illuminated either with an $s$ or a $p$ polarized incident electric field.

The absorption $Q_\text{abs}$ and scattering $Q_\text{sca}$ efficiencies of partially embedded  AuNP with a radius $a =12.5$ nm in a glass substrate ($n_\text{s} = 1.5$) and in an air matrix ($n_\text{m} = 1$) ---for different values of the incrustation parameter $h/a$, with $h$ the distance between the center of the AuNP and the interface--- are shown in Figs. \ref{fig:Inc:Abs} and \ref{fig:Inc:Sca}, respectively, as function of the wavelength $\lambda$ of the incident electric field $\vb{E}^\text{i}$ illuminating the AuNP from the substrate at an incidence angle $\theta_i < \theta_c = 41.8^\circ$ of $15^\circ$ [Figs. \ref{sfig:Inc:Abs:15} and \ref{sfig:Inc:Sca:15}] and  $38^\circ$ [Figs. \ref{sfig:Inc:Abs:38} and \ref{sfig:Inc:Sca:38}], and  at a value of $\theta_i>\theta_c$, thus forcing the interaction between an evanescent wave and the AuNP, equal to $42^\circ$  [Figs. \ref{sfig:Inc:Abs:42} and \ref{sfig:Inc:Sca:42}] and $75^\circ$  [Figs. \ref{sfig:Inc:Abs:75} and \ref{sfig:Inc:Sca:75}], considering an $s$ (filled circle/solid lines) and a $p$ (empty circle/dashed lines) polarization for $\vb{E}^\text{i}$. To compare to obtained results with the the Mie-limiting cases (AuNP embedded in air and in glass) ---green shaded region and red markers corresponding to the resonance of the absorption and scattering efficiencies---, the values of $Q_\text{abs}$ ($Q_\text{sca}$) evaluated at their wavelength of resonance $\lambda_\text{res}^\text{abs}$ ($\lambda_\text{res}^\text{sca}$) are signalized by the magenta markers and the numerical values of the later can be found in Table \ref{tab:Resonances}, where the saturation of the cell colors corresponds to a larger wavelength. Lastly, in Figs. \ref{fig:Inc:Abs} and \ref{fig:Inc:Sca} the gray continuous and dashed lines are a guides to the eye joining the resonances of the absorption and scattering efficiencies in each case considering an $s$ and a $p$ polarized incident electric field.

\begin{figure}[t!]\small \centering
    \hspace*{-.675\textwidth}%
        \begin{subfigure}{.735\textwidth}\caption{ }\label{sfig:Inc:Abs:15}\end{subfigure}%
        \begin{subfigure}{.25\textwidth}\caption{ }\label{sfig:Inc:Abs:38}\end{subfigure} \\[17em]
    \hspace*{-.675\textwidth}%
        \begin{subfigure}{.735\textwidth}\caption{ }\label{sfig:Inc:Abs:42}\end{subfigure}%
        \begin{subfigure}{.25\textwidth}\caption{ }\label{sfig:Inc:Abs:75}\end{subfigure} \\[-19.9em]
    \def\svgwidth{.95\textwidth}
    \includeinkscape{4-Inc-Obl/1-Efficiencies/1-Oblique-Inc-Abs}%
    \vspace*{-.5em}
    \caption[Absorption Efficiency of a partially embedded 12.5 nm AuNP into a substrate Illuminated in an internal configuration at oblique incidence]{%
    Absorption efficiency of a $12.5$ nm AuNP partially embedded in a glass substrate ($n_\text{s} = 1.5$) with an air matrix ($n_\text{m} = 1$) as function of the wavelength $\lambda$ of an \textit{s} (filled circle/solid lines) and a \textit{p} (empty circle/dashed lines) polarized incident electric plane wave propagating in the direction of the wave vector $\vb{k}^\text{i}$, in an internal configuration, at an angle of incidence $\theta_i$ of \textbf{a)} $15^\circ$, \textbf{b)} $38^\circ$, \textbf{c)} $42^\circ$ and \textbf{d)} $75^\circ$  relative to the normal direction to the glass-air interface. The green shaded region shows the two Mie-limiting cases of a AuNP embedded in air
and in glass; the magenta (partially embedded AuNP) and red (Mie-limiting) markers corresponds to the efficiencies evaluated at the wavelength of resonance for each case; the gray (gray dashed) line is a guide to the eye for the \textit{s} (\textit{p}) polarization case.
}
\label{fig:Inc:Abs}
\end{figure}

In Figs. \ref{fig:Inc:Abs} and \ref{fig:Inc:Sca} it can be seen that the absorption and scattering efficiencies, for each combination of $\theta_i$ and $h/a$, present a general redshift of the resonance wavelengths $\lambda_\text{res}^\text{abs}$ and $\lambda_\text{res}^\text{sca}$ as $h/a$ decreses, while preserving only one observable resonance in the visible range in the intervals $509\text{ nm} < \lambda_\text{res}^\text{abs} < 535 \text{ nm}$ and  $522\text{ nm} < \lambda_\text{res}^\text{abs} < 545 \text{ nm}$ , whose extreme values (red markers) corresponde to the resonance wavelengths for the Mie-limiting cases: AuNP in air and in glass. Besides the redshift of the resonances, it is observed an overall growth ---or decrease depending on the choice of incrustation parameter and angle of incidence--- for all wavelengths on the absorption and scattering efficiencies, which are integral optical properties and thus an average response of the system. Therefore, these changes can be identified as the combination of the effects due to  the AuNP interacting with a plane or an evanescent wave as discussed in Section\ref{s:Emb:Obl} and due to of the embedding of the AuNP discussed in Section \ref{s:Emb:Normal}.

A general effect of the choice of $\theta_i$ on $Q_\text{abs}$ and $Q_\text{sca}$ is their enhancement, in the visible range, for incidence angles near the critical angle $\theta_c = 41.8^\circ$, as it can be seen from the absorption and scattering efficiencies, for both the $s$ and $p$ polarization cases, at $\theta_i = 42^\circ$ [Figs. \ref{sfig:Inc:Abs:42} and \ref{sfig:Inc:Sca:42}], which are larger, for all $\lambda$, than the equivalent results for incidence angles of $15^\circ$, $38^\circ$ and $75^\circ$. This observation can be verified by comparing the vertical axis scale with a maximum value of $\sim 10.0$ in Fig. \ref{sfig:Inc:Abs:42} and  $\sim 10.1$ in Fig. \ref{sfig:Inc:Sca:42} ---absorption and scattering efficiencies, respectively--- with the maximum value of the axis scale for $\theta_i = 15^\circ$ in Figs. \ref{sfig:Inc:Abs:15} ($\sim 1.2$) and \ref{sfig:Inc:Sca:15} ($\sim 5.3$) and for $\theta_i = 38^\circ$ in  Figs. \ref{sfig:Inc:Abs:38} ($\sim 4.1$) and \ref{sfig:Inc:Sca:38} ($\sim 9.1$) ---both angles below $\theta_c$---; on a similar manner, the shown absorption and scattering efficiencies, for a fixed polarization state and incrustation parameter, are the smallest for $\theta_i = 75^\circ$ as shown by the maximum value of the vertical axes of $\sim 1.0$ in Figs. \ref{sfig:Inc:Abs:75} and \ref{sfig:Inc:Sca:75}. A similar analysis based on the Mie-limiting cases can be performed by comparing the values of the continuous and dashed curves relative to the green shaded region, yielding the same results.

To notice the effect of the  polarization state ---modulated by the AuNP's embedding--- of the incident electric field $\vb{E}^\text{i}$ in the absorption and scattering efficiencies, let us recall the behavior observed for a normally illuminated partially embedded AuNP in Section \ref{s:Emb:Normal}, where the values of $Q_\text{abs}$ and $Q_\text{sca}$ grew uniformly, for all wavelengths in the visible rage, as the incrustation parameter $h/a$ changes from $1$ to $-1$, that is, as the AuNP is buried into the substrate. Such behavior can be identified in the continuous curves, corresponding to an $s$ polarized $\vb{E}^\text{i}$, for any angle of incidence in Figs. \ref{fig:Inc:Abs} (absorption efficiency) and \ref{fig:Inc:Sca} (scattering efficiency). This can be explained by the direction of the electric field not changing across the glass-air interface due to the continuity of the parallel components of the electric filed at any boundary: for an $s$ polarized $\vb{E}^\text{i}$, the electric field illuminating the partially AuNP only changes in amplitude on its boundary, even for $\theta_i>\theta_c$ yielding a smooth increase in the average optical properties ($Q_\text{abs}$ and $Q_\text{sca}$) of the AuNP as its surroundings become optically denser.

\begin{figure}[t!]\small \centering
    \hspace*{-.675\textwidth}%
        \begin{subfigure}{.735\textwidth}\caption{ }\label{sfig:Inc:Sca:15}\end{subfigure}%
        \begin{subfigure}{.25\textwidth}\caption{ }\label{sfig:Inc:Sca:38}\end{subfigure} \\[17em]
    \hspace*{-.675\textwidth}%
        \begin{subfigure}{.735\textwidth}\caption{ }\label{sfig:Inc:Sca:42}\end{subfigure}%
        \begin{subfigure}{.25\textwidth}\caption{ }\label{sfig:Inc:Sca:75}\end{subfigure} \\[-19.9em]
    \def\svgwidth{.95\textwidth}
    \includeinkscape{4-Inc-Obl/1-Efficiencies/2-Oblique-Inc-Sca}%
    \vspace*{-.5em}
    \caption[Scattering Efficiency of a partially embedded 12.5 nm AuNP into a substrate Illuminated in an internal configuration at oblique incidence]{%
    Scattering efficiency of a $12.5$ nm AuNP partially embedded in a glass substrate ($n_\text{s} = 1.5$) with an air matrix ($n_\text{m} = 1$) as function of the wavelength $\lambda$ of an \textit{s} (filled circle/solid lines) and a \textit{p} (empty circle/dashed lines) polarized incident electric plane wave propagating in the direction of the wave vector $\vb{k}^\text{i}$, in an internal configuration, at an angle of incidence $\theta_i$ of \textbf{a)} $15^\circ$, \textbf{b)} $38^\circ$, \textbf{c)} $42^\circ$ and \textbf{d)} $75^\circ$  relative to the normal direction to the glass-air interface. The green shaded region shows the two Mie-limiting cases of a AuNP embedded in air
and in glass; the magenta (partially embedded AuNP) and red (Mie-limiting) markers corresponds to the efficiencies evaluated at the wavelength of resonance for each case; the gray (gray dashed) line is a guide to the eye for the \textit{s} (\textit{p}) polarization case.
}
\label{fig:Inc:Sca}
\end{figure}

Contrastingly, the efficiencies for the $p$ polarization case (empty circles/dashed lines) are uniformly enhanced as $h/a$ decreases ---as in the $s$ polarization case--- only for the angles of incidence of $15^\circ$ [Figs. \ref{sfig:Inc:Abs:15} and \ref{sfig:Inc:Sca:15}] and $75^\circ$ [Figs. \ref{sfig:Inc:Abs:15} and \ref{sfig:Inc:Sca:15}], while for values of $\theta_i$ near $\theta_c = 41.8^\circ$, the dependency of the overall growth of $Q_\text{abs}$ and $Q_\text{sca}$ on $h/a$ is as follows: The absorption efficiency for values of $\theta_i $ in the neighborhood of $\theta_c$, such as for $38^\circ$ and $42^\circ$ [Figs. \ref{sfig:Inc:Abs:38} and \ref{sfig:Inc:Abs:42}, respectively] present an uniform enhancement ---for all $\lambda$--- when $1 \leq h/a < 0.75$ and $0.00< h/a \leq -1.00$ and an uniform diminishment as $h/a$ changes from $0.75$ (orange dashed curve) to $0.50$  (blue dashed curve)  and $0.25$  (light orange dashed curve); a similar observation is made for the sactteringe efficiency for $\theta_i = 38 ^\circ$ and $42 ^\circ$   [Figs. \ref{sfig:Inc:Sca:38} and \ref{sfig:Inc:Sca:42} respectively] where the diminishment of $Q_\text{sca}$ occurs at a incrustation parameter of $0.50 \leq h/a \leq -0.50$ and the enhancement when $h/a$ changes from $1.00$ to $-0.75$ (black and orange dashed curves) and from $-0.75$ to $-1.00$  (light purple and brown dashed curves). Both of the past descriptions of the uniform enhancement and diminishment of $Q_\text{abs}$ and $Q_\text{sca}$ can be easily seen by following the dashed gray lines joining their resonances as the incrustation parameters is changed. Attributing such behavior to the change of the electric field illuminating the AuNP above and below the glass-air interface, it can be seen that for a $p$ polarized incident electric field, the partially embedded AuNP present an average optical responds more similar to that of the supported, due to the strength of the transmitted electric field,  when the incrustation parameter is such that $1.00<h/a<0.75$, that is, when at most one eight of the AuNP's volume/surface is embedded in the substrate; on the other hand, its average optical response resembles that of a totally embedded particle when at most one eight of the AuNP is in the matrix side, that is when  $-0.75>h/a>-1.00$.

\begin{table}[b!]\footnotesize\centering
    \caption{Wavelength of resonance for the absorption $\lambda_\text{res}^\text{abs}$ and the scattering $\lambda_\text{res}^\text{sca}$ efficiencies of a partially embedded 12.5 nm AuNP with a glass substrate ($n_\text{s} = 1.5$) and an air matrix ($n_\text{m} = 1$)    illuminated by an \textit{s} and a \textit{p} polarized electric plane wave traveling to the glass-air interface at an incidence angle of $15^\circ$,  $38^\circ$,  $42^\circ$ and  $75^\circ$ for several values of the incrustation parameter $h/a$ with $h$ the distance between the AuNP and its radius $a$. The values in this table correspond to the magenta markers in Figs. \ref{fig:Inc:Abs} and \ref{fig:Inc:Sca} while the saturation of the cell colors are a guide to the eye.}
    \label{tab:Resonances}
    

\begin{tabular}{l | r | ccccc || ccccc} \hline \hline
                                &       & \multicolumn{5}{c ||}{\small $\lambda_\text{res}^\text{abs}$ [nm]} & \multicolumn{5}{c}{\small $\lambda_\text{res}^\text{sca}$ [nm]}  \\ \hline \hline
                                & $h/a$ & $0^\circ$ & $15^\circ$     & $38^\circ$    & $42^\circ$    & $75^\circ$    & $0^\circ$ & $15^\circ$     & $38^\circ$    & $42^\circ$    & $75^\circ$    \\ \hline
\multirow{9}{*}{\rotatebox{90}{\emph{s} Polarization}}
    & 1.00  & \cellcolor{white!92!gray}510    & \cellcolor{white!92!gray}510    & \cellcolor{white!92!gray}510    & \cellcolor{white!92!gray}510    & \cellcolor{white!84!gray}512.5  & \cellcolor{white!89!orange}525    & \cellcolor{white!89!orange}525    & \cellcolor{white!89!orange}525    & \cellcolor{white!89!orange}525    & \cellcolor{white!89!orange}525    \\
    & 0.75  & \cellcolor{white!76!gray}515    & \cellcolor{white!76!gray}515    & \cellcolor{white!76!gray}515    & \cellcolor{white!76!gray}515    & \cellcolor{white!76!gray}515    & \cellcolor{white!89!orange}525    & \cellcolor{white!89!orange}525    & \cellcolor{white!89!orange}525    & \cellcolor{white!89!orange}525    & \cellcolor{white!89!orange}525    \\
    & 0.50  & \cellcolor{white!60!gray}520    & \cellcolor{white!60!gray}520    & \cellcolor{white!60!gray}520    & \cellcolor{white!60!gray}520    & \cellcolor{white!60!gray}520    & \cellcolor{white!67!orange}530    & \cellcolor{white!67!orange}530    & \cellcolor{white!67!orange}530    & \cellcolor{white!67!orange}530    & \cellcolor{white!67!orange}530    \\
    & 0.25  & \cellcolor{white!52!gray}522.5  & \cellcolor{white!52!gray}522.5  & \cellcolor{white!52!gray}522.5  & \cellcolor{white!52!gray}522.5  & \cellcolor{white!52!gray}522.5  & \cellcolor{white!45!orange}535    & \cellcolor{white!45!orange}535    & \cellcolor{white!45!orange}535    & \cellcolor{white!45!orange}535    & \cellcolor{white!45!orange}535    \\
    & 0.00  & \cellcolor{white!44!gray}525    & \cellcolor{white!44!gray}525    & \cellcolor{white!44!gray}525    & \cellcolor{white!44!gray}525    & \cellcolor{white!44!gray}525    & \cellcolor{white!23!orange}537.5  & \cellcolor{white!23!orange}537.5  & \cellcolor{white!23!orange}537.5  & \cellcolor{white!45!orange}535    & \cellcolor{white!45!orange}535    \\
    & -0.25 & \cellcolor{white!28!gray}527    & \cellcolor{white!28!gray}527    & \cellcolor{white!28!gray}527    & \cellcolor{white!28!gray}527    & \cellcolor{white!28!gray}527    & \cellcolor{white!45!orange}535    & \cellcolor{white!45!orange}535    & \cellcolor{white!45!orange}535    & \cellcolor{white!45!orange}535    & \cellcolor{white!45!orange}535    \\
    & -0.50 & \cellcolor{white!12!gray}530    & \cellcolor{white!12!gray}530    & \cellcolor{white!12!gray}530    & \cellcolor{white!12!gray}530    & \cellcolor{white!12!gray}530    & \cellcolor{white!23!orange}537.5  & \cellcolor{white!23!orange}537.5  & \cellcolor{white!23!orange}537.5  & \cellcolor{white!23!orange}537.5  & \cellcolor{white!23!orange}537.5  \\
    & -0.75 & \cellcolor{white!12!gray}530    & \cellcolor{white!12!gray}530    & \cellcolor{white!12!gray}530    & \cellcolor{white!12!gray}530    & \cellcolor{white!12!gray}530    & \cellcolor{white!12!orange}540    & \cellcolor{white!12!orange}540    & \cellcolor{white!12!orange}540    & \cellcolor{white!12!orange}540    & \cellcolor{white!12!orange}540    \\
    & -1.00 & \cellcolor{white!8!gray}532.5   & \cellcolor{white!8!gray}532.5   & \cellcolor{white!8!gray}532.5   & \cellcolor{white!8!gray}532.5   & \cellcolor{white!8!gray}532.5   & \cellcolor{white!1!orange}542.5   & \cellcolor{white!1!orange}542.5   & \cellcolor{white!1!orange}542.5   & \cellcolor{white!1!orange}542.5   & \cellcolor{white!12!orange}540    \\
\hline\hline
\multirow{9}{*}{\rotatebox{90}{\emph{p} Polarization}}
    & 1.00  & \cellcolor{white!92!gray}510    & \cellcolor{white!84!gray}512.5  & \cellcolor{white!84!gray}512.5  & \cellcolor{white!84!gray}512.5  & \cellcolor{white!84!gray}512.5  & \cellcolor{white!89!orange}525    & \cellcolor{white!89!orange}525    & \cellcolor{white!78!orange}527.5  & \cellcolor{white!78!orange}527.5  & \cellcolor{white!89!orange}525    \\
    & 0.75  & \cellcolor{white!76!gray}515    & \cellcolor{white!76!gray}515    & \cellcolor{white!68!gray}517.5  & \cellcolor{white!68!gray}517.5  & \cellcolor{white!68!gray}517.5  & \cellcolor{white!89!orange}525    & \cellcolor{white!89!orange}525    & \cellcolor{white!78!orange}527.5  & \cellcolor{white!78!orange}527.5  & \cellcolor{white!78!orange}527.5  \\
    & 0.50  & \cellcolor{white!60!gray}520    & \cellcolor{white!60!gray}520    & \cellcolor{white!68!gray}517.5  & \cellcolor{white!68!gray}517.5  & \cellcolor{white!60!gray}520    & \cellcolor{white!67!orange}530    & \cellcolor{white!67!orange}530    & \cellcolor{white!67!orange}530    & \cellcolor{white!78!orange}527.5  & \cellcolor{white!67!orange}530    \\
    & 0.25  & \cellcolor{white!52!gray}522.5  & \cellcolor{white!36!gray}525.5  & \cellcolor{white!60!gray}520    & \cellcolor{white!68!gray}517.5  & \cellcolor{white!36!gray}525.5  & \cellcolor{white!45!orange}535    & \cellcolor{white!45!orange}535    & \cellcolor{white!56!orange}532.5  & \cellcolor{white!56!orange}532.5  & \cellcolor{white!56!orange}532.5  \\
    & 0.00  & \cellcolor{white!44!gray}525    & \cellcolor{white!44!gray}525    & \cellcolor{white!52!gray}522.5  & \cellcolor{white!60!gray}520    & \cellcolor{white!52!gray}522.5  & \cellcolor{white!23!orange}537.5  & \cellcolor{white!45!orange}535    & \cellcolor{white!45!orange}535    & \cellcolor{white!45!orange}535    & \cellcolor{white!45!orange}535    \\
    & -0.25 & \cellcolor{white!28!gray}527    & \cellcolor{white!20!gray}527    & \cellcolor{white!44!gray}525    & \cellcolor{white!52!gray}522.5  & \cellcolor{white!44!gray}525    & \cellcolor{white!45!orange}535    & \cellcolor{white!45!orange}535    & \cellcolor{white!56!orange}532.5  & \cellcolor{white!34!orange}535.5  & \cellcolor{white!23!orange}537.5  \\
    & -0.50  & \cellcolor{white!12!gray}530    & \cellcolor{white!12!gray}530    & \cellcolor{white!20!gray}527    & \cellcolor{white!44!gray}525    & \cellcolor{white!20!gray}527    & \cellcolor{white!23!orange}537.5  & \cellcolor{white!23!orange}537.5  & \cellcolor{white!23!orange}537.5  & \cellcolor{white!45!orange}535    & \cellcolor{white!23!orange}537.5  \\
    & -0.75 & \cellcolor{white!12!gray}530    & \cellcolor{white!12!gray}530    & \cellcolor{white!12!gray}530    & \cellcolor{white!20!gray}527    & \cellcolor{white!12!gray}530    & \cellcolor{white!12!orange}540    & \cellcolor{white!12!orange}540    & \cellcolor{white!12!orange}540    & \cellcolor{white!45!orange}535    & \cellcolor{white!12!orange}540    \\
    & -1.00 & \cellcolor{white!8!gray}532.5   & \cellcolor{white!8!gray}532.5   & \cellcolor{white!12!gray}530    & \cellcolor{white!12!gray}530    & \cellcolor{white!12!gray}530    & \cellcolor{white!1!orange}542.5   & \cellcolor{white!1!orange}542.5   & \cellcolor{white!12!orange}540    & \cellcolor{white!23!orange}537.5  & \cellcolor{white!12!orange}540    \\
\hline\hline
\end{tabular}

\end{table}

The oblique illumination of a partially embedded AuNP of radius $a=12.5$ nm with different values of the incrustation parameter $h/a$ leads, besides to the uniform increase (decrease) of the absorption and scattering efficiencies discussed above, to a redshift of the absorption and scattering wavelengths of resonance, $\lambda_\text{res}^\text{abs}$ and $\lambda_\text{res}^\text{sca}$, which are easily visualized in Table \ref{tab:Resonances} ---for all considered values of $h/a$, $\theta_i$ and both polarization sates--- with aid of the saturation of the cell color, which is greater the larger the value of the resonance wavelength by case: $\lambda_\text{res}^\text{abs}$ and $\lambda_\text{res}^\text{sca}$ in either $s$ or $p$ polarization. From the reported values in Table \ref{tab:Resonances}, both the absorption and scattering resonances are spectrally located in between the resonances for the Mie-limiting results (red markers in Figs \ref{fig:Inc:Abs} and \ref{fig:Inc:Sca}) and that resonances for the partially embedded AuNP are, in general, redshifted as the embedding of the AuNP increases ($h/a$ changes from $1$ to $-1$). In particular, the redshift of $\lambda_\text{res}^\text{abs}$ (gray cells) and $\lambda_\text{res}^\text{sca}$ (orange cells) for an $s$ polarized incident electric field (upper block of Table \ref{tab:Resonances}) is independent of the angle of incidence and the rate of change of the redshift in relation to the incrustation parameter is a uniform function. The observed rate of change of the redshift for the $p$ polarization case (lower block of Table \ref{tab:Resonances}) as the incrustation parameter decrease have a similar behavior for the absorption and scattering resonances just as observed in the $s$ polarization case nevertheless, this rate for the $p$ polarized illumination of the AuNP is different for each incident angle unlike its counterpart: On the one hand, for values of $\theta_i$ far from $\theta_c$ the dependence of  $\lambda_\text{res}^\text{abs}$ and $\lambda_\text{res}^\text{sca}$ on $h/a$ for $p$ polarization resembles that for $s$ polarization as it can be seen not only in the color gradient in Table \ref{tab:Resonances} but also in the comparison between the gray continuous and dashed lines in Figs. \ref{sfig:Inc:Abs:15} and \ref{sfig:Inc:Sca:15} for $\theta_i = 15^\circ$ and in Figs. \ref{sfig:Inc:Abs:75} and \ref{sfig:Inc:Sca:75} for $\theta_i = 75^\circ$; on the other, for $\theta_i$ in the neighborhood of $\theta_c$ the rate of the redshift as the AuNP is buried into the substrate is larger for values of $h/a<0$ than for $h/a>$ and, additionally, this change in the rate is more notorious for $\theta_i = 42^\circ\gtrapprox \theta_c$ than for $38^\circ\lessapprox \theta_c$. Both the uniform redshift for $s$ polarization and the redshift growing faster for $h/a>0$ when the $p$ polarization case is considered, can be explained by the different directions of the electric field below and above the glass-substrate interface since only for the later there is a balance in the directions of the electric field illuminating yielding to similar optical properties for values of $h/a$  around zero: when there is no major part of the AuNP in one medium.

Both the uniform enhancement (diminishment) of $Q_\text{abs}$ and $Q_\text{sca}$  and the  redshift of their resonance wavelength  are expected to behave similarly, for a fixed angle of incidence and polarization, since the absorption and scattering efficiencies are quantities calculated by a volume and a surface integrals given by  Eqs. \eqref{eq:Cabs} and \eqref{eq:Csca}, respectively,  and since both the volume and surface fraction of the AuNP in the substrate, relative to its total volume or surface, are given by the same expression $(1-h/a)/2$. The uniform changes on $Q_\text{abs}$ and $Q_\text{sca}$ and their resonance wavelength for the $s$ polarization case as the AuNP is buried, and the contrary observed for the AuNP, suggests that the average optical properties of the partially embedded AuNP are determined, for the $s$ polarization case, solely by the fraction of the AuNP embedded in either medium between the substrate and the matrix, while for $p$ polarization the direction and magnitude of the transmitted electric field are to be take into account if more (less) than on eight of the AuNP is in the substrate, leading to a rapid (slow) change of its average optical properties as $h/a$ changes. The past summary of the discussion on Figs. \ref{fig:Inc:Abs} and \ref{fig:Inc:Sca}, and Table \ref{tab:Resonances} describes the average optical properties of the partially embedded 12.5 nm AuNP given by $Q_\text{abs}$ ad $Q_\text{sca}$; to have a better understanding of all the optical properties of such system, the spatial distribution of the induced electric field, in de far and near field regimes, is to be analyzed.

The radiation patterns of a partially embedded 12.5 AuNP shown in Fig. \ref{fig:Far:Inc:s} consider that the AuNP is illuminated at an angle of incidence ---above the critical angle--- of $42^\circ$ [Figs. \ref{sfig:Far:Inc:s:a} and \ref{sfig:Far:Inc:s:a}] and of $75^\circ$ [Figs. \ref{sfig:Far:Inc:s:c} and \ref{sfig:Far:Inc:s:d}] by an $s$ polarized incident electric field with a wavelength $\lambda = 525$ nm, the resonance wavelength for the incrustation parameter $h/a = 0$; in Figs. \ref{sfig:Far:Inc:s:a} and   \ref{sfig:Far:Inc:s:c} the radiation pattern is evaluated at a scattering plane perpendicular to the plane of incidence (vertical gray dotted line), while it overlaps to it in  Figs. \ref{sfig:Far:Inc:s:b} and   \ref{sfig:Far:Inc:s:d}. The radiation patterns for $\theta_i<\theta_c$ were omitted since they follow the same shape as the one presented in Fig. \ref{fig:Far:Inc:s} due to the incident electric field being parallel to the interface between the substrate and the matrix.

What is observed in Fig. \ref{fig:Far:Inc:s} is that the radiation patterns for different values of the incrustation parameters $h/a$, at fixed angle of incidence and $s$ polarization, presents the two and one-lobe shapes with an asymmetry due to the substrate discussed in the Section \ref{s:Emb:Normal} for a normally illuminated AuNP. As in the aforementioned, the average amplitude of the radiation pattern for oblique incidence is larger, the more embedded into the substrate  the AuNP is, for example as $h/a$ changes from $0.75$ (orange curves) to $-0.75$ (light purple curves). Additionally, the radiation pattern is modulated by the value of the efficiencies  thus it is expected that the scattered far-field has a shorter outreach for a angle of incidence farther from $\theta_c$, that is the case when Figs. \ref{sfig:Far:Inc:s:a} and  \ref{sfig:Far:Inc:s:c} ($\theta_i =42^\circ$) are compared with Figs. \ref{sfig:Far:Inc:s:c} and  \ref{sfig:Far:Inc:s:d} ($\theta_i =75^\circ$). These results are in agreement with the discussion on the optical properties of a partially embedded AuNP  illuminated with an $s$ polarization incident electric field: their properties change uniformly with the incrustation parameter including the far-field distribution as well absorption and scattering efficiencies, due to the direction of the transmitted electric field not changing above and below the glass-air interface.

\begin{figure}[h!]
    \centering
    \def\svgwidth{.8\textwidth}
    \includeinkscape[pretex = \footnotesize]{4-Inc-Obl/4-FarXY-S/4-5-Far-XY-S-42}\\[-16.7em]
    \hspace*{-.2\textwidth}%
        \begin{subfigure}{.4\textwidth}\caption{%
                    \footnotesize$\dfrac{\norm{\vb{E}^\text{sca}_\text{far}}}{\norm{\vb{E}^\text{i}}} \times 10^{-9}$  }\label{sfig:Far:Inc:s:a}\end{subfigure}%
        \begin{subfigure}{.4\textwidth}\caption{%
                    \footnotesize$\dfrac{\norm{\vb{E}^\text{sca}_\text{far}}}{\norm{\vb{E}^\text{i}}} \times 10^{-9}$  }\label{sfig:Far:Inc:s:b}\end{subfigure}\\[13em]
    %
    \def\svgwidth{.8\textwidth}
    \hspace*{-.21\textwidth}%
    \vspace*{-.85em}%
        \begin{subfigure}{.4\textwidth}\caption{%
                    \footnotesize$\dfrac{\norm{\vb{E}^\text{sca}_\text{far}}}{\norm{\vb{E}^\text{i}}} \times 10^{-9}$  }\label{sfig:Far:Inc:s:c}\end{subfigure}%
        \begin{subfigure}{.4\textwidth}\caption{%
                    \footnotesize$\dfrac{\norm{\vb{E}^\text{sca}_\text{far}}}{\norm{\vb{E}^\text{i}}} \times 10^{-9}$  }\label{sfig:Far:Inc:s:d}\end{subfigure}\\
    \includeinkscape[pretex = \footnotesize]{4-Inc-Obl/4-FarXY-S/4-5-Far-XY-S-75}%
    \caption[  Radiation pattern of a AuNP supported on a substrate illuminated at oblique incidence ]{%
    Radiation pattern of a AuNP (light yellow) of radius $a = 12.5$ nm partially embedded in a glass substrate (light blue, $n_\text{s} = 1.5$) with an air matrix ($n_\text{m} = 1$) illuminated by an \textit{s} polarized incident electric plane wave $\vb{E}^\text{i}$, with a wavelength $\lambda$, traveling in the $\vb{k}^\text{i}$ direction at an angle of incidence $\theta_i$ of \textbf{a,b)} $42^\circ$ and \textbf{c,d)} $75^\circ$ relative to the normal direction to the glass-air interface. The radiation patterns consider various values of the incrustation parameter $h/a$, with $a$ the AuNP's radius and $h$ the distance between its center and the interface, and an  incident electric field \textbf{a,c)} perpendicular to the incidence plane (vertical gray dotted lines) and \textbf{b,d)} equal to the incidence plane. In all cases the incident wave vector $\vb{k}^\text{i}$, the perpendicular $\vb{E}_\perp^\text{i}$ and the  parallel $\vb{E}_\parallel^\text{i}$ projection of the incident electric field relative to the scattering plane are schematized.%
     }
    \label{fig:Far:Inc:s}
\end{figure}

For a $p$ polarized incident electric field, the reflected and transmitted electric field  $\vb{E}^\text{i}$ have a direction dependence on the angle of incidence $\theta_i$, therefore the radiation patterns of a partially embedded  12.5 AuNP, illuminated by a  $p$ polarized  $\vb{E}^\text{i}$ is expected to beheve differently for each $\theta_i$. Therefore the radiation patterns for the described system, considering a wavelength $\lambda = 525$ nm for $\vb{E}^\text{i}$, are shown in Figs. \ref{fig:Far:Inc:p1} and  \ref{fig:Far:Inc:p2} for values of $\theta_i$ below and above the critical angle $41.8^\circ$, respectively. In particular, the radiation patterns in Figs. \ref{sfig:Far:Inc:p1:a} and  \ref{sfig:Far:Inc:p1:b} corresponds to $\theta_i = 15^\circ$ and in Figs. \ref{sfig:Far:Inc:p1:c} and  \ref{sfig:Far:Inc:p1:d} to  $\theta_i = 38^\circ$, while the radiation patterns for  $\theta_i = 42^\circ$ are shown in  Figs. \ref{sfig:Far:Inc:p2:a} and  \ref{sfig:Far:Inc:p2:b}, and for $\theta_i = 75^\circ$ in  Figs. \ref{sfig:Far:Inc:p2:c} and  \ref{sfig:Far:Inc:p2:d}. In both Fig. \ref{fig:Far:Inc:p1} and Fig. \ref{fig:Far:Inc:p2}, the radiation patterns are evaluated at \textbf{a,c)}  a scattering plane perpendicular to the incidence plane (vertical gray dotted line) and  at \textbf{b,d)}  a scattering plane overlapping the incidence plane.

\begin{figure}[h!]
    \centering
    \def\svgwidth{.8\textwidth}
    \includeinkscape[pretex = \footnotesize]{4-Inc-Obl/5-FarXY-P/4-5-Far-XY-P-15}\\[-16.5em]
    \hspace*{-.2\textwidth}%
        \begin{subfigure}{.375\textwidth}\caption{%
                    \footnotesize$\dfrac{\norm{\vb{E}^\text{sca}_\text{far}}}{\norm{\vb{E}^\text{i}}} \times 10^{-9}$  }\label{sfig:Far:Inc:p1:a}\end{subfigure}%
        \begin{subfigure}{.4\textwidth}\caption{%
                    \footnotesize$\dfrac{\norm{\vb{E}^\text{sca}_\text{far}}}{\norm{\vb{E}^\text{i}}} \times 10^{-9}$  }\label{sfig:Far:Inc:p1:b}\end{subfigure}\\[13em]
    %
    \def\svgwidth{.8\textwidth}
    \hspace*{-.21\textwidth}%
    \vspace*{-.7em}%
        \begin{subfigure}{.4\textwidth}\caption{%
                    \footnotesize$\dfrac{\norm{\vb{E}^\text{sca}_\text{far}}}{\norm{\vb{E}^\text{i}}} \times 10^{-9}$  }\label{sfig:Far:Inc:p1:c}\end{subfigure}%
        \begin{subfigure}{.4\textwidth}\caption{%
                    \footnotesize$\dfrac{\norm{\vb{E}^\text{sca}_\text{far}}}{\norm{\vb{E}^\text{i}}} \times 10^{-9}$  }\label{sfig:Far:Inc:p1:d}\end{subfigure}\\
    \includeinkscape[pretex = \footnotesize]{4-Inc-Obl/5-FarXY-P/4-5-Far-XY-P-38}%
    \caption[  Radiation pattern of a AuNP supported on a substrate illuminated at oblique incidence ]{
    Radiation pattern of a AuNP (light yellow) of radius $a = 12.5$ nm partially embedded in a glass substrate (light blue, $n_\text{s} = 1.5$) with an air matrix ($n_\text{m} = 1$) illuminated by an \textit{p} polarized incident electric plane wave $\vb{E}^\text{i}$, with a wavelength $\lambda$, traveling in the $\vb{k}^\text{i}$ direction at an angle of incidence $\theta_i$ of \textbf{a,b)} $15^\circ$ and \textbf{c,d)} $38^\circ$ relative to the normal direction to the glass-air interface. The radiation patterns consider various values of the incrustation parameter $h/a$, with $a$ the AuNP's radius and $h$ the distance between its center and the interface, and an  incident electric field \textbf{a,c)} perpendicular to the incidence plane (vertical gray dotted lines) and \textbf{b,d)} equal to the incidence plane. In all cases the incident wave vector $\vb{k}^\text{i}$, the perpendicular $\vb{E}_\perp^\text{i}$ and the  parallel $\vb{E}_\parallel^\text{i}$ projection of the incident electric field relative to the scattering plane are schematized.%
    }
    \label{fig:Far:Inc:p1}
\end{figure}

The radiation patterns for a partially embedded AuNP when it is illuminated at $\theta_i < \theta_c = 41.8^\circ$, so the transmitted electric field is a plane wave,  resembles  that of a supported AuNP ---discussed in Section \ref{s:TIR}---: an asymmetrical one-lobe shape and a two-lobe shape characteristic of a point dipole oriented perpendicularly to the propagating direction of the transmitted electric field. One effect of the AuNP's embedding is the enhancement of the average amplitude of the radiation patterns according to the absorption and scattering efficiencies, as already discussed for the $s$ polarization case shown in Fig. \ref{fig:Far:Inc:s} and also observed  for $p$ polarization at $\theta_i = 15^\circ$ in Figs. \ref{sfig:Far:Inc:s:a} and \ref{sfig:Far:Inc:s:b} nevertheless, a deformation of the radiation patterns as the AuNP is buried into the substrate can be observed, specially for $\theta_i = 38^\circ$ in Figs. \ref{sfig:Far:Inc:s:c} and \ref{sfig:Far:Inc:s:d}. In the first, the one-lobe shape present for $h/a\leq 0$ is deformed into a two lobe-shape if $h/a>0$  and in the later there is a rotation of $\sim 10^\circ$ clockwise of the two-lobe radiation pattern for $h/a>0$ relative to the patterns for $h/a\leq0$ which are aligned to the direction of the transmitted electric field.

\begin{figure}[h!]
    \centering
    \def\svgwidth{.8\textwidth}
    \includeinkscape[pretex = \footnotesize]{4-Inc-Obl/5-FarXY-P/4-5-Far-XY-P-42}\\[-16.7em]
    \hspace*{-.2\textwidth}%
        \begin{subfigure}{.4\textwidth}\caption{%
                    \footnotesize$\dfrac{\norm{\vb{E}^\text{sca}_\text{far}}}{\norm{\vb{E}^\text{i}}} \times 10^{-9}$  }\label{sfig:Far:Inc:p2:a}\end{subfigure}%
        \begin{subfigure}{.4\textwidth}\caption{%
                    \footnotesize$\dfrac{\norm{\vb{E}^\text{sca}_\text{far}}}{\norm{\vb{E}^\text{i}}} \times 10^{-9}$  }\label{sfig:Far:Inc:p2:b}\end{subfigure}\\[13em]
    %
    \def\svgwidth{.8\textwidth}
    \hspace*{-.21\textwidth}%
    \vspace*{-.7em}%
        \begin{subfigure}{.4\textwidth}\caption{%
                    \footnotesize$\dfrac{\norm{\vb{E}^\text{sca}_\text{far}}}{\norm{\vb{E}^\text{i}}} \times 10^{-9}$  }\label{sfig:Far:Inc:p2:c}\end{subfigure}%
        \begin{subfigure}{.4\textwidth}\caption{%
                    \footnotesize$\dfrac{\norm{\vb{E}^\text{sca}_\text{far}}}{\norm{\vb{E}^\text{i}}} \times 10^{-9}$  }\label{sfig:Far:Inc:p2:d}\end{subfigure}\\
    \includeinkscape[pretex = \footnotesize]{4-Inc-Obl/5-FarXY-P/4-5-Far-XY-P-75}%
    \caption[  Radiation pattern of a AuNP supported on a substrate illuminated at oblique incidence ]{
    Radiation pattern of a AuNP (light yellow) of radius $a = 12.5$ nm partially embedded in a glass substrate (light blue, $n_\text{s} = 1.5$) with an air matrix ($n_\text{m} = 1$) illuminated by an \textit{p} polarized incident electric plane wave $\vb{E}^\text{i}$, with a wavelength $\lambda$, traveling in the $\vb{k}^\text{i}$ direction at an angle of incidence $\theta_i$ of \textbf{a,b)} $42^\circ$ and \textbf{c,d)} $75^\circ$ relative to the normal direction to the glass-air interface. The radiation patterns consider various values of the incrustation parameter $h/a$,with $a$ the AuNP's radius and $h$ the distance between its center and the interface, and an  incident electric field \textbf{a,c)} perpendicular to the incidence plane (vertical gray dotted lines) and \textbf{b,d)} equal to the incidence plane. In all cases the incident wave vector $\vb{k}^\text{i}$, the perpendicular $\vb{E}_\perp^\text{i}$ and the  parallel $\vb{E}_\parallel^\text{i}$ projection of the incident electric field relative to the scattering plane are schematized.%
    }
    \label{fig:Far:Inc:p2}
\end{figure}

Another difference that arises between  the radiation patterns of a partially embedded AuNP and a supported AuNP (Section \ref{s:TIR}) when illuminated at a oblique incidence, is the ratio of the  average amplitude of the radiation pattern between polarization state of the incidente electric field. While the extent (\textcolor{red}{Quiero decir que "el alcance el campo lejano es mayor para...." pero creo que extent no es la palabra que busco}) of the induced far-field for supported AuNPs is larger for  $p$ polarization case than for $s$ ---see Fig. \ref{fig:Far:SuppObl})--- for $\theta_i \gtrapprox  \theta_c$, the contrary is observed for the partially embedded AuNP:







In Fig. \ref{fig:Near:Inc0:42} the spatial distribution of the induces electric field is shown for a partially embedded 12.5 nm AuNP, with an incrustation parameter of $h/a = 0$, when this is illuminated at an angle of incidence of $42^\circ>\theta_c$, in an internal configuration, by a $s$ [Figs. \ref{sfig:Near:Inc0:42:s1} and \ref{sfig:Near:Inc0:42:s2}] and a $p$ [Figs. \ref{sfig:Near:Inc0:42:p1} and \ref{sfig:Near:Inc0:42:p2}] polarized incident electric field with a wavelength o $\lambda  =525$ nm, considering that the scattering plane where, the radiation pattern is evaluated, is perpendicular to the incidence plane [Figs. \ref{sfig:Near:Inc0:42:s1} and \ref{sfig:Near:Inc0:42:p1}] and overlaps the incidence plane [Figs. \ref{sfig:Near:Inc0:42:s2} and \ref{sfig:Near:Inc0:42:p2}].


\begin{figure}[h!]\centering
   \def\svgwidth{.75\textwidth}
   \footnotesize
   \captionsetup[subfigure]{labelfont ={normal,bf,color = white}}
   \includeinkscape{4-Inc-Obl/2-Near-SP/2-NearYX-SP-42}\\[-32.6em]
   \hspace*{-.25\textwidth}
       \begin{subfigure}{.25\textwidth}\textcolor{red}{\caption{ } \label{sfig:Near:Inc0:42:s1}}\end{subfigure}%
       \begin{subfigure}{.34\textwidth}\caption{ }\label{sfig:Near:Inc0:42:s2}\end{subfigure}\\[13em]
    \hspace*{-.25\textwidth}
       \begin{subfigure}{.25\textwidth}\textcolor{red}{\caption{ } \label{sfig:Near:Inc0:42:p1}}\end{subfigure}%
       \begin{subfigure}{.34\textwidth}\caption{ }\label{sfig:Near:Inc0:42:p2}\end{subfigure}\\[15em]
   \caption[Induced Electric Field of a 12.5 nm Au NP on substrate illuminated at oblique incidence with a $s$ polarized electric field]{%
   Electric field $\vb{E}^\text{ind}$ induced by a partially embedded 12.5 nm AuNP (dashed black lines) illuminated by an \textbf{a,b)} $s$ polarized and  by a \textbf{c,d)} \textit{p} polarized incident electric plane wave $\vb{E}^\text{i}$ traveling in the $\vb{k}^\text{i}$ direction, in an internal configuration, at an angle of incidence of  $42^\circ$ relative to the normal direction to the interface ---white dashed lines--- between an air matrix ($n_\text{m} = 1$) and a glass substrate ($n_\text{s} = 1.5$). The incident electric plane wave is evaluated at $\lambda = 525$ nm and the norm $\norm{\vb{E}^\text{ind}}$ is evaluated at  \textbf{a,c)} a scattering plane perpendicular to the incidence plane (vertical gray dotted lines) and at \textbf{b,d)} a scattering plane overlapping the incidence plane. }
   \label{fig:Near:Inc0:42}
 \end{figure}
