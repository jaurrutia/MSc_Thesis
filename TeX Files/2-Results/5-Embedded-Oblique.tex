% !TeX root = ../tesis.tex

In the past section, the effect of the embedding of a spherical AuNP with radius $a = 12.5$ nm, located at the planer interface between a glass substrate and an air matrix,  on its optical properties was studied when the system was illuminated at at the normal direction to the interface. On the one hand, it was found that the AuNP can be still described by a mostly dipolar contrasted even though the homogeneity and symmetry of its surroundings is broken. On the other, the greatest near-field enhancement is localized on the surface of the AuNP in contact with the matrix, which was undesired if the partially embedded AuNP is to be used as the unit cell for a bidimensional array suited for biosensing. In order to find an optical configuration of the system suited for interactions above the substrae, in this Section the optical properties of a partially embedded AuNP are analyzed but considering an oblique incidence to the system (both below and above the critical angle), meaning that the system can be illuminated either with an $s$ or a $p$ polarized incident electric field.

The absorption $Q_\text{abs}$ and scattering $Q_\text{sca}$ efficiencies of partially embedded 12.5 nm AuNP in a glass substrate ($n_\text{s} = 1.5$) and in an air matrix ($n_\text{m} = 1$) ---for different values of the incrustation parameter--- are shown in Figs. \ref{fig:Inc:Abs} and \ref{fig:Inc:Sca}, respectively, as function of the wavelength $\lambda$ of the incident electric field $\vb{E}^\text{i}$ illuminating the AuNP from the substrate at an incidence angle $\theta_i < \theta_c = 41.8^\circ$ of $15^\circ$ [Figs. \ref{sfig:Inc:Abs:15} and \ref{sfig:Inc:Sca:15}] and  $38^\circ$ [Figs. \ref{sfig:Inc:Abs:38} and \ref{sfig:Inc:Sca:38}], and  at a value of $\theta_i>\theta_c$, thus forcing the interactino between an evanescent wave an the AuNP, equal to $42^\circ$  [Figs. \ref{sfig:Inc:Abs:42} and \ref{sfig:Inc:Sca:42}] and $75^\circ$  [Figs. \ref{sfig:Inc:Abs:75} and \ref{sfig:Inc:Sca:75}], considering an $s$ (filled circle/solid lines) and a $p$ (empty circle/dashed lines) polarization for $\vb{E}^\text{i}$. To compare to obtained results with the the Mie-limiting cases (AuNP embedded in air and in glass) ---green shaded region and red markers corresponding to the resonance of the absorption and scattering efficiencies---, the values of $Q_\text{abs}$ ($Q_\text{sca}$) evaluated at their wavelength of resonance $\lambda_\text{res}^\text{abs}$ ($\lambda_\text{res}^\text{sca}$) are signalized by the magenta markers and the numerical values of the later can be found in Table \ref{tab:Resonances}, where the saturation of the cell colors corresponds to a larger wavelength. Lastly, in Figs. \ref{fig:Inc:Abs} and \ref{fig:Inc:Sca} the gray continuous and dashed lines are a guides to the eye joining the resonances of the absorption and scattering efficiencies in each case considering an $s$ and a $p$ polarized incident electric field.

\begin{figure}[h!]\small \centering
    \hspace*{-.675\textwidth}%
        \begin{subfigure}{.735\textwidth}\caption{ }\label{sfig:Inc:Abs:15}\end{subfigure}%
        \begin{subfigure}{.25\textwidth}\caption{ }\label{sfig:Inc:Abs:38}\end{subfigure} \\[17em]
    \hspace*{-.675\textwidth}%
        \begin{subfigure}{.735\textwidth}\caption{ }\label{sfig:Inc:Abs:42}\end{subfigure}%
        \begin{subfigure}{.25\textwidth}\caption{ }\label{sfig:Inc:Abs:75}\end{subfigure} \\[-19.9em]
    \def\svgwidth{.95\textwidth}
    \includeinkscape{4-Inc-Obl/1-Efficiencies/1-Oblique-Inc-Abs}%
    \vspace*{-.5em}
    \caption[Absorption Efficiency of a partially embedded 12.5 nm AuNP into a substrate Illuminated in an internal configuration at oblique incidence]{%
    Absorption efficiency of a $12.5$ nm AuNP partially embedded in a glass substrate ($n_\text{s} = 1.5$) with an air matrix ($n_\text{m} = 1$) as function of the wavelength $\lambda$ of an \textit{s} (filled circle/solid lines) and a \textit{p} (empty circle/dashed lines) polarized incident electric plane wave propagating in the direction of the wave vector $\vb{k}^\text{i}$, in an internal configuration, at an angle of incidence $\theta_i$ of \textbf{a)} $15^\circ$, \textbf{b)} $38^\circ$, \textbf{c)} $42^\circ$ and \textbf{d)} $75^\circ$  relative to the normal direction to the glass-air interface. The green shaded region shows the two Mie-limiting cases of a AuNP embedded in air
and in glass; the magenta (partially embedded AuNP) and red (Mie-limiting) markers corresponds to the efficiencies evaluated at the wavelength of resonance for each case; the gray (gray dashed) line is a guide to the eye for the \textit{s} (\textit{p}) polarization case.
}
\label{fig:Inc:Abs}
\end{figure}








\begin{figure}[h!]\small \centering
    \hspace*{-.675\textwidth}%
        \begin{subfigure}{.735\textwidth}\caption{ }\label{sfig:Inc:Sca:15}\end{subfigure}%
        \begin{subfigure}{.25\textwidth}\caption{ }\label{sfig:Inc:Sca:38}\end{subfigure} \\[17em]
    \hspace*{-.675\textwidth}%
        \begin{subfigure}{.735\textwidth}\caption{ }\label{sfig:Inc:Sca:42}\end{subfigure}%
        \begin{subfigure}{.25\textwidth}\caption{ }\label{sfig:Inc:Sca:75}\end{subfigure} \\[-19.9em]
    \def\svgwidth{.95\textwidth}
    \includeinkscape{4-Inc-Obl/1-Efficiencies/2-Oblique-Inc-Sca}%
    \vspace*{-.5em}
    \caption[Scattering Efficiency of a partially embedded 12.5 nm AuNP into a substrate Illuminated in an internal configuration at oblique incidence]{%
    Scattering efficiency of a $12.5$ nm AuNP partially embedded in a glass substrate ($n_\text{s} = 1.5$) with an air matrix ($n_\text{m} = 1$) as function of the wavelength $\lambda$ of an \textit{s} (filled circle/solid lines) and a \textit{p} (empty circle/dashed lines) polarized incident electric plane wave propagating in the direction of the wave vector $\vb{k}^\text{i}$, in an internal configuration, at an angle of incidence $\theta_i$ of \textbf{a)} $15^\circ$, \textbf{b)} $38^\circ$, \textbf{c)} $42^\circ$ and \textbf{d)} $75^\circ$  relative to the normal direction to the glass-air interface. The green shaded region shows the two Mie-limiting cases of a AuNP embedded in air
and in glass; the magenta (partially embedded AuNP) and red (Mie-limiting) markers corresponds to the efficiencies evaluated at the wavelength of resonance for each case; the gray (gray dashed) line is a guide to the eye for the \textit{s} (\textit{p}) polarization case.
}
\label{fig:Inc:Sca}
\end{figure}












\begin{table}[h!]\footnotesize\centering
    \caption{Wavelength of resonance for the absorption $\lambda_\text{res}^\text{abs}$ and the scattering $\lambda_\text{res}^\text{sca}$ efficiencies of a partially embedded 12.5 nm AuNP with a glass substrate ($n_\text{s} = 1.5$) and an air matrix ($n_\text{m} = 1$)    illuminated by an \textit{s} and a \textit{p} polarized electric plane wave traveling to the glass-air interface at an incidence angle of $15^\circ$,  $38^\circ$,  $42^\circ$ and  $75^\circ$ for several values of the incrustation parameter $h/a$ with $h$ the distance between the AuNP and its radius $a$. The values in this table correspond to the magenta markers in Figs. \ref{fig:Inc:Abs} and \ref{fig:Inc:Sca} while the saturation of the cell colors are a guide to the eye.}
    \label{tab:Resonances}
    

\begin{tabular}{l | r | ccccc || ccccc} \hline \hline
                                &       & \multicolumn{5}{c ||}{\small $\lambda_\text{res}^\text{abs}$ [nm]} & \multicolumn{5}{c}{\small $\lambda_\text{res}^\text{sca}$ [nm]}  \\ \hline \hline
                                & $h/a$ & $0^\circ$ & $15^\circ$     & $38^\circ$    & $42^\circ$    & $75^\circ$    & $0^\circ$ & $15^\circ$     & $38^\circ$    & $42^\circ$    & $75^\circ$    \\ \hline
\multirow{9}{*}{\rotatebox{90}{\emph{s} Polarization}}
    & 1.00  & \cellcolor{white!92!gray}510    & \cellcolor{white!92!gray}510    & \cellcolor{white!92!gray}510    & \cellcolor{white!92!gray}510    & \cellcolor{white!84!gray}512.5  & \cellcolor{white!89!orange}525    & \cellcolor{white!89!orange}525    & \cellcolor{white!89!orange}525    & \cellcolor{white!89!orange}525    & \cellcolor{white!89!orange}525    \\
    & 0.75  & \cellcolor{white!76!gray}515    & \cellcolor{white!76!gray}515    & \cellcolor{white!76!gray}515    & \cellcolor{white!76!gray}515    & \cellcolor{white!76!gray}515    & \cellcolor{white!89!orange}525    & \cellcolor{white!89!orange}525    & \cellcolor{white!89!orange}525    & \cellcolor{white!89!orange}525    & \cellcolor{white!89!orange}525    \\
    & 0.50  & \cellcolor{white!60!gray}520    & \cellcolor{white!60!gray}520    & \cellcolor{white!60!gray}520    & \cellcolor{white!60!gray}520    & \cellcolor{white!60!gray}520    & \cellcolor{white!67!orange}530    & \cellcolor{white!67!orange}530    & \cellcolor{white!67!orange}530    & \cellcolor{white!67!orange}530    & \cellcolor{white!67!orange}530    \\
    & 0.25  & \cellcolor{white!52!gray}522.5  & \cellcolor{white!52!gray}522.5  & \cellcolor{white!52!gray}522.5  & \cellcolor{white!52!gray}522.5  & \cellcolor{white!52!gray}522.5  & \cellcolor{white!45!orange}535    & \cellcolor{white!45!orange}535    & \cellcolor{white!45!orange}535    & \cellcolor{white!45!orange}535    & \cellcolor{white!45!orange}535    \\
    & 0.00  & \cellcolor{white!44!gray}525    & \cellcolor{white!44!gray}525    & \cellcolor{white!44!gray}525    & \cellcolor{white!44!gray}525    & \cellcolor{white!44!gray}525    & \cellcolor{white!23!orange}537.5  & \cellcolor{white!23!orange}537.5  & \cellcolor{white!23!orange}537.5  & \cellcolor{white!45!orange}535    & \cellcolor{white!45!orange}535    \\
    & -0.25 & \cellcolor{white!28!gray}527    & \cellcolor{white!28!gray}527    & \cellcolor{white!28!gray}527    & \cellcolor{white!28!gray}527    & \cellcolor{white!28!gray}527    & \cellcolor{white!45!orange}535    & \cellcolor{white!45!orange}535    & \cellcolor{white!45!orange}535    & \cellcolor{white!45!orange}535    & \cellcolor{white!45!orange}535    \\
    & -0.50 & \cellcolor{white!12!gray}530    & \cellcolor{white!12!gray}530    & \cellcolor{white!12!gray}530    & \cellcolor{white!12!gray}530    & \cellcolor{white!12!gray}530    & \cellcolor{white!23!orange}537.5  & \cellcolor{white!23!orange}537.5  & \cellcolor{white!23!orange}537.5  & \cellcolor{white!23!orange}537.5  & \cellcolor{white!23!orange}537.5  \\
    & -0.75 & \cellcolor{white!12!gray}530    & \cellcolor{white!12!gray}530    & \cellcolor{white!12!gray}530    & \cellcolor{white!12!gray}530    & \cellcolor{white!12!gray}530    & \cellcolor{white!12!orange}540    & \cellcolor{white!12!orange}540    & \cellcolor{white!12!orange}540    & \cellcolor{white!12!orange}540    & \cellcolor{white!12!orange}540    \\
    & -1.00 & \cellcolor{white!8!gray}532.5   & \cellcolor{white!8!gray}532.5   & \cellcolor{white!8!gray}532.5   & \cellcolor{white!8!gray}532.5   & \cellcolor{white!8!gray}532.5   & \cellcolor{white!1!orange}542.5   & \cellcolor{white!1!orange}542.5   & \cellcolor{white!1!orange}542.5   & \cellcolor{white!1!orange}542.5   & \cellcolor{white!12!orange}540    \\
\hline\hline
\multirow{9}{*}{\rotatebox{90}{\emph{p} Polarization}}
    & 1.00  & \cellcolor{white!92!gray}510    & \cellcolor{white!84!gray}512.5  & \cellcolor{white!84!gray}512.5  & \cellcolor{white!84!gray}512.5  & \cellcolor{white!84!gray}512.5  & \cellcolor{white!89!orange}525    & \cellcolor{white!89!orange}525    & \cellcolor{white!78!orange}527.5  & \cellcolor{white!78!orange}527.5  & \cellcolor{white!89!orange}525    \\
    & 0.75  & \cellcolor{white!76!gray}515    & \cellcolor{white!76!gray}515    & \cellcolor{white!68!gray}517.5  & \cellcolor{white!68!gray}517.5  & \cellcolor{white!68!gray}517.5  & \cellcolor{white!89!orange}525    & \cellcolor{white!89!orange}525    & \cellcolor{white!78!orange}527.5  & \cellcolor{white!78!orange}527.5  & \cellcolor{white!78!orange}527.5  \\
    & 0.50  & \cellcolor{white!60!gray}520    & \cellcolor{white!60!gray}520    & \cellcolor{white!68!gray}517.5  & \cellcolor{white!68!gray}517.5  & \cellcolor{white!60!gray}520    & \cellcolor{white!67!orange}530    & \cellcolor{white!67!orange}530    & \cellcolor{white!67!orange}530    & \cellcolor{white!78!orange}527.5  & \cellcolor{white!67!orange}530    \\
    & 0.25  & \cellcolor{white!52!gray}522.5  & \cellcolor{white!36!gray}525.5  & \cellcolor{white!60!gray}520    & \cellcolor{white!68!gray}517.5  & \cellcolor{white!36!gray}525.5  & \cellcolor{white!45!orange}535    & \cellcolor{white!45!orange}535    & \cellcolor{white!56!orange}532.5  & \cellcolor{white!56!orange}532.5  & \cellcolor{white!56!orange}532.5  \\
    & 0.00  & \cellcolor{white!44!gray}525    & \cellcolor{white!44!gray}525    & \cellcolor{white!52!gray}522.5  & \cellcolor{white!60!gray}520    & \cellcolor{white!52!gray}522.5  & \cellcolor{white!23!orange}537.5  & \cellcolor{white!45!orange}535    & \cellcolor{white!45!orange}535    & \cellcolor{white!45!orange}535    & \cellcolor{white!45!orange}535    \\
    & -0.25 & \cellcolor{white!28!gray}527    & \cellcolor{white!20!gray}527    & \cellcolor{white!44!gray}525    & \cellcolor{white!52!gray}522.5  & \cellcolor{white!44!gray}525    & \cellcolor{white!45!orange}535    & \cellcolor{white!45!orange}535    & \cellcolor{white!56!orange}532.5  & \cellcolor{white!34!orange}535.5  & \cellcolor{white!23!orange}537.5  \\
    & -0.50  & \cellcolor{white!12!gray}530    & \cellcolor{white!12!gray}530    & \cellcolor{white!20!gray}527    & \cellcolor{white!44!gray}525    & \cellcolor{white!20!gray}527    & \cellcolor{white!23!orange}537.5  & \cellcolor{white!23!orange}537.5  & \cellcolor{white!23!orange}537.5  & \cellcolor{white!45!orange}535    & \cellcolor{white!23!orange}537.5  \\
    & -0.75 & \cellcolor{white!12!gray}530    & \cellcolor{white!12!gray}530    & \cellcolor{white!12!gray}530    & \cellcolor{white!20!gray}527    & \cellcolor{white!12!gray}530    & \cellcolor{white!12!orange}540    & \cellcolor{white!12!orange}540    & \cellcolor{white!12!orange}540    & \cellcolor{white!45!orange}535    & \cellcolor{white!12!orange}540    \\
    & -1.00 & \cellcolor{white!8!gray}532.5   & \cellcolor{white!8!gray}532.5   & \cellcolor{white!12!gray}530    & \cellcolor{white!12!gray}530    & \cellcolor{white!12!gray}530    & \cellcolor{white!1!orange}542.5   & \cellcolor{white!1!orange}542.5   & \cellcolor{white!12!orange}540    & \cellcolor{white!23!orange}537.5  & \cellcolor{white!12!orange}540    \\
\hline\hline
\end{tabular}

\end{table}











The radiation patterns of a partially embedded 12.5 AuNP shown in Fig. \ref{fig:Far:Inc:s} consider that the AuNP is illuminated at an angle of incidence ---above the critical angle--- of $42^\circ$ [Figs. \ref{sfig:Far:Inc:s:a} and \ref{sfig:Far:Inc:s:a}] and of $75^\circ$ [Figs. \ref{sfig:Far:Inc:s:c} and \ref{sfig:Far:Inc:s:d}] by an $s$ polarized incident electric field with a wavelength $\lambda = 525$ nm; in Figs. \ref{sfig:Far:Inc:s:a} and   \ref{sfig:Far:Inc:s:c} the radiation pattern is evaluated at a scattering plane perpendicular to the plane of incidence (vertical gray dotted line), while it overlaps to it in  Figs. \ref{sfig:Far:Inc:s:b} and   \ref{sfig:Far:Inc:s:d}. The radiation patterns for $\theta_i<\theta_c$ were omitted since they follow the same shape as the one presented in Fig. \ref{fig:Far:Inc:s} due to the incident electric field being parallel to the interface between the substrate and the matrix.

\begin{figure}[h!]
    \centering
    \def\svgwidth{.8\textwidth}
    \hspace*{-.2\textwidth}%
    \vspace*{-3.85em}%
        \begin{subfigure}{.4\textwidth}\caption{%
                    \footnotesize$\dfrac{\norm{\vb{E}^\text{sca}_\text{far}}}{\norm{\vb{E}^\text{i}}} \times 10^{-9}$  }\label{sfig:Far:Inc:s:a}\end{subfigure}%
        \begin{subfigure}{.4\textwidth}\caption{%
                    \footnotesize$\dfrac{\norm{\vb{E}^\text{sca}_\text{far}}}{\norm{\vb{E}^\text{i}}} \times 10^{-9}$  }\label{sfig:Far:Inc:s:b}\end{subfigure}\\
    \includeinkscape[pretex = \footnotesize]{4-Inc-Obl/4-FarXY-S/4-5-Far-XY-S-42}\\[-.75em]
    %
    \def\svgwidth{.8\textwidth}
    \hspace*{-.21\textwidth}%
    \vspace*{-.85em}%
        \begin{subfigure}{.4\textwidth}\caption{%
                    \footnotesize$\dfrac{\norm{\vb{E}^\text{sca}_\text{far}}}{\norm{\vb{E}^\text{i}}} \times 10^{-9}$  }\label{sfig:Far:Inc:s:c}\end{subfigure}%
        \begin{subfigure}{.4\textwidth}\caption{%
                    \footnotesize$\dfrac{\norm{\vb{E}^\text{sca}_\text{far}}}{\norm{\vb{E}^\text{i}}} \times 10^{-9}$  }\label{sfig:Far:Inc:s:d}\end{subfigure}\\
    \includeinkscape[pretex = \footnotesize]{4-Inc-Obl/4-FarXY-S/4-5-Far-XY-S-75}%
    \caption[  Radiation pattern of a AuNP supported on a substrate illuminated at oblique incidence ]{%
    Radiation pattern of a AuNP (light yellow) of radius $a = 12.5$ nm partially embedded in a glass substrate (light blue, $n_\text{s} = 1.5$) with an air matrix ($n_\text{m} = 1$) illuminated by an \textit{s} polarized incident electric plane wave $\vb{E}^\text{i}$, with a wavelength $\lambda$, traveling in the $\vb{k}^\text{i}$ direction at an angle of incidence $\theta_i$ of \textbf{a,b)} $42^\circ$ and \textbf{c,d)} $75^\circ$ relative to the normal direction to the glass-air interface. The radiation patterns consider various values of the incrustation parameter $h/a$, with $a$ the AuNP's radius and $h$ the distance between its center and the interface, and an  incident electric field \textbf{a,c)} perpendicular to the incidence plane (vertical gray dotted lines) and \textbf{b,d)} equal to the incidence plane. In all cases the incident wave vector $\vb{k}^\text{i}$, the perpendicular $\vb{E}_\perp^\text{i}$ and the  parallel $\vb{E}_\parallel^\text{i}$ projection of the incident electric field relative to the scattering plane are schematized.%
     }
    \label{fig:Far:Inc:s}
\end{figure}







Since the reflected and transmitted electric field for a $p$ polarized incident electric field $\vb{E}^\text{i}$ have a direction dependent on the angle of incidence $\theta_i$, the radiation patterns of a partially embedded  12.5 AuNP, illuminated by a  $p$ polarized  $\vb{E}^\text{i}$ with a wavelength $\lambda = 525$ nm, are shown in Figs. \ref{fig:Far:Inc:p1} and  \ref{fig:Far:Inc:p2} for values of $\theta_i$ below and above the critical angle $41.8^\circ$, respectively. In particular, the radiation patterns in Figs. \ref{sfig:Far:Inc:p1:a} and  \ref{sfig:Far:Inc:p1:b} corresponds to $\theta_i = 15^\circ$ and in Figs. \ref{sfig:Far:Inc:p1:c} and  \ref{sfig:Far:Inc:p1:d} to  $\theta_i = 38^\circ$, while the radiation patterns for  $\theta_i = 42^\circ$ are shown in  Figs. \ref{sfig:Far:Inc:p2:a} and  \ref{sfig:Far:Inc:p2:b}, and for $\theta_i = 75^\circ$ in  Figs. \ref{sfig:Far:Inc:p2:c} and  \ref{sfig:Far:Inc:p2:d}. In both Fig. \ref{fig:Far:Inc:p1} and Fig. \ref{fig:Far:Inc:p2}, the radiation patterns are evaluated at \textbf{a,c)}  a scattering plane perpendicular to the incidence plane (vertical gray dotted line) and  at \textbf{b,d)}  a scattering plane overlapping the incidence plane.

\begin{figure}[h!]
    \centering
    \def\svgwidth{.8\textwidth}
    \hspace*{-.2\textwidth}%
    \vspace*{-3.65em}%
        \begin{subfigure}{.375\textwidth}\caption{%
                    \footnotesize$\dfrac{\norm{\vb{E}^\text{sca}_\text{far}}}{\norm{\vb{E}^\text{i}}} \times 10^{-9}$  }\label{sfig:Far:Inc:p1:a}\end{subfigure}%
        \begin{subfigure}{.4\textwidth}\caption{%
                    \footnotesize$\dfrac{\norm{\vb{E}^\text{sca}_\text{far}}}{\norm{\vb{E}^\text{i}}} \times 10^{-9}$  }\label{sfig:Far:Inc:p1:b}\end{subfigure}\\
    \includeinkscape[pretex = \footnotesize]{4-Inc-Obl/5-FarXY-P/4-5-Far-XY-P-15}\\[-.75em]
    %
    \def\svgwidth{.8\textwidth}
    \hspace*{-.21\textwidth}%
    \vspace*{-.7em}%
        \begin{subfigure}{.4\textwidth}\caption{%
                    \footnotesize$\dfrac{\norm{\vb{E}^\text{sca}_\text{far}}}{\norm{\vb{E}^\text{i}}} \times 10^{-9}$  }\label{sfig:Far:Inc:p1:c}\end{subfigure}%
        \begin{subfigure}{.4\textwidth}\caption{%
                    \footnotesize$\dfrac{\norm{\vb{E}^\text{sca}_\text{far}}}{\norm{\vb{E}^\text{i}}} \times 10^{-9}$  }\label{sfig:Far:Inc:p1:d}\end{subfigure}\\
    \includeinkscape[pretex = \footnotesize]{4-Inc-Obl/5-FarXY-P/4-5-Far-XY-P-38}%
    \caption[  Radiation pattern of a AuNP supported on a substrate illuminated at oblique incidence ]{
    Radiation pattern of a AuNP (light yellow) of radius $a = 12.5$ nm partially embedded in a glass substrate (light blue, $n_\text{s} = 1.5$) with an air matrix ($n_\text{m} = 1$) illuminated by an \textit{p} polarized incident electric plane wave $\vb{E}^\text{i}$, with a wavelength $\lambda$, traveling in the $\vb{k}^\text{i}$ direction at an angle of incidence $\theta_i$ of \textbf{a,b)} $15^\circ$ and \textbf{c,d)} $38^\circ$ relative to the normal direction to the glass-air interface. The radiation patterns consider various values of the incrustation parameter $h/a$, with $a$ the AuNP's radius and $h$ the distance between its center and the interface, and an  incident electric field \textbf{a,c)} perpendicular to the incidence plane (vertical gray dotted lines) and \textbf{b,d)} equal to the incidence plane. In all cases the incident wave vector $\vb{k}^\text{i}$, the perpendicular $\vb{E}_\perp^\text{i}$ and the  parallel $\vb{E}_\parallel^\text{i}$ projection of the incident electric field relative to the scattering plane are schematized.%
    }
    \label{fig:Far:Inc:p1}
\end{figure}


\begin{figure}[h!]
    \centering
    \def\svgwidth{.8\textwidth}
    \hspace*{-.2\textwidth}%
    \vspace*{-3.65em}%
        \begin{subfigure}{.4\textwidth}\caption{%
                    \footnotesize$\dfrac{\norm{\vb{E}^\text{sca}_\text{far}}}{\norm{\vb{E}^\text{i}}} \times 10^{-9}$  }\label{sfig:Far:Inc:p2:a}\end{subfigure}%
        \begin{subfigure}{.4\textwidth}\caption{%
                    \footnotesize$\dfrac{\norm{\vb{E}^\text{sca}_\text{far}}}{\norm{\vb{E}^\text{i}}} \times 10^{-9}$  }\label{sfig:Far:Inc:p2:b}\end{subfigure}\\
    \includeinkscape[pretex = \footnotesize]{4-Inc-Obl/5-FarXY-P/4-5-Far-XY-P-42}\\[-.75em]
    %
    \def\svgwidth{.8\textwidth}
    \hspace*{-.21\textwidth}%
    \vspace*{-.7em}%
        \begin{subfigure}{.4\textwidth}\caption{%
                    \footnotesize$\dfrac{\norm{\vb{E}^\text{sca}_\text{far}}}{\norm{\vb{E}^\text{i}}} \times 10^{-9}$  }\label{sfig:Far:Inc:p2:c}\end{subfigure}%
        \begin{subfigure}{.4\textwidth}\caption{%
                    \footnotesize$\dfrac{\norm{\vb{E}^\text{sca}_\text{far}}}{\norm{\vb{E}^\text{i}}} \times 10^{-9}$  }\label{sfig:Far:Inc:p2:d}\end{subfigure}\\
    \includeinkscape[pretex = \footnotesize]{4-Inc-Obl/5-FarXY-P/4-5-Far-XY-P-75}%
    \caption[  Radiation pattern of a AuNP supported on a substrate illuminated at oblique incidence ]{
    Radiation pattern of a AuNP (light yellow) of radius $a = 12.5$ nm partially embedded in a glass substrate (light blue, $n_\text{s} = 1.5$) with an air matrix ($n_\text{m} = 1$) illuminated by an \textit{p} polarized incident electric plane wave $\vb{E}^\text{i}$, with a wavelength $\lambda$, traveling in the $\vb{k}^\text{i}$ direction at an angle of incidence $\theta_i$ of \textbf{a,b)} $42^\circ$ and \textbf{c,d)} $75^\circ$ relative to the normal direction to the glass-air interface. The radiation patterns consider various values of the incrustation parameter $h/a$,with $a$ the AuNP's radius and $h$ the distance between its center and the interface, and an  incident electric field \textbf{a,c)} perpendicular to the incidence plane (vertical gray dotted lines) and \textbf{b,d)} equal to the incidence plane. In all cases the incident wave vector $\vb{k}^\text{i}$, the perpendicular $\vb{E}_\perp^\text{i}$ and the  parallel $\vb{E}_\parallel^\text{i}$ projection of the incident electric field relative to the scattering plane are schematized.%
    }
    \label{fig:Far:Inc:p2}
\end{figure}








In Fig. \ref{fig:Near:Inc0:42} the spatial distribution of the induces electric field is shown for a partially embedded 12.5 nm AuNP, with an incrustation parameter of $h/a = 0$, when this is illuminated at an angle of incidence of $42^\circ>\theta_c$, in an internal configuration, by a $s$ [Figs. \ref{sfig:Near:Inc0:42:s1} and \ref{sfig:Near:Inc0:42:s2}] and a $p$ [Figs. \ref{sfig:Near:Inc0:42:p1} and \ref{sfig:Near:Inc0:42:p2}] polarized incident electric field with a wavelength o $\lambda  =525$ nm, considering that the scattering plane where, the radiation pattern is evaluated, is perpendicular to the incidence plane [Figs. \ref{sfig:Near:Inc0:42:s1} and \ref{sfig:Near:Inc0:42:p1}] and overlaps the incidence plane [Figs. \ref{sfig:Near:Inc0:42:s2} and \ref{sfig:Near:Inc0:42:p2}].


\begin{figure}[h!]\centering
   \def\svgwidth{.75\textwidth}
   \footnotesize
   \captionsetup[subfigure]{labelfont ={normal,bf,color = white}}
   \includeinkscape{4-Inc-Obl/2-Near-SP/2-NearYX-SP-42}\\[-32.6em]
   \hspace*{-.25\textwidth}
       \begin{subfigure}{.25\textwidth}\textcolor{red}{\caption{ } \label{sfig:Near:Inc0:42:s1}}\end{subfigure}%
       \begin{subfigure}{.34\textwidth}\caption{ }\label{sfig:Near:Inc0:42:s2}\end{subfigure}\\[13em]
    \hspace*{-.25\textwidth}
       \begin{subfigure}{.25\textwidth}\textcolor{red}{\caption{ } \label{sfig:Near:Inc0:42:p1}}\end{subfigure}%
       \begin{subfigure}{.34\textwidth}\caption{ }\label{sfig:Near:Inc0:42:p2}\end{subfigure}\\[15em]
   \caption[Induced Electric Field of a 12.5 nm Au NP on substrate illuminated at oblique incidence with a $s$ polarized electric field]{%
   Electric field $\vb{E}^\text{ind}$ induced by a partially embedded 12.5 nm AuNP (dashed black lines) illuminated by an \textbf{a,b)} $s$ polarized and  by a \textbf{c,d)} \textit{p} polarized incident electric plane wave $\vb{E}^\text{i}$ traveling in the $\vb{k}^\text{i}$ direction, in an internal configuration, at an angle of incidence of  $42^\circ$ relative to the normal direction to the interface ---white dashed lines--- between an air matrix ($n_\text{m} = 1$) and a glass substrate ($n_\text{s} = 1.5$). The incident electric plane wave is evaluated at $\lambda = 525$ nm and the norm $\norm{\vb{E}^\text{ind}}$ is evaluated at  \textbf{a,c)} a scattering plane perpendicular to the incidence plane (vertical gray dotted lines) and at \textbf{b,d)} a scattering plane overlapping the incidence plane. }
   \label{fig:Near:Inc0:42}
 \end{figure}
