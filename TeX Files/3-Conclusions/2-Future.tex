% !TeX root = ../tesis.tex

Based on numerical calculations performed with the Finite Element Method, in this thesis it was determined the conditions at which a single spherical AuNP of radius $12.5$~nm partially embedded in an air matrix and glass substrate is able to interact with its surrounding above the substrate and  at which its optical response is maximized. This feature does not only suit the 12.5~nm~AuNP as a candidate for the meta-atoms conforming a disordered metasurface tailored for biosensing, but also give rise to  methodologies to theoretically study the optical properties of such bidimensional systems and to  experimentally determine its average incrustation degree. The proposal for these methodologies are the following:

    \begin{itemize}
    \item \textbf{Theoretical description of the optical behavior of disordered metasurfaces of partially embedded AuNPs}\\
    To include the partial embedding of the nanosphere in theoretical models that describe the optical properties of bidimensional arrays of small particles supported on a substrate, like the Thin Island Theory \cite{bedeaux_optical_2004} or the Dipolar Model \cite{barrera1991optical} which are Effective Medium Theories (EMTs), it is proposed to modify a homogenization theory for the medium surrounding the nanosphere ---like the Bruggeman or power law formulas \cite{sihvola_electromagnetic_2008}---, so that the resonance of a single nanosphere is excited at the same wavelength than that of the partially embedded nanosphere, for a fixed angle, polarization and incrustation degree.  Once a homogenization theory for the surroundings of the partially embedded nanosphere is developed, an EMT can be used considering that the nanospheres are perfectly supported on a substrate and embedded in the homogenized medium, then the optical response of the real system can be described with a stratified system of three layers: substrate-homogenized EMT-matrix. It is expected that this approach can describe the optical properties of biosensing-aimed-metasurfaces with the embedding feature that can allow them to be used long-lastingly under realistic conditions.
    %
    \item \textbf{Methodology to measure the average incrustation degree of a disordered metasurfaces of partially embedded AuNPs}\\
    To determine the average incrustation degree of bidimensional arrays of identical spherical nanospheres (small compared with the wavelength of the incident light), it can be performed reflectivity measurements with $p$ polarized white light in Kretschmann configuration for different angles of incidence. By following the redshift of the resonance wavelength as a function of the angle of incidence it can be determined the average incrustation parameter by comparing the experimental data with theoretical calculations, as those presented in Fig.~\ref{tab:Resonances}.
    \end{itemize}
