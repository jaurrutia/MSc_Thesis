% !TeX root = ../tesis.tex

The problem of scattering and absorption of light by single spherical NP embedded into a matrix, with refractive index $n_\text{m}$, illuminated by a plane wave with wavelength $\lambda$  and traveling in the  $\vb{k}^\text{i}$ direction, has spherical symmetry, which was exploited to develop the Mie Theory as explained in Section \ref{s:Mie}. If a substrate, with refractive index $n_\text{m}$, is considered and the NP is located right above or below the interface ---but not at it---, there are four configurations into which the system can be found since the NP can be either embedded (E\_) in the substrate or supported (\_S) by it, and it can be illuminated either in an external (\_E) ---from the matrix to the substrate--- or an internal (\_I)--- from the substrate to matrix--- reflection as shown in Fig. \ref{sfig:TotallyNormal:1} where the four configurations are shown: Embedded-External (EE), Embedded-Internal (EI), Supported-External (SE) and Supported-Internal (SI)).

In Fig. \ref{sfig:TotallyNormal:2} the absorption $Q_\text{abs}$ and scattering $Q_\text{sca}$ efficiencies are shown as function of $\lambda$ for the Embedded-External (black), Embedded Internal (orange), Supported-External (blue) and Supported-Internal (light orange) scenario. The green region corresponds to the values between the efficiencies 

Each marker in \ref{sfig:TotallyNormal:2} corresponds to a value calculated by means of the FEM wile the solid lines are guides to the eye


\begin{figure}[h!]
    \hspace*{-19.15em}%
    \vspace*{-1.25em}%
        \begin{subfigure}{.385\textwidth}\caption{ }\label{sfig:TotallyNormal:1}\end{subfigure}%
        \begin{subfigure}{.25\textwidth}\caption{ }\label{sfig:TotallyNormal:2}\end{subfigure} \\
    \def\svgwidth{.95\textwidth}
    \small
    \centering
    \includeinkscape{1-Totally/1-Efficiencies/1-Normal-Eff}%
    \vspace*{0em}
    \caption[Absorption and Scattering Efficiencies of a 12.5 nm AuNP above and below a lanar Interface Illuminated at Normal Incidence]{\textbf{a)} Schematics of a AuNP embedded in (supported by) a glass substrate (with refractive index $n_\text{s} = 1.5$) forming a planar interface with an air matrix ($n_\text{m} = 1$)) and illuminated by a plane wave in the normal direction in an external and in an internal configuration. \textbf{b)} Absorption $Q_\text{abs}$ and scattering $Q_\text{sca}$ efficiencies as function of the wavelength $\lambda$
    }
\label{fig:TotallyNormal}
\end{figure}



\begin{figure}
    \centering
    \def\svgwidth{.9\textwidth}
    \hspace*{-.3\textwidth}%
    \vspace*{-.5em}%
        \begin{subfigure}{.395\textwidth}\caption{\small$\dfrac{\norm{\vb{E}^\text{sca}_\text{far}}}{\norm{\vb{E}^\text{i}}} \times 10^{-9}$  }\label{sfig:red:1}\end{subfigure}%
        \begin{subfigure}{.4\textwidth}\caption{\small$\dfrac{\norm{\vb{E}^\text{sca}_\text{far}}}{\norm{\vb{E}^\text{i}}} \times 10^{-9}$  }\label{sfig:red:1}\end{subfigure}\\
    \includeinkscape[pretex = \small]{1-Totally/4-5-Far-XY-Embedded/4-5-Far-XY-Embedded-External}\\
    %
    \def\svgwidth{.9\textwidth}
    \hspace*{-.3\textwidth}%
    \vspace*{-.5em}%
        \begin{subfigure}{.395\textwidth}\caption{\small$\dfrac{\norm{\vb{E}^\text{sca}_\text{far}}}{\norm{\vb{E}^\text{i}}} \times 10^{-9}$  }\label{sfig:red:1}\end{subfigure}%
        \begin{subfigure}{.4\textwidth}\caption{\small$\dfrac{\norm{\vb{E}^\text{sca}_\text{far}}}{\norm{\vb{E}^\text{i}}} \times 10^{-9}$  }\label{sfig:red:1}\end{subfigure}\\
    \includeinkscape[pretex = \small]{1-Totally/4-5-Far-XY-Embedded/4-5-Far-XY-Embedded-Internal}%
    \caption[Spectral redshift of the scattering and extinction of a spherical AuNP as function of its size and the embedding media]{Resonance wavelength $\lambda_\text{res}$ of the scattering (orange) and extinction (black) cross sections of a AuNP as function of the NP's radius when embedded \textbf{a)} into air ($n_\text{m} = 1)$ and \textbf{b)} into glass ($n_\text{m} = 1.5$), and as function of the refractive index of the matrix  $n_\text{m}$ for a AuNP of radius equal to  \textbf{c)} 12.5 nm and \textbf{d)} 50 nm, employing the dielectric function for the Au as reported by \citeauthor{johnson_optical_1972} (filled circle) and considering a size correction to it (empty circle).}
\end{figure}



\begin{figure}
    \centering
    \def\svgwidth{.9\textwidth}
    \hspace*{-.3\textwidth}%
    \vspace*{-.5em}%
        \begin{subfigure}{.395\textwidth}\caption{\small$\dfrac{\norm{\vb{E}^\text{sca}_\text{far}}}{\norm{\vb{E}^\text{i}}} \times 10^{-9}$  }\label{sfig:red:1}\end{subfigure}%
        \begin{subfigure}{.4\textwidth}\caption{\small$\dfrac{\norm{\vb{E}^\text{sca}_\text{far}}}{\norm{\vb{E}^\text{i}}} \times 10^{-9}$  }\label{sfig:red:1}\end{subfigure}\\
    \includeinkscape[pretex = \small]{1-Totally/4-5-Far-XY-Supported/4-5-Far-XY-Supported-External}\\
    %
    \def\svgwidth{.9\textwidth}
    \hspace*{-.3\textwidth}%
    \vspace*{-.5em}%
        \begin{subfigure}{.395\textwidth}\caption{\small$\dfrac{\norm{\vb{E}^\text{sca}_\text{far}}}{\norm{\vb{E}^\text{i}}} \times 10^{-9}$  }\label{sfig:red:1}\end{subfigure}%
        \begin{subfigure}{.4\textwidth}\caption{\small$\dfrac{\norm{\vb{E}^\text{sca}_\text{far}}}{\norm{\vb{E}^\text{i}}} \times 10^{-9}$  }\label{sfig:red:1}\end{subfigure}\\
    \includeinkscape[pretex = \small]{1-Totally/4-5-Far-XY-Supported/4-5-Far-XY-Supported-Internal}%
    \caption[Spectral redshift of the scattering and extinction of a spherical AuNP as function of its size and the embedding media]{Resonance wavelength $\lambda_\text{res}$ of the scattering (orange) and extinction (black) cross sections of a AuNP as function of the NP's radius when embedded \textbf{a)} into air ($n_\text{m} = 1)$ and \textbf{b)} into glass ($n_\text{m} = 1.5$), and as function of the refractive index of the matrix  $n_\text{m}$ for a AuNP of radius equal to  \textbf{c)} 12.5 nm and \textbf{d)} 50 nm, employing the dielectric function for the Au as reported by \citeauthor{johnson_optical_1972} (filled circle) and considering a size correction to it (empty circle).}
\end{figure}



\begin{figure}  \centering
	\def\svgwidth{\textwidth} \small
		\vspace*{4.em}
		\hspace*{-.5\textwidth}
	\begin{subfigure}{.525\textwidth}\caption{ } \label{fig:NearField:par}\end{subfigure}%
	\begin{subfigure}{.49\textwidth}\caption{ }\label{fig:NearField:perp}\end{subfigure}%
	\vspace*{-7.em}\\
	\includeinkscape{1-Totally/2-NearY/2-NearY-Embedded}\\
    \def\svgwidth{\textwidth}
        %\vspace*{5.em}
        \hspace*{-.5\textwidth}
    \begin{subfigure}{.525\textwidth}\caption{ } \label{fig:NearField:par}\end{subfigure}%
    \begin{subfigure}{.49\textwidth}\caption{ }\label{fig:NearField:perp}\end{subfigure}
        \vspace*{-1.75em}\\
    \includeinkscape{1-Totally/2-NearY/2-NearY-Supported}
	\vspace*{-2em}
	\caption[Induced Electric Field of a 12.5 nm Au Spherical NP Embedded into Air at the LSPR]{Induced electric field $\vb{E}^\text{int}$ evaluated at the planes \textbf{a)} $y = 0$ and \textbf{b)} $x = 0$  of a 12.5 nm Au spherical NP (dashed lines) embedded into air ($n_\text{mat} = 1$) when illuminated by an incident plane wave with an $x$-polarized electric field $\vb{E}^\text{i}$ traveling in the direction $\vb{k}^\text{i}$ along the $z$ axis with an	 excitation wavelength $\lambda = 509$ nm of the LSPR. At the plane $x = 0$, the incident electric field is parallel to the scattering plane, while it is perpendicular to it at $x = 0$. The optical response of the 12.5 nm AuNP was modeled using a size correction to the experimental data reported by \citeauthor{johnson_optical_1972} \cite{johnson_optical_1972}.}
	\label{fig:NearField}
 \end{figure}
