% !TeX root = ../tesis.tex

The Localized Surface Plasmon Resonance (LSPR) occurs when the scattered electric field of a particle illuminated by a plane wave is described by a stationary wave on the surface of a particle, meaning that it can observed in the near field regime. For a spherical particle, the LSPR occurs when the conditions in Eqs. \eqref{eq:E-LSPR} and \eqref{eq:M-LSPR} are met, which maximize the Mie coefficients  $a_\ell$ and $b_\ell$ [Eq. \eqref{eq:MieCoef}], respectively. There are optical properties in the far field regime, that is where only the radiative contributions to the electromagnetic fields are non neglectable, that depend on $a_\ell$ and $b_\ell$, for example,  the scattering amplitude matrix $\mathbb{F}$ [Eq. \eqref{eq:FscaS}]. Therefore, the LSPR can also be identified by analyzing experimental or theoretical results of optical properties such as the scattering $C_\text{sca}$ and extinction cross sections $C_\text{ext}$, which are related to $\mathbb{F}$.  Hereby, explicit expressions for the optical properties in the far field regime, for a scattering sphere, are obtained.

By substituting the scattering amplitude matrix for a spherical particle [Eq. \eqref{eq:FscaS}]  into Eqs. \eqref{eq:Csca} and \eqref{eq:Cext}  the scattering $C_\text{sca}$ and extinction $C_\text{ext}$ cross sections are obtained; the absorption
cross section $C_\text{abs}$ can be calculated by calculating $C_\text{ext} - C_\text{sca}$. Assuming an incident plane wave with an $x$-polarized electric field $\vb{E}^\text{i}$, and evaluating the scattering amplitude matrix in the forward direction $\theta = 0$, equivalent to $\cos\theta = 1$ , the extinction cross section $C_\text{ext}$ is given by
%
\begin{align}
	C_\text{ext} = \frac{4\pi}{k \norm{\vb{E}^\text{i}}^2}\Im[\frac{i}{k}S_2(\theta = 0)\vb{E}^\text{i}\cdot\vb{E}^\text{i*}]
	  			 = \frac{2 \pi}{k^2}\sum_{\ell = 1}^\infty (2 \ell + 1) \Re(a_\ell + b_\ell),
	\label{eq:CextSphere}
\end{align}
where the Eq. \eqref{eq:PiTau1} in Appendix \ref{app:MieCode} was employed to evaluate the angular functions $\pi_\ell(\cos\theta)$ and $\tau_\ell(\cos\theta)$. In a similar manner, the scattering cross section $C_\text{sca}$  can be written as
%
\begin{align}
C_\text{sca} = \int_0^{2\pi}\int_0^\pi  \frac{(iS_2(\theta)\vb{E}^\text{i})^*(iS_2(\theta)\vb{E}^\text{i})}{k^2\vb{E}^\text{i}} \sin\theta\dd{\varphi}\dd{\theta}
			 = \frac{2 \pi}{k^2}\sum_{\ell = 1}^\infty (2 \ell + 1) (\abs{a_\ell}^2 + \abs{b_\ell}^2),
	\label{eq:CscaSphere}
\end{align}
%
where the orthogonality relations of $\pi_\ell(\cos\theta)\pm\tau_\ell(\cos\theta)$ [Eq. \eqref{eq:(pipmtau)} in Appendix \ref{app:MieCode}] were used. In order to compare the absorption, scattering or extinction of light by a spherical particle, independently of its radius $a$ or embedding media (matrix), it is convenient to define the efficiencies of  absorption $Q_\text{abs}$, scattering $Q_\text{sca}$ and extinction $Q_\text{ext}$  by normalizing the  absorption, scattering and extinction cross sections by the geometrical cross section of the spherical particle, yielding the dimensionless expressions
%
\begin{align}
 	\frac{C_\text{ext}}{\pi a^2} =   \frac{C_\text{abs}}{\pi a^2}  + \frac{C_\text{sca}}{\pi a^2}
 		\qquad \longrightarrow \qquad
	Q_\text{ext} =    Q_\text{abs}  +  Q_\text{sca}.
	\label{eq:Efficiencies}
\end{align}
%

The Eq. \eqref{eq:Efficiencies}, along with optical theorem [Eq. \eqref{eq:CextFull}], states that the extinction of light considers  the combination of both absorption and scattering mechanisms. Since the analytical expression of $C_\text{ext}$ for a spherical particle [Eq. \eqref{eq:CextSphere} is proportional to the real parts of the Mie coefficients $a_\ell$ and $b_\ell$, then it is also maximized at the LSPR. Therefore, the LSPR can be observed in the far field by calculating or measuring the extinction cross section.

In order to study the LSPR of a spherical AuNP of radius $a = 12.5$ nm, the extinction  $Q_\text{ext}$ and the scattering  $Q_\text{sca}$ efficiencies are shown in Fig. \ref{fig:Mieefficiencies}, as function of the wavelength $\lambda$ of the incident plane wave illuminating the NP. Two different matrices were considered: a matrix of air with a refractive index of $n_\text{mat} = 1$ (black lines) and of glass with $n_\text{mat} = 1.5$ (orange lines). The optical response of the AuNP was modeled by a dielectric function considering the raw data (solid lines) from \citeauthor{johnson_optical_1972} \cite{johnson_optical_1972}, and a size correction to it (dashed lines). In all cases, the LSPR wavelength is indicated in the figure at the maximum of the extinction efficiencies and the wavelength of maximum scattering is also indicated.

% -------------------------------------- Extinctintion and scattering efficiencies -------------------------------
% --------------------------------------          12.5 nm AuNP @ Air and @ glass   -------------------------------
% --------------------------------------               fig:Mieefficiencies         -------------------------------
\begin{figure}[h!]
	\def\svgwidth{1\textwidth} \small
  \vspace*{3.0em}
  \hspace*{-.25\textwidth}
    \begin{subfigure}{.51\textwidth}\caption{ }\label{fig:Mieefficiencies:a}\end{subfigure}
    \begin{subfigure}{.49\textwidth}\caption{ }\label{fig:Mieefficiencies:b}\end{subfigure}
  \vspace*{-6.em}\\
  \includeinkscape{Mie-Au/1-Efficiencies}
  \vspace*{-2em}
  \caption[Extinction and Scattering Efficiency of a 12.5 nm Au Spherical NP embedded into Air and Glass]{ \textbf{a)} Extinction $Q_\text{ext}$ and \textbf{b)} scattering $Q_\text{sca}$ efficiencies of a 12.5 nm Au spherical NP embedded into air (black, $n_\text{mat} = 1$)  and into glass (orange, $n_\text{mat} = 1.33$), as function of the wavelength $\lambda$ of the incident plane wave.  The solid curves were calculated by considering no size effects on the dielectric function of the AuNP, while the dashed curves consider a size correction to it; the experimental data of \citeauthor{johnson_optical_1972} \cite{johnson_optical_1972} was employed.}
\label{fig:Mieefficiencies}
\end{figure}
% --------------------------------------               fig:Mieefficiencies         -------------------------------

By comparing Figs. \ref{fig:Mieefficiencies:a} and \ref{fig:Mieefficiencies:b}, it is determined that the main loss mechanism in the system is absorption since $Q_\text{sca}$ is two orders of magnitude smaller than $Q_\text{ext}$ for all $\lambda$ in the visible spectrum. Yet, another difference between  $Q_\text{sca}$ and  $Q_\text{ext}$ is the value of $\lambda$ that maximizes them: for the chosen system the $\lambda$ of maximum scattering is red shifted $\sim 12$ nm from  LSPR excitation wavelength in all cases. On the other size, an effect common for both the scattering and the extinction efficiencies is an overall enhancement  when the refractive index of the matrix increases, as well as a redshift of $\sim 25$ nm of the LSPR excitation wavelength and the wavelength of maximum scattering ---compare the black curves ($n_\text{mat} = 1$, air) with the orange ones ($n_\text{mat} = 1.5$, glass)---, which can be understood by analyzing the size parameter $ x = (2 \pi / \lambda ) a n_\text{mat} $. Since $x$ is a linear function of $n_\text{mat}$, the AuNP embedded into glass optically responds like a larger NP than what it is in air, thus having a more significant contribution from the scattering to the light extinction mechanism inside glass, as well as an increase in the absorption.

The effect of the size correction to the dielectric function of the AuNP can be understood by comparing the solid and dashed lines. On the one hand, there is a spectral shift of the LSPR excitation wavelength  of $\sim 2$ nm. On the other hand, the value of the efficiencies around the wavelength where the extinction and the scattering is maximized decreases in all cases shown if Fig. \ref{fig:Mieefficiencies}.  This behavior can be explained by how the size correction is performed: as explained in Appendix \ref{app:SizeCorrection}, the surface effects are taken into account by introducing a smaller mean free path for the free electrons inside the AuNP, therefore increasing the value of the damping constant and thus leading to a larger imaginary part for the dielectric functions employed, which is related to the absorption mechanisms \cite{ibach_solid-state_2009}. The decrease in the efficiencies due to a size corrected dielectric function is more evident for a matrix of glass than of air, since the AuNP is optically bigger in such matrix as explained above.  From this analysis it can be concluded that the most notable effect of a size correction to the dielectric function of a NP is the decrease in the extinction and scattering efficiencies, while there is still a spectral shift of the LSPR, whose effect is less relevant the larger the size parameter is.

While the scattering efficiency $Q_\text{sca}$ is an integral quantity, that is, it describes the scattering in all directions of a plane wave traveling in the  direction $\vb{k}^\text{i}$ due to the interaction with a NP, the scattering amplitude matrix elements $S_1(\theta)$,  given  for a spherical NP by Eq. \eqref{eq:S1}, and  $S_2(\theta)$, by Eq. \eqref{eq:S2}, depict the  electric field $\vb{E}^\text{s}$, at a measurement angle $\theta$,  scattered by a NP polarized in a direction perpendicular ($\perp$) to the scattering plane and parallel ($\parallel$) to it, respectively. A radiation pattern helps to visualize the behavior of $S_{1,2}(\theta)$, a dimensionless parameter such as the scattering efficiency, by plotting their squared modulus as function of $\theta$, as it is shown in for a 12.5 nm AuNP in Fig. \ref{fig:ScatteringMaps}, where $\abs{S_{1}(\theta)}^2$ (solid lines) and $\abs{S_2(\theta)}^2$ (dashed lines) are shown for two different scenarios: a AuNP embedded into air [Fig. \ref{fig:ScatteringMaps:a}, black curves]  illuminated  at a wavelength $\lambda = 522$ nm and a AuNP embedded into glass  [Fig. \ref{fig:ScatteringMaps:b}, orange curves]  illuminated at $\lambda = 543$ nm. The wavelengths of the incident plane wave corresponds to the value of $\lambda$ where $Q_\text{sca}$ is maximized for each matrix as seen in Fig. \ref{fig:Mieefficiencies}.

% --------------------------------------				 Scaattering Maps 	   ------------------------------
% --------------------------------------   12.5 nm AuNP @ Air and @ glass    ------------------------------
% --------------------------------------         fig:ScatteringMaps         -------------------------------
\begin{figure}[t!]
	\small\centering
	\def\svgwidth{.9\textwidth}
		\vspace*{1.5em}
		\hspace*{-.25\textwidth}
	\begin{subfigure}{.45\textwidth}%
		\caption{$\abs{S_{1,2}(\theta)}^2\times 10^4$} \label{fig:ScatteringMaps:a}%
		\end{subfigure}%
	\begin{subfigure}{.45\textwidth}%
		\caption{$\abs{S_{1,2}(\theta)}^2\times 10^3$}\label{fig:ScatteringMaps:b}%
		\end{subfigure}%
	\vspace*{-4.5em}\\
	\includeinkscape{Mie-Au/2-ScatteringMaps}
	\vspace*{-.5em}
	\caption[Radiation Pattern of a 12.5 nm Au Spherical NP embedded into Air and Glass]{Radiation pattern of a 12.5 nm Au spherical NP embedded into \textbf{a)} air illuminated by a plane wave at a wavelength of $\lambda = 522$ nm and \textbf{b)} glass illuminated at $\lambda = 543$ nm; the wavelength in each case corresponds to the wavelength of maximum scattering (see Fig. \ref{fig:Mieefficiencies}). The solid (dashed) lines corresponds to the scattering matrix element $S_1$ ($S_2$) related to an incident electric field $\vb{E}^\text{i}$ traveling in the $\vb{k}^\text{i}$ direction  and polarized perpendicularly (parallel) to the page. It was considered for both matrices a size correction to the experimental data of \citeauthor{johnson_optical_1972} \cite{johnson_optical_1972} for the electromagnetic response of the AuNP.}
	\label{fig:ScatteringMaps}
 \end{figure}
 % --------------------------------------         fig:ScatteringMaps         -------------------------------

The quantities $\abs{S_{1,2}(\theta)}^2$ for a 12.5 nm AuNP embedded into air ($n_\text{mat} = 1$, black curves) are one order of magnitude smaller  than into glass ($n_\text{mat} = 1.5$, orange curves), meaning that the AuNP scatters light less efficiently in air than in glass, which is consistent with the obtained values for the scattering efficiency $Q_\text{sca}$ in Fig. \ref{fig:Mieefficiencies}. On the angular dependency, the radiation pattern of the AuNP in both matrices follow the same tendencies: an homogeneous scattered electric field when the AuNP scatters the perpendicularly polarized incident electric field  $\vb{E}^\text{s}_\perp$ (continuous lines), and a two-lobes pattern when illuminated with a parallel polarized $\vb{E}^\text{i}$. The observed radiation pattern can be identify in the near field regime, see Fig. \ref{fig:NearField}, nevertheless within the radiation pattern analysis, the presence of the LSPR is lost, unlike within an analysis of the extinction cross section.
