% !TeX root = ../tesis.tex

The problem of scattering and absorption of light by single spherical NP embedded into a matrix, with refractive index $n_\text{m}$, illuminated by a plane wave with wavelength $\lambda$  and traveling in the  $\vb{k}^\text{i}$ direction, has spherical symmetry, which was exploited to develop the Mie Theory as explained in Section \ref{s:Mie}. If a substrate, with refractive index $n_\text{s}$, is considered and the NP is located right above or below the interface ---but not at it---, there are four combinations into which the system can be found since the NP can be either embedded  in the substrate or supported by it, and it can be illuminated either in an external ---from the matrix to the substrate--- or an internal ---from the substrate to matrix--- configuration as shown in Fig. \ref{sfig:TotallyNormal:1} where the four cases are shown: Embedded-External (EE), Embedded-Internal (EI), Supported-External (SE) and Supported-Internal (SI). In the  presence of the substrate, the electric field illuminating the AuNP is not the incoming plane wave but the sum of it with the reflected electric field (EI and SE) or the transmitted electric field (EE and SI), both of wich can calculated analytically through Fresnel's reflection and transmission amplitude coefficients as discussed in Appendix \ref{app:COMSOL}.

In Fig. \ref{sfig:TotallyNormal:2} the absorption $Q_\text{abs}$ and scattering $Q_\text{sca}$ efficiencies are shown as function of $\lambda$ for a AuNP of radius $a = 12.5$ nm in the Embedded-External (black), Embedded-Internal (orange), Supported-External (blue) and Supported-Internal (light orange) configurations; the green shaded areas corresponds to the values between the two limiting cases given by the Mie theory: the AuNP embedded in air (lower boundary) and embedded in glass (upper boundary). The magenta markers correspond to the values of the efficiencies evaluated at the wavelength of resonance considering the presence of a substrate while the red markers correspond to the efficiencies at the resonance wavelength for the Mie-limiting cases.

\begin{figure}[b!]
    \hspace*{-19.15em}%
    \vspace*{-1.25em}%
        \begin{subfigure}{.385\textwidth}\caption{ }\label{sfig:TotallyNormal:1}\end{subfigure}%
        \begin{subfigure}{.25\textwidth}\caption{ }\label{sfig:TotallyNormal:2}\end{subfigure} \\
    \def\svgwidth{.95\textwidth}
    \small
    \centering
    \includeinkscape{1-Totally/1-Efficiencies/1-Normal-Eff}%
    \vspace*{0em}
    \caption[Absorption and Scattering Efficiencies of a 12.5 nm AuNP above and below a planar Interface Illuminated at Normal Incidence]{\textbf{a)} Schematics of a AuNP embedded (E) in [supported (S) by] a glass substrate ($n_\text{s} = 1.5$) forming a planar interface with an air matrix ($n_\text{m} = 1$) and illuminated by a plane wave in the normal direction in an external (E) and in an internal (I) configuration. \textbf{b)} Absorption $Q_\text{abs}$ and scattering $Q_\text{sca}$ efficiencies of a $12.5$ nm AuNP as function of the illuminated wavelength $\lambda$ in different spatial configurations: EE (black), EI (orange), SE (blue) and SI (light orange). The green shaded region shows the two Mie-limiting cases of a  AuNP embedded in air and in glass; the magenta (AuNP and substrate) and red (Mie-limiting) markers corresponds to the efficiencies evaluated at the wavelength of resonance for each case.
    }
\label{fig:TotallyNormal}
\end{figure}

From the results in Fig. \ref{sfig:TotallyNormal:2} it can be seen that both the scattering and absorption efficiencies of the four spatial configurations with substrate are of the same order of magnitud as the Mie-limiting cases and, even more, the nominal values of the efficiencies for the embedded AuNP (black and orange lines) are around the Mie-limiting case of the AuNP in glass (upper boundary of the green shaded region) and the same is behavior is observed for the supported AuNP (blue and light orange lines) and the Mie-limiting case of a AuNP embedded in air (lower boundary of the green shaded region). The presence of a substrate yields an overall enhancement and damping of the scattering and the absorption efficiencies relative to the isolated NP, which depend on the illumination of the system since $Q_\text{abs}$ and $Q_\text{ext}$ are inversely proportional to the refractive index of the medium of incidence [Ecs. \eqref{eq:Csca} and \eqref{eq:Cabs}]: If the system is illuminated in an external configuration, the obtained efficiencies decrease relative to the Mie-limiting case as it can be seen from the black  and blue curves, which corresponds to the EE and SE cases; on the other hand, the calculated efficiencies for the internal illuminated cases, that is for EI (orange) and SI (light orange), are enhanced relative to the Mie-limiting cases.

Another effect of the substrate in the optical response of the system is a spectral shift of the scattering and absorption efficiencies, which depends on the medium where the AuNP is located. For example, in Fig. \ref{sfig:TotallyNormal:2} the wavelength of resonance (magenta markers) for both the absorption and the scattering efficiencies are redshifted $\sim 1$ nm, relative to the Mie-limiting case (red markers), for the AuNP supported on the substrate (blue and light orange curves) and blueshifted $\sim 2$ nm for the embedded AuNP (black and orange curves). These spectral shifts can be understood by considering the AuNPs as point dipoles parallel to the interface, as the radiation pattern and the electric near field distribution in the Mie-limiting cases suggests (see Figs. \ref{fig:ScatteringMaps} and  \ref{fig:NearField}), and its interaction with the image dipole induced due to the interface between the matrix and the substrate. Both the real and the image dipoles are parallel to the interface but its strength differs by  a factor of $A_\text{dip} = (\sqrt{n_j}-\sqrt{n_i}) / (\sqrt{n_j}+\sqrt{n_i})$ \cite{barrera1991optical}, where $n_{j}$ is the refractive index of the medium where the real dipole (the AuNP) is located and $n_i$ of the medium where the image dipole is induced. On the one hand, if the AuNP is embedded into the substrate, then $A_\text{dip}>0$ meaning that the induced dipole is parallel to the real dipole, which is a more energetic configuration that yields the spectral blueshift of the resonance. On the other hand, if the AuNP is supported by the substrate then $A_\text{dip}<0$ and the induced dipole is antiparallel to the real dipole, leading to a less energetic configuration and to the redshift observed in Fig. \ref{sfig:TotallyNormal:2}.%
\index{Dipole!Image!Strength}

The absorption and scattering efficiencies are integral quantities which describe the global behavior of the induced electric field $\vb{E}^\text{ind}$, which corresponds to the internal electric field $\vb{E}^\text{int}$ inside the AuNP and to the scattered electric field $\vb{E}^\text{sca}$ outside of it. The distribution of $\vb{E}^\text{ind}$, for a fixed wavelength, is studied in two spatial regimes: the far- and the near field. To analyze the optical response in the first regime, the radiation patterns of the AuNP are obtained numerically by plotting the norm of the scattered electric field in the far field regime\footnote{\label{fnote:Stratton:Chu}%
    The FEM returns the induced electric field by a scatterer in a neighborhood around it  and there is no guarantee that the returned electric field, even at the boundaries of the volume where the FEM simulation is performed, corresponds to the far field regime.  To calculate the radiation pattern from the obtained induced electric field, COMSOL Multiphysics\texttrademark{} Ver. 5.4  employs the  Stratton-Chu formula \cite{comsol_wave}, which is a near field to far field  transformation that  propagates the known electric near field  over a mathematical surface surrounding all the scatterers  to an arbitrary point \cite{anyutin_algorithm_2019}. The Stratton-Chu formula is obtained by employing the vectorial generalization of the Green's second identity with the electric and magnetic near fields and the Green's function to the scalar Helmholtz equation multiplied by a normal vector to the integration surface \cite{stratton_diffraction_1939}.%
    } %
 $\vb{E}^\text{sca}_\text{far}$ as function of the the angle relative to the normal to the interface. In Figs. \ref{fig:Far:Emb:Norm} and  \ref{fig:Far:Sup:Norm}, it is shown the radiation patterns of the embedded and the supported AuNP, respectively, for several values of the wavelength $\lambda$ of the incident plane wave, as well as considering an illumination of the system in an  [\textbf{a)} and \textbf{b})] external and in an  [\textbf{c)} and \textbf{d})] internal configuration; additionally, it is considered that the incident electric field is totally [\textbf{a)} and \textbf{c})] parallel to the scattering plane $\vb{E}^\text{i}_\parallel$ and [\textbf{b)} and \textbf{d})] perpendicular to the scattering plane $\vb{E}^\text{i}_\perp$.

\begin{figure}[b!]
    \centering
    \def\svgwidth{.8\textwidth}
    \hspace*{-.215\textwidth}%
    \vspace*{-.5em}%
        \begin{subfigure}{.32\textwidth}\caption{\footnotesize$\dfrac{\norm{\vb{E}^\text{sca}_\text{far}}}{\norm{\vb{E}^\text{i}}} \times 10^{-9}$  }\label{sfig:Far:Emb:Norm:a}\end{subfigure}%
        \begin{subfigure}{.4\textwidth}\caption{\footnotesize$\dfrac{\norm{\vb{E}^\text{sca}_\text{far}}}{\norm{\vb{E}^\text{i}}} \times 10^{-9}$  }\label{sfig:Far:Emb:Norm:b}\end{subfigure}\\
    \includeinkscape[pretex = \footnotesize]{1-Totally/4-5-Far-XY-Embedded/4-5-Far-XY-Embedded-External}\\
    %
    \def\svgwidth{.8\textwidth}
    \hspace*{-.215\textwidth}%
    \vspace*{-.5em}%
        \begin{subfigure}{.32\textwidth}\caption{\footnotesize$\dfrac{\norm{\vb{E}^\text{sca}_\text{far}}}{\norm{\vb{E}^\text{i}}} \times 10^{-9}$  }\label{sfig:Far:Emb:Norm:c}\end{subfigure}%
        \begin{subfigure}{.4\textwidth}\caption{\footnotesize$\dfrac{\norm{\vb{E}^\text{sca}_\text{far}}}{\norm{\vb{E}^\text{i}}} \times 10^{-9}$  }\label{sfig:Far:Emb:Norm:d}\end{subfigure}\\
    \includeinkscape[pretex = \footnotesize]{1-Totally/4-5-Far-XY-Embedded/4-5-Far-XY-Embedded-Internal}%
    \caption[ Radiation pattern of a AuNP totally embedded into a substrate illuminated at normal incidence ]{Radiation pattern of a AuNP (light yellow) of radius $a = 12.5$ nm embedded into a substrate (light blue) illuminated by an electric plane wave of  wavelength $\lambda$ traveling in the $\vb{k}^\text{i}$ direction, normal to the interface between the substrate ($n_\text{s} = 1.5$) and the matrix ($n_\text{m} = 1)$. The radiation patterns consider the illumination of the system  \textbf{a,b)} in an external and  \textbf{c,d)} in an internal configuration, and the incident electric field \textbf{a,c)} parallel to the scattering plane $\vb{E}^\text{i}_\parallel$ and \textbf{b,d)} perpendicular to the scattering plane $\vb{E}^\text{i}_\perp$.}
    \label{fig:Far:Emb:Norm}
\end{figure}

The radiation patterns of both the embedded  and the supported AuNP follow the sames shapes independently of the illuminating wavelength $\lambda$ but their amplitude is modulated by the scattering efficiencies shown in Fig. \ref{sfig:TotallyNormal:2}. For example, in the EE  and EI cases [Fig. \ref{fig:Far:Emb:Norm}] the scattered electric field (in the far field) decreases its amplitude as the wavelength increases from $400$ nm to $480$ nm (black, orange and blue curves) and from   $550$ nm to $480$  nm, while it increases from $505$ nm to $542$ nm ,near the wavelength of resonance for the scattering efficiency. Similarly, for the SE and SI  the amplitude of the far field is modulated by its sacttering efficiency as is can be seen from comparing the radiation patterns in Fig. \ref{fig:Far:Sup:Norm} at $400$ mn (black), $542$ nm (blue) and $480$ nm, $527$ nm (purple), with the the value of $Q_\text{sca}$ at those wavelengths, which corresponds to a global maximum, a global minimum and a local maximum at the wavelength of resonance, respectively [see Fig. \ref{sfig:TotallyNormal:2}].

\begin{figure}[t!]
    \centering
    \def\svgwidth{.8\textwidth}
    \hspace*{-.215\textwidth}%
    \vspace*{-.5em}%
        \begin{subfigure}{.32\textwidth}\caption{\footnotesize$\dfrac{\norm{\vb{E}^\text{sca}_\text{far}}}{\norm{\vb{E}^\text{i}}} \times 10^{-9}$  }\label{sfig:Far:Sup:Norm:a}\end{subfigure}%
        \begin{subfigure}{.4\textwidth}\caption{\footnotesize$\dfrac{\norm{\vb{E}^\text{sca}_\text{far}}}{\norm{\vb{E}^\text{i}}} \times 10^{-9}$  }\label{sfig:Far:Sup:Norm:b}\end{subfigure}\\
    \includeinkscape[pretex = \footnotesize]{1-Totally/4-5-Far-XY-Supported/4-5-Far-XY-Supported-External}\\
    %
    \def\svgwidth{.8\textwidth}
    \hspace*{-.215\textwidth}%
    \vspace*{-.5em}%
        \begin{subfigure}{.32\textwidth}\caption{\footnotesize$\dfrac{\norm{\vb{E}^\text{sca}_\text{far}}}{\norm{\vb{E}^\text{i}}} \times 10^{-9}$  }\label{sfig:Far:Sup:Norm:c}\end{subfigure}%
        \begin{subfigure}{.4\textwidth}\caption{\footnotesize$\dfrac{\norm{\vb{E}^\text{sca}_\text{far}}}{\norm{\vb{E}^\text{i}}} \times 10^{-9}$  }\label{sfig:Far:Sup:Norm:d}\end{subfigure}\\
    \includeinkscape[pretex = \footnotesize]{1-Totally/4-5-Far-XY-Supported/4-5-Far-XY-Supported-Internal}%
    \caption[  Radiation pattern of a AuNP supported into a substrate illuminated at normal incidence ]{Radiation pattern of a AuNP (light yellow) of radius $a = 12.5$ nm embedded into a substrate (light blue) illuminated by an electric plane wave of  wavelength $\lambda$ traveling in the $\vb{k}^\text{i}$ direction, normal to the interface between the substrate ($n_\text{s} = 1.5$) and the matrix ($n_\text{m} = 1)$. The radiation patterns consider the illumination of the system  \textbf{a,b)} in an external and  \textbf{c,d)} in an internal configuration, and the incident electric field \textbf{a,c)} parallel to the scattering plane $\vb{E}^\text{i}_\parallel$ and \textbf{b,d)} perpendicular to the scattering plane $\vb{E}^\text{i}_\perp$.}
    \label{fig:Far:Sup:Norm}
\end{figure}

The shape of the radiation pattern of a 12.5 nm AuNP in the presence of a substrate, either embedded or supported, resembles that of the isolated 12.5 AuNP discussed in Section \ref{sss:FarField} in that it follows a two-lobe and a one-lobe pattern depending on the orientation of $\vb{E}^\text{i}$ relative to the scattering plane. If the incident electric field is parallel to the scattering plane, a two-lobe pattern aligned to the direction $\vb{k}^\text{i}$ of the incident ---and transmitted--- plane wave arises as it can be seen in the Figs. \ref{sfig:Far:Emb:Norm:a} and  \ref{sfig:Far:Emb:Norm:c} for the EE case, and Figs.  \ref{sfig:Far:Sup:Norm:a} and \ref{sfig:Far:Sup:Norm:c} for the EI scenario. Contrastingly, when the incident electric field is perpendicular to the scattering plane, the one-lobe pattern can be identified [see Figs. \ref{sfig:Far:Emb:Norm:b} and \ref{sfig:Far:Emb:Norm:d} (SE), and \ref{sfig:Far:Sup:Norm:b} and \ref{sfig:Far:Sup:Norm:d} (SI)]. By comparing the Mie-limiting radiation pattern (see Fig. \ref{fig:ScatteringMaps}), to the radiation patterns considering a substrate, the later loses the polar symmetry observed in the Mie-limiting case.  In particular, the amplitude of $\vb{E}^\text{sca}_\text{far}$ is larger when evaluated at the medium of incidence  than at medium of transmission; this asymmetry is observed for both illuminating configurations (external and internal) and it does not depend on whether the AuNP is supported or embedded. Rather, the spatial configuration of the system determines the overall value of the far field: when the AuNP is embedded, the far field amplitude is greater by a factor of $2.5$  than when the AuNP is supported ---see scale axes in Figs. \ref{fig:Far:Emb:Norm} and \ref{fig:Far:Sup:Norm}---; this phenomena  is a consequence of the two following physical mechanisms. The first is the substrate having a greater refractive index than the matrix, thus making the optical response of the 12.5 nm AuNP as that of a larger NP ---but still small compared to the illuminating wavelength---, as in the Mie-limiting case. The second mechanism is the relative alignment of a point dipole ---small particle approximation to the AuNP--- and the induced dipole due to the interface, which is parallel when the AuNP is embedded onto the substrate and antiparallel when supported by it, thus leading to a more energetic configuration when the AuNP is located at the substrate than at the matrix.

The radiation pattern, an optical property observed in the far field regime, is a manifestation of the near field spatial distribution ---see the footnote on page \pageref{fnote:Stratton:Chu}--- which can be calculated numerically through the FEM for a AuNP of radius $a = 12.5$ nm. The scattered electric field in the  far field regime of a AuNP embedded or supported  [Figs. \ref{fig:Far:Emb:Norm} and \ref{fig:Far:Sup:Norm}] share some characteristic to the radiated field of an isolated AuNP (Mie-limiting case), and thus should the near field. In Fig. \ref{fig:Near:IntExt} it is shown the module of the induced electric field $\vb{E}^\text{ind}$ when the AuNP is illuminated by a $x$-polarized incident electric field $\vb{E}^\text{i}$ traveling in the $\vb{k}^\text{i}$ direction, perpendicular to the interface between air and glass; the induced electric field is evaluated at a plane parallel incident electric field ($y = 0$). The wavelength $\lambda$ of the incoming plane wave is $\lambda =535.2$ nm for a embedded AuNP either illuminated externally [Fig.  \ref{sfig:Near:EE}] or internally  [Fig.  \ref{sfig:Near:EI}] and  $\lambda =510$ nm for a supported AuNP either illuminated externally [Fig.  \ref{sfig:Near:SE}] or internally  [Fig.  \ref{sfig:Near:SI}], which corresponds to the wavelengths of the Localized Surface Plasmon Resonance (LSPR).

\begin{figure}[t!]\centering
   \def\svgwidth{.75 \textwidth}
   \footnotesize
   \captionsetup[subfigure]{labelfont ={normal,bf,color = white}}
   \includeinkscape{1-Totally/2-NearY/2-NearY-EmbSup}\\[-32.6em]
   \hspace*{-.25\textwidth}
       \begin{subfigure}{.25\textwidth}\textcolor{red}{\caption{ } \label{sfig:Near:EE}}\end{subfigure}%
       \begin{subfigure}{.36\textwidth}\caption{ }\label{sfig:Near:EI}\end{subfigure}\\[13em]
   \hspace*{-.25\textwidth}
       \begin{subfigure}{.25\textwidth}\caption{ } \label{sfig:Near:SE}\end{subfigure}%
       \begin{subfigure}{.36\textwidth}\caption{ }\label{sfig:Near:SI}\end{subfigure}\\[15em]
   \caption[Induced Electric Field of a 12.5 nm Au Spherical NP embbeded into (supported by) a substrate illuminated at a normal incidence]{Induced electric field $\vb{E}^\text{ind}$, evaluated at the plane $y = 0$, of a 12.5 nm AuNP (dashed black lines) \textbf{a,b)} embedded into a glass substrate ($n_\text{s} = 1.5$) and \textbf{c,d)} supported by it in an air matrix ($n_\text{m} = 1$) illuminated by plane wave with an $x$-polarized incident electric field $\vb{E}^\text{i}$ traveling in the direction $\vb{k}^\text{i}$ perpendicular to the interface between the air and the glass (dashed white lines). The system is illuminated \textbf{a,c)}  in an external and \textbf{b,d)} in an internal configuration at the resonance  wavelength for the absorption efficiency: $535.2$ nm for the embedded AuNP and $510$ nm for the supported AuNP. The incident electric field is parallel to the scattering plane at $y=0$.
   }
   \label{fig:Near:IntExt}
 \end{figure}

The spatial distribution of the near field, shown in Fig. \ref{fig:Near:IntExt}, is consistent with the description and explanation of both the absorption ans scattering efficiencies [Fig. \ref{sfig:TotallyNormal:2}] and the radiation patterns of the embedded [Fig. \ref{fig:Far:Emb:Norm}] and the supported [Fig. \ref{fig:Far:Sup:Norm}] AuNP. The induced electric field is in general, stronger when the AuNP is embedded into the substrate than when it is supported by it as it can be seen in the value of the hotspots around the AuNP: reddish regions in Figs. \ref{sfig:Near:EE} and \ref{sfig:Near:EI} and bluish in Figs. \ref{sfig:Near:SE} and \ref{sfig:Near:SI}. These hotspots also verifies that at the resonance wavelength, the main contribution to the electric fields is due to a dipolar moment since the characteristic two-lobe distribution of the near field can be easily identified nevertheless, the lobes are not horizontally aligned to the AuNP's equator but farther from the substrate for the embedded AuNP and closer to it for the supported AuNP, as if the induced dipole ---in the small particle approximation, where the AuNP is treated as a point dipole--- is parallel (perpendicular) to the dipolar moment induced in the AuNP when it is embedded into (supported by) the substrate, as discussed above.

Throughout this section, it was studied the optical properties of a 12.5 nm AuNP on the presence of a substrate considering four configurations: the AuNP either embedded or supported and the system illuminated from under the substrate or from above. The choice of normal incidence to the system allowed the obtained results to be compared with the Mie-limiting case, which lead to the identification of similarities and differences among the four configurations. The differences in the optical response are associated to the broken symmetry due to the two semiinfinte media now considered, while the similarities arise since the system is always illuminated by a plane wave independently of the choice of the medium of incidence, yielding a mostly dipolar electric field. Therefore, in the next section the oblique incidence case is addressed only when the AuNP is supported and illuminated in the internal configuration since it is the only case with a different illumination to the system: an evanescent wave when illuminated at an angle above the critical angle $\theta_c = (n_\text{m}/n_\text{s})$ \cite{born_max_principle_1999}.
