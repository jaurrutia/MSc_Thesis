% !TeX root = ../tesis.tex

 Let $\vb{E}^{\text{i}}$ be an $x$-polarized plane wave traveling in the  $\vb{e}_z$ direction; its representation in the spherical unit vector basis is
 % ---------------- index entries
 \index{Plane!Wave!$x$ Polarized,!Spherical Unit Vector Basis}%
 \index{Wave!Plane!$x$ Polarized,!Spherical Unit Vector Basis}%
 %
 \begin{align}
 \vb{E}^{\text{i}} (\vb{r})= E_0\qty(\sin\theta\cos\varphi \, \vu{e}_r +
					\cos\theta\cos\varphi \,\vu{e}_\theta
					- \sin\varphi \,\vu{e}_\varphi) \exp(ikr\cos\theta).
	\label{eq:ExPlane}
 \end{align}
 %
The monochromatic plane wave is a transversal wave, thus it can be written in terms of only $\vb{M}^{(1)}$ and $\vb{N}^{(1)}$, where the radial dependency is given by $j_\ell$ since the monochromatic plane wave is finite everywhere. Even more, due to the dependence on $\varphi$, it is only restricted to values of $m = 1$. By inspection on the radial component of $\vb{E}^\text{i}$, proportional to $\cos\varphi$, it depends only on $\vb{N}_{e1\ell}^{(1)}$, and on the azimuthal component, proportional to $\sin\varphi$, it depends only on $\vb{M}_{o1\ell}^{(1)}$. Thus, Eq. \eqref{eq:ExPlane} can be written as a linear combination of  $\vb{N}_{e1\ell}^{(1)}$ and $\vb{M}_{o1\ell}^{(1)}$. Through the orthogonality relations of the VSH [see Eqs. \eqref{eq:LL}--\eqref{eq:LM} from Appendix \ref{app:MieCode}], the $x$-polarized plane wave can be written as \cite{stratton_electromagnetic_2012}
%
\index{Plane!Wave!$x$ Polarized,!Vector Spherical Harmonics}%
\index{Wave!Plane!$x$ Polarized,!Vector Spherical Harmonics}%
%
  \begin{subequations}
 \begin{align}
 \vb{E}^\text{i} (\vb{r})=& E_0 \sum_{\ell = 1}^\infty\frac{i^\ell(2\ell+1)}{\ell(\ell+1)}\qty( \vb{M}_{o1\ell}^{(1)} -i \vb{N}_{e1\ell}^{(1)}),
\\
 \vb{H}^\text{i} (\vb{r})=& \frac{-kE_0}{\mu\omega} \sum_{\ell = 1}^\infty\frac{i^\ell(2\ell+1)}{\ell(\ell+1)}\qty( \vb{M}_{e1\ell}^{(1)} +i \vb{N}_{o1\ell}^{(1)}).
 \end{align}%
 \label{eq:plane waveMultipole}%
   \end{subequations}%
%
\indent
In the problem of scattering due to a spherical particle of radius $a$, the continuity conditions on the tangential components of the electric and magnetic fields are written as
  %
  \index{Boundary Conditions!Mie Problem}%
  %
 \begin{align}
 \qty(\vb{E}^\text{i} + \vb{E}^\text{sca}- \vb{E}^\text{int})\eval_{r=a}\times \vu{e}_r  =
  \qty(\vb{H}^\text{i} + \vb{H}^\text{sca} - \vb{H}^\text{int})\eval_{r=a} \times\vu{e}_r = 0,
  \label{eq:contuinity}
 \end{align}
 %
with  $\vb{E}^\text{sca}$ ( $\vb{E}^\text{int}$) the scattered (internal) electric field and  $\vb{H}^\text{sca}$ ( $\vb{H}^\text{int}$) the scattered (internal) magnetic field. If the incident field $\vb{E}^\text{i}$ is given by an $x$-polarized plane wave [Eq. \eqref{eq:ExPlane}] then the scattered and internal fields can be written also as a linear combination of $\vb{M}_{o1\ell}$ and $\vb{N}_{e1\ell}$. The internal field is finite inside the particle, thus the radial dependence is given by the function $j_\ell(k_\text{p} a)$, with $k_\text{p}$ the wave number inside the particle. For the scattered fields, it is chosen the spherical Hankel function of first kind $h^{(1)}(ka)$ due to its asymptotic behavior of a spherical outgoing wave. Such a choice for the radial dependence is denoted by the superscript $(3)$ over the VSH. To simplify the following steps, the scattered and the internal electric fields are written as
  %
  \begin{subequations}
 \begin{align}
 \vb{E}^\text{sca} (\vb{r})=& E_0 \sum_{\ell = 1}^\infty\frac{i^\ell(2\ell+1)}{\ell(\ell+1)}\qty(i a_\ell \vb{N}_{e1\ell}^{(3)} -b_\ell \vb{M}_{o1\ell}^{(3)}),
 \label{eq:EscaLC}\\
 \vb{E}^\text{int} (\vb{r})=& E_0 \sum_{\ell = 1}^\infty\frac{i^\ell(2\ell+1)}{\ell(\ell+1)}\qty( c_\ell\vb{M}_{o1\ell}^{(1)} - i d_\ell \vb{N}_{e1\ell}^{(1)}),
 \label{eq:EscaInt}
  \end{align}%
   \label{eqs:EscaInt}
 \end{subequations}
 %
% where the Hankel spherical functions has $\rho =$
with the respective magnetic fields
  %
    \begin{subequations}
 \begin{align}
 \vb{H}^\text{sca} (\vb{r})=& \frac{-kE_0}{\mu\omega} \sum_{\ell = 1}^\infty\frac{i^\ell(2\ell+1)}{\ell(\ell+1)}\qty(i b_\ell \vb{N}_{o1\ell}^{(3)} +a_\ell \vb{M}_{e1\ell}^{(3)}),\\
 \vb{H}^\text{int} (\vb{r})=& \frac{-kE_0}{\mu_\text{p}\omega} \sum_{\ell = 1}^\infty\frac{i^\ell(2\ell+1)}{\ell(\ell+1)}\qty( d_\ell\vb{M}_{e1\ell}^{(1)} + i c_\ell \vb{N}_{o1\ell}^{(1)}).
 \end{align}%
\label{eq:HscaInt}%
 \end{subequations}%
 %
Since only the term $m=1$ is taken into account, it is convenient to define the angular functions
%
\begin{align}
 \pi_\ell(\cos\theta )  = \frac{P_\ell^1(\cos\theta)}{\sin\theta}\qqtext{and}
 \tau_\ell(\cos\theta) = \dv{P_\ell^1(\cos\theta)}{\theta},
 \label{eq:PiTau}
\end{align}
%
 \index{Scattering!Mie!Angular Functions@{Angular Functions $\pi$ and $\tau$}}%
%
which are not orthogonal but their addition and subtraction are, that is $\pi_\ell \pm \tau_\ell$ are orthogonal functions \cite{bohren_absorption_1983}; see Eq. \eqref{eq:(pipmtau)} in Appendix \ref{app:MieCode}. After substitution  of Eqs. \eqref{eq:plane waveMultipole}, \eqref{eqs:EscaInt} and \eqref{eq:HscaInt} into Eq. \eqref{eq:contuinity} and considering the orthogonality between $\vb{M}$ and $\vb{N}$,  between even and odd VSH, and between $\pi_\ell \pm \tau_\ell$, it is shown that the coefficients $a_\ell$, $b_\ell$, $c_\ell$ and $d_\ell$ are given by two decoupled equation systems
%
\begin{align}
\mqty([x h_\ell^{(1)}(x)]  & (\mu/\mu_\text{p}) [(mx) j_\ell(mx)] \\
		m[x h_\ell^{(1)}(x)]' & [(mx) j_\ell(mx)]')
		\mqty(a_\ell \\ d_\ell) = \mqty([xj_\ell(x)] \\ m[x j_\ell(x)]') ,
	\label{eq:syst1}
		\\
\intertext{and}
\mqty( m[xh_\ell^{(1)}(x)]  &  [(mx)j_\ell(mx)] \\
		[x h_\ell^{(1)}(x)]' & (\mu/\mu_\text{p})[(mx) j_\ell(mx)]' )
		\mqty(b_\ell \\ c_\ell) = \mqty(m[xj_\ell(x)] \\ [x j_\ell(x)]'),
	\label{eq:syst2}
\end{align}
%
where $m = k_\text{p} / k = n_\text{p} / n_\text{mat}$ is the contrast between the sphere and the matrix, $x= ka = 2\pi n_\text{mat} (a/\lambda)$ is the size parameter and  $'$ denotes  derivative with respect to the argument of the spherical Bessel or Hankel functions; the size parameter compares the size of the scatterer with the incident wavelength in the matrix: the larger the value of $x$, the bigger the scatterer can be considered inside such a matrix. Equations \eqref{eq:syst1} and \eqref{eq:syst2} are simplified when the Riccati-Bessel functions $\psi_\ell( \rho) = \rho j_\ell(\rho)$ and $\xi(\rho) = \rho h_\ell^{(1)}(\rho)$ are introduced.

When a non-magnetic particle nor matrix are assumed  ($\mu_\text{p} = \mu = \mu_0$), the coefficients $a_\ell$ and $b_\ell$ are known as the Mie coefficients, whose expressions are calculated by inverting  Eqs. \eqref{eq:syst1} and \eqref{eq:syst2}, yielding
%
 \index{Scattering!Mie!Coefficients}%
 %
% -----------------------------
\begin{subequations}
	\begin{tcolorbox}[title = Mie Coefficients, ams align, breakable ]
	a_\ell &= \frac{\psi_\ell(x)\psi_\ell' (mx)-m\psi_\ell(mx)\psi_\ell'(x)}
				{\xi_\ell(x)\psi_\ell'(mx)-m\psi_\ell(mx)\xi_\ell'(x)},
				\label{eqs:a_ell}\\[.5em]
	b_\ell &= \frac{m\psi_\ell(x)\psi_\ell' (mx)-\psi_\ell(mx)\psi_\ell'(x)}
			{m\xi_\ell(x)\psi_\ell'(mx)-\psi_\ell(mx)\xi_\ell'(x)}.
			 \label{eqs:b_ell}
	\end{tcolorbox}\label{eq:MieCoef}
\end{subequations}
% ------------------------------
\noindent
Likewise, the coefficients $c_\ell$ and $d_\ell$ are given by
% -----------------------------
%
 \index{Scattering!Mie!Coefficients}%
 %
\begin{subequations}
\begin{align}
	c_\ell &= \frac{-m\xi_\ell'(x)\psi_\ell(x)+m\xi_\ell(x)\psi_\ell'(x)}
			{m\xi_\ell(x)\psi_\ell'(mx)-\psi_\ell(mx)\xi_\ell'(x)},\\[.5em]
	d_\ell &= \frac{-m\xi_\ell'(x)\psi_\ell(x)+m\xi_\ell(x)\psi_\ell'(x)}
				{\xi_\ell(x)\psi_\ell'(mx)-m\psi_\ell(mx)\xi_\ell'(x)}.
\end{align}%
\label{eq:coeffInt}%
\end{subequations}\noindent%
% ------------------------------
%
Even though the coefficients of the linear combination for the scattered and internal fields were obtained assuming an $x$-polarized incident field, due to the spherical symmetry of the problem, by applying the transformation $\varphi \to \varphi + \pi/2$  the same procedure is valid for a $y$-polarized incident field  \cite{bohren_absorption_1983}, therefore all quantities related to the scattered and the internal field can be expressed in terms of Eqs. \eqref{eq:MieCoef} and \eqref{eq:coeffInt}.

As discussed in Section \ref{s:AmpMatCrossSect}, the optical properties of a particle are encoded into the scattering, absorption, and extinction cross sections. These quantities can be calculated by means of the scattering amplitude matrix [Eq. \eqref{eq:ScatAmpMat}] and the Optical Theorem [Eq. \eqref{eq:Cext}]. Since the particle is spherical, it is convenient to exploit the symmetry of the problem  by decomposing the scattered electric field [Eq. \eqref{eq:EscaLC}] into components parallel and perpendicular to the scattering plane. To obtain the scattering amplitude matrix  expressed in an orthogonal base relative to the scattering plane ($\vu{e}_\parallel^s =\vu{e}_\theta$ and $\vu{e}_\perp^s=-\vu{e}_\varphi$) let us substitute the Mie coefficients [Eqs. \eqref{eq:MieCoef}] into Eq. \eqref{eq:EscaLC}  while rewriting the VSH $\vb{M}^{(3)}_{o1\ell}$  and $\vb{N}^{(3)}_{e1\ell}$ [Eqs. \eqref{eq:VSHEvenOdd}] in terms of the Riccati-Bessel function  $\xi$ and its derivative:
%
\begin{subequations}\label{eq:Esca}%
\begin{align}
\vb{E}^\text{sca}\cdot\vu{e}_r &=  \frac{\cos\varphi}{(kr)^2}
								\sum_{\ell = 1}^\infty E_0i^\ell(2\ell+1)
								ia_\ell \xi(kr)\pi_\ell(\cos\theta) \sin\theta ,
\label{eq:EscaR}\\
\vb{E}^\text{sca}\cdot\vu{e}^\text{sca}_\parallel &=  \frac{\cos\varphi}{kr}
								\sum_{\ell = 1}^\infty E_0i^\ell\frac{2\ell+1}{\ell(\ell+1)}
						[i a_\ell\xi_\ell'(kr)\tau_\ell(\cos\theta)-b_\ell\xi_\ell(kr)\pi_\ell(\cos\theta)],
\label{eq:EscaTheta}\\
\vb{E}^\text{sca}\cdot\vu{e}^\text{sca}_\perp &=  \frac{\sin\varphi}{-kr}
								\sum_{\ell = 1}^\infty E_0i^\ell\frac{2\ell+1}{\ell(\ell+1)}
						[i a_\ell\xi_\ell'(kr)\pi_\ell(\cos\theta)-b_\ell\xi_\ell(kr)\tau_\ell(\cos\theta)].
\label{eq:EscaMPhi}
\end{align}%
\end{subequations}
%
Following a similar procedure but substituting Eq. \eqref{eq:coeffInt} into Eq. \eqref{eq:EscaInt}, the internal electric field $\vb{E}^\text{int}$ can be rewritten also in terms of the Riccati-Bessel functions as
%
\begin{subequations}\label{eq:Eint}%
\begin{align}
\vb{E}^\text{int}\cdot\vu{e}_r &=  -\frac{\cos\varphi}{(k_\text{p}r)^2}
               \sum_{\ell = 1}^\infty E_0i^\ell(2\ell+1)
               id_\ell \psi(k_\text{p}r)\pi_\ell(\cos\theta) \sin\theta ,
\label{eq:EintR}\\
\vb{E}^\text{int}\cdot\vu{e}^\text{sca}_\parallel &=  \frac{\cos\varphi}{k_\text{p}r}
               \sum_{\ell = 1}^\infty E_0i^\ell\frac{2\ell+1}{\ell(\ell+1)}
           [-i d_\ell\psi_\ell'(k_\text{p}r)\tau_\ell(\cos\theta)+c_\ell\psi_\ell(k_\text{p}r)\pi_\ell(\cos\theta)],
\label{eq:EintTheta}\\
\vb{E}^\text{int}\cdot\vu{e}^\text{sca}_\perp &=  \frac{\sin\varphi}{-k_\text{p}r}
               \sum_{\ell = 1}^\infty E_0i^\ell\frac{2\ell+1}{\ell(\ell+1)}
           [-i d_\ell\psi_\ell'(k_\text{p}r)\pi_\ell(\cos\theta)+c_\ell\psi_\ell(k_\text{p}r)\tau_\ell(\cos\theta)].
\label{eq:EintMPhi}
\end{align}%
\end{subequations}


The scattering amplitude matrix relates the incident electric field to the scattered electric field in the far-field regime, that is when $kr\gg 1$. Considering that the series of Eqs. \eqref{eq:EscaR}-\eqref{eq:EscaMPhi} converge uniformly, so all contributions after the sufficiently large  term $\ell_c$ of the sum can be neglected for all values of $kr$, the asymptotic  expressions for the $\xi$ Riccati-Bessel function and its derivative can be employed, which are given by \cite{bohren_absorption_1983}
%
\index{Functions!Spherical Hankel@{Spherical Hankel $h^{(1)}_\ell(\rho)$ and $h^{(2)}_\ell(\rho)$}!Asymptotic Limit}%
\index{Wave!Spherical}
%
\begin{align}
\xi(kr)\approx (-i)^\ell \frac{\exp(ikr)}{i},
\qqtext{and}
\dv{\xi(kr)}{(kr)} = (-i)^\ell \exp(ikr)\qty(\frac{1}{i kr} + 1),
\qqtext{when}
\ell_c^2\ll kr.
\label{eq:RBXiAssym}
\end{align}
%
%
Substituting Eq. \eqref{eq:RBXiAssym} into  Eqs. \eqref{eq:EscaR}-\eqref{eq:EscaMPhi} and neglecting terms proportional to $(kr)^{-2}$, it yields a zero radial electric field, while
%
\begin{subequations}%
\begin{align}
\vb{E}^\text{sca}\cdot\vu{e}^\text{sca}_\parallel & \approx \frac{\exp(ikr)}{r}
\left\{\frac{i}{k}\sum_{\ell = 1}^\infty \frac{2\ell+1}{\ell(\ell+1)}
						[a_\ell\tau_\ell(\cos\theta)+b_\ell\pi_\ell(\cos\theta)]
				\right\}E_0\cos\varphi,
\label{eq:EthetaSca}\\
\vb{E}^\text{sca}\cdot\vu{e}^\text{sca}_\perp &\approx \frac{\exp(ikr)}{r}
\left\{\frac{i}{k}\sum_{\ell = 1}^\infty \frac{2\ell+1}{\ell(\ell+1)}
						[a_\ell\pi_\ell(\cos\theta)+b_\ell\tau_\ell(\cos\theta)]
				\right\}E_0(-\sin\varphi),
\label{eq:EphiSca}%
\end{align}\label{eq:EFar}%
\end{subequations}
%
\noindent
where it can be identified that $\vb{E}^\text{i}\cdot \vu{e}^\text{i}_\parallel = E_0\cos\varphi$ and $\vb{E}^\text{i}\cdot \vu{e}^\text{i}_\perp = -E_0\sin\varphi$ for $\vb{E}^\text{i}$ a plane wave traveling along the $z$ direction with an arbitrary polarization. Finally, the scattering amplitude matrix for a spherical particle can be obtained by comparing Eqs. \eqref{eq:EthetaSca} and \eqref{eq:EphiSca} with Eq. \eqref{eq:ScatAmpMat}, yielding
%
\index{Scattering!Amplitude Matrix@{Amplitude Matrix $\mathbb{F}$}!for Spherical Particles}%
%
 \begin{subequations}
     \label{eq:FS1S2}
\begin{tcolorbox}[title = Scattering Amplitude Matrix for Spherical Particles, ams align, breakable ]
\mathbb{F}(\vu{k}^\text{sca},\vu{k}^\text{i})
            &= \mqty(\dfrac{i}{k}S_2(\theta) & 0 \\
			0 & \dfrac{i}{k}S_1(\theta)  ),
            \label{eq:FscaS}
\intertext{with  $\vu{k}^\text{sca} = \vu{e}_r$, $\vu{k}^\text{i} = \vu{e}_z$, $\cos\theta = \vu{k}^\text{sca}\cdot\vu{k}^\text{i}$  and}
S_1(\theta)  &= \sum_{\ell = 1}^\infty \frac{2\ell+1}{\ell(\ell+1)}
						[a_\ell\tau_\ell(\cos\theta)+b_\ell\pi_\ell(\cos\theta)],
            \label{eq:S1}
\\
S_2(\theta) &= \sum_{\ell = 1}^\infty \frac{2\ell+1}{\ell(\ell+1)}
						[a_\ell\pi_\ell(\cos\theta)+b_\ell\tau_\ell(\cos\theta)],
            \label{eq:S2}
\end{tcolorbox}
\end{subequations}
%
\noindent
which depend entirely on the angular functions $\pi_\ell(\cos\theta)$ and $\tau_\ell(\cos\theta)$ modulated by the Mie coefficients $a_\ell$ and $b_\ell$. The scattering amplitude matrix $\mathbb{F}$ shows the symmetry of the system when it is written in a base relative to the scattering plane, as stated in Section \ref{s:AmpMatCrossSect}. Since the scatterer is a spherical isotropic NP,  $\mathbb{F}$ is a diagonal matrix in the base  $\{\vu{e}^\text{sca}_\parallel, \vu{e}^\text{sca}_\perp  \}$, that is, that the scattered electric field $\vb{E}^\text{sca}$ maintains the same polarization degree relative to the scattering plane than the incident electric field $\vb{E}^\text{i}$ that illuminates the spherical particle.
