% !TeX root = ../tesis.tex

The problem of scattering and absorption of light by single spherical NP embedded into a matrix, with refractive index $n_\text{m}$, illuminated by a plane wave with wavelength $\lambda$  and traveling in the  $\vb{k}^\text{i}$ direction, has spherical symmetry, which was exploited to develop the Mie Theory as explained in Section \ref{s:Mie}. If a substrate, with refractive index $n_\text{s}$, is considered and the NP is located right above or below the interface ---but not at it---, there are four combinations into which the system can be found since the NP can be either embedded  in the substrate or supported by it, and it can be illuminated either in an external ---from the matrix to the substrate--- or an internal --- from the substrate to matrix--- configuration as shown in Fig. \ref{sfig:TotallyNormal:1} where the four cases are shown: Embedded-External (EE), Embedded-Internal (EI), Supported-External (SE) and Supported-Internal (SI).

In Fig. \ref{sfig:TotallyNormal:2} the absorption $Q_\text{abs}$ and scattering $Q_\text{sca}$ efficiencies are shown as function of $\lambda$ for a AuNP of radius $a = 12.5$ nm in the Embedded-External (black), Embedded-Internal (orange), Supported-External (blue) and Supported-Internal (light orange) scenarios; the green shaded areas corresponds to the values between the two limiting cases given by the Mie theory: the AuNP embedded in air (lower boundary) and embedded in glass (upper boundary). The blue markers corresponds to the values of the efficiencies evaluated at the wavelength of resonance considering the presence of a substrate while the red markers corresponds to the efficiencies at the resonance wavelength for the Mie-limiting cases.

\begin{figure}[h!]
    \hspace*{-19.15em}%
    \vspace*{-1.25em}%
        \begin{subfigure}{.385\textwidth}\caption{ }\label{sfig:TotallyNormal:1}\end{subfigure}%
        \begin{subfigure}{.25\textwidth}\caption{ }\label{sfig:TotallyNormal:2}\end{subfigure} \\
    \def\svgwidth{.95\textwidth}
    \small
    \centering
    \includeinkscape{1-Totally/1-Efficiencies/1-Normal-Eff}%
    \vspace*{0em}
    \caption[Absorption and Scattering Efficiencies of a 12.5 nm AuNP above and below a planar Interface Illuminated at Normal Incidence]{\textbf{a)} Schematics of a AuNP embedded (E) in [supported (S) by] a glass substrate ($n_\text{s} = 1.5$) forming a planar interface with an air matrix ($n_\text{m} = 1$) and illuminated by a plane wave in the normal direction in an external (E) and in an internal (I) configuration. \textbf{b)} Absorption $Q_\text{abs}$ and scattering $Q_\text{sca}$ efficiencies as function of the illuminated wavelength $\lambda$ in different spatial configurations: EE (black), EI (orange), SE (blue) and SI (light orange). The green shaded region shows the two Mie-limiting casesof a  AuNP embedded
     in air and in glass, and the blue a red markers corresponds to the efficiencies evaluated at the wavelength of resonance for each case.
    }
\label{fig:TotallyNormal}
\end{figure}

From the results in Fig. \ref{sfig:TotallyNormal:2} it can be seen that both the scattering and absorption efficiencies of the four spatial configurations with substrate are of the same order of magnitud as the Mie-limiting cases and, even more, the nominal values of the efficiencies for the embedded AuNP (black and orange lines) are around the Mie-limiting case of the AuNP in glass (upper boundary of the green shaded region) and the same is true for the supported AuNP (blue and light orange lines) and the Mie-limiting case of a AuNP embedded in air (lower boundary of the green shaded region). The presence of a substrate yields an overall enhancement and damping of $Q_\text{abs}$ and $Q_\text{ext}$ relative to the isolated NP, which is related to the illumination of the system: If the system is illuminated in an external configuration, the obtained efficiencies decreases relative to the Mie-limiting case as it can be seen from the black  and blue curves, which corresponds to the EE and SE cases; on the other hand, the calculated efficiencies for the internal illuminated cases, that is for EI (orange) and SI (light orange), are enhanced relative to the Mie-limiting cases. Another effect of the substrate in the optical response of the system is a spectral shift on scattering and absorption efficiencies, which depends on the medium where the AuNP is located. For example, the wavelength of resonance (blue markers) for both the absorption and the scattering efficiencies are redshifted, relative to the Mie-limiting case (red markers), for the AuNP supported on the substrate (blue and light orange curves) and blueshifted for the embedded AuNP (black and orange curves).


\begin{figure}[h!]
    \centering
    \def\svgwidth{.9\textwidth}
    \hspace*{-.3\textwidth}%
    \vspace*{-.5em}%
        \begin{subfigure}{.395\textwidth}\caption{\small$\dfrac{\norm{\vb{E}^\text{sca}_\text{far}}}{\norm{\vb{E}^\text{i}}} \times 10^{-9}$  }\label{sfig:red:1}\end{subfigure}%
        \begin{subfigure}{.4\textwidth}\caption{\small$\dfrac{\norm{\vb{E}^\text{sca}_\text{far}}}{\norm{\vb{E}^\text{i}}} \times 10^{-9}$  }\label{sfig:red:1}\end{subfigure}\\
    \includeinkscape[pretex = \small]{1-Totally/4-5-Far-XY-Embedded/4-5-Far-XY-Embedded-External}\\
    %
    \def\svgwidth{.9\textwidth}
    \hspace*{-.3\textwidth}%
    \vspace*{-.5em}%
        \begin{subfigure}{.395\textwidth}\caption{\small$\dfrac{\norm{\vb{E}^\text{sca}_\text{far}}}{\norm{\vb{E}^\text{i}}} \times 10^{-9}$  }\label{sfig:red:1}\end{subfigure}%
        \begin{subfigure}{.4\textwidth}\caption{\small$\dfrac{\norm{\vb{E}^\text{sca}_\text{far}}}{\norm{\vb{E}^\text{i}}} \times 10^{-9}$  }\label{sfig:red:1}\end{subfigure}\\
    \includeinkscape[pretex = \small]{1-Totally/4-5-Far-XY-Embedded/4-5-Far-XY-Embedded-Internal}%
    \caption[Spectral redshift of the scattering and extinction of a spherical AuNP as function of its size and the embedding media]{Resonance wavelength $\lambda_\text{res}$ of the scattering (orange) and extinction (black) cross sections of a AuNP as function of the NP's radius when embedded \textbf{a)} into air ($n_\text{m} = 1)$ and \textbf{b)} into glass ($n_\text{m} = 1.5$), and as function of the refractive index of the matrix  $n_\text{m}$ for a AuNP of radius equal to  \textbf{c)} 12.5 nm and \textbf{d)} 50 nm, employing the dielectric function for the Au as reported by \citeauthor{johnson_optical_1972} (filled circle) and considering a size correction to it (empty circle).}
\end{figure}



\begin{figure}
    \centering
    \def\svgwidth{.9\textwidth}
    \hspace*{-.3\textwidth}%
    \vspace*{-.5em}%
        \begin{subfigure}{.395\textwidth}\caption{\small$\dfrac{\norm{\vb{E}^\text{sca}_\text{far}}}{\norm{\vb{E}^\text{i}}} \times 10^{-9}$  }\label{sfig:red:1}\end{subfigure}%
        \begin{subfigure}{.4\textwidth}\caption{\small$\dfrac{\norm{\vb{E}^\text{sca}_\text{far}}}{\norm{\vb{E}^\text{i}}} \times 10^{-9}$  }\label{sfig:red:1}\end{subfigure}\\
    \includeinkscape[pretex = \small]{1-Totally/4-5-Far-XY-Supported/4-5-Far-XY-Supported-External}\\
    %
    \def\svgwidth{.9\textwidth}
    \hspace*{-.3\textwidth}%
    \vspace*{-.5em}%
        \begin{subfigure}{.395\textwidth}\caption{\small$\dfrac{\norm{\vb{E}^\text{sca}_\text{far}}}{\norm{\vb{E}^\text{i}}} \times 10^{-9}$  }\label{sfig:red:1}\end{subfigure}%
        \begin{subfigure}{.4\textwidth}\caption{\small$\dfrac{\norm{\vb{E}^\text{sca}_\text{far}}}{\norm{\vb{E}^\text{i}}} \times 10^{-9}$  }\label{sfig:red:1}\end{subfigure}\\
    \includeinkscape[pretex = \small]{1-Totally/4-5-Far-XY-Supported/4-5-Far-XY-Supported-Internal}%
    \caption[Spectral redshift of the scattering and extinction of a spherical AuNP as function of its size and the embedding media]{Resonance wavelength $\lambda_\text{res}$ of the scattering (orange) and extinction (black) cross sections of a AuNP as function of the NP's radius when embedded \textbf{a)} into air ($n_\text{m} = 1)$ and \textbf{b)} into glass ($n_\text{m} = 1.5$), and as function of the refractive index of the matrix  $n_\text{m}$ for a AuNP of radius equal to  \textbf{c)} 12.5 nm and \textbf{d)} 50 nm, employing the dielectric function for the Au as reported by \citeauthor{johnson_optical_1972} (filled circle) and considering a size correction to it (empty circle).}
\end{figure}



\begin{figure}  \centering
	\def\svgwidth{\textwidth} \small
		\vspace*{4.em}
		\hspace*{-.5\textwidth}
	\begin{subfigure}{.525\textwidth}\caption{ } \label{fig:NearField:par}\end{subfigure}%
	\begin{subfigure}{.49\textwidth}\caption{ }\label{fig:NearField:perp}\end{subfigure}%
	\vspace*{-7.em}\\
	\includeinkscape{1-Totally/2-NearY/2-NearY-Embedded}\\
    \def\svgwidth{\textwidth}
        %\vspace*{5.em}
        \hspace*{-.5\textwidth}
    \begin{subfigure}{.525\textwidth}\caption{ } \label{fig:NearField:par}\end{subfigure}%
    \begin{subfigure}{.49\textwidth}\caption{ }\label{fig:NearField:perp}\end{subfigure}
        \vspace*{-1.75em}\\
    \includeinkscape{1-Totally/2-NearY/2-NearY-Supported}
	\vspace*{-2em}
	\caption[Induced Electric Field of a 12.5 nm Au Spherical NP Embedded into Air at the LSPR]{Induced electric field $\vb{E}^\text{int}$ evaluated at the planes \textbf{a)} $y = 0$ and \textbf{b)} $x = 0$  of a 12.5 nm Au spherical NP (dashed lines) embedded into air ($n_\text{mat} = 1$) when illuminated by an incident plane wave with an $x$-polarized electric field $\vb{E}^\text{i}$ traveling in the direction $\vb{k}^\text{i}$ along the $z$ axis with an	 excitation wavelength $\lambda = 509$ nm of the LSPR. At the plane $x = 0$, the incident electric field is parallel to the scattering plane, while it is perpendicular to it at $x = 0$. The optical response of the 12.5 nm AuNP was modeled using a size correction to the experimental data reported by \citeauthor{johnson_optical_1972} \cite{johnson_optical_1972}.}
	\label{fig:NearField}
 \end{figure}
