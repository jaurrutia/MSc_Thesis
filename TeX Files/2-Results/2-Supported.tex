% !TeX root = ../tesis.tex

On the past section, the AuNP of radius $a = 12.5$ nm was illuminated at normal incidence in four different spatial configurations considering the  presence of a substrate: the AuNP either embedded in the substrate (with a refractive index $n_\text{s}$) or supported on it embedded in an air matrix (with refractive index $n_\text{s}$), and the incident electric field  illuminating the system from the substrate (internal configuration) or from the matrix (external configuration). By considering that the incident electric plane wave $\vb{E}^\text{i}$  propagates from the substrate to the matrix at an angle $\theta_i$, relative to the normal direction to the interface between the two media, the electric field interacting with the AuNP is the transmitted field, that propagates at a transmission angle $\theta_t = \asin(n_\text{m}\sin\theta_i/n_\text{s})$ \cite{born_max_principle_1999} and two differences arises compared to the normal incidence cases: there are two different polarization states for $\vb{E}^\text{i}$ ---$s$ and $p$ polarization\footnote{%
    The $s$ and $p$ polarization states of the electric field are defined by considering its oscillations perpendicular and parallel, respectively, to the incidence plane, defined by the the propagating direction of the incident electric plane wave and the normal direction to the interface between the substrate and the matrix \cite{bohren_absorption_1983}.}%
--- and if $\theta_i$ is greater than the critical angle $\theta_c = \asin(n_\text{m}/n_\text{s})$ the transmitted electric field ceases to be a plane wave and it is now described by an evanescent wave propagating along the interface  \cite{born_max_principle_1999}. Similarly to the past section, the optical properties of the supported AuNP illuminated at an oblique incidence in an internal configuration are studied by analyzing the absorption and scattering efficiencies, the radiation pattern and, lastly, the induced electric field on the AuNP.

\begin{figure}[b!]
    \def\svgwidth{.95\textwidth}
    \centering
    \hspace*{-28.5em}%
    \vspace*{-1.25em}%
        \begin{subfigure}{.71\textwidth}\caption{ }\label{sfig:SuppObl:Eff:Abs}\end{subfigure}%
        \begin{subfigure}{.25\textwidth}\caption{ }\label{sfig:SuppObl:Eff:Sca}\end{subfigure} \\
    \includeinkscape[pretex = \small]{2-SuppObl/1-Efficiencies/1-Oblique-Supp-Eff}%
    \vspace*{-.5em}
    \caption[Absorption and Scattering Efficiencies of a 12.5 nm AuNP on a Interface Illuminated in an internal configuration at oblique incidence]{\textbf{a)} Absorption and \textbf{b)} scattering efficiencies of a $12.5$ nm AuNP in an air matrix ($n_\text{m} = 1$) and supported on a glass substrate ($n_\text{s} = 1.5$) as function of the wavelength $\lambda$ of an  $s$ (filled circle/solid lines) and a $p$ (empty circle/dashed lines) polarized incident electric plane wave propagating in the direction of the wave vector $\vb{k}^\text{i}$, in an internal configuration, at an angle of incidence $\theta_i$ of $15^\circ$ (black),  $38^\circ$ (orange),  $42^\circ$ (blue) and  $75^\circ$ (light orange) relative to the normal direction to the glass-air interface. The green shaded region shows the two Mie-limiting cases of a AuNP embedded in air and in glass; the magenta (supported AuNP) and red (Mie-limiting) markers corresponds to the efficiencies evaluated at the wavelength of resonance for each case.}
\label{fig:SuppObl:Eff}
\end{figure}

In Fig. \ref{fig:SuppObl:Eff} the absorption $Q_\text{abs}$ [Fig.\ref{sfig:SuppObl:Eff:Abs}] and scattering $Q_\text{sca}$ [Fig. \ref{sfig:SuppObl:Eff:Sca}] efficiencies of a 12.5 nm AuNP in air ($n_\text{m} = 1$) supported on a glass substrate ($n_\text{s} = 1.5$) are shown as function of the wavelength $\lambda$ of the incident electric field $\vb{E}^\text{i}$ illuminating the AuNP from the substrate at an incidence angle $\theta_i$ of $15^\circ$ (black),  $38^\circ$ (orange),   $42^\circ$ (blue) and  $75^\circ$ (light orange) considering an $s$ (filled circle/solid lines) and a $p$ (empty circle/dashed lines) polarization for $\vb{E}^\text{i}$. Since the critical angle for a glass-air interface is $\theta_c = 41.8^\circ$, the blue and light orange curves corresponds to the interaction between an evanescent wave and the AuNP. The magenta markers correspond to the values of $Q_\text{abs}$ and $Q_\text{sca}$ evaluated at the wavelength of resonance; the Mie-limiting cases (AuNP embedded in air and in glass)  are signalized by the boundaries of the green shaded region  and the red markers correspond to the resonances of their efficiencies.

A general behavior on both the absorption and scattering efficiencies is that their value for all considered values of $\lambda$, with a fixed polarization state, increases as the angle of incidence reaches the critical angle $41.8^\circ$ as it can be seen by comparing the black, orange ---$\theta_i = 15^\circ,\, 38^\circ < \theta_c$---  and blue ---$\theta_c<\theta_i = 42^\circ$--- curves, and they decrease in the interval  $\theta_c<\theta_i<90^\circ$, which it it noticeable by comparing the aforementioned curves with the light orange one ---$\theta_i = 75^\circ$---. This tendency on the overall value of $Q_\text{abs}$ and $Q_\text{abs}$ is due to the transmitted electric field which illuminates the AuNP and it is described by a plane wave for $\theta_i<\theta_c$ and by an evanescent wave for $\theta_i>\theta_c$ accoriding to the Fresnel's transmission amplitude coefficients \textcolor{red}{\textbf{Sería buena idea ponerlas en el apéndice de COMSOL?}}, whose real part are monotonically increasing (decreasing) functions of $\theta_i$ for values smaller (greater) to the critical angle. Yet, another explanation for te decreasing behavior after the critical angle is due to the penetration depth of the evanescent electric field, which is given by $\lambda/(2\pi n_\text{s}\sin\theta_i)$, meaning that evanescent wave is carries more energy for values of $\theta_i$ above and near $\theta_c$.

When comparing the absorption and scattering efficiencies based on the polarization of the incident plane wave, it can be observed that their enhancement, relative to the Mie-limiting case of a AuNP in a air matrix, is greater for a  \textit{p} than for an \textit{s} polarized incident electric field for a fixed angle of incidence (see continuos and dashed curves of the same color). Another notable phenomena is that the  efficiencies for $\theta_i=75^\circ$ (light orange curves)  are bellow the Mie-limiting case for all wavelengths, while only the efficiencies for  \text{p} polarized incident electric field with $\theta_i = 42^\circ \gtrsim \theta_c$ (dashed blue curves) are greater than the glass Mie-limiting case at the resonance wavelength. The last difference between the \textit{s} and the \textit{p} polarization cases arises by analyzing the spectral shift of the LSPR (magenta markers). For a fixed polarization state, all the LSPR are excited at the same wavelength independently of the angle of incidence up to the wavelength discretization ---$\Delta \lambda = 2.5$ nm in the LSPR's neighborhood --- employed in the FEM simulations from Fig. \ref{fig:SuppObl:Eff}: The absorption extinction wavelength for the \textit{s} polarized incident electric field is $\sim 510$ nm, which is the same as the equivalent system illuminated at normal incidence [see Fig. \ref{sfig:TotallyNormal:2}], while the LSPR for the $p$ polarization case is excited at the larger wavelength $\sim 512.5$ nm. To better understand the spectral shift between the two polarization states let analyze the scattered induced field in the far and near-field regimes.

The radiation pattern of the 12.5 nm AuNP embedded in air and supported on a glass substrate is shown in Fig. \ref{fig:Far:SuppObl} when an $s$ polarized [Figs. \ref{sfig:Far:SuppObl:s:a} and \ref{sfig:Far:SuppObl:s:b}] and a  $p$ polarized [Figs. \ref{sfig:Far:SuppObl:p:c} and \ref{sfig:Far:SuppObl:p:d}] incident electric field $\vb{E}^\text{i}$ illuminates the AuNP at an incidence angle of $\theta_i = 15^\circ$ (black) and $\theta_i = 38^\circ$ ---below the critical angle $\theta_c = 41.8^\circ$---, and $\theta_i = 42^\circ$ (black) and $\theta_i = 75^\circ$ ---above  $\theta_c$--- at the wavelengths of resonance of the absorption efficiency ---magenta markers in Fig. \ref{sfig:SuppObl:Eff:Abs}--- for each case. The scattering plane where the radiation pattern is shown in Figs. \ref{sfig:Far:SuppObl:s:a} and \ref{sfig:Far:SuppObl:p:c}  is perpendicular to the incidence plane (vertical gray dotted lines) while the scattering plane equals the incidence plane in Figs. \ref{sfig:Far:SuppObl:s:b} and \ref{sfig:Far:SuppObl:p:d}; the incident wave vector $\vb{k}^\text{i}$ and the components of the incident electric field parallel $\vb{E}^\text{i}_\parallel$ and perpendicular $\vb{E}^\text{i}_\perp$ to the scattering plane are schematized in all cases.

\begin{figure}[t!]
    \centering
    \def\svgwidth{.8\textwidth}
    \hspace*{-.215\textwidth}%
    \vspace*{-.5em}%
        \begin{subfigure}{.32\textwidth}\caption{\footnotesize$\dfrac{\norm{\vb{E}^\text{sca}_\text{far}}}{\norm{\vb{E}^\text{i}}} \times 10^{-9}$  }\label{sfig:Far:SuppObl:s:a}\end{subfigure}%
        \begin{subfigure}{.4\textwidth}\caption{\footnotesize$\dfrac{\norm{\vb{E}^\text{sca}_\text{far}}}{\norm{\vb{E}^\text{i}}} \times 10^{-9}$  }\label{sfig:Far:SuppObl:s:b}\end{subfigure}\\
    \includeinkscape[pretex = \footnotesize]{2-SuppObl/4-5-FarXY/4-5-Far-XY-S}\\
    %
    \def\svgwidth{.8\textwidth}
    \hspace*{-.215\textwidth}%
    \vspace*{-.5em}%
        \begin{subfigure}{.32\textwidth}\caption{\footnotesize$\dfrac{\norm{\vb{E}^\text{sca}_\text{far}}}{\norm{\vb{E}^\text{i}}} \times 10^{-9}$  }\label{sfig:Far:SuppObl:p:c}\end{subfigure}%
        \begin{subfigure}{.4\textwidth}\caption{\footnotesize$\dfrac{\norm{\vb{E}^\text{sca}_\text{far}}}{\norm{\vb{E}^\text{i}}} \times 10^{-9}$  }\label{sfig:Far:SuppObl:p:d}\end{subfigure}\\
    \includeinkscape[pretex = \footnotesize]{2-SuppObl/4-5-FarXY/4-5-Far-XY-P}%
    \caption[  Radiation pattern of a AuNP supported on a substrate illuminated at oblique incidence]{Radiation pattern of a AuNP (light yellow) of radius $a = 12.5$ nm supported on a glass substrate (light blue, $n_\text{s} = 1.5$) with an air matrix ($n_\text{m} = 1$) illuminated by an incident electric plane wave $\vb{E}^\text{i}$, with a wavelength $\lambda$, traveling in the $\vb{k}^\text{i}$ direction at an angle of incidence $\theta_i$ of $15^\circ$ (black),  $38^\circ$ (orange),  $42^\circ$ (blue) and  $75^\circ$ (light orange) relative to the normal direction to the glass-air interface. The radiation patterns consider an \textbf{a,b)} $s$ polarized and  a \textbf{c,d}) $p$ polarized incident electric field and the scattering plane \textbf{a,c)} perpendicular to the incidence plane (vertical gray dotted lines) and \textbf{b,d)} equal to the incidence plane. In all cases the incident wave vector $\vb{k}^\text{i}$, the perpendicular $\vb{E}_\perp^\text{i}$ and the  parallel $\vb{E}_\parallel^\text{i}$ projection of the incident electric field relative to the scattering plane are schematized.%
    }
    \label{fig:Far:SuppObl}
\end{figure}

As expected from the results in Section \ref{s:Totally:Normal}, the amplitudes of the radiation patterns for AuNP supported on a substrate and illuminated in an internal configuration at an oblique incidence are modulated by the absorption and scattering efficiencies, for example, the maximum value of the radiation pattern for the $s$ polarization case is $1.2 \text{ nV m}^{-1}$ [see Figs. \ref{sfig:Far:SuppObl:s:a} and \ref{sfig:Far:SuppObl:s:b}] while this value is two folded for the $p$ polarization case [see Figs. \ref{sfig:Far:SuppObl:p:c} and \ref{sfig:Far:SuppObl:p:d}]. Additionally, the radiation pattern is in average greater for a fixed angle of incidence, the greater the absorption and scattering efficiencies are, as it can be seen by comparing blue and the light orange curves ---corresponding to $\theta_i = 42^\circ$ and $\theta_i = 75^\circ$, respectively--- in Fig. \ref{fig:Far:SuppObl} with the curves in Fig. \ref{fig:SuppObl:Eff}.

If the shape of the radiation patterns is studied, the $s$ polarization case shows a two and a one-lobe shapes if the incident electric field is parallel [Fig. \ref{sfig:Far:SuppObl:s:a}] or perpendicular [Fig. \ref{sfig:Far:SuppObl:s:b}] to the scattering plane, respectively. Such radiation patterns are observed for all values of $\theta_i$ in a $s$ polarization configuration due to the continuity of the components of the electric field parallel the substrate where the the AuNP is onto. On the other hand, for a $p$ polarized incident electric field, the transmitted electric field illuminating the AuNP has a different component perpendicular to the substrate depending on the angle of incidence; the orientation of the transmitted electric field, relative to the normal direction to the substrate, is obtained by adding $-\pi/2$ to the angle of transmission $\theta_t = \arcsin(n_\text{m}\sin\theta_i/n_\text{s})$, that is $\theta_t - \pi/2$. The direction of the transmitted electric field for an incidence angle $\theta_i = 15^\circ<\theta_c$ and $\theta_i = 38^\circ<\theta_c$ ---black and orange curves--- result on the two-lobe shapes  in Fig. \ref{sfig:Far:SuppObl:p:c} ---scattering plane perpendicular to the incidence plane--- and Fig. \ref{sfig:Far:SuppObl:p:d} ---scattering plane overlapping the incidence plane---. When the scattering and incidence plane are perpendicular to each other, the transmitted electric field have components both parallel and perpendicular to the scattering plane for $\theta_i<\theta_c$ yielding radiation patterns without non-radiation directions, while the radiation patterns when the scattering and the incidence plane overlap are two-lob shapes rotated with no-radiation directions given by the transmission angle $\theta_t$. For $\theta_i>\theta_c$, the transmitted electric field is described by an evanescent wave traveling along the interface and its direction is perpendicular to the the substrate, thus the two-lobe shape of the AuNP's radiation pattern is aligned to the interface; this can be observed when $\theta_i = 42^\circ$ (blue curve) and $\theta_i = 75^\circ$ (light orange curve) when the scattering and the incidence plane are perpendicular between them [Fig. \ref{sfig:Far:SuppObl:p:c}] and when they overlap [Fig. \ref{sfig:Far:SuppObl:p:d}].

The behavior of the scattered electric field by the AuNP, in the far-field regime, described above suggests that a 12.5 nm AuNP on a substrate can be studied in the small particle approximation and be treated, even at oblique incidence, as a point dipole oriented  parallel to the substrate for an $s$ polarized incident electric field and in a perpendicular direction to that given by the transmission angle for the $p$ polarization case. Under such schema, alongside the interaction between the point dipole and an image dipole induced due to the substrate, the difference in magnitud of the efficiencies, and thus of the far-field, for different polarization states of the incident electric field  is understood since for $s$ polarization the point and the image dipole are parallel to each other, while for  $p$ polarization the dipoles are collinear to each other, that is, they have a stronger response \textcolor{red}{\bf Buscar referencia}.

To further analyze the optical response of the AuNP on a substrate, the spatial distribution of the induced electric field $\vb{E}^\text{ind}$ ---the internal and the scattered electric field in the near-field regime--- is needed. Since the radiation patterns observed when an $s$ polarized incident electric field illuminates the AuNP follow the same one and a two-lobe shapes for all the considered combinations of $\theta_i$ and $\lambda$ only differing on the magnitud, the distribution of $\vb{E}^\text{ind}$  have the same qualitative behavior for all incident angles at this polarization state. Therefore, the norm of the induced electric field $\norm{\vb{E}^\text{ind}}$, evaluated at a scattering plane perpendicular to the incidence plane (vertical gray dashed lines) is shown in Fig. \ref{fig:Near:SuppObl:s} for a 12.5 nm AuNP (black dashed lines) supported on the interface between a glass substrate and an air matrix (white dashed lines) when the incident electric field $\vb{E}^\text{i}$ illuminates the system at an incident angle of $15^\circ$ [Fig. \ref{sfig:Near:SuppObl:s:15}], $38^\circ$ [Fig. \ref{sfig:Near:SuppObl:s:38}], $42^\circ$ [Fig. \ref{sfig:Near:SuppObl:s:42}]  and $75^\circ$ [Fig. \ref{sfig:Near:SuppObl:s:75}].

\begin{figure}[t!]\centering
   \def\svgwidth{.75\textwidth}
   \footnotesize
   \captionsetup[subfigure]{labelfont ={normal,bf,color = white}}
   \includeinkscape{2-SuppObl/2-Near-sP/2-NearYX-sPol}\\[-32.6em]
   \hspace*{-.25\textwidth}
       \begin{subfigure}{.25\textwidth}\textcolor{red}{\caption{ } \label{sfig:Near:SuppObl:s:15}}\end{subfigure}%
       \begin{subfigure}{.34\textwidth}\caption{ }\label{sfig:Near:SuppObl:s:38}\end{subfigure}\\[13em]
    \hspace*{-.25\textwidth}
       \begin{subfigure}{.25\textwidth}\textcolor{red}{\caption{ } \label{sfig:Near:SuppObl:s:42}}\end{subfigure}%
       \begin{subfigure}{.34\textwidth}\caption{ }\label{sfig:Near:SuppObl:s:75}\end{subfigure}\\[15em]
\caption[Induced Electric Field of a 12.5 nm Au NP on substrate illuminated at oblique incidence with a $s$ polarized electric field]{%
Electric field $\vb{E}^\text{ind}$ induced by a supported 12.5 nm AuNP (dashed black lines) illuminated by an $s$ polarized incident electric plane wave $\vb{E}^\text{i}$ traveling in the $\vb{k}^\text{i}$ direction, in an internal configuration, at an angle of incidence of \textbf{a)} $15^\circ$, \textbf{b)} $38^\circ$, \textbf{c)} $42^\circ$ and \textbf{d)} $75^\circ$, relative to the normal direction to the interface ---white dashed lines--- between an air matrix ($n_\text{m} = 1$) and a glass substrate ($n_\text{s} = 1.5$). The incident electric plane wave is evaluated at $\lambda = 510$ nm ---resonance wavelength of the absorption efficiency--- and in all shown cases the scattering plane is perpendicular to the incidence plane (vertical gray dotted lines).
}
 \label{fig:Near:SuppObl:s}
 \end{figure}

 \begin{figure}[!H]\centering
    \def\svgwidth{.7\textwidth}
    \scriptsize
    \captionsetup[subfigure]{labelfont ={small,bf,color = white}}
    \includeinkscape{2-SuppObl/3-Near-pP/2-NearYX-pPol}\\[-62.75em]
    \hspace*{-.255\textwidth}
        \begin{subfigure}{.25\textwidth}\textcolor{red}{\caption{ } \label{sfig:Near:SuppObl:p:15:perp}}\end{subfigure}%
        \begin{subfigure}{.3\textwidth}\caption{ }\label{sfig:Near:SuppObl:p:15:par}\end{subfigure}\\[12.8em]
    \hspace*{-.225\textwidth}
        \begin{subfigure}{.225\textwidth}\textcolor{red}{\caption{ } \label{sfig:Near:SuppObl:p:38:perp}}\end{subfigure}%
        \begin{subfigure}{.34\textwidth}\caption{ }\label{sfig:Near:SuppObl:p:38:par}\end{subfigure}\\[12.8em]
    \hspace*{-.225\textwidth}
        \begin{subfigure}{.225\textwidth}\textcolor{red}{\caption{ } \label{sfig:Near:SuppObl:p:42:perp}}\end{subfigure}%
        \begin{subfigure}{.34\textwidth}\caption{ }\label{sfig:Near:SuppObl:p:42:par}\end{subfigure}\\[12.8em]
    \hspace*{-.225\textwidth}
        \begin{subfigure}{.225\textwidth}\textcolor{red}{\caption{ } \label{sfig:Near:SuppObl:p:75:perp}}\end{subfigure}%
        \begin{subfigure}{.34\textwidth}\caption{ }\label{sfig:Near:SuppObl:p:75:par}\end{subfigure}\\[15.75em]
    \caption[Induced Electric Field of a 12.5 nm Au NP on substrate illuminated at oblique incidence with a $p$ polarized electric field]{\footnotesize%
    Electric field $\vb{E}^\text{ind}$ induced by a supported 12.5 nm AuNP (dashed black lines) illuminated by a $p$ polarized incident electric plane wave $\vb{E}^\text{i}$ traveling in the $\vb{k}^\text{i}$ direction, in an internal configuration, at an angle of incidence of \textbf{a,b)} $15^\circ$, \textbf{c,d)} $38^\circ$, \textbf{e,f)} $42^\circ$ and \textbf{g,h)} $75^\circ$, relative to the normal direction to the interface ---white dashed lines--- between an air matrix ($n_\text{m} = 1$) and a glass substrate ($n_\text{s} = 1.5$). The incident electric plane wave is evaluated at the resonance wavelength of the absorption efficiency ---see Fig. \ref{sfig:SuppObl:Eff:Abs}--- and in all shown cases the norm $\norm{\vb{E}^\text{ind}}$ is evaluated at  \textbf{a,c,e,g)} a scattering plane perpendicular to the incidence plane (vertical gray dotted lines) and at \textbf{b,d,f,h)} a scattering plane equal to the incidence plane.
    }
    \label{fig:Near:SuppObl:p}
  \end{figure}

Contrastingly to the $s$ polarized incident electric field case, the radiation patterns of a supported 12.5 nm AuNP illuminated with a $p$ polarized electric field have a different qualitative behavior depending on the angle of incidence $\theta_i$ as shown in Figs. \ref{sfig:Far:SuppObl:p:c} and  \ref{sfig:Far:SuppObl:p:d}. Thus, the norm of $\vb{E}^\text{ind}$ is shown in Fig. \ref{fig:Near:SuppObl:p} and it is evaluated at scattering plane perpendicular to the incidence plane (vertical gray dotted line) [Figs. \ref{sfig:Near:SuppObl:p:15:perp}, \ref{sfig:Near:SuppObl:p:38:perp}, \ref{sfig:Near:SuppObl:p:42:perp} and \ref{sfig:Near:SuppObl:p:75:perp}] and at the incidence  plane [Figs. \ref{sfig:Near:SuppObl:p:15:par}, \ref{sfig:Near:SuppObl:p:38:par}, \ref{sfig:Near:SuppObl:p:42:par} and \ref{sfig:Near:SuppObl:p:75:par}] for an incidence angle $\theta_i = 15^\circ$ [Figs. \ref{sfig:Near:SuppObl:p:15:perp}  and \ref{sfig:Near:SuppObl:p:15:par}] and  $\theta_i = 38^\circ$ [Figs. \ref{sfig:Near:SuppObl:p:38:perp}  and \ref{sfig:Near:SuppObl:p:38:par}], both below the critical angle $\theta_c = 41.8^\circ$, and $\theta_i = 42^\circ$ [Figs. \ref{sfig:Near:SuppObl:p:42:perp}  and \ref{sfig:Near:SuppObl:p:42:par}] and  $\theta_i = 75^\circ$ [Figs. \ref{sfig:Near:SuppObl:p:75:perp}  and \ref{sfig:Near:SuppObl:p:75:par}] above $\theta_c$.

In both Fig. \ref{fig:Near:SuppObl:s} and Fig. \ref{fig:Near:SuppObl:p} it can be seen that the greater enhancement of electric field occurs when the angle of incidence is $\theta_i = 42^ \circ \gtrsim \theta_c$ [see Figs. \ref{sfig:Near:SuppObl:s:42}, \ref{sfig:Near:SuppObl:p:42:perp} and \ref{sfig:Near:SuppObl:p:42:par}] for each polarization and the lesser enhancement for $\theta_i$ close to $90^\circ$ [see Figs. \ref{sfig:Near:SuppObl:s:75}, \ref{sfig:Near:SuppObl:p:75:perp} and \ref{sfig:Near:SuppObl:p:75:par}], in agreement with the tendency of the absorption and scattering efficiencies and the radiation patterns presented above. On the spatial distribution, the induced electric field for an $s$ polarized incident electric field [Fig. \ref{fig:Near:SuppObl:s}], the characteristic dipolar distribution with hotspots aligned to the substrate is shown with a deviation nearest to the substrate, as observed in the normal incidence case, while for the $p$ polarization case [Fig. \ref{fig:Near:SuppObl:p}] the hotspots of the near-field spatial distribution are rotated according to the orientation of the transmitted electric field. This rotation of the spatial distributions of $\vb{E}^\text{ind}$ leads to an enhancement of $\sim 30$ when $\theta_i \gtrsim \theta_c$ since one hotspot is in contact with the substrate, nevertheless on the diametral hotspot is $\sim 12.5$, which is still larger than for the equivalent $s$ polarization case.

In this Section, the optical properties of a 12.5 AuNP in air and supported on a glass matrix was studied when illuminated at normal and oblique incidence for both polarization states of the incident electric field. It was observed that there is an enhancement of the absorption and scattering efficiencies of the AuNP relative to the Mie-limiting case and a redshift of the LSPR due to the substrate which is polarization dependent but not angle of incidence dependent. On the other hand, the near and far-field induced by the AuNP interacting with the transmitted electric field showed a different behavior according to the polarization of the incident electric field and on the angle of incidence since the transmitted electric field is described by a plane or by an evanescent wave if the angle of incidence is smaller or greater than the critical angle, respectively. Lastly, it was shown that when the AuNP is illuminated by an evanescent wave, the overall optical response of the AuNP is stronger and there is an induced dipolar moment perpendicular to the substrate which describes this optical response. In the next Section, a similar analysis is performed but now considering that the AuNP is partially embedded in the substrate, which reproduces a more realistic experimental system.
