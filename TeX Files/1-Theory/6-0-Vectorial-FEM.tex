% !TeX root = ../tesis.tex

 In order to perform the vectorial finite element approximation in Eq. \eqref{eq:Vec-FEM} equivalently to the scalar method in Eq. \eqref{eq:GalerkinInt}, an approximation $\tilde{\vb{E}}$ to the electric field $\vb{E}$ is proposed as a linear combination of the vectorial interpolating functions $\boldsymbol{\eta}_j$ with scalar coefficients $e_i$. If the nodal approximation [Eq. \eqref{eq:Nodal}] were used for the linear combination setting the coefficients $e_i$ as the exact solution of each component of  electric field at some nodal points into the finite element $\Omega_k$, not only the boundary condtion in Eq. \eqref{eq:Dirichlet-weak} might be diffuclt to meet \cite{larson_finite_2013,jin_theory_2010}, but also   non-physical solutions may arise \cite{bondeson_computational_2005}. Therefore, instead of employing the nodal approximation, the expression of $\tilde{\vb{E}}$ in a finite element $\Omega_k$ of $\Omega$ is given by \cite{larson_finite_2013}
%
\index{Approximation!Nodal!Vectorial}%
\index{Partial Differential Equation (PDE)!Nodal Approximation!Vectorial}
\index{Partial Differential Equation (PDE)!Nodal Approximation!Vectorial!Interpolating Function $\boldsymbol{\eta}$}
\index{Functions!Interpolating $\boldsymbol{\eta}$}
%
\begin{align}
    \tilde{\vb{E}}(\vb{r}) = \sum_{i_k} e_{i_k} \boldsymbol{\eta}_{i_k}(\vb{r}),
        \qquad
        \text{with}
        \qquad
    e_{i} = \vu{t}_{i_k}\cdot \vb{E}(\vb{r}\in E_{i_k}),
\end{align}
%
where $E_{i_k}$ is the ${i_k}$-th edge in $\Omega_k$, $\vu{t}_{i_k}$ is a unitary vector  parallel to $E_{i_k}$ and $e_{i_k}$ is the exact value of the tangential component of the electric field on it. If the interpolating functions  $\boldsymbol{\eta}_{i_k}$ in the element $\Omega_k$ are to follow Eq. \eqref{eq:Dirichlet}, they  must have a continuous tangential components across $\partial\Omega_k$, while no special requirement is asked for their normal component on $\partial\Omega_k$. A special family of vectorial ---or edge--- reference finite elements are given by the linear functional \cite{larson_finite_2013}
%
\index{Finite Element!Linear Functional!Nédélec}
%
\begin{align}
    \mathcal{F}^{N}_{i_k}[\boldsymbol{\eta}_{\ell_k}(\boldsymbol{\xi})] = \frac{1}{\abs{E_{i_k}}}\qty(\int_{E_{i_k}} \vu{t}_{i_k}\cdot\boldsymbol{\eta}_{\ell_k}(\boldsymbol{\xi})   \dd{(\partial\Omega_k)} )^{1/2}  = \delta_{{i_k}{\ell_k}},
    \label{eq:Nedelec}
\end{align}
%
which is the square-root mean value of the interpolating function along the edge $E_{i_k}$. The reference finite elements obtained from Eq. \eqref{eq:Nedelec} are known as the Nédélec finite element of lowest order since their only degree of freedom is on the edges of the finite element \cite{bergot_generation_2010}. For triangular and tetrahedral finite elements, a closed formula for $\boldsymbol{\eta}_{i_k}$ is given by \cite{jin_theory_2010,larson_finite_2013,bondeson_computational_2005}
%
\begin{align}
    \boldsymbol{\eta}_{i_k} = \abs{E_{i_k}} \qty(\phi_{j_k}\nabla\phi_{\ell_k}
                                                    - \phi_{\ell_k}\nabla\phi_{j_k} ),
    \label{eq:NedelecTRi}
\end{align}
%
\index{Functions!Interpolating $\phi$!Lagrange}%
%
with cyclic permutation of the indices $\{i_k, j_k, \ell_k\}$, corresponding to the vertices of triangular surfaces forming  the edges of the reference element. The scalar functions $\phi_{j_k}$ in Eq. \eqref{eq:NedelecTRi}  are the interpolating functions obtained with the Lagrange linear operator [Eq. \eqref{eq:Lag}] applied on $\Omega_k$ at the vertices. Nédélec finite elements of higher orders can be obtained if, additionally, more degrees of freedom are set to the faces and the volume of $\Omega_k$ \cite{bergot_generation_2010} nevertheless, their implementation is of greater difficulty and thus, less common \cite{larson_finite_2013,jin_theory_2010}. An example of Nédélec finite elements of lowest order are shown in Fig. \ref{fig:Nedelec:a} for a triangular finite element and in Fig. \ref{fig:Nedelec:b} for a tetrahedral finite element; the edge where Eqs. \eqref{eq:Nedelec} and \eqref{eq:NedelecTRi} are applied to is shown with a thick magenta line in all cases. In all interpolating functions (2D and 3D) it can be seen that all functions $\boldsymbol{\eta}_i$ are divergenceless inside the finite elements and their components parallel to the edges are zero except at the associated $i_k$-th edge. Therefore, the Nédélec finite elements are suited to solve Eq. \eqref{eq:Scatt-Weak-All}  by following an exact evaluation for a given boundary condition, and by following the electric Gauss's law with no sources.

\begin{figure}[t!]
    \hspace*{-.7\textwidth}
    \vspace*{7.5em}%
     \begin{subfigure}{\textwidth}\caption{ }\label{fig:Nedelec:a}\end{subfigure}\\
    \hspace*{-.7\textwidth}
     \begin{subfigure}{\textwidth}\caption{ }\label{fig:Nedelec:b}\end{subfigure}
     \vspace*{-11.5em}\\
    \centering
    \scriptsize
    \def\svgwidth{.875\textwidth}
    \includeinkscape{FEM-Theory/3-Example-Nedelec-2}
\caption[Pyramidal and Tetrahedral Nédélec Finite Element Family of lowest Order]{Interpolating functions for the lowest order Nédélec \textbf{a)} pyramidal (2D) and \textbf{b)} tetrahedral (3D) reference finite elements. The thick magenta lines correspond to the edges where the linear functional of the Nédélec family [Eq. \eqref{eq:Nedelec}] is evaluated.}
\label{fig:Nedelec}
\end{figure}
