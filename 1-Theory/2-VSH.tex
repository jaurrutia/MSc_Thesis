% !TeX root = ../tesis.tex

The electric and magnetic field, denoted as $\vb{E}$ and $\vb{B}$, respectively, are a solution to the homogeneous vectorial Helmholtz\index{Helmholtz!Equation, Vectorial} when an harmonic time dependence and a spacial domain with no external charge nor current densities is assumed, that is,
%
% -----------------------------
\begin{subequations}
\begin{tcolorbox}[title = Vectorial Helmholtz Equation,	ams align, breakable]
	\grad^2 \vb{E}(\vb{r},\omega) + k_\text{m}^2 \vb{E}(\vb{r},\omega) &= \vb{0},\\
  \grad^2 \vb{B}(\vb{r},\omega) + k_\text{m}^2 \vb{B}(\vb{r},\omega) &= \vb{0}.
\end{tcolorbox}
\label{eq:Helmholtz}
\end{subequations}
% ------------------------------
%
\noindent where the vectorial operator $\grad^2$ must be understood as $\grad^2 = \nabla(\nabla\cdot) - \nabla\times\nabla\times $, and $k_\text{m}$ is the wave number in the matrix. It is possible to build a basis set for the electric and magnetic fields as long as the elements of this basis are also solution to Eq. \eqref{eq:Helmholtz}. One alternative is to employ the following set of vector functions
%
% -----------------------------
\begin{subequations}
\begin{align}
	\vb{L} =& \nabla \psi,
	\label{eq:L}\\
	\vb{M} =& \nabla\times(\vb{r}\psi),
	\label{eq:M}\\
	\vb{N} =&  \frac{1}{k_\text{m}}\nabla\times\vb{M},
	\label{eq:N}
\end{align}
\label{eq:VSH}
\end{subequations}
% ------------------------------
%
that are solution to the homogeneous vectorial Helmholtz equation as long as the scalar function $\psi$ is solution to the scalar Helmholtz equation\footnote{%
	This result can be proven by considering the following: Let $f$ be $\mathcal{C}^3$ and $\vb{F}$ a $\mathcal{C}^2$. Then, it is true that $\nabla^2(\nabla f) = \nabla(\nabla^2 f)$, and $\curl(\grad^2\vb{F}) = \grad^2(\curl\vb{F})$. }\index{Helmholtz!Equation, Scalar}
%
% -----------------------------
\begin{align}
	\nabla^2 \psi + k_\text{m}^2 \psi = 0.
\label{eq:HelmoltzScalar}
\end{align}
% ------------------------------
%
The triad $\left\{\vb{L},\vb{M},\vb{N}\right\}$ is a set of orthogonal vectors\footnote{%
	Employing the Einstein sum convention with $\epsilon_{ijk}$ the Levi-Civita symbol, Eq. \eqref{eq:M} can be the written as follows:%
	 	$M_i = [\nabla\times(\vb{r}\psi)]_i
	 	=  \epsilon_{ijk}\partial_j(r_k\psi)
	 	=\psi\epsilon_{ijk}\partial_j(r_k) -\epsilon_{ikj}r_k\partial_j\psi
	 	=\psi[\nabla\times\vb{r}]_i - [\vb{r}\times\nabla\psi]_i
	 	= - [\vb{r}\times\nabla\psi]_i
	 	= [\vb{L}\times\vb{r}]_i$,%
	 therefore $\vb{M}$ is orthogonal to $\vb{L}$ and $\vb{r}$. From Eq. \eqref{eq:N} $\vb{M}\cdot\vb{N}=0$.
	 \textcolor{red}{Falta probar $\vb{L}\cdot\vb{N} = 0$.}
	}%
that obey Helmholtz equation \textit{i.e.}, they can be directly identify as electric or magnetic fields. The the elements of the vector basis from Eq. \eqref{eq:VSH}   are known as the Vectorial Spherical Harmonics (VSH) as defined by  \citeauthor{stratton_electromagnetic_2012} \cite{stratton_electromagnetic_2012}, and \citeauthor{bohren_absorption_1983} \cite{bohren_absorption_1983}. The scalar function $\psi$ is known as the generating function of the VSH.


If spherical coordinates are chosen, and it is assume that $\psi(r,\theta,\varphi) = R(r)\Theta(\theta)\Phi(\varphi)$, then Eq. \eqref{eq:HelmoltzScalar} can be decouple into three ordinary differential equations:
%
% ------------------------------
 \begin{align}
	\frac{1}{\Phi}\pdv[2]{\psi}{\varphi} &+ m^2 \Phi =0,
 \label{eq:Phi}\\
	\frac{1}{\sin\theta}\dv{\theta}\qty(\sin\theta\dv{\Theta}{\theta}) &+ \qty[\ell(\ell+1)- \frac{m^2}{\sin^2\theta}]\Theta =0,
	\label{eq:Theta}\\
	\dv{r}\qty(r^2\dv{R}{r}) &+ \qty[ (k_\text{m} r)^2 - \ell (\ell +1)] R =0,
 \label{eq:Req}
\end{align}	
% ------------------------------
%
where $\ell$ can takes natural values and zero, and $\abs{m}\leq \ell$ so $\Phi$ and $\Theta$ are univaluated and finite on a sphere. Eqs. \eqref{eq:Theta} and \eqref{eq:Req} can be rewritten as
%
% ------------------------------
 \begin{align}
(1-\mu^2)\dv[2]{\Theta}{\mu} - 2\mu\dv{\Theta}{\mu} + \qty[\ell(\ell+1)-\frac{m^2}{1-\mu^2}]\Theta &= 0, \qqtext{ with $\mu = \cos\theta$,}
	\label{eq:ThetaMu}\\
	\rho\dv{\rho}\qty(\rho\dv{Z}{\rho}) +  \qty[\rho^2 - \qty(\ell + \frac12)^2]Z  &= 0,  \qqtext{ with $Z = R\sqrt{\rho}$ and $\rho = k_\text{m}r$.}
\label{eq:Reqkr}
\end{align}	
% ------------------------------
%
The solution to Eq. \eqref{eq:ThetaMu} are the Legendre Associated Functions $ P_\ell^m(\mu)$ and to Eq. \eqref{eq:Reqkr} the solution is given by the Spherical Bessel Functions $z_\ell = j_\ell, y_\ell$. Following the convention from most literature on Mie Scattering \cite{zangwill_modern_2013}, the solution to Eq. \eqref{eq:Phi} will be decompose in an odd ($o$) and an even ($e$) solution, that is, as sine and cosine functions. After this procedure, it is determined that the generating function of the VSH is given by
%
% -----------------------------
\begin{subequations}
\begin{tcolorbox}[title = $\psi$: Generating function of the vectorial spherical harmonics,	ams align, breakable]
	\psi_{e\ell m}(r,\theta,\varphi) =& \cos(m\varphi)P_\ell^m(\cos\theta)z_\ell(k_\text{m}r), 
	\\ 	
	\psi_{o\ell m}(r,\theta,\varphi) =& \sin(m\varphi)P_\ell^m(\cos\theta)z_\ell(k_\text{m}r).
\end{tcolorbox}
\label{eq:Helmholtz}
\end{subequations}
% ------------------------------
%
\noindent
























