% !TeX root = ../tesis.tex

Any PDE system describing a physical system in either equilibrium or in a steady-state, with a set of boundary conditions, can be described by \cite{dhatt_finite_2012}
%
\begin{align}
    \mathcal{L}[\vb{u}(\vb{r})] - \vb{f}_\Omega = 0,
    \qquad\qquad
    \mathcal{C}[\vb{u}(\vb{r})] = \vb{f}_{\partial\Omega},
\label{eq:PDESystem}
\end{align}
%
where $\mathcal{L}$ and $\mathcal{C}$ are differential operators applied on the unknown functions $\vb{u}$ ---with $D$ components--- on a domain $\Omega$ and its boundary $\partial\Omega$, respectively, and  $\vb{f}_\Omega$ and $\vb{f}_{\partial\Omega}$ are known functions related to the sources of $\vb{u}$ and to the boundary conditions each.  The PDE system as described by Eq. \eqref{eq:PDESystem} is known as the strong formulation problem since  the unknown functions $\vb{u}$ must be $m$-times differentiable on $\Omega$ if $\mathcal{L}$ is a differential operator of order $m$ \cite{dhatt_finite_2012,larson_finite_2013}. It is possible to relax such differentiability condition on $\vb{u}$ by employing the weak formulation of the PDE system, which in an integral representation of Eq. \eqref{eq:PDESystem} obtained by multiplying it by a trial function $\psi$ and integrating on $\Omega$ \cite{dhatt_finite_2012,larson_finite_2013,fletcher_computational_1984}
%
\begin{align}
    W(\vb{u}) = \int_\Omega \psi_j(\vb{r}) \left\{ \mathcal{L}[\vb{u}(\vb{r})] - \vb{f}_\Omega   \right\} \dd{\Omega} = \vb{0}.
    \label{eq:Weak}
\end{align}
%
The weak formulation of the PDE system yields a weak solution to $\vb{u}$ since Eq. \eqref{eq:Weak} can be rewritten by performing $s$-fold integration by parts and then either employing Gauss' Theorem or Green's first identity \cite{larson_finite_2013}:
%
\begin{align}
    \int_\Omega \psi \nabla\cdot \vb{u}\dd{\Omega} &=  - \int_\Omega \nabla\psi \cdot \vb{u} \dd{\Omega} + \oint_{\partial\Omega} \psi \vb{u}\cdot \vu{n}\dd{(\partial\Omega)},
        & \mbox{(Gauss' Theorem),}
        \label{eq:GT}
    \\
    \int_\Omega \vb{u}\cdot\nabla\psi_n \dd{\Omega} &=  - \int_\Omega \psi \nabla\cdot\vb{u}\dd{\Omega}  + \int_{\partial \Omega}\psi \vb{u}\cdot\vu{n}\dd{(\partial\Omega)},
        & \mbox{(Green's first identity).}
        \label{eq:G1I}
\end{align}
%
If  Eqs. \eqref{eq:GT} or \eqref{eq:G1I} are used $s$-fold on Eq. \eqref{eq:Weak} then $\vb{u}$ must be differentiable $m-s$ times instead of $m$, while $\psi$ must be $s$ times differentiable. More over, the boundary conditions imposed to $\vb{u}$ must be satisfied only if they contain  derivatives up to the order $m-s-1$ since the conditions with derivatives of order bigger than $m-s$ are taken into account in the integrals of Eqs. \eqref{eq:GT} and \eqref{eq:G1I} ---and on such boundary conditions $\psi$ must equal zero---. Thus, any solution $\vb{u}$ to Eq. \eqref{eq:Weak} is known as a weak solution  since it does not holds the differentiability condition as they are required by the equivalent strong formulation \cite{dhatt_finite_2012}.

    \subsection{The Nodal Approximation and the Weighted Residual Method}

         One way to find an approximated solution to $\vb{u}$ it to employ the weighted residual method, which changes the PDE system to an algebraic equation system by proposing an approximation  $\tilde{\vb{u}}$ as a linear combination of the known functions $\{\phi_i\}_{i\leq N}$, evaluated in $\Omega$, which differs from the exact solution $\vb{u}$ by an error $e_{\vb{u}}$, that is \cite{dhatt_finite_2012,larson_finite_2013,fletcher_computational_1984}
        %
        \begin{align}
            \vb{u}(\vb{r}) = \tilde{\vb{u}}(\vb{r}) + e_{\vb{u}}(\vb{r}),
                \qquad\qquad
                \text{with}
                \qquad\qquad
                \tilde{\vb{u}}(\vb{r}) = \sum_{i = 1}^{N} \vb{a}_i\phi(\vb{r}),
            \label{eq:uapprox}
        \end{align}
        %
         where $\tilde{\vb{u}}$ follows the same boundary conditions as $\vb{u}$ at $\partial\Omega$ and $\vb{a}_i$ are $N$ parameters to be determined\footnote{The $N$ parameters $a_i$ are constant for equilibrium and steady-state problems while they may depend on time for transport problems \cite{dhatt_finite_2012}.} with $D$ components. The values of $a_i$ are chosen such that $e_{\vb{u}}\ll  \tilde{\vb{u}} $, which may be achieved by increasing the cardinality $N$ of the known functions set  or choosing values of $\vb{a}_i$ that match the exact value of $\vb{u}$ at determined points.

         One particular form to the approximated solution $\tilde{\vb{u}}$ in Eq. \eqref{eq:uapprox} is known as the nodal approximation \cite{dhatt_finite_2012, fletcher_computational_1984}:
        %
        \begin{align}
            \tilde{\vb{u}}(\vb{r}) = \sum_{i = 1}^N \vb{u}_i \phi_i(\vb{r}),
                \qquad
                \text{with}
                \qquad
            \vb{u}_i = \vb{u}(\vb{r}_i),
        \label{eq:Nodal}
        \end{align}
        %
        where  $\phi_i$ are the interpolating ---or shape--- functions and $\vb{u}_i$ are coefficients that equals the exact value of the function $\vb{u}$ at some points $\vb{r}_j \in \Omega$, called the nodal points. From Eq. \eqref{eq:Nodal} it can be seen that the error $e_{\vb{u}}$ between the exact and the approximated solutions vanishes at the nodes $\vb{r}_j$ and thus $\phi_i(\vb{r}_j) = \delta_{ij}$, with $\delta_{ij}$ the Kronecker delta.

         Since $\tilde{\vb{u}}$ is an approximated solution, the evaluation fo Eq. \eqref{eq:Nodal} into Eq. \eqref{eq:PDESystem} equals to a residual $  R_{\tilde{\vb{u}}}(\vb{r},\{\vb{u}_i\}_{i\leq N}) $ which in general is different to zero \cite{fletcher_computational_1984,larson_finite_2013}, that is,
        %
        \begin{align}
            \mathcal{L}[\tilde{\vb{u}}(\vb{r})] - \vb{f}_\Omega = R_{\tilde{\vb{u}}}(\vb{r},\{\vb{u}_i\}_{i\leq N}) \neq 0.
        \label{eq:PDESystem-app}
        \end{align}
        %
        To determine the coefficients $\vb{u}_i$, the residual $R_{\tilde{\vb{u}}}$ is multiplied by a weighting ---or trial--- function $\psi_j$ and integrated over $\Omega$ imposing that the integral goes to zero, that is
        %
        \begin{align}
            W(\tilde{\vb{u}}) = \int_\Omega \psi_j(\vb{r}) R_{\tilde{\vb{u}}}(\vb{r},\{\vb{u}_i\}_{i\leq N}) \dd{\Omega} = \vb{0}.
                \qquad\qquad
                \text{with}
                \qquad\qquad
                \psi_j \in \{\psi_j\}_{j\leq N},
            \label{eq:WRM}
        \end{align}
        %
        which is a set of $N\times D$ independent algebraic equations with $N\times D$ variables, where $D$ is the dimensionality of $\vb{u}$.

        The weighted residual method is a family of numerical methods defined by the election of the trial functions set $\{\psi_j\}_{j\leq N}$. Some of the most common election for the trial functions set yield the collocation method, the least-squares method, the method of moments\footnote{The collocation method employes Dirac deltas as trial functions $\psi_j(\vb{r}) = \delta(\vb{r} - \vb{r}_j)$, the least-squares method minimizes integral in Eq. \eqref{eq:WRM} by setting $\psi_j = \partial R_{\tilde{u}}/\partial a_j$, which $a_j$ the parameters to determined, and the method of moments  sets $\psi_j(x) = x^j$ \cite{fletcher_computational_1984}. } and the Galerkin method, which sets the trial functions equal to the interpolating functions, that is, $\{\psi_j\}_{j\leq N} = \{\phi_i\}_{i\leq N}$ \cite{fletcher_computational_1984}.  It is worth noting that Eq. \eqref{eq:WRM} equals Eq. \eqref{eq:Weak}, and thus $\vb{u} = \tilde{\vb{u}}$, if the trail functions are elements of an infinite set, that is, $N \to \infty$ \cite{dhatt_finite_2012}.
