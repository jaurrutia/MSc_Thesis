% !TeX root = ../tesis.tex

To compare the optical response of a NP in the presence of a substrate with that of a NP in a totally homogeneous environment, let us first analyze the spectral response given by the Mie Theory when the matrix and the size of the NP varies. In Fig. \ref{fig:Mie:redshift} it is shown the wavelength of resonance $\lambda_\text{res}$, that is, the wavelength at which the scattering (orange) and extinction (black) efficiencies are maximized,  as a  function of the radius $a$ of a AuNP embedded in a matrix of air [Fig. \ref{sfig:red:1}] and of glass [Fig. \ref{sfig:red:2}], with a refractive index of $n_\text{m} = 1$ and $n_\text{m} = 1.5$, respectively, and as a function of the refractive index of the matrix $n_\text{m}$ for a AuNP with a radius of {$a = 12.5$ nm} [Fig. \ref{sfig:red:3}] and with a radius of $a = 50$ nm [Fig. \ref{sfig:red:4}]. For the optical response of the AuNP it was employed the experimental data as reported by \citeauthor{johnson_optical_1972} \cite{johnson_optical_1972} (filled circles) and by considering a size correction ---see Appendix \ref{app:SizeCorrection}--- to it (empty circles).

\begin{figure}[h!]
    \def\svgwidth{\textwidth}
    \includeinkscape[pretex = \small]{Redshift/redshift}
    \vspace*{-20.9em} \\
    \hspace*{1em}%
        \begin{subfigure}{.225\textwidth}\caption{AuNP$@$Air}\label{sfig:red:1}\end{subfigure}%
        \begin{subfigure}{.28\textwidth}\caption{AuNP$@$Glass}\label{sfig:red:2}\end{subfigure}%
        \begin{subfigure}{.225\textwidth}\caption{12.5 nm AuNP}\label{sfig:red:3}\end{subfigure}%
        \begin{subfigure}{.24\textwidth}\caption{50 nm AuNP}\label{sfig:red:4}\end{subfigure}
    \vspace*{16.5em}\\
    \caption[Spectral redshift of the scattering and extinction  efficiencies of a spherical AuNP as a function of its size and the embedding media]{Resonance wavelength $\lambda_\text{res}$ of the scattering (orange) and extinction (black) efficiencies of a AuNP as a function of the NP's radius when embedded \textbf{a)} in air ($n_\text{m} = 1)$ and \textbf{b)} in glass ($n_\text{m} = 1.5$), and as a function of the refractive index of the matrix  $n_\text{m}$ for a AuNP of radius \textbf{c)} 12.5 nm and \textbf{d)} 50 nm, using the dielectric function for gold as reported by \citeauthor{johnson_optical_1972} \cite{johnson_optical_1972} (filled circles) and considering a size correction to it (empty circles).}
    \label{fig:Mie:redshift}
\end{figure}

From the results shown in Fig. \ref{fig:Mie:redshift} it can be seen that the wavelength of resonance  $\lambda_\text{res}$ for the extinction, considering the bulk dielectric function for Au (filled circles), is smaller than that of the scattering  and that the distance between them decreases as either the size of the AuNP or the refractive index of the matrix increases. This behavior arises from a redshift of $\lambda_\text{res}$ for increasing values of $a$ and $n_\text{m}$ and it shows that, for particles small compared to the wavelength of the incident light in the matrix, the main contribution to the extinction of light  is due to absorption processes and as the size of the AuNP grows, the extinction is dominated by its other contribution: the scattering, as discussed in Section \ref{ss:AuMie} and supported by Eq. \eqref{eq:Cext}. The redshift of   $\lambda_\text{res}$ can also be observed when considering a size corrected dielectric function (empty circles). Remarkably, for values of radius $\lesssim 15/n_\text{m}$ there is a blueshift of $\lambda_\text{res}$, as it can be seen in Figs. \ref{sfig:red:1} and \ref{sfig:red:2}, which is a consequence of a greater imaginary part of the dielectric function for the AuNP due to the size correction. On the other hand, an increase in $n_\text{m}$ for a fixed radius presents only redshifts either with or without a size corrected dielectric function [see Figs. \ref{sfig:red:3} and  \ref{sfig:red:4}].

The spectral behavior of the scattering and extinction of light due to a spherical NP summarized in Fig. \ref{fig:Mie:redshift} was calculated by assuming a homogeneous medium (the matrix) where the NP is embedded and thus allowing the direction of the illuminating plane wave to be arbitrary, yet yielding the same results. In the following Sections, the homogeneity of the surroundings of the NP is substituted by two semiinfinite media and thus modifying the optical response of the system depending on how it is illuminated.
