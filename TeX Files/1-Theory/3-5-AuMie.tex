% !TeX root = ../tesis.tex

\textbf{Falta emplear las relacinoes de ortogonalidad para calcular las secciones de extincion y edmás}
\clearpage


In Fig. \ref{fig:Mieefficiencies} the extinction  $Q_\text{ext}$ and the scattering  $Q_\text{sca}$ efficiencies of a gold NP (AuNP) with a radius of $a = 12.5$ nm are shown as function of the wavelength $\lambda$ of the incident planewave illuminating the NP. Two matrices were considered in Fig. \ref{fig:Mieefficiencies}: a matrix of air with a refractive index of $n_\text{mat} = 1$ (black lines) and a matrix of glass with $n_\text{mat} = 1.5$ (orange lines).  The experimental dielectric function for Au reported by \citeauthor{johnson_optical_1972} \cite{johnson_optical_1972}  was employed to model the electromagnetic response of the spherical AuNP nevertheless, this data corresponds to a bulk sample, meaning that it may not reproduce the optical behavior of a NP since surface effects cease to be neglectable  due to their spacial dimensions \cite{noguez_surface_2007}.   In order to study  the optical properties of AuNP while considering  surface effects,  a size correction to the dielectric function of the AuNP was performed as described in Appendix \ref{app:SizeCorrection}. The efficiencies of the $12.5$ nm AuNP  taking into account a dielectric function with (dashed lines) and without (solid lines) a size correction were compared for both considered matrices; on each curve the wavelength of their maximum value is indicated. The value of $\lambda$ at which the extinction efficiency is maximum corresponds to the excitation wavelength of the Localized Surface Pasmon Resonance (LSPR), that is, the wavelength where the extinction of light occurs the most.

% -------------------------------------- Extinctintion and scattering efficiencies -------------------------------
% --------------------------------------          12.5 nm AuNP @ Air and @ glass   -------------------------------
% --------------------------------------               fig:Mieefficiencies         -------------------------------
\begin{figure}[h!]
	\def\svgwidth{1\textwidth} \small
  \vspace*{3.0em}
  \hspace*{-9.em}
    \begin{subfigure}{.46\textwidth}\caption{ }\label{fig:Mieefficiencies:a}\end{subfigure}
    \begin{subfigure}{.49\textwidth}\caption{ }\label{fig:Mieefficiencies:b}\end{subfigure}
  \vspace*{-6.em}\\
  \includeinkscape{1-Theory-Figs/Mie-Au/1-Efficiencies}
  \vspace*{-2em}
  \caption[Extinction and Scattering Efficency of a 12.5 nm Au Spherical NP embeded into Air and Glass]{ \textbf{a)} Extinction $Q_\text{ext}$ and \textbf{b)} scattering $Q_\text{sca}$ efficiencies of a 12.5 nm Au spherical NP embeded into air (black, $n_\text{mat} = 1$)  and into galss (orange, $n_\text{mat} = 1.33$), as function of the wavelength $\lambda$ of the incident planewave.  The solid curves were calculated by considering no size effects on the dielectric function of the AuNP, while the dashed curves consider a size correction to it; the experimental data of \citeauthor{johnson_optical_1972} \cite{johnson_optical_1972} was employed.}
\label{fig:Mieefficiencies}
\end{figure}
% --------------------------------------               fig:Mieefficiencies         -------------------------------

From the extinction and scattering efficiencies in Fig. \ref{fig:Mieefficiencies}, two main spectral tendencies arises between these quantities: on the overall value of the efficiencies and on the spectral position of their maximum. Since the scattering efficiency is two orders of magnitude smaller than the extinction efficiency within the visible range, the main energy loss mechanism is absorption, as stated by the Optical Theorem [Eq. \eqref{eq:Cext}]. Even though the 12.5 nm AuNP absorbs more light than what it scatters, both phenomena are present. For example, when the  12.5 nm AuNP is embedded into air, the wavelengths at which it absorbs and scatters the most are $\sim 509$ nm and $\sim 522$ nm, respectively. When the AuNP is embedded into glass, the absorption of light is optimized at a wavelength of $\sim 535$ nm, while the scattering is optimized at $\sim 543$ nm. For both an air and a glass matrix, the wavelength of most scattering is redshifted relative to the wavelength of most absorption, which is the excitation wavelength of the LSPR for each case: $\sim 15$ nm for air and $\sim 7$ nm for glass. The extinction and scattering have different spectral tendencies as already discussed, yet they share common characteristics   when the dependency on the embedding media is studied.

Both the scattering and the absorption efficiencies of a 12.5 nm AuNP present an overall enhancement within the visible range when the refractive index of the matrix, the embedding media,  increases. This can be seen in Fig. \ref{fig:Mieefficiencies} by comparing the black curves ($n_\text{mat} = 1$, air) with the orange curves ($n_\text{mat} = 1.5$, glass). The value of the extinction efficiency of a 12.5 nm AuNP  at the wavelength of the LSPR for a matrix of glass is around five times larger than for a matrix of air, as it can bee seen in Fig. \ref{fig:Mieefficiencies:a}, while the maximum value of the scattering efficiency for a glass matrix is around eight times larger compared to the case of an air matrix, as shown in Fig. \ref{fig:Mieefficiencies:b}. The enhancement of the extinction and the scattering efficiency as function of the refractive index of the matrix can be understood by analyzing the size paramter $ x = (2 \pi / \lambda ) a n_\text{mat} $, which compares the size of the NP relative to the incident wavelength in the matrix: the larger the value of $x$, the bigger a NP can be considered inside such matrix. Since the size parameter is a linear function of $n_\text{mat}$, the AuNP embedded into glass optically responds like a larger NP than what it is in air, thus having a more significant contributiom from the scattering to the light extinction mechanism inside glass, as well as an increse in the absorption.

The extinction [Fig. \ref{fig:Mieefficiencies:a}] and the scattering [Fig. \ref{fig:Mieefficiencies:b}] efficiencies of the same system, a 12.5 nm AuNP embedded into a matrix of either air or glass, are different whether a size correction to the dielectric function of the AuNP is considered (dashed lines) or not (solid lines). On the one hand, there is a spectral shift of LSPR excitation wavelength for both matrices of $\sim 2$ nm: the AuNP embedded into air redifts from $\sim 507$ nm (no size correction) to $\sim 509$ nm (size correction) and the AuNP embedded into glass blueshifts from $\sim 506$ nm to $\sim 535$ nm. The maximum value of $Q_\text{sca}$ redifts $\sim 1$ nm when considering the size correction of the dielectric function when the AuNP is embeded into air nevertheless, the wavelength of maximum scattering does not present a red nor blueshift larger than $1$ nm. On the other hand, the value of the efficiencies around the wavelength where the absorption and the scattering is maximized decreases in all cases shown if Fig. \ref{fig:Mieefficiencies}. The descrease in the efficiencies due to a size corrected dielectric function is more evident for a matrix out of glass than out of air. This behavior can be explained once again by the size paramter $x$ since two different dielectric functions are amployed into a AuNP which has the opical properties of a larger NP when it is embedded into glass rather than into air. From this analysis it can be concluded that the most notable effect of a size correction to the dielectric function of a NP is the decrease in the extinction and scattering efficiencies, while there is still a spectral shift of the LSPR, whose effect is less relevant the larger the size parameter is.

While the scattering efficiency $Q_\text{sca}$ is an integral quantity, that is, it describes the scattering in all directions of a planewave traveling in the  direction $\vb{k}^\text{i}$ due to the interaction with a NP, the scattering amplitude matrix element $S_1(\theta)$,  given  for a spherical NP by Eq. \eqref{eq:S1}, depict the scattering of an incident electric field $\vb{E}^\text{i}$ polarized perpendicularly to the scattering plane for a particular observation angle $\theta$; in a similar manner, $S_2(\theta)$ [Eq. \eqref{eq:S2}] corresponds to the scattering of a parallel polarized $\vb{E}^\text{i}$ relative to the scattering plane at $\theta$. A scattering map Fig. \ref{fig:ScatteringMaps} the squared modulus of the scattering matrix elements $\abs{S_{1}(\theta)}^2$ (solid lines) and $\abs{S_2(\theta)}^2$ (dashed lines) are shown as function of the observation angle  $\theta$ for a AuNP with a radius equal to 12.5 nm embedded into air [Fig. \ref{fig:ScatteringMaps:a}] illuminated by a planewave at a wavelength $\lambda = 522$ nm and a 12.5 nm AuNP embedded into glass  [Fig. \ref{fig:ScatteringMaps:b}]  illuminated at $\lambda = 543$ nm; the illuminating wavelength corresponds to the vaud of $\lambda$ where $Q_\text{sca}$ is maximized for each matrix as seen in Fig. \ref{fig:Mieefficiencies}.

% --------------------------------------				 Scaattering Maps 	   ------------------------------
% --------------------------------------   12.5 nm AuNP @ Air and @ glass    ------------------------------
% --------------------------------------         fig:ScatteringMaps         -------------------------------
\begin{figure}[h!]
	\def\svgwidth{.9\textwidth} \small\centering
		\vspace*{3.25em}
		\hspace*{-8.5em}
	\begin{subfigure}{.45\textwidth}%
		\caption{$\abs{S_{1,2}(\theta)}^2\times 10^4$} \label{fig:ScatteringMaps:a}%
		\end{subfigure}%
	\begin{subfigure}{.45\textwidth}%
		\caption{$\abs{S_{1,2}(\theta)}^2\times 10^3$}\label{fig:ScatteringMaps:b}%
		\end{subfigure}%
	\vspace*{-6.25em}\\
	\includeinkscape{1-Theory-Figs/Mie-Au/2-ScatteringMaps}
	\vspace*{-.5em}
	\caption[Scattering Map of a 12.5 nm Au Spherical NP embeded into Air and Glass]{Scattering map of a 12.5 nm Au spherical NP embeded into \textbf{a)} air illuminatated by a planewave at a wavelength of $\lambda = 522$ nm and \textbf{b)} glass illuminated at $\lambda = 543$ nm; the wavlength in each case corresponds to the wavelength of maximum scattering (see Fig. \ref{fig:Mieefficiencies}). The solid (dashed) lines corresponds to the scattering matrix element $S_1$ ($S_2$) related to an incident electric field $\vb{E}^\text{i}$ travelling in the $\vb{k}^\text{i}$ direction  and polarized perpendicularly (prallel) to the page. It was considered for both matrices a size corretion to the experimental data of \citeauthor{johnson_optical_1972} \cite{johnson_optical_1972} for the electromagnetic response of the AuNP.}
	\label{fig:ScatteringMaps}
 \end{figure}
 % --------------------------------------         fig:ScatteringMaps         -------------------------------

The quantities $\abs{S_{1,2}(\theta)}^2$ for a 12.5 nm AuNP embedded into air ($n_\text{mat} = 1$, black curves) are one order of magnitude smaller  than into glass ($n_\text{mat} = 1.5$, orange curves), meaning that the AuNP into air scatters light less efficiently than the same particle embedded into glass. This result is consistent with the obtained values for the scattering efficiency $Q_\text{sca}$ in Fig. \ref{fig:Mieefficiencies}. On the angular dependency of the scattering, it can be seen that the 
    
    




























