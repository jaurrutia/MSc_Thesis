% !TeX root = ../tesis.tex


The weighted residual method returns an approximated solution $\tilde{\vb{u}}$ to $\vb{u}$, in the weak sense, as a linear combination of the interpolating functions $\{\phi_i\}_{i\leq N}$ so long the error $e_{\vb{u}}$ can be depreciated for all points in the domain $\Omega$. From a computational approach, the required computing time and resources increase as $\Omega$ does, therefore requiring the cardinality of the sets $\{\psi_j\}_{j\leq N}$ and  $\{\phi_i\}_{i\leq N}$ to increase as well \cite{dhatt_finite_2012}. To overcome such problem, one option is to employ the finite element approximation.

The finite element approximation allows the use of low order interpolating functions by defining the subdomains $\Omega_k$ such that
%
\begin{align}
    \bigcup_{k=1}^M \Omega_k = \Omega,
        \qquad
        \text{and}
        \qquad
    \bigcap_{k=1}^M \Omega_k = \emptyset,
\label{eq:finite}
\end{align}
%
that is that, all $\Omega_k$ together represent the original domain $\Omega$ and that they do not overlap nevertheless, the boundaries of the finite elements are shared by neighbouring elements. Then, the finite element approximation restricts  $\tilde{\vb{u}}_k$  ---the nodal approximation  on each subdomain--- to depend only on the nodal points on $\Omega_k$ and on its boundary $\delta\Omega_k$, while all $\tilde{\vb{u}}_k$ must be continuos across $\partial \Omega_k$ and obey the  differentiability condition they are bound to, whether the strong or weak formulation is employed \cite{dhatt_finite_2012}.

    \subsubsection{The Finite Element}

    A finite element $\Omega_k$ is a subdomain of $\Omega$ following Eq. \eqref{eq:finite} but its formal definition requieres $\Omega_k$ to be a manifold embedded into $\Omega$, as well as to chose a polynomial function space on $\Omega_k$, and to define a collection of $N_k$ linear functionals $\mathcal{F}_{\ell_k}[\cdot]$ on $\Omega_k$. The description of $\Omega_k$ as a manifold determines its geometrical properties such as dimensionality, shape and curvature, while the polynomials function space sets the order of the interpolating functions $\{\phi_{i_k}\}_{i_k\leq N_k}$. By applying $\phi_{i_k}$ into  $\mathcal{F}_{\ell_k}[\cdot]$, a system of $N_k$ algebraic  equations is obtained:
    %
    \begin{align}
        \mathcal{F}_{\ell_k}[\phi_{i_k}] = \delta_{{\ell_k} {i_k}},
        \label{eq:linfunc}
    \end{align}
    %
    from which the interpolating functions are obtained. Since the $N_k$ linear functional imposes conditions on the interpolating functions, the $N_k$ correspondes to the number of degrees of freedom of the finite element.

     The finte element corresponds to a particular way of discretitation of the domain $\Omega$ and thus it is convinient to define the manifold $\Omega_k$ by its geometrical nodes $\vb{r}_{n_k}$, which are a finite collection of  points in both $\Omega_k$ and its boundary. In general, the domain $\Omega$ does not have cartesian symmetry, therefore the finite elements are in general curvilinear; some examples of straight and curvilinear finite elements can be found in Fig. \ref{fig:FiniteElement-Sys}  for one, two and three dimensional $\Omega_k$, as well as equivalent elements with straight boundaries; eventhough elements shown are triangular (2D) and  traingular piramids (3D), their shapes can also be composed by squares and prisms (see \cite{dhatt_finite_2012} for a mor extended list of shapes). The markers on each finite element corresponds to their geometrical nodes and the red markers corresponds to the edges of the element, which are geometrical nodes shared among all the neighbouring finite elements.
      %
     \begin{figure}[h!]
         \centering
        \def\svgwidth{.8\textwidth} \small
       \includeinkscape{FEM-Theory/1-systems-Elements-orders}
        \caption[Extinction and Scattering Efficency of a 12.5 nm Au Spherical NP embeded into Air and Glass]{ \textbf{a)} Extinction $Q_\text{ext}$ and \textbf{b)} scattering $Q_\text{sca}$ efficiencies of a 12.5 nm Au spherical NP embeded into air (black, $n_\text{mat} = 1$)  and into galss (orange, $n_\text{mat} = 1.33$), as function of the wavelength $\lambda$ of the incident plane wave.  The solid curves were calculated by considering no size effects on the dielectric function of the AuNP, while the dashed curves consider a size correction to it; the experimental data of \citeauthor{johnson_optical_1972} \cite{johnson_optical_1972} was employed.}
     \label{fig:FiniteElement-Sys}
     \end{figure}
     %

     The choice of finite elements with straight or curvilinear boundaries is related to the discretitation method of the domain $\Omega$: Were $\Omega$ a cilinder in 3D, the use of finite elements with straight boundaries arises an error due to truncation of $\Omega$ at its boundary, which can be minimized by increasing the number of finite elements, while curvilinear finite elements may fill such space without increasing the number of finite elements.One disadvantage of curvilinear finite elements is that the determination of the interpolating functions through Eq. \eqref{eq:linfunc} might be performed for each finite element. To avoid such complication, a transformation $T$ is performed on the finite elements of $\Omega$ ---refered as the real-space elements--- in order to reshape it into a reference element, which are finite elements with the same number of geometrical nodes and edges but with straight boundaries. The transformation $T$ is a change of coordinates from the real-space system points $\vb{r}$ into a coordinate system $\boldsymbol{\xi}$ where the finite element has a simplier geometry. The use of Eq. \eqref{eq:linfunc} in the reference elements yield the different kind of interpolating functions, which can be employed to solve the weak formulation of a PDE system [Eq. \eqref{eq:WRM}] by transforming the therivatives in the real space coordinate system by means of the jacobian matrix $\mathbbm{J}$, whose elements are $J_{ij} = \partial \xi_i/\partial r_j$, and its determinant, the jacobian, that is:
     %
    \begin{align}
        \pdv{r_i} = \pdv{\xi_i}{r_j} \pdv{\xi_j},
        \qquad
        \text{and}
        \qquad
        \dd{\Omega_k} \to \det[\mathbbm{J}]\dd{\Omega_k}.
    \label{eq:jac}
    \end{align}
     %
     The Eq. \eqref{eq:jac} sets a constriction into the discretitation of the original domain $\Omega$ and its partition into finite elements, since the jacobian must be non singular ---different from zero--- in all points in $\Omega_k$, meaning that the transformation $T$ of the real-space finite element into the reference element is bijective. To avoid singular points in $\Omega_k$, the real-space finite element must not be deformed considerably when transformed into the reference element.

     In order to build the interpolating functions in the reference finite elemente, $\psi_{i_k}(\boldsymbol{\xi})$, one must  first choose the polynomial functions space on the manifold $\Omega_k$, which is defined by the number of geometrical nodes found in between two edges so that the boundary conditions are met. For example, if there are $m$ geometrical nodes between two edges, the finite element is said to be of order $m+1$ since a polynomial of order $m+1$ is garanteed to pass through the values given to  the two edges and the $m$ nodes. For example, the finite elements in Fig. \ref{fig:FiniteElement-Sys} are a cuibc order 1D finte element (two geometrical nodes between edges), a quadratic order triangular finite element (2D shape with one node between edges), and a liniear tetrahedral finite element (3D volume with four triangular faces and no nodes between edges).

     Once the polynomial functions space on the manifold $\Omega_k$ is set, this can be spanned by the set of interpolating functions $\phi_{i_k}(\boldsymbol{\xi})$, that are determined by means of the $N_k$ linear functional  $\mathcal{F}_\ell[]\cdot]$.   The election of the  linear functionals give rise to different sets of interpolating functions and thus different families of finite elements. For example, the linear functional given by
     %
     \begin{align}
    \mathcal{F}^{L}_{\ell_k}[f(\boldsymbol{\xi})] = f(\boldsymbol{\xi_{\ell_k}}) ,
    \label{eq:Lag}
     \end{align}
     %
     with $f$ an arbitrary function and $\boldsymbol{\xi}_{\ell_k}$ the geometrical nodes in the reference finite element, defines the Lagrange finite element family since the  interpolating functions obtained by employing Eqs. \eqref{eq:linfunc} and \eqref{eq:Lag} are the Lagrange polynomials. Such interpolating functions arises since the Eq.  \eqref{eq:Lag} considers that there are $N_k$ geometrical nodes and since the functional $\mathcal{F}^{L}[\cdot]$ corresponds to an evaluation point on the geometrical nodes, while Eq. \eqref{eq:linfunc} warranties that only one interpolating function is non-vanishig at such points. The functional $\mathcal{F}^{L}[\cdot]$ do not imposes any condition on the derivatives of $\phi_m$, specifically, in the boundary of $\Omega_k$, therefore the Lagrange finite element family returns a solution on $\Omega$ with discontinuois derivatives. One linear functional which returns interpolating functions with continuos first derivatives is
     %
     \begin{align}
         \mathcal{F}^{H}_\ell[f(\vb{r})] = f(\vb{r}_\ell) ,
         \qquad
         \text{and}
         \qquad
         \mathcal{F}^{H}_{\ell'}[f(\vb{r})] = \vb{t}\cdot\nabla f(\vb{r}_{\ell'}) = 0 ,
     \label{eq:Her}
    \end{align}
         %
     with $\vb{r}_\ell$ the geometric nodes, $\vb{r}_{\ell'}$ the geometry nodes at the edges of $\Omega_k$, and $\vb{t}$ a vector parallel to its boundary. The functional in Eq. \eqref{eq:Her} give rise to the Hermite finte element family since the resulting interpolating functions are the Hermite polynomial, which allows for a solution on $\Omega$ 1-differentiable\footnote{Another difference between the Lagrange and the Hermite finite elements family is the degrees of freedom. In the first one, the degrees of freedom corresponds to the number of geometrical nodes while the later considers, additionally, the directional derivative along the boundary at the edges of the finite element. }.  The Lagrange and the Hermite finite element families are two of the most common and simple nevertheless, one can build yet another family known as the serendipity finte element family  if the geometrical nodes outside de boundary of $\Omega_k$ are not considered in Eq. \eqref{eq:Lag}, which decreases the degrees of freedom $N_k$.

     \begin{figure}[h!]
         \centering
         \def\svgwidth{.9\textwidth} \scriptsize
       \includeinkscape{FEM-Theory/2-Example-Elements-2D}
        \caption[Example of interpolating functions for 1D and 2D finite element]{Interpolating functions of \textbf{a)} a Lagrange cubic 1D reference finite element and \textbf{b)} a Lagrange quadratic 2D triangular reference element method. The blue markers corresponds to the evaluation of the linear functionals [Eqs. \eqref{eq:linfunc} and $\eqref{eq:Lag}$] on the interpolating functions of each case.}
     \label{fig:ExampleElements}
     \end{figure}
