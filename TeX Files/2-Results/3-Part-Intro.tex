% !TeX root = ../tesis.tex

When studying the optical properties of ensambles of NPs, a series of assumptions and simplifications are done in order to describe their optical response theoretically. For example, the van de Hulst  \cite{hulst_light_1981} and the Maxwell Garnett \cite{sihvola_electromagnetic_2008} models for an effective dielectric function of colloidal three dimensional mixtures requiere not only the NPs to be perfectly spherical (and small compared to the incident wavelength for the later), but also a low concentration of NPs in the ensamble, which yields the Single Scattering Approximation to be valid and thus the results from the Mie Theory ---Section \ref{s:Mie}--- can be exploited. When the optical properties of bidimensional arrays of NPs are of interest, one assumption common to several models, besides the aforementioned, is that the NPs are  perfectly supported on the substrate, as it is the case example in the Thin Island Theory \cite{bedeaux_optical_2004}, the Dipolar Model \cite{barrera1991optical}, and the Coherent Scattering Model \cite{garcia2012multiple}. While some of these assumptions can be experimentally achieved, such as the size of the NPs and their low concentration in an ensamble \textcolor{red}{\textbf{Se puede citar el artículo con las muestras de francia?}}, a partial embedding of the NP into a substrate is not only a possible outcome of the fabrication process \cite{meng_anisotropic_2015} but sometimes it is a desirable feature of the bidimensional array of NPs for its use in applications, for example, in low-cost plasmonic biosensors \cite{moirangthem_enhanced_2012}, where the partial embedding diminishes the washability of the sample.

The theoretical approach employed by the Thin Island Theory, the Dipolar Model and the Coherent Scattering Model involves the optical properties of a single scatterer in a surrounding isotropic medium to later introduce the effect of both the multiple scattering due to the ensamble of NPs and the presence of the substrate. \textcolor{red}{Therefore, the optical properties of a single partially embedded nanospheres can be of use to extend the validity  of such models for more realistic conditions}. In the following sections, the absorption and scattering efficiencies, as well as the induced electric field in the far and near-field regime, are studied for a 12.5 nm AuNP partially embedded considering an air matrix and a glass substrate.
