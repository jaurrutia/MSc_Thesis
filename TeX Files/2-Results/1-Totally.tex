% !TeX root = ../tesis.tex

%
%
%
%\clearpage



\begin{figure}
    \hspace*{-18.75em}%
    \vspace*{-1.25em}%
        \begin{subfigure}{.375\textwidth}\caption{ }\label{sfig:red:1}\end{subfigure}%
        \begin{subfigure}{.25\textwidth}\caption{ }\label{sfig:red:2}\end{subfigure} \\
    \def\svgwidth{.95\textwidth}
    \small
    \centering
    \includeinkscape{1-Totally/1-Efficiencies/1-Normal-Eff}%
    \vspace*{0em}
    \caption[Spectral redshift of the scattering and extinction of a spherical AuNP as function of its size and the embedding media]{Resonance wavelength $\lambda_\text{res}$ of the scattering (orange) and extinction (black) cross sections of a AuNP as function of the NP's radius when embedded \textbf{a)} into air ($n_\text{m} = 1)$ and \textbf{b)} into glass ($n_\text{m} = 1.5$), and as function of the refractive index of the matrix  $n_\text{m}$ for a AuNP of radius equal to  \textbf{c)} 12.5 nm and \textbf{d)} 50 nm, employing the dielectric function for the Au as reported by \citeauthor{johnson_optical_1972} (filled circle) and considering a size correction to it (empty circle).}
\end{figure}


\begin{figure}
    \hspace*{-18.75em}%
    \vspace*{-1.25em}%
        \begin{subfigure}{.375\textwidth}\caption{ }\label{sfig:red:1}\end{subfigure}%
        \begin{subfigure}{.25\textwidth}\caption{ }\label{sfig:red:2}\end{subfigure} \\
    \def\svgwidth{.9\textwidth}
    \centering
    \includeinkscape[pretex = \small]{1-Totally/4-5-Far-XY-Embedded/4-5-Far-XY-Embedded}%
    \vspace*{0em}
    \caption[Spectral redshift of the scattering and extinction of a spherical AuNP as function of its size and the embedding media]{Resonance wavelength $\lambda_\text{res}$ of the scattering (orange) and extinction (black) cross sections of a AuNP as function of the NP's radius when embedded \textbf{a)} into air ($n_\text{m} = 1)$ and \textbf{b)} into glass ($n_\text{m} = 1.5$), and as function of the refractive index of the matrix  $n_\text{m}$ for a AuNP of radius equal to  \textbf{c)} 12.5 nm and \textbf{d)} 50 nm, employing the dielectric function for the Au as reported by \citeauthor{johnson_optical_1972} (filled circle) and considering a size correction to it (empty circle).}
\end{figure}


\begin{figure}
    \hspace*{-18.75em}%
    \vspace*{-1.25em}%
        \begin{subfigure}{.375\textwidth}\caption{ }\label{sfig:red:1}\end{subfigure}%
        \begin{subfigure}{.25\textwidth}\caption{ }\label{sfig:red:2}\end{subfigure} \\
    \def\svgwidth{.9\textwidth}
    \centering
    \includeinkscape[pretex = \small]{1-Totally/4-5-Far-XY-Supported/4-5-Far-XY-Supported}%
    \vspace*{0em}
    \caption[Spectral redshift of the scattering and extinction of a spherical AuNP as function of its size and the embedding media]{Resonance wavelength $\lambda_\text{res}$ of the scattering (orange) and extinction (black) cross sections of a AuNP as function of the NP's radius when embedded \textbf{a)} into air ($n_\text{m} = 1)$ and \textbf{b)} into glass ($n_\text{m} = 1.5$), and as function of the refractive index of the matrix  $n_\text{m}$ for a AuNP of radius equal to  \textbf{c)} 12.5 nm and \textbf{d)} 50 nm, employing the dielectric function for the Au as reported by \citeauthor{johnson_optical_1972} (filled circle) and considering a size correction to it (empty circle).}
\end{figure}
