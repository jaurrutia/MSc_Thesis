% !TeX root = ../tesis.tex

The Localized Surface Plasmon Resonance (LSPR) occurs when the scattered electric field of a particle illuminated by a plane wave is described by an stationary wave on the surface of a particle when there is a coupling between the particle acting as an scatterer and the incident plane wave illuminating it. For a spherical particle, the LSPR occurs when the conditions in Eqs. \eqref{eq:E-LSPR} and \eqref{eq:M-LSPR} are met, which maximize the Mie coefficients [Eq. \eqref{eq:MieCoef}] $a_\ell$ and $b_\ell$, respectively. Since the optical quantities such as the scattering and extinction cross sections can be derived from the amplitude scattering matrix $\mathbb{F}$, Eq. \eqref{eq:FscaS}, which depends on the Mie coefficients  $a_\ell$ and $b_\ell$, the LSPR can also be observed in this spacial regime, where the only non neglectable contribution of the scattered electric field is proportional to $r^{-1}$, where $r$ is the distance from the center of the scatterer to the evaluation point. Hereby, the explicit expressions for the optical quantities in the far field regime, for a scattering sphere, are obtained.

The optical properties of  a spherical particle in the far field regime are  the scattering $C_\text{sca}$ and extinction $C_\text{ext}$ cross sections, which are obtained by substituting the amplitude scattering matrix for a spherical particle [Eq. \eqref{eq:FscaS}]  into Eqs. \eqref{eq:Csca} and \eqref{eq:Cext}, respectively; the absorption cross $C_\text{abs}$ section can be calculated by subtraction of the past two. Thus, assuming an incident plane wave with an $x$ polarized electric field $\vb{E}^\text{i}$, and evaluating the scattering amplitude matrix in the forward direction $\theta = 0$, equivalent to $\cos\theta = 1$ , the extinction cross section $C_\text{ext}$ is given by
%
\begin{align}
	C_\text{ext} = \frac{4\pi}{k \norm{\vb{E}^\text{i}}^2}\Im[\frac{i}{k}S_2(\theta = 0)\vb{E}^\text{i}\cdot\vb{E}^\text{i*}]
	  			 = \frac{2 \pi}{k^2}\sum_{\ell = 1}^\infty (2 \ell + 1) \Re(a_\ell + b_\ell),
	\label{eq:CextSphere}
\end{align}
where the Eq. \eqref{eq:PiTau1} in Appendix \ref{app:MieCode} was employed to tevaluate the angular functions $\pi_\ell(\cos\theta)$ and $\tau_\ell(\cos\theta)$ . In a similar manner, the scattering cross section $C_\text{sca}$  can be written as
%
\begin{align}
C_\text{sca} = \int_0^{2\pi}\int_0^\pi  \frac{(iS_2(\theta)\vb{E}^\text{i})^*(iS_2(\theta)\vb{E}^\text{i})}{k^2\vb{E}^\text{i}} \sin\theta\dd{\varphi}\dd{\theta}
			 = \frac{2 \pi}{k^2}\sum_{\ell = 1}^\infty (2 \ell + 1) (\abs{a_\ell}^2 + \abs{b_\ell}^2).
	\label{eq:CscaSphere}
\end{align}
%
where the orthogonality relations of $\pi_\ell(\cos\theta)\pm\tau_\ell(\cos\theta)$ [Eq. \eqref{eq:(pipmtau)} in Appendix \ref{app:MieCode}] were used. In order to compare the absorption, scattering or extinction of light of a spherical particle, independently of its radius $a$ or embedding media (matrix), the efficiencies of  absorption $Q_\text{abs}$, scattering $Q_\text{sca}$ and extinction $Q_\text{ext}$ are defined by normalizing the  absorption, scattering and extinction cross sections by the geometrical cross section of the spherical particle yielding the dimensionless expressions
%
\begin{align}
 	\frac{C_\text{ext}}{\pi a^2} =   \frac{C_\text{abs}}{\pi a^2}  + \frac{C_\text{sca}}{\pi a^2}
 		\qquad \longrightarrow \qquad
	Q_\text{ext} =    Q_\text{abs}  +  Q_\text{sca}.
	\label{eq:Efficiencies}
\end{align}
%

The Eq. \eqref{eq:Efficiencies}, along with optical theorem [Eq. \eqref{eq:CextFull}] states that the extinction of light considers  the combination of both absorption and scattering mechanism. Since the analytical expression of $C_\text{eq}$ for a spherical particle [Eq. \eqref{eq:CextSphere} is proportional to the real parts of the Mie coefficients $a_\ell$ and $b_\ell$, then it is also maximized at the LSPR. Therefore, the LSPR can be observed in the far field by calculating or measuring the extinction cross section.


In order to study the LSPR of a gold nanoparticle (AuNP) of radius $a = 12.5$ nm (12.5 nm AuNP), the extinction  $Q_\text{ext}$ and the scattering  $Q_\text{sca}$ efficiencies are shown in Fig. \ref{fig:Mieefficiencies} as function of the wavelength $\lambda$ of the incident plane wave illuminating the NP. Two different matrices were considered in Fig. \ref{fig:Mieefficiencies}: a matrix of air with a refractive index of $n_\text{mat} = 1$ (black lines) and a matrix of glass with $n_\text{mat} = 1.5$ (orange lines). The optical response of the AuNP was modeled by a dielectric function for the AuNP with (dashed lines) and without (solid lines) a size correction (see Appendix \ref{app:SizeCorrection}) to the experimental data reported by  \citeauthor{johnson_optical_1972} \cite{johnson_optical_1972} and it effect were compared for both considered matrices. On each curve of the scattering and extinction efficiencies, the wavelength of their maximum value is indicated and the value of $\lambda$ at which the $Q_\text{ext}$ is maximum corresponds to the excitation wavelength of the dipolar LSPR ($\ell = 1$).

% -------------------------------------- Extinctintion and scattering efficiencies -------------------------------
% --------------------------------------          12.5 nm AuNP @ Air and @ glass   -------------------------------
% --------------------------------------               fig:Mieefficiencies         -------------------------------
\begin{figure}[h!]
	\def\svgwidth{1\textwidth} \small
  \vspace*{3.0em}
  \hspace*{-9.em}
    \begin{subfigure}{.46\textwidth}\caption{ }\label{fig:Mieefficiencies:a}\end{subfigure}
    \begin{subfigure}{.49\textwidth}\caption{ }\label{fig:Mieefficiencies:b}\end{subfigure}
  \vspace*{-6.em}\\
  \includeinkscape{Mie-Au/1-Efficiencies}
  \vspace*{-2em}
  \caption[Extinction and Scattering Efficency of a 12.5 nm Au Spherical NP embeded into Air and Glass]{ \textbf{a)} Extinction $Q_\text{ext}$ and \textbf{b)} scattering $Q_\text{sca}$ efficiencies of a 12.5 nm Au spherical NP embeded into air (black, $n_\text{mat} = 1$)  and into galss (orange, $n_\text{mat} = 1.33$), as function of the wavelength $\lambda$ of the incident plane wave.  The solid curves were calculated by considering no size effects on the dielectric function of the AuNP, while the dashed curves consider a size correction to it; the experimental data of \citeauthor{johnson_optical_1972} \cite{johnson_optical_1972} was employed.}
\label{fig:Mieefficiencies}
\end{figure}
% --------------------------------------               fig:Mieefficiencies         -------------------------------

From the extinction and scattering efficiencies in Fig. \ref{fig:Mieefficiencies}, two main spectral tendencies arises between these quantities: on the overall value of the efficiencies and on the spectral position of their maximum. Since the scattering efficiency is two orders of magnitude smaller than the extinction efficiency within the visible range, the main energy loss mechanism is absorption, as stated by the Optical Theorem [Eq. \eqref{eq:Cext}]. Even though the 12.5 nm AuNP absorbs more light than what it scatters, both phenomena are present. For example, when the  12.5 nm AuNP is embedded into air, the wavelengths at which it absorbs and scatters the most are $\sim 509$ nm and $\sim 522$ nm, respectively. When the AuNP is embedded into glass, the absorption of light is optimized at a wavelength of $\sim 535$ nm, while the scattering is optimized at $\sim 543$ nm. For both an air and a glass matrix, the wavelength of most scattering is redshifted relative to the wavelength of most absorption for each case: $\sim 15$ nm for air and $\sim 7$ nm for glass. The extinction and scattering have different spectral tendencies as already discussed, yet they share common characteristics   when the dependency on the embedding media is studied.

Both the scattering and the absorption efficiencies of a 12.5 nm AuNP present an overall enhancement within the visible range when the refractive index of the matrix, the embedding media,  increases. This can be seen in Fig. \ref{fig:Mieefficiencies} by comparing the black curves ($n_\text{mat} = 1$, air) with the orange curves ($n_\text{mat} = 1.5$, glass). The value of the extinction efficiency of a 12.5 nm AuNP  at the excitation wavelength of the LSPR for a matrix of glass is around five times larger than for a matrix of air, as it can bee seen in Fig. \ref{fig:Mieefficiencies:a}, while the maximum value of the scattering efficiency for a glass matrix is around eight times larger compared to the case of an air matrix, as shown in Fig. \ref{fig:Mieefficiencies:b}. The enhancement of the extinction and the scattering efficiency as function of the refractive index of the matrix can be understood by analyzing the size paramter $ x = (2 \pi / \lambda ) a n_\text{mat} $, which compares the size of the NP relative to the incident wavelength in the matrix: the larger the value of $x$, the bigger a NP can be considered inside such matrix. Since the size parameter is a linear function of $n_\text{mat}$, the AuNP embedded into glass optically responds like a larger NP than what it is in air, thus having a more significant contributiom from the scattering to the light extinction mechanism inside glass, as well as an increse in the absorption.

The extinction [Fig. \ref{fig:Mieefficiencies:a}] and the scattering [Fig. \ref{fig:Mieefficiencies:b}] efficiencies of the same system, a 12.5 nm AuNP embedded into a matrix of either air or glass, are different whether a size correction to the dielectric function of the AuNP is considered (dashed lines) or not (solid lines). On the one hand, there is a spectral shift of the LSPR excitation wavelength for both matrices of $\sim 2$ nm: the AuNP embedded into air redifts from $\sim 507$ nm (no size correction) to $\sim 509$ nm (size correction) and the AuNP embedded into glass blueshifts from $\sim 506$ nm to $\sim 535$ nm. The maximum value of $Q_\text{sca}$ redifts $\sim 1$ nm when considering the size correction of the dielectric function when the AuNP is embeded into air nevertheless, the wavelength of maximum scattering does not present neither a red nor a blueshift larger than $1$ nm. On the other hand, the value of the efficiencies around the wavelength where the absorption and the scattering is maximized decreases in all cases shown if Fig. \ref{fig:Mieefficiencies}.  This behavior can be explained by how the size correction is performed: as explained in Appendix \ref{app:SizeCorrection}, the surface effects are taken into account by introuducing a smaller mean free path for the free electrons inside the AuNP, therefore increasing the value of the damping constant and thus leading to a larger imaginary part for the dielectric functiones employed, which is related to the absorption mechanisms \cite{ibach_solid-state_2009}. The descrease in the efficiencies due to a size corrected dielectric function is more evident for a matrix out of glass than out of air, which is an effect explained by the size parameter $x$ since a more light absorbing dielectric function is employed into a AuNP which has the opical properties of a larger NP when it is embedded into glass than into air. From this analysis it can be concluded that the most notable effect of a size correction to the dielectric function of a NP is the decrease in the extinction and scattering efficiencies, while there is still a spectral shift of the LSPR, whose effect is less relevant the larger the size parameter is.

While the scattering efficiency $Q_\text{sca}$ is an integral quantity, that is, it describes the scattering in all directions of a plane wave traveling in the  direction $\vb{k}^\text{i}$ due to the interaction with a NP, the scattering amplitude matrix elements $S_1(\theta)$,  given  for a spherical NP by Eq. \eqref{eq:S1}, and  $S_2(\theta)$, by Eq. \eqref{eq:S2}, depict the  electric field $\vb{E}^\text{s}$, at a measurement angle $\theta$,  scattered by a NP polarized in a direction perpendicular ($\perp$) to the scattering plane and parallel ($\parallel$) to it, respectively. A radiation pattern helps to vizualize the behavior of $S_{1,2}(\theta)$, a dimensionless parameter such as the scattering efficiency, by plotting their squared modulus as function of $\theta$, as it is shown in for a 12.5 nm AuNP in Fig. \ref{fig:ScatteringMaps}, where $\abs{S_{1}(\theta)}^2$ (solid lines) and $\abs{S_2(\theta)}^2$ (dashed lines) are shown for two different scenarios: a AuNP embedded into air [Fig. \ref{fig:ScatteringMaps:a}, black curves]  illuminated  at a wavelength $\lambda = 522$ nm and a AuNP embedded into glass  [Fig. \ref{fig:ScatteringMaps:b}, orange curves]  illuminated at $\lambda = 543$ nm. The wavelengths of the incident plane wave corresponds to the value of $\lambda$ where $Q_\text{sca}$ is maximized for each matrix as seen in Fig. \ref{fig:Mieefficiencies}.

% --------------------------------------				 Scaattering Maps 	   ------------------------------
% --------------------------------------   12.5 nm AuNP @ Air and @ glass    ------------------------------
% --------------------------------------         fig:ScatteringMaps         -------------------------------
\begin{figure}[t!]
	\def\svgwidth{.9\textwidth} \small\centering
		\vspace*{3.25em}
		\hspace*{-.25\textwidth}
	\begin{subfigure}{.25\textwidth}%
		\caption{$\abs{S_{1,2}(\theta)}^2\times 10^4$} \label{fig:ScatteringMaps:a}%
		\end{subfigure}%
	\begin{subfigure}{.45\textwidth}%
		\caption{$\abs{S_{1,2}(\theta)}^2\times 10^3$}\label{fig:ScatteringMaps:b}%
		\end{subfigure}%
	\vspace*{-6.25em}\\
	\includeinkscape{Mie-Au/2-ScatteringMaps}
	\vspace*{-.5em}
	\caption[Radiation Pattern of a 12.5 nm Au Spherical NP embeded into Air and Glass]{Radiation pattern of a 12.5 nm Au spherical NP embeded into \textbf{a)} air illuminatated by a plane wave at a wavelength of $\lambda = 522$ nm and \textbf{b)} glass illuminated at $\lambda = 543$ nm; the wavlength in each case corresponds to the wavelength of maximum scattering (see Fig. \ref{fig:Mieefficiencies}). The solid (dashed) lines corresponds to the scattering matrix element $S_1$ ($S_2$) related to an incident electric field $\vb{E}^\text{i}$ travelling in the $\vb{k}^\text{i}$ direction  and polarized perpendicularly (prallel) to the page. It was considered for both matrices a size corretion to the experimental data of \citeauthor{johnson_optical_1972} \cite{johnson_optical_1972} for the electromagnetic response of the AuNP.}
	\label{fig:ScatteringMaps}
 \end{figure}
 % --------------------------------------         fig:ScatteringMaps         -------------------------------

The quantities $\abs{S_{1,2}(\theta)}^2$ for a 12.5 nm AuNP embedded into air ($n_\text{mat} = 1$, black curves) are one order of magnitude smaller  than into glass ($n_\text{mat} = 1.5$, orange curves), meaning that the AuNP scatters light less efficiently in air than in glass, which is consistent with the obtained values for the scattering efficiency $Q_\text{sca}$ in Fig. \ref{fig:Mieefficiencies}. On the angular dependency, the radiation pattern of the AuNP in both matrices follow the same tendencies: an homogeneous scattered electric field when the AuNP scatters the perpendicularly polarized incident electric field  $\vb{E}^\text{s}_\perp$ (continuous lines), and a two-lobes pattern when illuminated with a parallel polarized $\vb{E}^\text{i}$. The observed radiation pattern can be identify in the near field regime, see Fig. \ref{fig:NearField}, nevertheless within the radiation pattern analysis, the presence of the LSPR is lost, unlike within an analysis of the extinction cross section.
