% !TeX root = ../tesis.tex

\textbf{Falta emplear las relacinoes de ortogonalidad para calcular las secciones de extincion y edmás}
\clearpage


In Fig. \ref{fig:Mieefficiencies} the extinction  $Q_\text{ext}$ and the scattering  $Q_\text{sca}$ efficiencies of a gold NP (AuNP) with a radius of $a = 12.5$ nm are shown as function of the wavelength $\lambda$ of the incident planewave illuminating the NP. Two matrices were considered in Fig. \ref{fig:Mieefficiencies}: a matrix of air with a refractive index of $n_\text{mat} = 1$ (black lines) and a matrix of glass with $n_\text{mat} = 1.5$ (orange lines).  The experimental dielectric function for Au reported by \citeauthor{johnson_optical_1972} \cite{johnson_optical_1972}  was employed to model the electromagnetic response of the spherical AuNP nevertheless, this data corresponds to a bulk sample, meaning that it may not reproduce the optical behavior of a NP since surface effects cease to be neglectable  due to their spacial dimensions \cite{noguez_surface_2007}.   In order to study  the optical properties of AuNP while considering  surface effects,  a size correction to the dielectric function of the AuNP was performed as described in Appendix \ref{app:SizeCorrection}. The efficiencies of the $12.5$ nm AuNP  taking into account a dielectric function with (dashed lines) and without (solid lines) a size correction were compared for both considered matrices; on each curve the wavelength of their maximum value is indicated.

% -------------------------------------- Extinctintion and scattering efficiencies -------------------------------
% --------------------------------------          12.5 nm AuNP @ Air and @ glass   -------------------------------
% --------------------------------------               fig:Mieefficiencies         -------------------------------
\begin{figure}[h!]
	\def\svgwidth{1\textwidth} \small
  \vspace*{3.25em}
  \hspace*{-10.75em}
    \begin{subfigure}{.49\textwidth}\caption{ }\label{fig:Mieefficiencies:a}\end{subfigure}
    \begin{subfigure}{.49\textwidth}\caption{ }\label{fig:Mieefficiencies:b}\end{subfigure}
  \vspace*{-6.25em}\\
  \includeinkscape{1-Theory-Figs/Mie-Au/1-Efficiencies}
  \vspace*{-2em}
  \caption[Extinction and Scattering Corss Section of a 12.5 nm Au spherical NP embeded into a vacuum- and into a waterlike environment]{ \textbf{a)} Extinction $Q_\text{ext}$ and \textbf{b)} Scattering $Q_\text{sca}$ a Efficiencies of a 12.5 nm Au spherical NP embeded into a vacuum-like matrix (black, $n_\text{mat} = 1$)  and into a water-like matrix (orange, $n_\text{mat} = 1.33$), as function of  the wavelength $\lambda$ of the incident plane wave.  The solid curves were calculated by considering no size effects on the dielectric function of the AuNP, while the dashed curves considers a size correction to it; the experimental data of \citeauthor{johnson_optical_1972} \cite{johnson_optical_1972} was employed.}
\label{fig:Mieefficiencies}
\end{figure}
% --------------------------------------               fig:Mieefficiencies         -------------------------------

From the extinction and scattering efficiencies in Fig. \ref{fig:Mieefficiencies}, two main spectral tendencies arises between these quantities: on the overall value of the efficiencies and on the spectral position of their maximum. Since the scattering efficiency is two orders of magnitude smaller than the extinction efficiency within the visible range, the main energy loss mechanism is absorption, as stated by the Optical Theorem [Eq. \eqref{eq:Cext}]. Even though the 12.5 nm AuNP absorbs more light than what it scatters, both phenomena are present. For example, when the  studied AuNP is embedded in air, the wavelengths at which it absorbs and scatters the most are $\sim 509$ nm and $\sim 522$ nm, respectively. When the AuNP is embedded into glass, the absorption of light is optimized at a wavelength of $\sim 535$ nm, while the scattering is optimized at $\sim 543$ nm. For both an air and a glass matrix, the wavelength of most scattering is redshifted relative to the wavelength of most absorption: $\sin 15$ nm for air and $\sim 7$ nm for glass. The extinction and scattering have different spectral tendencies as already discussed, yet they share common characteristics   when the dependency on the embedding media is studied.

Both the scattering and the absorption efficiencies of a 12.5 nm AuNP present an overall enhancement within the visible range when the refractive index of the matrix, the embedding media,  increases. This can be seen in Fig. \ref{fig:Mieefficiencies} by comparing the black curves ($n_\text{mat} = 1$, air) with the orange curves ($n_\text{mat} = 1.5$, glass). The maximum value of the extinction efficiency of a 12.5 nm AuNP for a matrix of glass is around five times bigger than for a matrix of air, as it can bee seen in Fig. \ref{fig:Mieefficiencies:a}, while te maximum value of the scattering efficiency for a glass matrix is around eigth times bigger compared to the case of an air matrix, as shown in Fig. \ref{fig:Mieefficiencies:b}. The enhancement of the scattering efficiency as function of the refractive index of the matrix can be understood by analyzing the size paramter $ x = (2 pi / \lambda ) a n_\text{mat} $, which compares the size of the NP relative to the incident wavelength in the matrix: the larger the value of $x$, the bigger a NP can be considered inside such matrix. Since the size paramter is a linear function of $n_\text{mat}$, the AuNP embedded into glass will behave more like a bigger NP, thus having a more significant contributiom from the scattering to the  light extinction mechanism inside  glass than inside air.

The last effect that can be analyzed from the extinction and scattering efficiencies in Fig. \fig{fig:Mieefficiencies} is their dependency on the size correction to the dielectric function.

The extinction [Fig. \ref{fig:Mieefficiencies:a}] and the scattering [Fig. \ref{fig:Mieefficiencies:b}] efficiencies of the same system, a 12.5 nm AuNP embedded into a matrix either of air or glass, are different


 considering no size correction (solid lines) and a size correction (dashed lines) to the dielectric function









\begin{figure}[h!]
	\def\svgwidth{1\textwidth} \small
\vspace*{3.25em}
\hspace*{-6.5em}
\begin{subfigure}{.49\textwidth}\caption{$\norm{S_{1,2}(\theta)}^2\times 10^5$}\label{fig:Mieefficiencies:a}\end{subfigure}
\begin{subfigure}{.49\textwidth}\caption{$\norm{S_{1,2}(\theta)}^2\times 10^4$}\label{fig:Mieefficiencies:b}\end{subfigure}
\vspace*{-6.25em}\\
\includeinkscape{1-Theory-Figs/Mie-Au/2-ScatteringMaps}
\vspace*{-2em}
\caption[Extinction and Scattering Corss Section of a 12.5 nm Au spherical NP embeded into a vacuum- and into a waterlike environment]{ \textbf{a)} Extinction $Q_\text{ext}$ and \textbf{b)} Scattering $Q_\text{sca}$ a Efficiencies of a 12.5 nm Au spherical NP embeded into a vacuum-like matrix (black, $n_\text{mat} = 1$)  and into a water-like matrix (orange, $n_\text{mat} = 1.33$), as function of  the wavelength $\lambda$ of the incident plane wave.  The solid curves were calculated by considering no size effects on the dielectric function of the AuNP, while the dashed curves considers a size correction to it; the experimental data of \citeauthor{johnson_optical_1972} \cite{johnson_optical_1972} was employed.}
 \end{figure}
