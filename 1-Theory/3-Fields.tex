% !TeX root = ../tesis.tex

 Let $\vb{E}_{0,x}$ be a plane wave traveling in the vertical direction $\vb{e}_z$; its representation in the canonical spherical basis is
 %
 \begin{align}
 \vb{E}_{0,x} (\vb{r})= E_0\qty(\sin\theta\cos\varphi \vu{e}_r +
					\cos\theta\cos\varphi \vu{e}_\theta +
					- \sin\varphi \vu{e}_\varphi) \exp(ikr\cos\theta).
	\label{eq:ExPlane}
 \end{align}
 %
The monochromatic plane wave is a transversal wave, thus it can be written in terms of only the VSH $\vb{M}^{(1)}$ and $\vb{N}^{(1)}$, where the radial dependency is given by $j_\ell$ since the monochromatic plane wave is finite everywhere. Even more, due to the dependency on $\varphi$, it is only restricted to values of $m = 1$. By inspection on the radial component of $\vb{E}_{0,x}$, proportional to $\cos\varphi$ it depends only on $\vb{N}_{e1\ell}^{(1)}$, and on the azimuthal component, proportional to $\sin\varphi$, it can depend on $\vb{M}_{o1\ell}^{(1)}$. Thus, Eq. \eqref{eq:ExPlane} can be written as the linear combination
 %
 \begin{align}
 \vb{E}_{0,x} (\vb{r})= \sum_\ell\qty(a_\ell \vb{M}_{o1\ell}^{(1)} + b_\ell \vb{M}_{e1\ell}^{(1)}).
	\label{eq:ExPlane}
 \end{align}
 %




























































































