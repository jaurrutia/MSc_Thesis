% !TeX root = ../tesis.tex

Let $\vb{E}^\text{i} = \vb{E}^\text{i}_0 \exp(i\vb{k}^\text{i}\cdot\vb{r})$ be the electric field of an incident monochromatic plane wave with constant amplitude $\vb{E}_0^\text{i}$  traveling through a non absorbing medium with refractive index $n_\text{mat}$, denominated as the matrix, in the direction $\vb{k}^\text{i} = k\vu{k}^\text{i}$, with $k = (\omega/c)n_\text{mat}$ the wave number of the plane wave \textcolor{red}{in} the matrix, and let $\vb{E}^\text{sca}$ be the scattered electric field due to a particle with arbitrary shape embedded in the matrix. In general, the scattered electric field propagates in all directions but for an observation point $\vb{r} = r\vu{e}_r$, the traveling direction is defined by the vector $\vb{k}^\text{sca} = k\vu{k}^\text{sca} = k\vu{e}_r$.  Due to the linearity of the Maxwell's equations,   the incident and scattered electric fields  in the far-field regime are related by the following linear relation \cite{tsang_scattering_2000}:
% --------------------------------- index entries ----------------------------------
\index{Plane!Wave}%
\index{Wave!Plane}%
% ------------------------------------------------------------------------------------
% ---------------------------------- eq: ScatAmpMat ----------------------------------
 \begin{equation}
	\vb{E}^\text{sca} =   \frac{\exp(i\vb{k}^\text{sca}\cdot\vb{r})}{r} \mathbb{F}(\vu{k}^\text{sca}, \vu{k}^\text{i}) \vb{E}^\text{i},
 \label{eq:ScatAmpMat}
 \end{equation}
% ---------------------------------- eq: ScatAmpMat ----------------------------------
%
where $\mathbb{F}(\vu{k}^\text{sca}, \vu{k}^\text{i})$ is the scattering  amplitude matrix from direction $\vu{k}^\text{i}$ into $\vu{k}^\text{sca}$. Since only the far-field is considered, both the incident and the scattered electric fields can be decomposed into two linearly independent components perpendicular to $\vb{k}^\text{i}$ and $\vb{k}^\text{sca}$, respectively, each forming a right-handed orthonormal system. If the particle acting as a scatterer has a symmetric shape, it is convenient to define an orthonormal system relative to the scattering plane, which is the plane containing $\vb{k}^\text{i}$ and $\vb{k}^\text{sca}$, since the elements of $\mathbb{F}(\vu{k}^\text{sca}, \vu{k}^\text{i})$ are simplified when represented in these bases \cite{tsang_scattering_2000}. In Fig. \ref{fig:ScatPlane} a plane wave traveling in the $z$ direction illuminates an arbitrary particle centered at the origin of the coordinate system and the scattering plane is depicted in green. By defining the directions perpendicular  ($\perp$) and parallel ($\parallel$) to the scattering plane,  the incident and scattered electric fields can be written as
% --------------------------------- index entries ----------------------------------
\index{Scattering!Amplitude Matrix@{Amplitude Matrix $\mathbb{F}$}}%
\index{Scattering!Plane!Unit Vector System}%
\index{Plane!Scattering!Coordinate System}%
\index{Scattering!Plane!Coordinate System}%
 \index{Coordinate System!Relative to Scattering Plane}%
% ------------------------------------------------------------------------------------
% ---------------------------------- eq:Ei // eq:Es ------------------------------
 \begin{align}
	\vb{E}^\text{i} & =  \qty(E_\parallel^\text{i}\vu{e}^\text{i}_\parallel + E_\perp^\text{i} \vu{e}_\perp^\text{i}) \exp(i\vb{k}^\text{i}\cdot\vb{r}),
 \label{eq:Ei} \\
	\vb{E}^\text{sca} & = \qty(E_\parallel^\text{sca}\vu{e}^\text{sca}_\parallel + E_\perp^\text{sca} \vu{e}_\perp^\text{sca}) \frac{\exp(i\vb{k}^\text{sca}\cdot\vb{r})}{r},
 \label{eq:Es}
 \end{align}
% ---------------------------------- eq:Ei // eq:Es ------------------------------
%
where a harmonic time dependence $\exp(-i\omega t)$ has been omitted, and it has been assumed that the scattered field is described by a spherical wave; the superscript `$\text{i}$' (`$\text{sca}$') denotes the orthonormal system defined by the incident plane wave (scattered fields).  Since $\{\vu{e}_\perp^\text{i}, \vu{e}_\parallel^\text{i},\vu{k}^\text{i} \}$ and $\{\vu{e}_\perp^\text{sca}, \vu{e}_\parallel^\text{sca},\vu{k}^\text{sca} \}$ ---shown in purple in Fig. \ref{fig:ScatPlane}  along with the Cartesian (blue) and spherical (black) unit vector bases--- are right-handed orthonormal systems, they are related as follows
%
% ---------------------------------- eq:eParaPerpPerp ------------------------------
 \begin{align}
	\vu{e}_\perp^\text{i} = \vu{e}_\perp^\text{sca}  & =  \vu{k}^\text{sca} \times \vu{k}^\text{i},
		\qquad
	\vu{e}^\text{i}_\parallel = \vu{k}^\text{i}\times \vu{e}^\text{i}_\perp,
		\qquad\text{and}\qquad
	\vu{e}^\text{sca}_\parallel = \vu{k}^\text{sca} \times \vu{e}_\perp^\text{sca}.
 \label{eq:eParaPerp}
 \end{align}
% ---------------------------------- eq:eParaPerpPerp ------------------------------
%
% -------------------------------------- Scattering plane vector system  -------------------------------
% -------------------------------------- Scattering plane vector system  -------------------------------
% --------------------------------------          fig:ScatPlane          -------------------------------
\begin{figure}[!bht]\centering
	\tdplotsetmaincoords{60}{110}
 \pgfmathsetmacro{\rvec}{1. 3}
 \pgfmathsetmacro{\thetavec}{30}
 \pgfmathsetmacro{\varphivec}{60}
\begin{tikzpicture}[scale=3.5,tdplot_main_coords]
 %draw the NP
 %\draw[tdplot_screen_coords,ball color=yellow, opacity = 1] (0,0,0) circle (.05);
 %\draw[tdplot_screen_coords, color=yellow, opacity = 1] (0,0,0) circle (.05);

 \pgfmathsetseed{3}
     \draw[tdplot_screen_coords, ball color=yellow, opacity = 1,scale =.075]
     plot [smooth cycle, samples=8,domain={1:8}]
         (\x*360/8+5*rnd:0.5cm+1cm*rnd) node at (0,0) {};
 \pgfmathsetseed{3}
     \draw[tdplot_screen_coords, color=yellow, opacity = 1,scale =.075]
      plot [smooth cycle, samples=8,domain={1:8}]
      (\x*360/8+5*rnd:0.5cm+1cm*rnd) node at (0,0) {};


 % Set up some coordinates
  \coordinate (O) at (0,0,0);

 %determine a coordinate (P) using (r,\theta,\varphi) coordinates.   This command
 %also determines (Pxy), (Pxz), and (Pyz): the xy-, xz-, and yz-projections
 %of the point (P).
 %syntax: \tdplotsetcoord{Coordinate name without parentheses}{r}{\theta}{\varphi}
 \tdplotsetcoord{P}{\rvec}{\thetavec}{\varphivec}

 %draw figure contents
 %--------------------
 %draw the main coordinate system axes
     \draw[thick,- latex] (0,0,0) -- (1. 5,0,0) node[anchor=north east]{$x$};
     \draw[thick,- latex] (0,0,0) -- (0,1. 5,0) node[anchor=north west]{$y$};
     \draw[thick,- latex] (0,0,0) -- (0,0,1. 5) node[anchor=south]{$z$};

 %draw the main cartesian vector system
     \draw[thick,- latex, blue] (0,0,0) -- (1,0,0) node[anchor= south east]{$\vu{e}_x$};
     \draw[thick,- latex, blue] (0,0,0) -- (0,1,0) node[anchor=north west]{$\vu{e}_y$};
     \draw[thick,- latex, blue] (0,0,0) -- (0,0,1) node[anchor= east]{$\vu{e}_z$};

 %draw a vector from origin to point (P)
     \draw[thick,color=green, - latex] (O) -- (P);
     \node at (1,. 5,1. 1) {\color{green} $\vb{r}$};

 %draw projection on xy plane, and a connecting line
     \draw[dashed, color=green] (O) -- (Pxy);
     \draw[dashed, color=green] (P) -- (Pxy);
     \fill[green, opacity = .3] (O) --(Pxy)-- (P)--(O);
     \draw[- latex, tdplot_screen_coords,green](.42,.2)--(.8,.2);
     \node[tdplot_screen_coords] at (1.2,.2) {\color{green}\small Scattering plane};


 %draw the angle \varphi, and label it
     %syntax: \tdplotdrawarc[coordinate frame, draw options]{center point}{r}{angle}{label options}{label}
     \tdplotdrawarc[- latex]{(O)}{0. 5}{0}{\varphivec}{anchor=south}{$\varphi$}


 %set the rotated coordinate system so the x'-y' plane lies within the
     %"theta plane" of the main coordinate system
     %syntax: \tdplotsetthetaplanecoords{\varphi}
     \tdplotsetthetaplanecoords{\varphivec}

 %draw theta arc and label, using rotated coordinate system
     \tdplotdrawarc[tdplot_rotated_coords, - latex]{(0,0,0)}{0. 45}{0}{\thetavec}{anchor=north}{$\theta$}

 %draw some dashed arcs, demonstrating direct arc drawing
     \draw[dashed,tdplot_rotated_coords] (\rvec,0,0) arc (0:90:\rvec);
     \draw[dashed] (\rvec,0,0) arc (0:90:\rvec);

 %set the rotated coordinate definition within display using a translation
 %coordinate and Euler angles in the "z(\alpha)y(\beta)z(\gamma)" euler rotation convention
 %syntax: \tdplotsetrotatedcoords{\alpha}{\beta}{\gamma}
     \tdplotsetrotatedcoords{\varphivec}{\thetavec}{0}

 %translate the rotated coordinate system
 %syntax: \tdplotsetrotatedcoordsorigin{point}
     \tdplotsetrotatedcoordsorigin{(P)}

 %use the tdplot_rotated_coords style to work in the rotated, translated coordinate frame
     \draw[thick,tdplot_rotated_coords,- latex, purple] (0,0,0) -- (. 3,0,0) node[anchor=north west]{{\color{black}$\vu{e}_\theta,$}$\vu{e}_{\parallel}^\text{sca}$};
     \draw[thick,tdplot_rotated_coords,- latex,black] (0,0,0) -- (0,. 3,0) node[anchor=west]{$\vu{e}_\varphi$};
     \draw[thick,tdplot_rotated_coords,- latex,purple] (0,0,0) -- (0,-. 3,0) node[anchor= north west]{$\vu{e}_{\perp}^\text{sca}$};
     \draw[thick,tdplot_rotated_coords,- latex] (0,0,0) -- (0,0,. 3) node[anchor=south]{$\vu{k}^\text{sca}, \vu{e}_r$ };

 %set the rotated coordinate definition within display using a translation
 %coordinate and Euler angles in the "z(\alpha)y(\beta)z(\gamma)" euler rotation convention
 %syntax: \tdplotsetrotatedcoords{\alpha}{\beta}{\gamma}
     \tdplotsetrotatedcoords{\varphivec}{0}{0}

 %translate the rotated coordinate system
 %syntax: \tdplotsetrotatedcoordsorigin{point}
     \tdplotsetrotatedcoordsorigin{(Pxy)}

     \draw[thick,tdplot_rotated_coords,- latex, purple] (0,0,0) -- (. 3,0,0) node[anchor= west]{$\vu{e}_{\parallel}^\text{i}$};
     \draw[thick,tdplot_rotated_coords,- latex, blue] (0,0,0) -- (0,0,. 3) node[anchor= west]{$\vu{e}_z$};
     \draw[thick,tdplot_rotated_coords,- latex, purple] (0,0,0) -- (0,-. 3,0) node[anchor= north west]{$\vu{e}_{\perp}^\text{i}$};

 % plane wave
     \foreach \i in {-7,...,-2}{
         \draw[thick,tdplot_screen_coords,red, - latex] (\i/10,0,0)--(\i/10,1,0);}
     \node[tdplot_screen_coords] at (-4.5/10,1.1,0){\color{red}$\vb{k}^\text{i}$};
     \node[tdplot_screen_coords] at (-4.5/10,-.15,0){\begin{minipage}{2.cm}\centering\small \color{red}Incident plane wave\end{minipage}};
\end{tikzpicture}%

    \caption[Scattering plane unit vector systems]{The scattering plane (green) is defined by the vector $\vu{k}^\text{i}$ (red) parallel to $\vu{e}_z$ ---the direction of the incident plane wave--- and the vector $\vu{k}^\text{sca}$ ---the direction of the scattered field in a given point $\vec{r}$---. The parallel and perpendicular components of the incident field relative to the scattering plane are $\vu{e}_\parallel^\text{i} = \cos\varphi\,\vu{e}_x +\sin\varphi\,\vu{e}_y$ and  $\vu{e}_\perp^\text{i} = -\vu{e}_\varphi$, while the components of the scattering field relative to the scattering plane are $\vu{e}_\parallel^\text{sca} = \vu{e}_\theta$, $\vu{e}_\perp^\text{sca} = - \vu{e}_\varphi$. The Cartesian unit vector basis is shown in blue, the spherical unit vector basis in black, while the basis of the orthonormal systems relative to the scattering plane are shown in purple. }
    \label{fig:ScatPlane}
\end{figure}
% --------------------------------------          fig:ScatPlane          -------------------------------
%
\noindent
As Eqs. \eqref{eq:eParaPerp} suggest, the unit vector bases of the orthonormal systems relative to the scattering plane depend on the scattering direction. For example, if the incident plane wave travels along the $z$ axis (Fig. \ref{fig:ScatPlane}), then $\vu{k}^\text{i} = \vu{e}_z$ and $\vu{k}^\text{sca} = \vu{e}_r$. Thus the unit vector bases of the systems relative to the scattering plane are   $\vu{e}_\parallel^\text{i} = \cos\varphi\, \vu{e}_x +\sin\varphi\, \vu{e}_y$, $\vu{e}_\parallel^\text{sca} = \vu{e}_\theta$ and $\vu{e}_\perp^\text{i} = \vu{e}_\perp^\text{sca}  = - \vu{e}_\varphi$, with $\theta$ the polar angle and $\varphi$ the azimuthal angle.

When an incident plane wave interacts with a particle with a complex refractive index $n_\text{p}(\omega)$, the total electric field outside the particle is given by the sum of the incident and the scattered fields. Therefore, the time averaged Poynting vector $\ev{\vb{S}}_t$, denoting the power flow per unit area, of the total field is given by
% --------------------- index entries----------------------
\index{Vector!Time Averaged Poynting@{Averaged Poynting $\ev{\vb{S}}_t$}}%
% ------------------------------------  eq:Stot   ---------------------
\begin{align}
	\ev{\vb{S}}_t
		= \underbrace{\frac12 \Re \qty(\vb{E}^\text{i}\times\vb{H}^\text{i*})}_{\text{\normalsize $\ev{\vb{S}^\text{i}}_t $}} +
		  \underbrace{\frac12 \Re \qty(\vb{E}^\text{sca}\times\vb{H}^\text{sca*})}_{\text{\normalsize $\ev{\vb{S}^\text{sca}}_t $}}+
		   \underbrace{	\frac12 \Re\qty(\vb{E}^\text{i}\times\vb{H}^\text{sca*} + \vb{E}^\text{sca}\times\vb{H}^\text{i*})}_{\text{\normalsize$\ev{\vb{S}^\text{ext}}_t$}},
 \label{eq:Stot}
\end{align}
% ------------------------------------  eq:Stot   ---------------------
%
with $^*$  the complex conjugate operation and where the total Poynting vector is separated in three terms: the contribution from the incident field $\ev{\vb{S}^\text{i}}_t$, from the scattered field $\ev{\vb{S}^\text{sca}}_t$ and from their cross product denoted by $\ev{\vb{S}^\text{ext}}_t$. By means of the Faraday-Lenz's law and Eqs. \eqref{eq:ScatAmpMat}--\eqref{eq:Es}, the  contribution to the Poynting vector from the incident and the scattered fields can be rewritten as
% --------------------- index entries------------------------------------
\index{Faraday-Lenz's!Law}%
\index{Law!Faraday-Lenz's}
% ------------------ eq:AvePoyntingISca ---------------------------------
\begin{equation}
	\ev{\vb{S}^\text{i}}_t = \frac{\norm{\vb{E}_0^\text{i}}^2}{2 Z_\text{mat}}\vu{k}^\text{i},
		\qquad\text{and}\qquad
	\ev{\vb{S}^\text{sca}}_t = \frac{\norm{\vb{E}^\text{sca}}^2}{2 Z_\text{mat}}\vu{k}^\text{sca}
						=  \frac{\norm{\mathbb{F}(\vu{k}^\text{sca},\vu{k}^\text{i})\vb{E}^\text{i}}^2}{2 Z_\text{mat}r^2}\vu{k}^\text{sca},
 \label{eq:AvePoyntingISca}
\end{equation}
% ------------------ eq:AvePoyntingISca ---------------------------------
%
with $Z_\text{mat} = \sqrt{\mu_\text{mat}/\varepsilon_\text{mat}}$ the impedance of the non-absorbing matrix, while the crossed contribution is given by
%
% ------------------ eq:AvePoyntingExt ----------------------------------
 \begin{align}
 \ev{\vb{S}^\text{ext}}_t = &\Re\left\{
								\frac{\exp[-i(\vb{k}^\text{sca}-\vb{k}^\text{i})\cdot\vb{r}]}{2 Z_\text{mat}r^2}
								\qty[\vu{k}^\text{sca}\qty(\vb{E}_0^\text{i}\cdot \mathbb{F}^*\vb{E}^\text{i*})
									-\mathbb{F}^*\vb{E}^\text{i*}	\qty(\vb{E}^\text{i}_0\cdot\vu{k}^\text{sca})]
							 \right.\notag	\\
							&\hspace{2em}\left.
								+\frac{\exp[i(\vb{k}^\text{sca}-\vb{k}^\text{i})\cdot\vb{r}]}{2 Z_\text{mat}r^2}
								\qty[\vu{k}^\text{i}\qty(\mathbb{F}\vb{E}^\text{i}\cdot\vb{E}^\text{i*}_0)
									-\vb{E}^\text{i*}_0 \qty(\mathbb{F}\vb{E}^\text{i}\cdot\vu{k}^\text{i})]	\right\},
 \label{eq:AvePoyntingExt}
\end{align}%
% ------------------ eq:AvePoyntingExt ----------------------------------%
where the scattering amplitude matrix is evaluated as $\mathbb{F}(\vu{k}^\text{sca},\vu{k}^\text{i})$.

The power scattered by the particle can be calculated by integrating $\ev{\vb{S}^\text{sca}}_t$ in a closed surface surrounding the particle; if the scattered power is normalized by the irradiance of the incident field $\norm{\ev{\vb{S}^\text{i}}_t}$, it is obtained a quantity with units of area, known as the scattering cross section $C_\text{sca}$, given by \cite{bohren_absorption_1983}%
% -------------------- index entries ------------------------------
\index{Plane!Wave!Irradiance}%
\index{Cross Section!Scattering@{Scattering $C_\text{sca}$}}%
\index{Scattering!Cross Section}%
% ------------------- eq:Csca --------------------------------------
 \begin{tcolorbox}[title = Scattering Cross Section,	ams align, breakable]
	C_\text{sca} = \frac{2Z_\text{mat}}{\norm{\vb{E}_0}^2} \oint_\mathcal{S} \ev{\vb{S}^\text{sca}}_t \cdot\dd{\vb{a}}
				= \oint_\mathcal{S} \frac{\norm{\mathbb{F}(\vu{k}^\text{sca},\vu{k}^\text{i})\vb{E}^\text{i}}^2}
									{\norm{\vb{E}^\text{i}_0}^2}\dd{\Omega},
 \label{eq:Csca}
 \end{tcolorbox}
% ------------------- eq:Csca --------------------------------------
%
\noindent
where $\dd{\Omega}$ is the differential solid angle.

Similarly, an absorption cross section $C_\text{abs}$ can be defined as well. On the one side, the absorption cross section is given by the integral on a closed surface of \textcolor{red}{$\ev{-\vb{S}}_t$}  [Eq. \eqref{eq:Stot}] divided by the irradiance of the incident field, where the minus sign is chosen so that $C_\text{abs}>0$ if the particle absorbs energy  \cite{bohren_absorption_1983}. On the other side, if an Ohmic material with conductivity $\sigma(\omega) = i\omega n_\text{p}^2(\omega)$ \cite{jackson_classical_1999} for the particle is assumed, through Joule's heating law \cite{tsang_scattering_2000}, the absorption cross section can be computed as
% ------------------------------------index entries ----------------
\index{Joule!Heating Law}%
\index{Ohm!Law}%
\index{Law!Joule Heating}%
\index{Law!Ohm}%
\index{Cross Section!Absorption@{Scattering $C_\text{abs}$}}%
\index{Absorption!Cross Section}%
% ------------------- eq:Cabs --------------------------------------
 \begin{tcolorbox}[title = Ohmic Particle - Absorption Cross Section,	ams align, breakable]
 	C_\text{abs} =	 \frac12\int_\mathcal{V} \frac{\Re(\vb{J}\cdot \vb{E}^\text{int*})}
 									{\norm{\vb{E}_0^\text{i}}^2/2Z_\text{mat}}\dd{V}
				= \int_\mathcal{V} \omega Z_\text{mat}\Im(n_\text{p}^2) \frac{\norm{\vb{E}^\text{int}}^2}{\norm{\vb{E}^\text{i}_0}^2} \dd{V},
 \label{eq:Cabs}
 \end{tcolorbox}%
% ------------------- eq:Cabs --------------------------------------
%
\noindent
where the integration is performed inside the particle, and $\vb{J}$  and $\vb{E}^\text{int}$ are the volumetric electric current density and the total electric field in this region, respectively. Both the  scattering and the absorption cross sections are quantities related to the optical signature of a particle \cite{pellarin_forward_2019}, and their relation can be made explicit by performing the surface integral representation of $C_\text{abs}$ and defining $C_\text{ext}$, that is,
%
% ------------------- eq:CabsScaInt ----------
\begin{align}
C_\text{abs} = & - \frac{2Z_\text{mat}}{\norm{\vb{E}^\text{i}_0}^2}\int_\mathcal{S}
                        \Big(
                                \ev{\vb{S}^\text{i}}_t + \Bigl\langle\vb{S}^\text{sca}\Bigr\rangle_t + \ev{\vb{S}^\text{ext}}_t
                        \Big)\cdot\dd{\vb{a}}
					\notag \\
			=  & - C_\text{sca} - \frac{2Z_\text{mat}}{\norm{\vb{E}_0^\text{i}}^2}\int_\mathcal{S}
                        \ev{\vb{S}^\text{ext}}_t\cdot \vu{e}_r\dd{\Omega}
					\notag \\
			= & -C_\text{sca} + C_\text{ext},
\label{eq:CabsScaInt}
\end{align}
% -------------------- eq:CabsScaInt ----------
%
where the contribution of $\ev{\vb{S}^\text{i}}_t$ to the integral is zero since a non-absorbing matrix was assumed. From Eq. \eqref{eq:CabsScaInt} it can be seen that $C_\text{ext}$ takes into account both mechanisms for energy losses (scattering and absorption), thus it is called the extinction cross section. To solve the integral in Eq. \eqref{eq:CabsScaInt} let us define $\theta$ as the angle between $\vu{k}^\text{sca}$ and $\vu{k}^\text{i}$ as the polar angle  and  $\varphi$ as the azimuthal angle, as shown in Fig \ref{fig:ScatPlane}. With this choice of coordinates,  the extinction cross section can be computed as
% --------------------------- index entries ------
\index{Extinction!Cross Section}%
\index{Cross Section!Extinction}%
% ----------------------------- eq:CextFull ------
\begin{align}
C_\text{ext} = - &\Re \left\{
			 \frac{\exp(-ikr) }{\norm{\vb{E}_0^\text{i}}^2}
			 			\oint_\mathcal{S} \exp(ikr\cos\theta)\qty(\vb{E}^\text{i}\cdot \mathbb{F}^*\vb{E}^\text{i*})  \dd{\Omega} \right.	\notag\\
			&\hspace*{1.5em}+\frac{\exp(ikr) }{\norm{\vb{E}_0^\text{i}}^2}
						\oint_\mathcal{S} \exp(-ikr\cos\theta)\cos\theta \qty(\vb{E}^\text{i*}\cdot \mathbb{F}\vb{E}^\text{i})     \dd{\Omega}
\label{eq:CextFull}\\
			&\hspace*{1.5em}+\left.\frac{\exp(ikr) }{\norm{\vb{E}^\text{i}_0}^2}
						\oint_\mathcal{S} \exp(-ikr\cos\theta)\sin\theta(E_{0,x}^\text{i}\cos\varphi+E_{0,y}^\text{i}\sin\varphi)
									\qty(\mathbb{F}\vb{E}^\text{i}\cdot\vb{k}^\text{i})    \dd{\Omega}  \right\}, \notag
\end{align}
% ------------------------------ eq:CextFull -------
%
using that $\vu{k}^\text{sca}\cdot\vu{e}_r = 1$, $\vu{k}^\text{i}\cdot\vu{e}_r = \cos\theta$ and  $\vb{E}^\text{sca}\cdot\vu{e}_r = 0$. The integrals in Eq. \eqref{eq:CextFull} can be solved by a twofold integration by parts in the polar angle $\theta$ and by neglecting terms proportional to $r^{-2}$. This process leads to a zero contribution from the integrand proportional to $\sin\theta$  in Eq. \eqref{eq:CextFull} and, after rearranging the other terms in their real and imaginary parts, it follows that $C_\text{ext}$ depends only on the forward direction  $\vu{k}^\text{sca} = \vu{k}^\text{i}$ ($\theta =0$). This result is known as the Optical Theorem  whose mathematical expression is given by \cite{tsang_scattering_2000,pellarin_forward_2019,newton_optical_1976}:
% ------------------------index entries----
\index{Optical!Theorem}%
\index{Theorem!Optical}%
% --------------------------eq:Cext  ---
\begin{tcolorbox}[title = Optical Theorem - Extinction Cross Section,	ams align, breakable]
		C_\text{ext} = C_\text{abs} + C_\text{sca}
					=  \frac{4\pi}{k \norm{\vb{E}_0^\text{i}}^2}&\Im\qty[ \vb{E}_0^\text{i}\cdot \mathbb{F}^*(\vu{k}^\text{i},\vu{k}^\text{i}) \vb{E}^\text{i*} ].
\label{eq:Cext}
\end{tcolorbox}
% ------------------------ eq:Cext ------
%
\noindent The Optical Theorem is a general result applicable to general scattering phenomena, both quantum and classical  \cite{bohren_absorption_1983,newton_optical_1976}, and its derivation rely in the incident field being a plane wave [see Eq. \eqref{eq:CextFull}] and more precisely, in the lack of longitudinal components of the incident field \cite{krasavin_generalization_2018,born_max_principle_1999}.

From Eqs. \eqref{eq:Stot} and  \eqref{eq:Cext} it can be seen that the extinction of light, the combined result of scattering and absorption as energy loss mechanisms, is also a manifestation of the interference between the incident and the scattered fields and, remarkably,  that the overall effect of the light extinction can be fully understood by analyzing the  amplitude of the scattering field in the forward direction.  It is worth noting that Eq. \eqref{eq:Cext} is an exact relation but its usefulness is bond to the correct evaluation of the scattering amplitude matrix $\mathbb{F}$ \cite{tsang_scattering_2000}. Thus, in the following Sections a scattering problem with spherical symmetry will be assumed, so that the exact solution to the scattering amplitude matrix can be developed; this solution is known as the \emph{Mie Theory}.
