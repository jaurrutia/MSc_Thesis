% !TeX root = ../tesis.tex

The Vector Spherical Harmonics (VSH) where defined in Section \ref{ss:VSH} in terms of their generating function $\psi(r,\theta,\varphi)$ which must satisfy the scalar Helmholtz equation [Eq.  \eqref{eq:HelmoltzScalar}]. By employing the separation of variables method,  it was determined that $\psi$ is the product of either $\sin(m\varphi)$ or $\cos(m\varphi)$, the associated Legendre functions $P_\ell^m(\cos\theta)$ and the spherical Bessel/Hankel functions $z_\ell(kr)$, which are solutions to Eqs. \eqref{eq:Phi}-\eqref{eq:Reqkr}. In this Section, it is discussed the chosen definitions for $P_\ell^m$, $z_\ell$ and related functions, as well as how to calculate them.

\section*{Radial Dependency: Spherical Bessel/Hankel Functions}

\index{Functions!Spherical Bessel@{Spherical Bessel $j_\ell(\rho)$ and $y_\ell(\rho)$}}%
\index{Functions!Spherical Hankel@{Spherical Hankel $h^{(1)}_\ell(\rho)$ and $h^{(2)}_\ell(\rho)$}}%
\index{Bessel!Spherical Functions@{Spherical Functions $j_\ell(\rho)$ and $y_\ell(\rho)$}}%
\index{Hankel!Spherical Functions@{Spherical Functions $h^{(1)}_\ell(\rho)$ and $h^{(2)}_\ell(\rho)$}}%
%
The radial dependency of the VSH is given by the two linearly independent solutions to Eq. \eqref{eq:Reqkr} which are the spherical Bessel function of first and second kind $j_\ell(\rho)$ and $y_\ell(\rho)$, respectively, related to the regular Bessel function of fractional order $J_{\ell+1/2}(\rho)$ and  $Y_{\ell+1/2}(\rho)$ by  \cite{abramowitz_handbook_2013}
%
\begin{align}
j_\ell(\rho) = \sqrt{\frac{\pi}{2 \rho}}J_{\ell+1/2}(\rho),\qqtext{ and}
y_\ell(\rho) = \sqrt{\frac{\pi}{2\rho}}Y_{\ell+1/2}(\rho).
\label{eq:bessel}
\end{align}
%
Another set of two linear independent solutions to  Eq. \eqref{eq:Reqkr} are the spherical Hankel functions  of first ($h^{(1)}_\ell$)  and second kind ($h^{(1)}_\ell$) given by \cite{abramowitz_handbook_2013}
%
\begin{align}
h^{(1)}_\ell(\rho) = j_\ell(\rho) + i y_\ell(\rho),\qqtext{ and}
h^{(2)}_\ell(\rho) = j_\ell(\rho) - i y_\ell(\rho).
\label{eq:Hankel}
\end{align}
%
Since the spherical Hankel functions are a linear combination of the Bessel spherical functions, they four obey the following recurrence relations \cite{abramowitz_handbook_2013}
%
\begin{align}
\frac{z_\ell(\rho)}{\rho} =& \frac{z_{\ell-1}(\rho) + z_{\ell+1}(\rho)}{2\ell + 1},
\label{eq:recBessel1}\\
\dv{z_\ell(\rho)}{\rho} =& \frac{\ell z_{\ell-1}(\rho) - (\ell+1)z_{\ell+1}(\rho)}{2\ell + 1} ,
\label{eq:recBessel2}
\end{align}
%
with $z_\ell$ any of the functions in Eqs. \eqref{eq:bessel} and \eqref{eq:Hankel}.


\section*{Azimuthal Angular Dependency $\varphi$: Sine, Cosine}

Within this text, it was chosen the azimuthal solution to the scalar Helmholtz equation to be sines and cosines, so $m$ can only take non negative integer values. These functions obey the orthogonality relations
%
\begin{align}
\int_0^{2\pi} \sin(m\varphi)\sin(m'\varphi) \dd{\varphi} &= \delta_{m,m'}( 1 - \delta_{0,m}) \pi,
\label{eq:SinOrth}\\
\int_0^{2\pi} \cos(m\varphi)\cos(m'\varphi) \dd{\varphi} & =\delta_{m,m'}( 1+ \delta_{0,m}) \pi,
\label{eq:CosOrth}\\
\int_0^{2\pi} \cos(m\varphi)\sin(m'\varphi) \dd{\varphi} &=0,
\label{eq:SinCosOrth}
\end{align}
%
with $\delta_{m,m'}$ the Kronecker delta.

\section*{Polar Angular Dependence: Associated Legendre Functions and the Angular Functions $\pi_\ell$ and $\tau_\ell$}

The solution to the polar angle equation [Eq. \eqref{eq:Theta}] are the associated Legendre functions and in this work they are defined as by \citeauthor{arfken_mathematical_2001} \cite{arfken_mathematical_2001}, that is,
%
\index{Legendre!Associated Functions@{Associated Functions $P_\ell^m(\mu)$}}%
\index{Functions!Legendre Associated}%
\index{Scattering!Mie!Angular Functions@{Angular Functions $\pi$ and $\tau$}}%
%
\begin{equation}
P_\ell^m(\mu) = (1-\mu^2)^{m/2}\dv[m]{\,}{\mu}\hspace*{.05em} P_\ell(\mu),
\qqtext{ with}
P_\ell(\mu) = \frac{1}{2^\ell \ell!}\dv[\ell]{\,}{\mu}\hspace*{.05em}  (\mu^2-1)^\ell ,
\label{eq:Plm}
\end{equation}
%
where $\mu = \cos\theta$ and $P_\ell(\mu)$ are the Legendre polynomials with $\ell$ a non negative integer. With such definition, the  associated Legendre functions follows the orthogonality relation
%
\begin{align}
\int_{-1}^1 P_\ell^m(\mu)P_{\ell'}^m(\mu)\dd{\mu} = \frac{2\delta_{\ell,\ell'}}{2\ell+1}\frac{(\ell+m)!}{(\ell-m)!}.
\label{eq:PlmOrtho}
\end{align}
%
It was shown in Section \ref{ss:Fields} that  a plane wave can be written as a linear combination of the VSH with only $m = 1$, which lead to the definition of the angular functions $\pi_\ell$ and $\tau_\ell$ given by
%
\index{Scattering!Mie!Angular Functions $\pi_\ell$ and $\tau_\ell$}
%
\begin{align*}
 \pi_\ell(\cos\theta )  = \frac{P_\ell^1(\cos\theta)}{\sin\theta},\qqtext{and}
 \tau_\ell(\cos\theta) = \dv{P_\ell^1(\cos\theta)}{\theta},
\end{align*}
%
that can be calculated recursively with Eq. \eqref{eq:Plm}  and the recurrence relations of the Legendre polynomials
%
\begin{align}
(2\ell-1)\mu P_{\ell-1}(\mu) =&\, (\ell-1) P_{\ell}(\mu) + \ell P_{\ell-2}(\mu),\\
(1-\mu)^2\dv{P_\ell(\mu)}{\mu} =&\, \ell P_{\ell-1}(\mu) - \ell\mu P_\ell(\mu),
\end{align}
%
leading to
%
\begin{align}
\pi_\ell(\mu) =&\, \frac{2\ell-1}{\ell-1}\mu \pi_{\ell-1}(\mu) - \frac{\ell}{\ell-1}\pi_{\ell-2},\\
\tau_ \ell (\mu) =&\, \ell\mu\pi_\ell(\mu) - (\ell+1)\pi_{\ell-2}(\mu),
\end{align}
%
where $\pi_1(\mu) = 1$ according to  Eq. \eqref{eq:Plm} and where $\pi_0(\mu)=0$ is defined. Another result from Eq. \eqref{eq:Plm} is that the angular functions $\pi_\ell(\mu)$ and $\tau_\ell(\mu)$, when evaluated at $\theta =0$ ($\mu = 1$),  follows
%
\begin{align}
\pi_\ell(\mu=1) & =  \dv{P_\ell(\mu)}{\mu}\eval_{\mu=1},\\
\tau_\ell (\mu = 1) & = \qty[\dv{P_\ell^1(\mu)}{\mu} + (1-\mu^2)^{1/2}\dv[2]{P_\ell(\mu)}{\mu}]\eval_{\mu=1} = \dv{P_\ell(\mu)}{\mu}\eval_{\mu=1},
\end{align}
%
which can be obtained from the Legendre equation  by setting $m = 1$ and $\mu = 1$ in Eq. \eqref{eq:ThetaMu}, leading to
%
\begin{align}
    \pi_\ell(\mu=1) = \tau_\ell(\mu=1) = \frac{\ell(\ell+1)}{2} P_\ell(\mu = 1) = \frac{\ell(\ell+1)}{2},
    \label{eq:PiTau1}
\end{align}
%
where the last equality arises from the chosen definition of the Legendre polynomials [Eq. \eqref{eq:Plm}].

The angular functions $\pi_\ell$  and $\tau_\ell$ are not orthogonal in general, nevertheless  $\pi_\ell(\mu)\pm\tau_\ell(\mu)$ are. To prove the orthogonality of $\pi_\ell\pm\tau_\ell$ let us apply the Legendre equation [Eq. \eqref{eq:Theta}] to $P_\ell^m$ and multiply it by $P_{\ell'}^m$; repeating this procedure inverting $\ell$ and $\ell'$ and adding both equations it is obtained that
%
\begin{align}
\dv{\theta}&\qty(\sin\theta P_{\ell'}^m(\mu)\dv{P_\ell^m(\mu)}{\theta}) +
\dv{\theta}\qty(\sin\theta P_{\ell}^m(\mu)\dv{P_{\ell'}^m(\mu)}{\theta}) +
\label{eq:PlPl'}
\\
&\qty[\ell(\ell+1)+\ell'(\ell'+1)]P_{\ell'}^m(\mu)P_{\ell}^m(\mu) \sin\theta
=
 2\qty(\frac{mP_\ell^m(\mu)}{\sin\theta}\frac{mP_{\ell'}^m(\mu)}{\sin\theta}+ \dv{P_\ell^m(\mu)}{\theta}\dv{P_{\ell'}^m(\mu)}{\theta})\sin\theta,\notag
\end{align}
%
where  it was added $2\dv*{P_\ell^m}{\theta}\dv*{P_{\ell'}^m}{\theta}$ on both sides to complete the derivatives. Integrating Eq. \eqref{eq:PlPl'} in the interval $\theta \in (0,\pi)$, or $\mu \in(-1,1)$, and employing Eqs. \eqref{eq:Plm} and \eqref{eq:PlmOrtho}, one obtains that
%
\begin{align}
\int_{-1}^1 \qty(\frac{mP_\ell^m(\mu)}{\sin\theta}\frac{mP_{\ell'}^m(\mu)}{\sin\theta}+ \dv{P_\ell^m(\mu)}{\theta}\dv{P_{\ell'}^m(\mu)}{\theta})\dd{\mu} =
\delta_{\ell,\ell'}\frac{2\ell(\ell+1)}{2\ell+1}\frac{(\ell+m)!}{(\ell-m)!}.
\label{eq:PimlTauml}
\end{align}
%
Additionally
%
\begin{align}
\int_{-1}^1\frac{mP_\ell^m(\mu)}{\sin\theta}\dv{P_{\ell'}^m(\mu)}{\theta}\dd{\mu}
 = \int_0^{\pi} mP_\ell^m(\mu)\dv{P_{\ell'}^m(\mu)}{\theta}\dd{\theta} =
 -\int_{-1}^1\frac{mP_{\ell'}^m(\mu)}{\sin\theta}\dv{P_{\ell}^m(\mu)}{\theta}\dd{\mu},
 \label{eq:taupiCross}
\end{align}
%
where Eq. \eqref{eq:Plm} was employed along integration by parts. Thus, combining Eqs. \eqref{eq:PimlTauml} and \eqref{eq:taupiCross}, it leads to
%
\begin{align}
\int_{-1}^{1}\qty(\frac{mP_\ell^m(\mu)}{\sin\theta}\pm\dv{P_\ell^m(\mu)}{\theta})\qty(\frac{mP_{\ell'}^m(\mu)}{\sin\theta}\pm\dv{P_{\ell'}^m(\mu)}{\theta})\dd{\mu}
 =  \delta_{\ell,\ell'}\frac{2\ell(\ell+1)}{2\ell+1}\frac{(\ell+m)!}{(\ell-m)!}.
 \label{eq:(pipmtau)}
\end{align}
%
The Eq. \eqref{eq:(pipmtau)}  is the orthogonality of $\pi_\ell(\mu)\pm\tau_\ell(\mu)$ when $m = 1$, which also simplifies the right hand side to $\delta_{\ell,\ell'} 2\ell^2(l+1)^2/(2\ell+1)$.


\section*{Vector Spherical Harmonics Orthogonality Relations}

The VSH follow orthogonality relations inherited from the orthogonality of sine, cosine and the associated Legendre functions. Let us define the inner product as the integral in the solid angle between two vector functions as
%
\index{Vector!Spherical Harmonics!Orthogonality}%
%
\begin{align}
\ev{\vb{A},\vb{A}'}_\Omega = \int_0^{2\pi}\int_0^{\pi} \vb{A}\cdot\vb{A}'\sin\theta\dd{\theta}\dd{\varphi}.
\label{eq:inner}
\end{align}
%
Under this inner product, all even VSH are orthogonal to the odd VSH, as well as all VSH  with $m\neq m'$, due to the orthogonality of $\sin(m\varphi)$ and $\cos(m'\varphi)$. The remaining orthogonality relations  can be obtained by employing Eq. \eqref{eq:PimlTauml}, leading to
%
\begin{align}
&\ev{\vb{L}_{em'\ell},\vb{L}_{em'\ell'}}_\Omega = \ev{\vb{L}_{om\ell},\vb{L}_{om\ell'}}_\Omega \notag\\
	 &\hspace*{2em} =
 	\delta_{m,m'}\delta_{\ell,\ell'} (1\pm\delta_{m,0})
 	\frac{2\pi}{2\ell+1}\frac{(\ell+m)!}{(\ell-m)!}
 	\qty[\qty(k\dv{z_\ell(kr)}{(kr)})^2 + \ell(\ell+1)\qty(k\frac{z_\ell(kr)}{kr})^2],
    \label{eq:LL}\\
%
&\ev{\vb{M}_{em\ell},\vb{M}_{em\ell'}}_\Omega = \ev{\vb{M}_{om\ell},\vb{M}_{om\ell'}}_\Omega \notag\\
		 &\hspace*{2em} =
	\delta_{m,m'}\delta_{\ell,\ell'} (1\pm\delta_{m,0})
	\pi \frac{2  \ell(\ell+1)}{2\ell+1}\frac{(\ell+m)!}{(\ell-m)!}
	z_{\ell}^2(kr),\\
%
&\ev{\vb{N}_{em\ell},\vb{N}_{em\ell'}}_\Omega = \ev{\vb{N}_{om\ell},\vb{N}_{om\ell'}}_\Omega  \notag\\
	 &\hspace*{2em} =
	 \delta_{m,m'}\delta_{\ell,\ell'} (1\pm\delta_{m,0})
	\pi \frac{2  \ell(\ell+1)}{2\ell+1}\frac{(\ell+m)!}{(\ell-m)!}
	\qty[\qty(\frac{ z_{\ell}}{kr})^2 + \qty(\frac{1}{kr}\dv{[kr z_\ell(kr)]}{(kr)})^2], \\
%%
&\ev{\vb{L}_{em\ell},\vb{N}_{em\ell'}}_\Omega = \ev{\vb{L}_{om\ell},\vb{N}_{om\ell'}}_\Omega \notag\\
	 &\hspace*{2em} =
	 \delta_{m,m'}\delta_{\ell,\ell'} (1\pm\delta_{m,0})
	\pi \frac{2  \ell(\ell+1)}{2\ell+1}\frac{(\ell+m)!}{(\ell-m)!}
	\qty[\frac{ z_{\ell}}{kr}\dv{z_\ell(kr)}{(kr)} + \qty(\frac{1}{kr}\dv{[kr z_\ell(kr)]}{(kr)})^2],
    \label{eq:LM}
\end{align}
%
where $(1+\delta_{m,0}) $ is for odd VSH and $(1-\delta_{m,0}) $ for even VSH. The orthogonality relations of the VSH can be further simplified by means of the recurrence relations of the spherical Bessel/Hankel functions [Eqs. \eqref{eq:recBessel1} and \eqref{eq:recBessel2}], which imply that
%
\begin{align}
\qty[\qty(k\dv{z_\ell(kr)}{(kr)})^2 + \ell(\ell+1)\qty(k\frac{z_\ell(kr)}{kr})^2] =& k^2
		\qty[\ell z_{\ell-1}^2(kr) + \ell(\ell+1)z_{\ell+1}^2(kr)],\\
\qty[\qty(\frac{ z_{\ell}}{kr})^2 + \qty(\frac{1}{kr}\dv{[kr z_\ell(kr)]}{(kr)})^2] =&
 	\ell(\ell+1)\qty[(\ell+1) z_{\ell-1}^2(kr)+ \ell z_{\ell+1}^2(kr)],  \\
\qty[\frac{ z_{\ell}}{kr}\dv{z_\ell(kr)}{(kr)} + \qty(\frac{1}{kr}\dv{[kr z_\ell(kr)]}{(kr)})^2] =&
	 	\ell(\ell+1)\qty[z_{\ell-1}^2(kr)- z_{\ell+1}^2(kr)].
\end{align}
%
