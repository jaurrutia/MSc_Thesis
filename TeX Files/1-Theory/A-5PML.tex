\begin{frame}{\textit{Perfect Matching Layer}$^{1,2}$ (PML)}
{Propiedades geométricas de $\Omega_\text{PML}$, dominio que rodea a $\Omega$}

\begin{itemize}
    \item Elongaciones geométrias$^3$ y aproximación de campo lejano considerada como válida
         \begin{align*}
         \nabla_\text{s} \equiv \qty(\frac{\vu{e}_x }{s_x}\pdv{x} + \frac{\vu{e}_y}{s_y}\pdv{y} + \frac{\vu{e}_z}{s_z}\pdv{z}) \to \vb{k} = \frac{k_x}{s_x}\vu{e}_x + \frac{k_y}{s_y}\vu{e}_y  +\frac{k_y}{s_z}\vu{e}_z
     \end{align*}
%
    \item Factores de deformación complejos:  $s_{x_i} = s_{x_i}(x_i)$; en $\Omega,\, s_{x_i} = 1$
        \begin{align*}
        \nabla_\text{s}\cdot \vb{v} =&
         %            \frac{1}{s_x s_y s_z}\qty[\pdv{x}\big( s_y s_z v_x\big) + \pdv{y} \big( s_x s_z v_y\big)+ \pdv{z}\big( s_x s_y v_z\big)]
                   \frac{1}{s_x s_y s_z} \nabla\cdot \qty[\text{diag}(s_y s_z,s_x s_z,s_x s_y)\vb{v}]
         \\
        \nabla_\text{s}\times \vb{v} =& %\frac{1}{s_x s_y s_z}\mqty|s_x\vu{e}_x & s_y\vu{e}_y & s_z\vu{e}_z \\
                                         %                    \pdv*{x} & \pdv*{y} & \pdv*{z} \\
                                         %                    s_x v_x & s_y v_y & s_z v_z|
                                     = \text{diag}\qty(\frac{1}{s_y s_z},\frac{1}{s_x s_z},\frac{1}{s_x s_y}) \nabla\times
                                     \qty[\text{diag}(s_x,s_y,s_z) \vb{v}]
        \end{align*}
    \item Relación de dispersión de una onda plana:
        \begin{align*}
        \vb{k} \cdot \vb{k}  = k^2 = \mu\varepsilon\omega^2 =
            \qty(\frac{k_x}{s_x}^2 )+ \qty(\frac{k_y}{s_y})^2 + \qty(\frac{k_z}{s_z})^2\\
        k_x = k s_x \sin\theta\cos\varphi \qquad
            k_y = k s_y \sin\theta\sin\varphi \qquad
                k_z = k s_z \cos\theta
        \end{align*}
\end{itemize}
%
	\noindent\rule{.25\textwidth}{0.4pt}
 \begin{spacing}{0}\fontsize{4}{12} \selectfont
	$^1$ \fullcite{fletcher_computational_1984}\\
	$^2$ \fullcite{jin_theory_2010}\\
	$^3$ \fullcite{bergot_generation_2010}
	\end{spacing}
\end{frame}

\begin{frame}{\textit{Perfect Matching Layer} (PML)}
{Condiciones para nulas reflexiones}
\small
\begin{itemize}
     \item Coeficientes de amplitud de reflexión$^1$:
        \begin{align*}
       r_\text{s} = \frac{k^\text{(PML)}_z s^\text{(PML)}_z\mu_{{}_\text{PML}} - k^{(\Omega)}_z s^{(\Omega)}_z\mu_{{}_\Omega}}
                        {k^\text{(PML)}_z s^\text{(PML)}_z\mu_{{}_\text{PML}} + k^{(\Omega)}_z s^{(\Omega)}_z\mu_{{}_\Omega}}
           \quad
       r_\text{p} = \frac{k^\text{(PML)}_z s^\text{(PML)}_z\varepsilon_{{}_\text{PML}} - k^{(\Omega)}_z s^{(\Omega)}_z\varepsilon_{{}_\Omega}}
                        {k^\text{(PML)}_z s^\text{(PML)}_z\varepsilon_{{}_\text{PML}} + k^{(\Omega)}_z s^{(\Omega)}_z\varepsilon_{{}_\Omega}}
        \end{align*}
    \item Condciones de empatamiento de fase (componentes paralelas a la interfaz entre $\Omega$ y $\Omega_\text{PML}$)
        \begin{align*}
             k^\text{(PML)} s^\text{(PML)}_x \sin\theta_{{}_\text{PML}}\cos\varphi_{{}_\text{PML}} =
                        k^{(\Omega)} s^{(\Omega)}_x \sin\theta_{{}_\Omega}\cos\varphi_{{}_\Omega}\\
        k^\text{(PML)} s^\text{(PML)}_y \sin\theta_{{}_\text{PML}}\sin\varphi_{{}_\text{PML}} =
                     k^{(\Omega)} s^{(\Omega)}_y \sin\theta_{{}_\Omega}\sin\varphi_{{}_\Omega}
        \end{align*}
\end{itemize}

    \begin{alertblock}{Condiciones para una PML$^2$}
    \begin{align*}
            \left. \mqty{\varepsilon_{{}_\Omega}=\varepsilon_{{}_\text{PML}} ,
                                                &\mu_{{}_\Omega}=\mu_{{}_\text{PML}}
                                                \\
                                                \\
                                            s^{(\Omega)}_x  = s^\text{(PML)}_x,
                                            &  s^{(\Omega)}_y  = s^\text{(PML)}_y} \right\}
                    \qquad\implies \qquad
           r_\text{s} = r_\text{p} = 0 .
    \end{align*}

    Adicionalmente $\Im[s_z^{(PML)}]<0$ para que sea un material absorbente.
    \end{alertblock}
%
	\noindent\rule{.25\textwidth}{0.4pt}
 \begin{spacing}{0}\fontsize{4}{12} \selectfont
	$^1$ \fullcite{jackson_classical_1999}\\
	$^2$ \fullcite{jin_theory_2010}
	\end{spacing}
\end{frame}



\begin{frame}{\textit{Perfect Matching Layer} (PML)}
{Campos electromagnéticos$^1$}\small
\begin{itemize}
\item Relación entre los campos en $\Omega$ y $\Omega_\text{PML}$
\begin{align*}
        \vb{E}^{(\Omega)} = \text{diag}(s_x,s_y,s_z)\vb{E}^\text{(PML)}
            \qquad
            &\iff
            \qquad
         \vb{E}^\text{(PML)} = \text{diag}\qty(\frac{1}{s_x},\frac{1}{s_y},\frac{1}{s_z})\vb{E}^{(\Omega)}
    \\
       \vb{H}^{(\Omega)} = \text{diag}(s_x,s_y,s_z)\vb{H}^\text{(PML)}
            \qquad
             &\iff
            \qquad
       \vb{H}^\text{(PML)} = \text{diag}\qty(\frac{1}{s_x},\frac{1}{s_y},\frac{1}{s_z})\vb{H}^{(\Omega)}
   \end{align*}
%
\item Transormación de las ecuaciones de Maxwell
\begin{align*}
        \nabla_\text{s} \cdot \qty(\varepsilon \vb{E}^\text{(PML)})  = 0
            \qquad &\implies\qquad
            \nabla \cdot \Big[\Big(\varepsilon \mathbb{\Lambda}\Big)\vb{E}^{(\Omega)}\Big] =  0
             \\
        \nabla_\text{s}  \cdot    \Big(\mu \vb{H}^\text{(PML)}\Big) = 0
            \qquad &\implies\qquad
            \nabla \cdot \Big[\Big(\mu \mathbb{\Lambda} \Big)\vb{H}^{(\Omega)}\Big] =  0
            \\
        \nabla_\text{s} \times \vb{E}^\text{(PML)}  = i\omega \mu \vb{H}^\text{(PML)}
            \qquad &\implies\qquad
            \nabla \times \Big(\vb{E}^{(\Omega)}\Big) = i\omega\Big( \mu\mathbb{\Lambda}\Big) \vb{H}^{(\Omega)}
            \\
        \nabla_\text{s}  \times\Big(\mu \vb{H}^\text{(PML)} \Big) =   - i\omega \varepsilon \vb{E}^\text{(PML)}
            \qquad &\implies\qquad
            \nabla \times \Big[\Big(\mu \mathbb{\Lambda}\Big) \vb{E}^{(\Omega)}\Big] =  -i\omega\Big(\varepsilon\mathbb{\Lambda}\Big)\vb{E}^{(\Omega)}
    \end{align*}
\item Matriz de transformación:
        \begin{align*}
        \mathbb{\Lambda} = \text{diag}\qty(\frac{s_y s_z}{s_x},\frac{s_x s_z}{s_y},\frac{s_x s_y}{s_z})
    \end{align*}

\end{itemize}

	\noindent\rule{.25\textwidth}{0.4pt}
 \begin{spacing}{0}\fontsize{4}{12} \selectfont
	$^1$ \fullcite{jin_theory_2010}
	\end{spacing}
\end{frame}
