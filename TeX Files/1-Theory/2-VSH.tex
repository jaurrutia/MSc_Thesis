% !TeX root = ../tesis.tex

The electric and magnetic field, denoted as $\vb{E}$ and $\vb{B}$, respectively, are a solution to the homogeneous vectorial Helmholtz\index{Helmholtz!Equation, Vectorial} when an harmonic time dependence and a spacial domain with no external charge nor current densities is assumed, that is,
%
% -----------------------------
\begin{subequations}
\begin{tcolorbox}[title = Vectorial Helmholtz Equation,	ams align, breakable]
	\grad^2 \vb{E}(\vb{r},\omega) + k^2 \vb{E}(\vb{r},\omega) &= \vb{0},\\
  \grad^2 \vb{B}(\vb{r},\omega) + k^2 \vb{B}(\vb{r},\omega) &= \vb{0}.
\end{tcolorbox}
\label{eq:Helmholtz}
\end{subequations}
% ------------------------------
%
\noindent where the vectorial operator $\grad^2$ must be understood as $\grad^2 = \nabla(\nabla\cdot) - \nabla\times\nabla\times $, and $k$ is the wave number in the matrix, which must follow the relation of dispersion $k = (\omega/c) n_\text{mat}$, with $n_\text{mat}=\sqrt{\mu_\text{mat}\varepsilon_\text{mat} /\mu_0\varepsilon_0}$ the refractive index of the matrix, $\mu_\text{mat}$ its magnetic permeability and $\varepsilon_\text{mat}$ its dielectric function. It is possible to build a basis set for the electric and magnetic fields as long as the elements of this basis are also solution to Eq. \eqref{eq:Helmholtz}. One alternative is to employ the following set of vector functions
%
% -----------------------------
\begin{subequations}
\begin{align}
	\vb{L} =& \nabla \psi,
	\label{eq:L}\\
	\vb{M} =& \nabla\times(\vb{r}\psi),
	\label{eq:M}\\
	\vb{N} =&  \frac{1}{k}\nabla\times\vb{M},
	\label{eq:N}
\end{align}%
\label{eq:VSH}%
\end{subequations}
% ------------------------------
%
that are solution to the homogeneous vectorial Helmholtz equation as long as the scalar function $\psi$ is solution to the scalar Helmholtz equation\footnote{%
	This result can be proven by considering the following: Let $f$ be $\mathcal{C}^3$ and $\vb{F}$ a $\mathcal{C}^2$. Then, it is true that $\nabla^2(\nabla f) = \nabla(\nabla^2 f)$, and $\curl(\grad^2\vb{F}) = \grad^2(\curl\vb{F})$. }\index{Helmholtz!Equation, Scalar}
%
% -----------------------------
\begin{align}
	\nabla^2 \psi + k^2 \psi = 0.
\label{eq:HelmoltzScalar}
\end{align}
% ------------------------------
%
The triad $\left\{\vb{L},\vb{M},\vb{N}\right\}$ is a set of vectors\footnote{%
	Employing the Einstein sum convention with $\epsilon_{ijk}$ the Levi-Civita symbol, Eq. \eqref{eq:M} can be the written as follows:%
	 	$M_i = [\nabla\times(\vb{r}\psi)]_i
	 	=  \epsilon_{ijk}\partial_j(r_k\psi)
	 	=\psi\epsilon_{ijk}\partial_j(r_k) -\epsilon_{ikj}r_k\partial_j\psi
	 	=\psi[\nabla\times\vb{r}]_i - [\vb{r}\times\nabla\psi]_i
	 	= - [\vb{r}\times\nabla\psi]_i
	 	= [\vb{L}\times\vb{r}]_i$,%
	 therefore $\vb{M}$ is orthogonal to $\vb{L}$ and $\vb{r}$. From Eq. \eqref{eq:N} $\vb{M}\cdot\vb{N}=0$, so $\vb{M}$ is orthogonal to $\vb{N}$. As it will be shown, not necessarily $\vb{L}$ is orthogonal to $\vb{N}$ in a geometrical sense.
	}
 that obey Helmholtz equation \textit{i.e.}, they can be directly identify as electric or magnetic fields. The elements of the vector basis from Eq. \eqref{eq:VSH}   are known as the Vectorial Spherical Harmonics (VSH) as defined by  \citeauthor{stratton_electromagnetic_2012} \cite{stratton_electromagnetic_2012}, and \citeauthor{bohren_absorption_1983} \cite{bohren_absorption_1983} and the scalar function $\psi$ is known as the generating function of the VSH. From the definition of the VSH in Eqs. \eqref{eq:VSH} it can be seen that $\vb{L}$ has only a longitudinal component while $\vb{M}$ and $\vb{N}$ have only transversal components; specifically $\vb{M}$ is tangential to any sphere of radius $\norm{\vb{r}}$.


If spherical coordinates are chosen, and it is assumed that $\psi(r,\theta,\varphi) = R(r)\Theta(\theta)\Phi(\varphi)$, then Eq. \eqref{eq:HelmoltzScalar} can be decouple into three ordinary differential equations:
%
% ------------------------------
 \begin{align}
	\dv[2]{\Phi}{\varphi} &+ m^2 \Phi =0,
 \label{eq:Phi}\\
	\frac{1}{\sin\theta}\dv{\theta}\qty(\sin\theta\dv{\Theta}{\theta}) &+ \qty[\ell(\ell+1)- \frac{m^2}{\sin^2\theta}]\Theta =0,
	\label{eq:Theta}\\
	\dv{r}\qty(r^2\dv{R}{r}) &+ \qty[ (k r)^2 - \ell (\ell +1)] R =0,
 \label{eq:Req}
\end{align}
% ------------------------------
%
where $\ell$ can take natural values and zero, and $\abs{m}\leq \ell$ so $\Phi$ and $\Theta$ are univalued and finite on a sphere. Eqs. \eqref{eq:Theta} and \eqref{eq:Req} can be rewritten as
%
% ------------------------------
 \begin{align}
(1-\mu^2)\dv[2]{\Theta}{\mu} - 2\mu\dv{\Theta}{\mu} + \qty[\ell(\ell+1)-\frac{m^2}{1-\mu^2}]\Theta &= 0, \qqtext{ with $\mu = \cos\theta$,}
	\label{eq:ThetaMu}\\
	\rho\dv{\rho}\qty(\rho\dv{Z}{\rho}) +  \qty[\rho^2 - \qty(\ell + \frac12)^2]Z  &= 0,  \qqtext{ with $Z = R\sqrt{\rho}$ and $\rho = kr$.}
\label{eq:Reqkr}
\end{align}
% ------------------------------
%
The solution to Eq. \eqref{eq:ThetaMu} are the associated Legendre functions $ P_\ell^m(\mu)$ and to Eq. \eqref{eq:Reqkr} the solution is given by the spherical Bessel functions of the first ($j_\ell$)  and second ($y_\ell$) kind, and the spherical Hankel functions of first ($h_\ell^{(1)} = j_\ell + iy_\ell$) and second ( $h_\ell^{(2)} = j_\ell - iy_\ell$)  kind. Following the convention from most literature on Mie Scattering \cite{zangwill_modern_2013}, the solution to Eq. \eqref{eq:Phi} will be decompose into an odd ($o$) and an even ($e$) solution, that is, as sine and cosine functions, thus restricting the values of $m$ to non-negative integers. After this procedure, it is determined that the generating function of the VSH is given by
%
% -----------------------------
\begin{subequations}
\begin{tcolorbox}[title = $\psi$: Generating function of the vectorial spherical harmonics,	ams align, breakable]
	\psi_{e\ell m}(r,\theta,\varphi) =& \cos(m\varphi)P_\ell^m(\cos\theta)z_\ell(kr),
	\label{eq:psiE}\\
	\psi_{o\ell m}(r,\theta,\varphi) =& \sin(m\varphi)P_\ell^m(\cos\theta)z_\ell(kr).
	\label{eq:psiO}
\end{tcolorbox}
\label{eq:psi}
\end{subequations}
% ------------------------------
%
\noindent%
where $z_\ell$ stands for any of the four solutions to the radial equation [Eq. \eqref{eq:Reqkr}]. Substituting Eq. \eqref{eq:psiE} in Eqs. \eqref{eq:L}--\eqref{eq:N} one finds the even VSH
%
% -----------------------------
\begin{subequations}
\begin{tcolorbox}[title = Even vectorial spherical harmonics,	ams align, breakable]
	\vb{L}_{em\ell} =& k \cos(m\varphi)P_\ell^m(\cos\theta)\dv{z_\ell(kr)}{(kr)}\,\vu{e}_r
					 +  k\cos(m\varphi) \frac{z_\ell(kr)}{kr}\dv{P_\ell^m(\cos\theta)}{\theta} \,\vu{e}_\theta \notag \\
					& - km \sin(m\varphi) \frac{P_\ell^m(\cos\theta)}{\sin\theta}\frac{z_\ell(kr)}{kr} \,\vu{e}_\varphi
	\label{eq:Leml}\\
	\vb{M}_{em\ell} = &-m\sin(m\varphi)z_\ell(kr) \frac{P_\ell^m(\cos\theta)}{\sin\theta}\,\vu{e}_\theta
					-\cos(m\theta)z_\ell(kr) \dv{P_\ell^m(\cos\theta)}{\theta}(\cos\theta)\,\vu{e}_\varphi,
	\label{eq:Meml} \\
	\vb{N}_{em\ell} = &\cos(m\varphi) \frac{z_\ell(kr)}{kr} \ell(\ell+1)P_\ell^m(\cos\theta)\,\vu{e}_r
						+ \cos(m\varphi)  \frac{1}{kr} \dv{[kr\, z_\ell(kr)] }{(kr)}
						\dv{P_\ell^m(\cos\theta)}{\theta}(\cos\theta)\,\vu{e}_\theta \notag\\
						&- m \sin(m\varphi) \frac{1}{kr} \dv{[kr\, z_\ell(kr)] }{(kr)}\frac{P_\ell^m(\cos\theta)}{\sin\theta}
		 \,\vu{e}_\varphi,
	\label{eq:Neml}
\end{tcolorbox}
\label{eq:VSHEven}
\end{subequations}
% ------------------------------
%
\noindent where the term $\ell( \ell+1)P_\ell^m$ arises since the associated Legendre functions obeys Eq. \eqref{eq:ThetaMu}. Likewise, the odd VSH are given by
%
% -----------------------------
\begin{subequations}
\begin{tcolorbox}[title = Odd vectorial spherical harmonics,	ams align, breakable]
	\vb{L}_{om\ell} =& k \sin(m\varphi)P_\ell^m(\cos\theta)\dv{z_\ell(kr)}{(kr)}\,\vu{e}_r
					 +  k\sin(m\varphi) \frac{z_\ell(kr)}{kr}\dv{P_\ell^m(\cos\theta)}{\theta} \,\vu{e}_\theta \notag \\
					& +  km\cos(m\varphi) \frac{P_\ell^m(\cos\theta)}{\sin\theta}\frac{z_\ell(kr)}{kr} \,\vu{e}_\varphi
	\label{eq:Loml}\\
	\vb{M}_{om\ell} = & m\cos(m\varphi)z_\ell(kr) \frac{P_\ell^m(\cos\theta)}{\sin\theta}\,\vu{e}_\theta
					-\sin(m\theta)z_\ell(kr) \dv{P_\ell^m(\cos\theta)}{\theta}(\cos\theta)\,\vu{e}_\varphi,
	\label{eq:Moml} \\
	\vb{N}_{om\ell} =&\sin(m\varphi)\frac{z_\ell(kr)}{kr} \ell(\ell+1)P_\ell^m(\cos\theta)\,\vu{e}_r +
					 \sin(m\varphi)  \frac{1}{kr} \dv{[kr\, z_\ell(kr)]}{(kr)} \dv{P_\ell^m(\cos\theta)}{\theta}(\cos\theta) \,\vu{e}_\theta \notag\\
					 & + m \cos(m\varphi) \frac{1}{kr} \dv{[kr\, z_\ell(kr)]}{(kr)} \frac{P_\ell^m(\cos\theta)}{\sin\theta}\, \vu{e}_\varphi.
	 \label{eq:Noml}
\end{tcolorbox}
\label{eq:VSHOdd}
\end{subequations}
% ------------------------------
%
\noindent%
The election on $z_\ell$ in Eqs. \eqref{eq:VSHEven} and  \eqref{eq:VSHOdd} is due to the physical constrains of the scattering problem. The spherical Bessel function of first kind, unlike the other three proposed solution to the radial equation, is finite at $r = 0$, thus it is appropriate for the internal electric field and plane waves. This election of $z_\ell$ will be denoted in the VSH with the superscript $(1)$. On the other hand, the asymptotic behavior ($\ell\ll \rho$) of the Hankel function of first kind $h^{(1)} = j_\ell + i y_\ell$ and its derivative are outgoing spherical waves \cite{bohren_absorption_1983}
\begin{align}
h_\ell^{(1)}(\rho)\approx (-i)^\ell \frac{\exp(i\rho)}{i\rho}
\qqtext{and}
\dv{h_\ell^{(1)}(\rho)}{\rho} \approx (-i)^\ell \frac{\exp(i\rho)}{\rho}
\end{align}
which are suited for the scattered field; the VSH with $z_\ell = h^{(1)}_\ell$ will be then, denoted with the superscript $(3)$.

Within this text, the VSH were define in Eq. \eqref{eq:VSH} under the condition of being a solution to the vectorial Helmholtz equation, which lead to the generating function $\psi$ to be a solution to the scalar Helmholtz equation, nevertheless there are other definitions as discussed by \citeauthor{barrera_vector_1985} \cite{barrera_vector_1985}. The chosen definition of the VSH allows the VSH to be interpreted directly as electric and magnetic fields, specifically identifying $\vb{N}$ with the electric contribution and $\vb{M}$ with the magnetic, as it will be shown in the following sections.
