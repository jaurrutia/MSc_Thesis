% !TeX root = ../tesis.tex



\subsubsection{Sommerfeld's Radiation Condition}

Arnold Sommerfeld firts stated its radiation condition in order to guarantee the uniqueness of the solution to the scalar Helmholtz equation [Eq. \eqref{eq:HelmoltzScalar}] applied to the scalar potential $\phi = \phi(\vb{r})$. Its condition for uniqueness can be found by substituting the scalar potential, generated by bounded sources, and the Green's function to the scalar Helmholtz equation in a 3D domain $\Omega$, given by \cite{zangwill_modern_2013,jackson_classical_1999}
%
\begin{align}
    \qty(\nabla^2 + k^2) g(\vb{r}|\vb{r}')   = \delta(\vb{r}-\vb{r}')  \iff
    g(\vb{r}|\vb{r}')   = \frac{\exp[i\vb{k}\cdot(\vb{r}-\vb{r}')]}{\norm{\vb{r}-\vb{r}'}},
\end{align}
%
into Green's second identity\footnote{%
        Let $\psi$ and $\phi$ be $\mathcal{C}^2$ in $\Omega$, then %
        $\int_\Omega (\psi\nabla^2\psi-\phi\nabla^2\psi)\dd[3]{r} = %
        \oint_{\partial\Omega}\vu{n}\cdot(\psi\nabla\phi-\phi\nabla\psi)\dd[2]{r}$, with $\vu{n}$ a normal vector to the boundary $\delta\Omega$  \cite{zangwill_modern_2013}.} %
 leading to
\begin{align}
    4\pi \phi(\vb{r}) =& \oint_{\partial\Omega}  \frac{\exp[i\vb{k}\cdot(\vb{r}-\vb{r}')]}{\norm{\vb{r}-\vb{r}'}^2} \phi(\vb{r}) \dd[2]{r}
                 \oint_{\partial\Omega} \frac{\dd[2]{r}}{\norm{\vb{r}-\vb{r}'}^2} \qty[\norm{\vb{r}-\vb{r}'}\qty(\pdv{ \phi(\vb{r})}{r}-ik\cdot\vu{e}_r \phi(\vb{r})) ].
    \label{eq:pot-Green}
\end{align}
%
The boundary condition that the potential $\phi$ must decay to zero as $r\to\infty$ sets Eq. \eqref{eq:pot-Green} equal to zero, which is  hold immediately by its the left hand side. The first integral in the right hand side of Eq. \eqref{eq:pot-Green} equals zero since $\phi(r\to 0 ) = 0$ while $\dd[2]{r}/\norm{\vb{r}-\vb{r}}\approx\dd[2]{r}/r^2$ remains bounded, thus the term between brackets in the second integral must equal zero, that is
%
\begin{align}
    \lim_{r\to \infty} r\qty(\pdv{\phi(\vb{r})}{r} - ik\phi(\vb{r})) = 0,
    \label{eq:SommScal}
\end{align}
%
where it was assumed that $\norm{\vb{r}-\vb{r}'} \approx r$ and $\vb{k} = k \vu{e}_r$. The Eq. \eqref{eq:SommScal} is known as the Sommerfeld Radiation condition since it states that in the far field there can only be outgoing waves that decay uniformly in all directions.

A generalization of the Eq. \eqref{eq:SommScal} to the electric field $\vb{E}$ and the $\vb{H}$ field, both of which are solution to the vectorial Helmholtz equation [Eq. \eqref{eq:Helmholtz}], can be obtained as follows: Substitution of  $\vb{E}$ ---or $\vb{H}$--- and the vectorial Green's function into the generalization for vector fields of Green's second identity\footnote{This can be obtained if  Eq. \eqref{eq:curlcurlE} is subtracted by itself interchanging $\boldsymbol{\eta}_j $ and $\vb{E}$ and setting $\boldsymbol{\eta}_i(\vb{r}) = \vb{n}g(\vb{r}|\vb{r}')$ as explained in \cite{zangwill_modern_2013}}. Yet, an equivalent derivation is achieved by separating the electromagnetic fields into two contributions and introducing  two vectorial potentials according to the origin of their sources: magnetic and electric  charges and currents induced into a bounded volume \cite{jin_theory_2010,bondeson_computational_2005}. Under such considerations, the electromagnetic fields can be written as:
%
\begin{align}
\vb{E} = \vb{E}_\text{e} + \vb{E}_\text{m},
    \qquad
    \text{and}
    \qquad
\vb{H} = \vb{H}_\text{e} + \vb{H}_\text{m},
\label{eq:EemHem}
\end{align}
%
where the subscript 'e' ('m') stands for the electric (magnetic) sources. Substitution of Eqs. \eqref{eq:EemHem} into the time harmonic Maxwells' equation [Eqs. \eqref{eq:MaxwellsEq}] leads to
%
\begin{subequations}
    \label{eq:MaxwellsEq}
\begin{align}
    \nabla \cdot \qty(\varepsilon \vb{E}_\text{e})  &= \rho_\text{ext},
                    &\nabla \cdot \qty(\varepsilon \vb{E}_\text{m})  &= 0,\\
    \nabla \cdot  (\mu\vb{H}_\text{e})  &= 0,
                    & \nabla \cdot  (\mu\vb{H}_\text{m})  &= \rho_\text{m},\\
    \nabla \times \vb{E}_\text{e}  &= i\omega \mu \vb{H}_\text{e},
                    & \nabla \times \vb{E}_\text{m}  &= i\omega \mu \vb{H}_\text{m} + \vb{J}_\text{m},\\
    \nabla \times \vb{H}_\text{e}  &= \vb{J}_\text{ext} - i\omega \varepsilon \vb{E}_\text{e},
                & \nabla \times \vb{H}_\text{m}  &=- i\omega \varepsilon \vb{E}_\text{m},
\end{align}
\end{subequations}
where $\rho_\text{m}$ and $\vb{J}_\text{m}$ are  induced charge and current densities due to the magnetization of the sources \cite{jin_theory_2010}. From the magnetic Gauss law apply on $\mu\vb{H}_\text{e}$ it is defined the vector potential $\vb{A}$ and, analogously, the vector potential $\vb{F}$ arises from the electric Gauss law on $\varepsilon\vb{E}\text{m}$. Then, scalar potentials $\phi_\text{e}$ and $\phi_\text{m}$ for $\vb{E}_\text{e} $ and $\vb{H}_\text{m}$ are obtained from the Faraday-Lenz law and the Ampère-Maxwell law applied on them, accordingly. If the electric scalar and vector potentials are chosen so they follow the Lorenz gauge, and the same is imposed for the magnetic scalar and vector potential, that is $\nabla\cdot\vb{A} = -i\omega\mu\varepsilon \phi_\text{e}$ and $\nabla\cdot\vb{F} = -i\omega\mu\varepsilon \phi_\text{m}$, then the electromagnetic fields are given by
%
\begin{align}
    \vb{E} = - \frac{\nabla[\nabla\cdot\vb{A}]}{i\omega\varepsilon\mu} + i \omega\vb{A} + \frac{1}{\varepsilon}\nabla\times\vb{F},
    \qquad
    \text{and}
    \qquad
    \vb{H} =- \frac{\nabla[\nabla\cdot\vb{F}]}{i\omega\varepsilon\mu} + i \omega\vb{F} + \frac{1}{\mu}\nabla\times\vb{A}
    \label{eq:EMFields}
\end{align}
%
where the vectorial potentials $\vb{A}$ and $\vb{F}$ are also solution to Helmholtz equation on each component and thus can be expressed as
%
\begin{align}
    \vb{A} = \frac{\mu}{4\pi} \int_\Omega \vb{J}_\text{ext}  \frac{\exp[i\vb{k}\cdot(\vb{r}-\vb{r}')]}{\norm{\vb{r}-\vb{r}'}}\dd{\Omega'},
    \qquad
    \text{and}
    \qquad
    \vb{F} = \frac{\varepsilon}{4\pi} \int_\Omega \vb{J}_\text{m}  \frac{\exp[i\vb{k}\cdot(\vb{r}-\vb{r}')]}{\norm{\vb{r}-\vb{r}'}}\dd{\Omega'},
\end{align}
%
In the far field regime it follows that $\norm{\vb{r}-\vb{r}'}^{-1}\approx r^{-1}$ and $\vb{k}\cdot \vb{r} = kr$, therefore
%
\begin{subequations}
    \label{eq:AF}
\begin{align}
    \vb{A} &= \frac{\mu\exp(ikr)}{4\pi r}\vb{N},         &\text{with} \qquad \vb{N} = \int_\Omega \vb{J}_\text{ext}  \exp(-i\vb{k}\cdot\vb{r}')\dd{\Omega'}\\
    \vb{F} &= \frac{\varepsilon\exp(ikr)}{4\pi r}\vb{L}, &\text{with} \qquad \vb{L}= \int_\Omega \vb{J}_\text{m}  \exp(-i\vb{k}\cdot\vb{r}')\dd{\Omega'}
\end{align}
\end{subequations}
%
and that also the nabla operator $\nabla$ acts as $\nabla\to i\vb{k} = ik\vb{e}_r$ since the electric field can be written as a plane travelling in the $\vb{k}$ direction. Substituting Eqs. \eqref{eq:AF} into Eq. \eqref{eq:EMFields} leads to the following expressions for the electromagnetic field in the far field regime:
%
\begin{subequations}
    \label{eq:EHFar}
\begin{align}
    \lim_{r\to\infty}\vb{E} &= -i k\frac{\exp(ikr)}{4\pi r}
                \left[ \vu{e}_r\times\vb{L}-\sqrt{\frac{\mu}{\varepsilon}}  \Big(\vb{N}-(\vu{e}_r\cdot\vb{N})\vu{e}_r\Big) \right]
            \label{eq:EFar}\\
    \lim_{r\to\infty}\vb{H} &=i k\frac{\exp(ikr)}{4\pi r}
                \left[\sqrt{\frac{\varepsilon}{\mu}}  \Big(\vb{L}-(\vu{e}_r\cdot\vb{L})\vu{e}_r + \vu{e}_r\times\vb{N}\Big) \right]
             \label{eq:HFar}
\end{align}
\end{subequations}
where the dispersion relation $k^2 = \omega^2\mu\varepsilon$ was employed. From Eq. \eqref{eq:EHFar} it can be seen that the electromagnetic fields in the far field have no radial components. Also, by calculation the cross product $\vu{e}\times\vb{E}$ in the far field, an comparing with Eq. \eqref{eq:HFar} one obtains
%
\begin{align}
    \lim_{r\to\infty} \qty(\vu{e}_r\times\vb{E} - \sqrt{\frac{\mu}{\varepsilon}} \vb{H}) = 0,
    \label{eq:FarLimit}
\end{align}
%
which states that the electric field is perpendicular to the direction of propagation and to the $\vb{H}$ field in the far field and that their amplitudes hace a fixed ratio of $\sqrt{{\mu}/{\varepsilon}}$, known as the impedance of the medium. Lastly, Eq. \eqref{eq:FarLimit} can be rewritten in terms of only $\vb{E}$ with aide of the Faraday-Lenz law and relatio of dispersion for a plane wave, which yield to the generalization of the Eq. \eqref{eq:SommScal},
%
\begin{tcolorbox}[title = Generzlized Sommerfeld or Silver-Müller Radiation Condition, ams align, breakable ]
    \lim_{r\to \infty} r\qty(\nabla\times \vb{E} - i k \vu{e}_r\times\vb{E}) = 0
    \label{eq:SommVec}
\end{tcolorbox}%
 \noindent%
which is also known as the Silver-Müller radiation condition.

The implementation of the light scattering problem [Eq. \eqref{eq:Scatt-Weak-All}] into numerical methods, such as the FEM, has the disadvantage that it is a problem solved in an unbounded, or an open, domain. Nevertheless, the evaluation of Eq. \eqref{eq:SommVec} into $\partial\Omega$ guarantees that the obtained solution reproduces that of the light scattering. Since the Sommerfeld radiation  condition is a non-homogeneous Neumann boundary condition, it  requieres Eq. \eqref{eq:SommVec} to be evaluated at a surface, with a normal vector $\vu{n}$. Due to its integration on $\partial\Omega$, Sommerfeld radiation condition is mostly used when the scatterer into $\Omega$ is small relative to it and when the scattered electric field incides normally to the boundary.  Were any of these conditions are not met, another ABC be implemented.

    \subsubsection{Perfect Matching Layer}

    The PML is a domain that surrounds  $\Omega$, where Eq. \eqref{eq:Scatt-Weak-All} is to be solved, which has the property that any reflection on it is damped due to its geometric properties. In order to determine which conditions are needed to create such virtual material, let us use \textit{stretch} coordinates on the PML domain $\Omega_\text{PML}$ where it is assumed that the far field approximation of the electromagnetic fields is valid. Thus, the gradient in such coordinate system can be written as
    %
     \begin{align}
         \nabla_\text{s} \equiv \qty(\frac{\vu{e}_x }{s_x}\pdv{x} + \frac{\vu{e}_y}{s_y}\pdv{y} + \frac{\vu{e}_z}{s_z}\pdv{z}) \to \vb{k} = \frac{k_x}{s_x}\vu{e}_x + \frac{k_y}{s_y}\vu{e}_y  +\frac{k_y}{s_z}\vu{e}_z,
     \label{eq:kstretch}
     \end{align}
    %
    where the subscript 's' stands for \textit{stretch} and $\vb{k}$ is the wave vector of the traveling electric plane wave in the far field. The scale factors $s_{x_i}$, with $x_i \in \{x, y z\}$, depend only on the coordinate of its stretching direction, that is, $s_{x_i} = s_{x_i}(x_i)$. On the domain $\Omega$, outside the PML, the scale factors are equal to one. From Eq. \eqref{eq:kstretch}, the dispersion relation of a plane wave  is
    %
    \begin{align}
        \vb{k} \cdot \vb{k}  = k^2 = \mu\varepsilon\omega^2 =
            \qty(\frac{k_x}{s_x}^2 )+ \qty(\frac{k_y}{s_y})^2 + \qty(\frac{k_z}{s_z})^2
     \label{eq:k-reldisp}
    \end{align}
    %
    whose solution is given by
    %
    \begin{align}
        k_x = k s_x \sin\theta\cos\varphi, \qquad
            k_y = k s_y \sin\theta\sin\varphi, \qquad \text{and}\qquad
                k_z = k s_z \cos\theta.
     \label{eq:kstretchcomp}
    \end{align}
    %
    with $\varphi$ and $\theta$ the azimuthal and polar angles.

    Let us note that in between the domains $\Omega$ and $\Omega_\text{PML}$ there is a boundary which will be  locally assumed as a plane interface. The Fresnel's reflection amplitude  coefficient can be defined as usual since the ratio of the incident and the reflected electric field in the stretched coordinate system does not depend on any stretch coefficient $s_{x_i}$. From the continuity of the tangential component of the electric field, the reflection amplitude coefficients are
    %
    \begin{align}
       r_\text{s} = \frac{k^\text{(PML)}_z s^\text{(PML)}_z\mu_{{}_\text{PML}} - k^{(\Omega)}_z s^{(\Omega)}_z\mu_{{}_\Omega}}
                        {k^\text{(PML)}_z s^\text{(PML)}_z\mu_{{}_\text{PML}} + k^{(\Omega)}_z s^{(\Omega)}_z\mu_{{}_\Omega}},
           \quad
           \text{and}
           \quad
       r_\text{p} = \frac{k^\text{(PML)}_z s^\text{(PML)}_z\varepsilon_{{}_\text{PML}} - k^{(\Omega)}_z s^{(\Omega)}_z\varepsilon_{{}_\Omega}}
                        {k^\text{(PML)}_z s^\text{(PML)}_z\varepsilon_{{}_\text{PML}} + k^{(\Omega)}_z s^{(\Omega)}_z\varepsilon_{{}_\Omega}}.
       \label{eq:refl-Fresnel}
    \end{align}
   %
   where the $z$ component of the wave vector $\vb{k}$ is perpendicular to the interface.   Also, the phase matching condition on the interface between $\Omega$ and $\Omega_k$, states that
   %
   \begin{subequations}
       \label{eq:PhaseMatch}
    \begin{align}
       k^\text{(PML)} s^\text{(PML)}_x \sin\theta_{{}_\text{PML}}\cos\varphi_{{}_\text{PML}} =
                        k^{(\Omega)} s^{(\Omega)}_x \sin\theta_{{}_\Omega}\cos\varphi_{{}_\Omega},\\
        k^\text{(PML)} s^\text{(PML)}_y \sin\theta_{{}_\text{PML}}\sin\varphi_{{}_\text{PML}} =
                     k^{(\Omega)} s^{(\Omega)}_y \sin\theta_{{}_\Omega}\sin\varphi_{{}_\Omega}.
    \end{align}
    \end{subequations}
    \noindent
   %
   From Eqs. \eqref{eq:k-reldisp}--\eqref{eq:PhaseMatch} it can be seen that the reflection amplitude coefficient for both polarization states vanish if the magnetic permeability and the electric permittivity, as well as the stretch coefficient $s_x$ and $s_y$, of the PML matches those of the domain $\Omega$ since such conditions leads to  $\varphi_{{}_\text{PML}} = \varphi_{{}_\Omega}$ and  $\theta_{{}_\text{PML}} = \theta_{{}_\Omega}$.
