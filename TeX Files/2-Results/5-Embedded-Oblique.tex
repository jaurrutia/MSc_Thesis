% !TeX root = ../tesis.tex

In the past Section, the optical properties of a spherical AuNP with radius $a = 12.5$~nm were studied when the AuNP was located at the interface between a glass substrate and an air matrix, and illuminated in the normal direction to the glass-air interface. On the one hand, it was found that the AuNP can be still described by a mostly dipolar contribution, as in the Mie-limiting case (AuNP in a homogeneous medium), even though the homogeneity and symmetry of its surroundings is broken. On the other hand, the near-field enhancement on  the surface of the AuNP was greater in the side of the glass substrate than in the air matrix, which is undesired if the partially embedded AuNP is to be used as the unit cell for a bidimensional array suited for biosensing. In order to find an optimal configuration of the system suited for interactions above the substrate, in this Section the optical properties of a partially embedded AuNP are analyzed considering an oblique incidence to the system (both below and above the critical angle), meaning that the AuNP is illuminated either with an $s$ or a $p$ polarized incident electric field.

The absorption $Q_\text{abs}$ and scattering $Q_\text{sca}$ efficiencies of a partially embedded  AuNP with radius $a =12.5$~nm in a glass substrate ($n_\text{s} = 1.5$) and in an air matrix ($n_\text{m} = 1$) ---for different values of the incrustation parameter $h/a$, with $h$ the distance between the center of the AuNP and the interface--- are shown in Figs. \ref{fig:Inc:Abs} and \ref{fig:Inc:Sca}, respectively, as a function of the wavelength $\lambda$ of the incident electric field $\vb{E}^\text{i}$ illuminating the AuNP from the substrate at an incidence angle $\theta_i < \theta_c = 41.8^\circ$ of $15^\circ$ [Figs. \ref{sfig:Inc:Abs:15} and \ref{sfig:Inc:Sca:15}] and  $38^\circ$ [Figs. \ref{sfig:Inc:Abs:38} and \ref{sfig:Inc:Sca:38}], and  at a value of $\theta_i>\theta_c$, thus forcing the interaction between an evanescent wave and the AuNP, equal to $42^\circ$  [Figs. \ref{sfig:Inc:Abs:42} and \ref{sfig:Inc:Sca:42}] and $75^\circ$  [Figs. \ref{sfig:Inc:Abs:75} and \ref{sfig:Inc:Sca:75}], considering an $s$ (filled circles/solid lines) and a $p$ (empty circles/dashed lines) polarization for $\vb{E}^\text{i}$. To compare the obtained results with the Mie-limiting cases (AuNP embedded in air and in glass) ---green shaded region and cyan markers corresponding to the resonance of the absorption and scattering efficiencies---, the values of $Q_\text{abs}$ ($Q_\text{sca}$) evaluated at their wavelength of resonance $\lambda_\text{res}^\text{abs}$ ($\lambda_\text{res}^\text{sca}$) are signalized by the magenta markers and the numerical values of the later can be found in Table \ref{tab:Resonances}, where the saturation of the cell colors corresponds to a larger wavelength. Lastly, in Figs. \ref{fig:Inc:Abs} and \ref{fig:Inc:Sca} the gray continuous lines are a guide to the eye joining the resonances of the absorption and scattering efficiencies in each case considering an $s$ and a $p$ polarized incident electric field.

\begin{figure}[t!]\footnotesize \centering
	\vspace*{3.5em}
    \hspace*{-.65\textwidth}%
        \begin{subfigure}{.735\textwidth}\caption{ }\label{sfig:Inc:Abs:15}\end{subfigure}%
        \begin{subfigure}{.25\textwidth}\caption{ }\label{sfig:Inc:Abs:38}\end{subfigure} \\[19.5em]
    \hspace*{-.65\textwidth}%
        \begin{subfigure}{.735\textwidth}\caption{ }\label{sfig:Inc:Abs:42}\end{subfigure}%
        \begin{subfigure}{.25\textwidth}\caption{ }\label{sfig:Inc:Abs:75}\end{subfigure} \\[-28em]
    \def\svgwidth{.95\textwidth}
    \includeinkscape{4-Inc-Obl/1-Efficiencies/2-Oblique-Inc-Abs}%
    \vspace*{1.25em}
    \caption[Absorption Efficiency of a partially embedded 12.5 nm AuNP into a substrate Illuminated in an internal configuration at oblique incidence]{%
    Absorption efficiency of a $12.5$ nm AuNP partially embedded in a glass substrate ($n_\text{s} = 1.5$) with an air matrix ($n_\text{m} = 1$), as a function of the wavelength $\lambda$ of an \textit{s} (filled circles/solid lines) and a \textit{p} (empty circles/dashed lines) polarized incident electromagnetic plane wave propagating in the direction of the wave vector $\vb{k}^\text{i}$, in an internal configuration, at an angle of incidence $\theta_i$ of \textbf{a)} $15^\circ$, \textbf{b)} $38^\circ$, \textbf{c)} $42^\circ$, and \textbf{d)} $75^\circ$  relative to the normal direction to the glass-air interface. The green shaded region shows the two Mie-limiting cases of a AuNP embedded either in air or in glass; the magenta (partially embedded AuNP) and cyan (Mie-limiting) markers correspond to the efficiencies evaluated at the wavelength of resonance for each case; the gray  line is a guide  to the eye.
}
\label{fig:Inc:Abs}
\end{figure}

In Figs. \ref{fig:Inc:Abs} and \ref{fig:Inc:Sca} it can be seen that the absorption and scattering efficiencies, for each combination of $\theta_i$ and $h/a$, present a general redshift of the resonance wavelengths $\lambda_\text{res}^\text{abs}$ and $\lambda_\text{res}^\text{sca}$ as $h/a$ decreses, while preserving only one appreciable resonance in the visible range in the intervals $509\text{ nm} < \lambda_\text{res}^\text{abs} < 535 \text{ nm}$ and  $522\text{ nm} < \lambda_\text{res}^\text{sca} < 545 \text{ nm}$, whose extreme values (cyan markers) correspond to the resonance wavelength for the Mie-limiting cases: AuNP in air and in glass. Besides the redshift of the resonances, it is observed an overall growth ---or decrease, depending on the choice of incrustation parameter and angle of incidence--- for all wavelengths on the absorption and scattering efficiencies, which are integral optical properties and thus an average response of the system. Therefore, these changes can be identified as the combination of the effects due to  the AuNP interacting with a plane or an evanescent wave, as discussed in Section \ref{s:Emb:Obl}, and due to the embedding of the AuNP, discussed in Section \ref{s:Emb:Normal}.

A general effect of the choice of $\theta_i$ on $Q_\text{abs}$ and $Q_\text{sca}$ is their enhancement, in the visible range, for incidence angles near the critical angle $\theta_c = 41.8^\circ$, as it can be seen from the absorption and scattering efficiencies, for both the $s$ and $p$ polarization cases, at $\theta_i = 42^\circ$ [Figs. \ref{sfig:Inc:Abs:42} and \ref{sfig:Inc:Sca:42}], which are larger, for all $\lambda$, than the equivalent results for incidence angles of $15^\circ$, $38^\circ$, and $75^\circ$. This observation can be verified by comparing the vertical axis scale with a maximum value of $\sim 10.0$ in Fig. \ref{sfig:Inc:Abs:42} and  $\sim 10.1 \times 10^{-2} $ Fig. \ref{sfig:Inc:Sca:42} ---absorption and scattering efficiencies, respectively--- with the maximum value of the axis scale for $\theta_i = 15^\circ$ in Figs. \ref{sfig:Inc:Abs:15} ($\sim 1.2$) and \ref{sfig:Inc:Sca:15} ($\sim 5.3 \times 10^{-2}$) and for $\theta_i = 38^\circ$ in  Figs. \ref{sfig:Inc:Abs:38} ($\sim 4.1$) and \ref{sfig:Inc:Sca:38} ($\sim 9.1  \times 10^{-2}$) ---both angles below $\theta_c$---; on a similar manner, the shown absorption and scattering efficiencies, for a fixed polarization state and incrustation parameter, are the smallest for $\theta_i = 75^\circ$ as shown by the maximum value of the vertical axes of $\sim 1.0$ in Fig. \ref{sfig:Inc:Abs:75} and of $\sim 1.0\times 10^{-2}$ in Fig. \ref{sfig:Inc:Sca:75}. A similar analysis based on the Mie-limiting cases can be performed by comparing the values of the continuous curves relative to the green shaded region, yielding the same results.

To notice the effect of the  polarization state ---modulated by the AuNP's embedding--- of the incident electric field $\vb{E}^\text{i}$ in the absorption and scattering efficiencies, let us recall the behavior observed for a normally illuminated partially embedded AuNP in Section \ref{s:Emb:Normal}, where the values of $Q_\text{abs}$ and $Q_\text{sca}$ grew uniformly, for all wavelengths in the visible rage, as the incrustation parameter $h/a$ changes from $1$ to $-1$, that is, as the AuNP is buried into the substrate. Such behavior can be identified in the continuous gray curves for any angle of incidence in Figs. \ref{fig:Inc:Abs} (absorption efficiency) and \ref{fig:Inc:Sca} (scattering efficiency). This can be explained by the direction of the electric field not changing across the glass-air interface due to the continuity of the parallel component of the electric filed at any boundary: for an $s$ polarized $\vb{E}^\text{i}$, the electric field illuminating the partially AuNP only changes in amplitude on its boundary, even for $\theta_i>\theta_c$, yielding a smooth increase in the average optical properties ($Q_\text{abs}$ and $Q_\text{sca}$) of the AuNP as its surroundings become optically denser.

\begin{figure}[t!]\footnotesize \centering
	\vspace*{3em}
    \hspace*{-.67\textwidth}%
        \begin{subfigure}{.735\textwidth}\caption{ }\label{sfig:Inc:Sca:15}\end{subfigure}%
        \begin{subfigure}{.25\textwidth}\caption{ }\label{sfig:Inc:Sca:38}\end{subfigure} \\[21.5em]
    \hspace*{-.67\textwidth}%
        \begin{subfigure}{.735\textwidth}\caption{ }\label{sfig:Inc:Sca:42}\end{subfigure}%
        \begin{subfigure}{.25\textwidth}\caption{ }\label{sfig:Inc:Sca:75}\end{subfigure} \\[-29.5em]
    \def\svgwidth{.95\textwidth}
    \includeinkscape{4-Inc-Obl/1-Efficiencies/3-Oblique-Inc-Sca}%
    \vspace*{1.25em}
    \caption[Scattering Efficiency of a partially embedded 12.5 nm AuNP into a substrate Illuminated in an internal configuration at oblique incidence]{%
    Scattering efficiency of a $12.5$ nm AuNP partially embedded in a glass substrate ($n_\text{s} = 1.5$) with an air matrix ($n_\text{m} = 1$), as a function of the wavelength $\lambda$ of an \textit{s} (filled circles/solid lines) and a \textit{p} (empty circles/dashed lines) polarized incident electromagnetic plane wave propagating in the direction of the wave vector $\vb{k}^\text{i}$, in an internal configuration, at an angle of incidence $\theta_i$ of \textbf{a)} $15^\circ$, \textbf{b)} $38^\circ$, \textbf{c)} $42^\circ$, and \textbf{d)} $75^\circ$  relative to the normal direction to the glass-air interface. The green shaded region shows the two Mie-limiting cases of a AuNP embedded either in air or in glass; the magenta (partially embedded AuNP) and cyan (Mie-limiting) markers correspond to the efficiencies evaluated at the wavelength of resonance for each case; the gray  line is a guide  to the eye.
}
\label{fig:Inc:Sca}
\end{figure}

Contrastingly, the efficiencies for the $p$ polarization case (empty circles/dashed lines) are uniformly enhanced as $h/a$ decreases ---as in the $s$ polarization case--- only for the angles of incidence of $15^\circ$ [Figs. \ref{sfig:Inc:Abs:15} and \ref{sfig:Inc:Sca:15}] and $75^\circ$ [Figs. \ref{sfig:Inc:Abs:15} and \ref{sfig:Inc:Sca:15}], while for values of $\theta_i$ near $\theta_c = 41.8^\circ$, the dependency of the overall growth of $Q_\text{abs}$ and $Q_\text{sca}$ on $h/a$ is as follows: the absorption efficiency for values of $\theta_i $ in the neighborhood of $\theta_c$, such as for $38^\circ$ and $42^\circ$ [Figs. \ref{sfig:Inc:Abs:38} and \ref{sfig:Inc:Abs:42}, respectively] present an uniform enhancement ---for all $\lambda$--- when $1 \leq h/a < 0.75$ and $0< h/a \leq -1$ and an uniform diminishment as $h/a$ changes from $0.75$ (orange dashed curve) to $0.50$  (blue dashed curve)  and $0.25$  (light orange dashed curve); a similar observation is made for the sactteringe efficiency for $\theta_i = 38 ^\circ$ and $42 ^\circ$   [Figs. \ref{sfig:Inc:Sca:38} and \ref{sfig:Inc:Sca:42}, respectively] where the diminishment of $Q_\text{sca}$ occurs at a incrustation parameter of $0.50 \leq h/a \leq -0.50$ and the enhancement when $h/a$ changes from $1$ to $0.75$ (black and orange dashed curves) and from $-0.75$ to $-1$  (light purple and brown dashed curves). Both of the aforementioned descriptions of the uniform enhancement and diminishment of $Q_\text{abs}$ and $Q_\text{sca}$ can be easily seen by following the gray lines joining their resonances as the incrustation parameters is changed. Attributing such behavior to the change of the electric field illuminating the AuNP above and below the glass-air interface, it can be seen that for a $p$ polarized incident electric field, the scattering and absorption efficiencies of the partially embedded AuNP are similar to that of the supported (totally embedded) AuNP, due to the strength of the transmitted electric field, under determined values of the incrustation parameter: when $1<h/a<0.75$  ($-0.75>h/a>-1$), that is, when one eight of the AuNP's volume/surface is embedded in the substrate (matrix).

\begin{table}[b!]\footnotesize\centering
    \caption{Wavelength of resonance for the absorption $\lambda_\text{res}^\text{abs}$ and the scattering $\lambda_\text{res}^\text{sca}$ efficiencies of a partially embedded 12.5 nm AuNP with a glass substrate ($n_\text{s} = 1.5$) and an air matrix ($n_\text{m} = 1$), illuminated by an \textit{s} and a \textit{p} polarized electromagnetic plane wave traveling to the glass-air interface at an incidence angle of $0^\circ$, $15^\circ$,  $38^\circ$,  $42^\circ$ and  $75^\circ$, for several values of the incrustation parameter $h/a$ with $h$ the distance between the AuNP and its radius $a$. The values in this table correspond to the magenta markers in Figs. \ref{fig:Inc:Eff}, \ref{fig:Inc:Abs} and \ref{fig:Inc:Sca} while the saturation of the cell colors are a guide to the eye for the wavelength shift.}
    \label{tab:Resonances}
    

\begin{tabular}{l | r | ccccc || ccccc} \hline \hline
                                &       & \multicolumn{5}{c ||}{\small $\lambda_\text{res}^\text{abs}$ [nm]} & \multicolumn{5}{c}{\small $\lambda_\text{res}^\text{sca}$ [nm]}  \\ \hline \hline
                                & $h/a$ & $0^\circ$ & $15^\circ$     & $38^\circ$    & $42^\circ$    & $75^\circ$    & $0^\circ$ & $15^\circ$     & $38^\circ$    & $42^\circ$    & $75^\circ$    \\ \hline
\multirow{9}{*}{\rotatebox{90}{\emph{s} Polarization}}
    & 1.00  & \cellcolor{white!92!gray}510    & \cellcolor{white!92!gray}510    & \cellcolor{white!92!gray}510    & \cellcolor{white!92!gray}510    & \cellcolor{white!84!gray}512.5  & \cellcolor{white!89!orange}525    & \cellcolor{white!89!orange}525    & \cellcolor{white!89!orange}525    & \cellcolor{white!89!orange}525    & \cellcolor{white!89!orange}525    \\
    & 0.75  & \cellcolor{white!76!gray}515    & \cellcolor{white!76!gray}515    & \cellcolor{white!76!gray}515    & \cellcolor{white!76!gray}515    & \cellcolor{white!76!gray}515    & \cellcolor{white!89!orange}525    & \cellcolor{white!89!orange}525    & \cellcolor{white!89!orange}525    & \cellcolor{white!89!orange}525    & \cellcolor{white!89!orange}525    \\
    & 0.50  & \cellcolor{white!60!gray}520    & \cellcolor{white!60!gray}520    & \cellcolor{white!60!gray}520    & \cellcolor{white!60!gray}520    & \cellcolor{white!60!gray}520    & \cellcolor{white!67!orange}530    & \cellcolor{white!67!orange}530    & \cellcolor{white!67!orange}530    & \cellcolor{white!67!orange}530    & \cellcolor{white!67!orange}530    \\
    & 0.25  & \cellcolor{white!52!gray}522.5  & \cellcolor{white!52!gray}522.5  & \cellcolor{white!52!gray}522.5  & \cellcolor{white!52!gray}522.5  & \cellcolor{white!52!gray}522.5  & \cellcolor{white!45!orange}535    & \cellcolor{white!45!orange}535    & \cellcolor{white!45!orange}535    & \cellcolor{white!45!orange}535    & \cellcolor{white!45!orange}535    \\
    & 0.00  & \cellcolor{white!44!gray}525    & \cellcolor{white!44!gray}525    & \cellcolor{white!44!gray}525    & \cellcolor{white!44!gray}525    & \cellcolor{white!44!gray}525    & \cellcolor{white!23!orange}537.5  & \cellcolor{white!23!orange}537.5  & \cellcolor{white!23!orange}537.5  & \cellcolor{white!45!orange}535    & \cellcolor{white!45!orange}535    \\
    & -0.25 & \cellcolor{white!28!gray}527    & \cellcolor{white!28!gray}527    & \cellcolor{white!28!gray}527    & \cellcolor{white!28!gray}527    & \cellcolor{white!28!gray}527    & \cellcolor{white!45!orange}535    & \cellcolor{white!45!orange}535    & \cellcolor{white!45!orange}535    & \cellcolor{white!45!orange}535    & \cellcolor{white!45!orange}535    \\
    & -0.50 & \cellcolor{white!12!gray}530    & \cellcolor{white!12!gray}530    & \cellcolor{white!12!gray}530    & \cellcolor{white!12!gray}530    & \cellcolor{white!12!gray}530    & \cellcolor{white!23!orange}537.5  & \cellcolor{white!23!orange}537.5  & \cellcolor{white!23!orange}537.5  & \cellcolor{white!23!orange}537.5  & \cellcolor{white!23!orange}537.5  \\
    & -0.75 & \cellcolor{white!12!gray}530    & \cellcolor{white!12!gray}530    & \cellcolor{white!12!gray}530    & \cellcolor{white!12!gray}530    & \cellcolor{white!12!gray}530    & \cellcolor{white!12!orange}540    & \cellcolor{white!12!orange}540    & \cellcolor{white!12!orange}540    & \cellcolor{white!12!orange}540    & \cellcolor{white!12!orange}540    \\
    & -1.00 & \cellcolor{white!8!gray}532.5   & \cellcolor{white!8!gray}532.5   & \cellcolor{white!8!gray}532.5   & \cellcolor{white!8!gray}532.5   & \cellcolor{white!8!gray}532.5   & \cellcolor{white!1!orange}542.5   & \cellcolor{white!1!orange}542.5   & \cellcolor{white!1!orange}542.5   & \cellcolor{white!1!orange}542.5   & \cellcolor{white!12!orange}540    \\
\hline\hline
\multirow{9}{*}{\rotatebox{90}{\emph{p} Polarization}}
    & 1.00  & \cellcolor{white!92!gray}510    & \cellcolor{white!84!gray}512.5  & \cellcolor{white!84!gray}512.5  & \cellcolor{white!84!gray}512.5  & \cellcolor{white!84!gray}512.5  & \cellcolor{white!89!orange}525    & \cellcolor{white!89!orange}525    & \cellcolor{white!78!orange}527.5  & \cellcolor{white!78!orange}527.5  & \cellcolor{white!89!orange}525    \\
    & 0.75  & \cellcolor{white!76!gray}515    & \cellcolor{white!76!gray}515    & \cellcolor{white!68!gray}517.5  & \cellcolor{white!68!gray}517.5  & \cellcolor{white!68!gray}517.5  & \cellcolor{white!89!orange}525    & \cellcolor{white!89!orange}525    & \cellcolor{white!78!orange}527.5  & \cellcolor{white!78!orange}527.5  & \cellcolor{white!78!orange}527.5  \\
    & 0.50  & \cellcolor{white!60!gray}520    & \cellcolor{white!60!gray}520    & \cellcolor{white!68!gray}517.5  & \cellcolor{white!68!gray}517.5  & \cellcolor{white!60!gray}520    & \cellcolor{white!67!orange}530    & \cellcolor{white!67!orange}530    & \cellcolor{white!67!orange}530    & \cellcolor{white!78!orange}527.5  & \cellcolor{white!67!orange}530    \\
    & 0.25  & \cellcolor{white!52!gray}522.5  & \cellcolor{white!36!gray}525.5  & \cellcolor{white!60!gray}520    & \cellcolor{white!68!gray}517.5  & \cellcolor{white!36!gray}525.5  & \cellcolor{white!45!orange}535    & \cellcolor{white!45!orange}535    & \cellcolor{white!56!orange}532.5  & \cellcolor{white!56!orange}532.5  & \cellcolor{white!56!orange}532.5  \\
    & 0.00  & \cellcolor{white!44!gray}525    & \cellcolor{white!44!gray}525    & \cellcolor{white!52!gray}522.5  & \cellcolor{white!60!gray}520    & \cellcolor{white!52!gray}522.5  & \cellcolor{white!23!orange}537.5  & \cellcolor{white!45!orange}535    & \cellcolor{white!45!orange}535    & \cellcolor{white!45!orange}535    & \cellcolor{white!45!orange}535    \\
    & -0.25 & \cellcolor{white!28!gray}527    & \cellcolor{white!20!gray}527    & \cellcolor{white!44!gray}525    & \cellcolor{white!52!gray}522.5  & \cellcolor{white!44!gray}525    & \cellcolor{white!45!orange}535    & \cellcolor{white!45!orange}535    & \cellcolor{white!56!orange}532.5  & \cellcolor{white!34!orange}535.5  & \cellcolor{white!23!orange}537.5  \\
    & -0.50  & \cellcolor{white!12!gray}530    & \cellcolor{white!12!gray}530    & \cellcolor{white!20!gray}527    & \cellcolor{white!44!gray}525    & \cellcolor{white!20!gray}527    & \cellcolor{white!23!orange}537.5  & \cellcolor{white!23!orange}537.5  & \cellcolor{white!23!orange}537.5  & \cellcolor{white!45!orange}535    & \cellcolor{white!23!orange}537.5  \\
    & -0.75 & \cellcolor{white!12!gray}530    & \cellcolor{white!12!gray}530    & \cellcolor{white!12!gray}530    & \cellcolor{white!20!gray}527    & \cellcolor{white!12!gray}530    & \cellcolor{white!12!orange}540    & \cellcolor{white!12!orange}540    & \cellcolor{white!12!orange}540    & \cellcolor{white!45!orange}535    & \cellcolor{white!12!orange}540    \\
    & -1.00 & \cellcolor{white!8!gray}532.5   & \cellcolor{white!8!gray}532.5   & \cellcolor{white!12!gray}530    & \cellcolor{white!12!gray}530    & \cellcolor{white!12!gray}530    & \cellcolor{white!1!orange}542.5   & \cellcolor{white!1!orange}542.5   & \cellcolor{white!12!orange}540    & \cellcolor{white!23!orange}537.5  & \cellcolor{white!12!orange}540    \\
\hline\hline
\end{tabular}

\end{table}

The oblique illumination of a partially embedded AuNP of radius $a=12.5$ nm with different values of the incrustation parameter $h/a$ leads, besides to the uniform increase (decrease) of the absorption and scattering efficiencies discussed above, to a redshift of the absorption and scattering wavelengths of resonance, $\lambda_\text{res}^\text{abs}$ and $\lambda_\text{res}^\text{sca}$, which are shown in Table \ref{tab:Resonances} ---for all considered values of $h/a$, $\theta_i$ and both polarization sates--- with aid of the saturation of the cell color, which is greater the larger the value of the resonance wavelength by case: $\lambda_\text{res}^\text{abs}$ and $\lambda_\text{res}^\text{sca}$ in either $s$ or $p$ polarization. From the reported values in Table \ref{tab:Resonances}, both the absorption and scattering resonances are spectrally located in between the resonances for the Mie-limiting results (cyan markers in Figs. \ref{fig:Inc:Abs} and \ref{fig:Inc:Sca}) and that resonances for the partially embedded AuNP are, in general, redshifted as the embedding of the AuNP increases ($h/a$ changes from $1$ to $-1$). In particular, the redshift of $\lambda_\text{res}^\text{abs}$ (gray cells) and $\lambda_\text{res}^\text{sca}$ (orange cells) for an $s$ polarized incident electric field (upper block in Table \ref{tab:Resonances}) is independent of the angle of incidence and the rate of change of the redshift in relation to the incrustation parameter is a uniform function. The observed rate of change of the redshift for the $p$ polarization case (lower block in Table \ref{tab:Resonances}) as the incrustation parameter decreases, has a similar behavior for the absorption and scattering resonances just as the one observed for the $s$ polarization case nevertheless, this rate for the $p$ polarized illumination of the AuNP is different for each incident angle unlike its counterpart. On the one hand, for values of $\theta_i$ far from $\theta_c$ the dependence of  $\lambda_\text{res}^\text{abs}$ and $\lambda_\text{res}^\text{sca}$ on $h/a$ for $p$ polarization resembles that for $s$ polarization, as it can be seen not only in the color gradient in Table \ref{tab:Resonances} but also in the comparison between the gray continuous lines in Figs. \ref{sfig:Inc:Abs:15} and \ref{sfig:Inc:Sca:15} for $\theta_i = 15^\circ$ and in Figs. \ref{sfig:Inc:Abs:75} and \ref{sfig:Inc:Sca:75} for $\theta_i = 75^\circ$. On the other hand, for $\theta_i$ in the neighborhood of $\theta_c$, the rate of the redshift as the AuNP is buried into the substrate is larger for values of $h/a<0$ than for $h/a>0$ and, additionally, this change in the rate is more notorious for $\theta_i = 42^\circ\gtrsim \theta_c$ than for $38^\circ\lesssim \theta_c$. The redshift for both $s$ and $p$ polarizations can be explained by the directions of electric field below and above the glass-substrate interface. For $s$ polarization the electric field is parallel to the interface and due to its continuity across boundaries there is no change in its direction for any $\theta_i$, thus the uniform redshift arises. For $p$ polarization, there are both parallel and perpendicular components of the electric field relative to the glass-air interface; the perpendicular component of the electric field in the substrate is antiparallel to that in the matrix, and thus there is a competition among the perpendicular components of the electric field when it is integrated at the surface of the AuNP (to calculate $Q_\text{sca}$) and in its volume (to calculate $Q_\text{abs}$) when $h/a$  is around zero, that is, when there is no major part of the AuNP in one medium.

Both the uniform enhancement  of $Q_\text{abs}$ and $Q_\text{sca}$  and the  redshift of their resonance wavelength  are expected to behave similarly, for a fixed angle of incidence and polarization, since the absorption and scattering efficiencies are quantities calculated by a volume and a surface integral given by  Eqs. \eqref{eq:Cabs} and \eqref{eq:Csca}, respectively,  and since both the volume and surface fraction of the AuNP in the substrate, relative to its total volume or surface, are given by the same expression $(1-h/a)/2$. The uniform changes on $Q_\text{abs}$ and $Q_\text{sca}$ and their resonance wavelength for the $s$ polarization case as the AuNP is buried suggest that the average optical properties of the partially embedded AuNP are determined, for the $s$ polarization case, solely by the fraction of the AuNP embedded in either medium between the substrate and the matrix, while for $p$ polarization the direction and magnitude of the transmitted electric field are to be take into account if more (less) than one eight of the AuNP is in the substrate, leading to a rapid (slow) change of its average optical properties as $h/a$ changes. The past summary of the discussion on Figs. \ref{fig:Inc:Abs} and \ref{fig:Inc:Sca}, and Table \ref{tab:Resonances} describes the average optical properties of a partially embedded 12.5 nm AuNP given by $Q_\text{abs}$ and $Q_\text{sca}$; to have a better understanding of all the optical properties of such system, the spatial distribution of the induced electric field, in the far and near-field regimes, is to be analyzed.

The radiation patterns of a partially embedded 12.5 AuNP shown in Fig. \ref{fig:Far:Inc:s} consider that the AuNP is illuminated at an angle of incidence ---above the critical angle--- of $42^\circ$ [Figs. \ref{sfig:Far:Inc:s:a} and \ref{sfig:Far:Inc:s:b}] and of $75^\circ$ [Figs. \ref{sfig:Far:Inc:s:c} and \ref{sfig:Far:Inc:s:d}] by an $s$ polarized incident electric field with a wavelength $\lambda = 525$ nm, the resonance wavelength for the incrustation parameter $h/a = 0$; in Figs. \ref{sfig:Far:Inc:s:a} and   \ref{sfig:Far:Inc:s:c} the radiation patterns are evaluated at a scattering plane perpendicular to the plane of incidence (vertical gray dotted line), while it overlaps to it in  Figs. \ref{sfig:Far:Inc:s:b} and   \ref{sfig:Far:Inc:s:d}. The radiation patterns for $\theta_i<\theta_c$ were omitted since they follow the same tendency as the one presented in Fig. \ref{fig:Far:IncNorm} due to the incident electric field being parallel to the interface between the substrate and the matrix.

What is observed in Fig. \ref{fig:Far:Inc:s} is that the radiation patterns for different values of the incrustation parameters $h/a$, at fixed angle of incidence and $s$ polarization, present the two and one-lobe shapes with an asymmetry due to the substrate, as discussed in the Section \ref{s:Emb:Normal} for a normally illuminated AuNP. As mentioned, the average amplitude of the radiation pattern for oblique incidence is larger the more embedded into the substrate  the AuNP is, for example, as $h/a$ changes from $0.75$ (orange curves) to $-0.75$ (light purple curves). Additionally, the radiation pattern is modulated by the magnitude of the efficiencies,  thus it is expected that the scattered far-field has a shorter extent for an angle of incidence farther from $\theta_c$, that is the case when Figs. \ref{sfig:Far:Inc:s:a} and  \ref{sfig:Far:Inc:s:b} ($\theta_i =42^\circ$) are compared with Figs. \ref{sfig:Far:Inc:s:c} and  \ref{sfig:Far:Inc:s:d} ($\theta_i =75^\circ$). These results are in agreement with the discussion on the optical properties of a partially embedded AuNP  illuminated with an $s$ polarization incident electric field: their properties change uniformly with the incrustation parameter including the far-field distribution as well as the absorption and scattering efficiencies, due to the direction of the transmitted electric field not changing above and below the glass-air interface.

\begin{figure}[h!]
    \centering
    \def\svgwidth{.8\textwidth}
    \includeinkscape[pretex = \footnotesize]{4-Inc-Obl/4-FarXY-S/4-5-Far-XY-S-42}\\[-16.7em]
    \hspace*{-.2\textwidth}%
        \begin{subfigure}{.4\textwidth}\caption{%
                    \footnotesize$\dfrac{\norm{\vb{E}^\text{sca}_\text{far}}}{\norm{\vb{E}^\text{i}}} \; [10^{-9}]$  }\label{sfig:Far:Inc:s:a}\end{subfigure}%
        \begin{subfigure}{.4\textwidth}\caption{%
                    \footnotesize$\dfrac{\norm{\vb{E}^\text{sca}_\text{far}}}{\norm{\vb{E}^\text{i}}} \; [10^{-9}]$  }\label{sfig:Far:Inc:s:b}\end{subfigure}\\[13.75em]
    %
    \def\svgwidth{.8\textwidth}
    \hspace*{-.21\textwidth}%
    \vspace*{-.85em}%
        \begin{subfigure}{.4\textwidth}\caption{%
                    \footnotesize$\dfrac{\norm{\vb{E}^\text{sca}_\text{far}}}{\norm{\vb{E}^\text{i}}} \; [10^{-9}]$  }\label{sfig:Far:Inc:s:c}\end{subfigure}%
        \begin{subfigure}{.4\textwidth}\caption{%
                    \footnotesize$\dfrac{\norm{\vb{E}^\text{sca}_\text{far}}}{\norm{\vb{E}^\text{i}}} \; [10^{-9}]$  }\label{sfig:Far:Inc:s:d}\end{subfigure}\\
    \includeinkscape[pretex = \footnotesize]{4-Inc-Obl/4-FarXY-S/4-5-Far-XY-S-75}%
    \caption[  Radiation pattern of a AuNP supported on a substrate illuminated at oblique incidence ]{%
    Radiation patterns of a AuNP (light yellow) of radius $a = 12.5$ nm partially embedded in a glass substrate (light blue, $n_\text{s} = 1.5$) with an air matrix ($n_\text{m} = 1$), illuminated by an \textit{s} polarized incident electromagnetic plane wave $\vb{E}^\text{i}$, with a wavelength $\lambda_\text{abs}^{res}$ (see Table \ref{tab:Resonances}) and traveling in the $\vb{k}^\text{i}$ direction at an angle of incidence $\theta_i$ of \textbf{a,b)} $42^\circ$ and \textbf{c,d)} $75^\circ$ relative to the normal direction to the glass-air interface. The radiation patterns consider various values of the incrustation parameter $h/a$, with $a$ the AuNP's radius and $h$ the distance between its center and the interface, and an  incident electric field \textbf{a,c)} perpendicular to the incidence plane (vertical gray dotted lines) and \textbf{b,d)} equal to the incidence plane. In all cases, the incident wave vector $\vb{k}^\text{i}$, the perpendicular $\vb{E}_\perp^\text{i}$ and the  parallel $\vb{E}_\parallel^\text{i}$ projections of the incident electric field relative to the scattering plane are schematized.%
     }
    \label{fig:Far:Inc:s}
\end{figure}

For a $p$ polarized incident electric field $\vb{E}^\text{i}$, the reflected and transmitted electric fields have a direction dependent on the angle of incidence $\theta_i$, therefore the radiation patterns of a partially embedded  12.5 AuNP is expected to behave differently for each $\theta_i$. The radiation patterns for the described system, considering a wavelength $\lambda = 525$ nm for $\vb{E}^\text{i}$, are shown in Figs. \ref{fig:Far:Inc:p1} and  \ref{fig:Far:Inc:p2} for values of $\theta_i$ below and above the critical angle $41.8^\circ$, respectively. The radiation patterns in Figs. \ref{sfig:Far:Inc:p1:a} and  \ref{sfig:Far:Inc:p1:b} correspond to $\theta_i = 15^\circ$ and in Figs. \ref{sfig:Far:Inc:p1:c} and  \ref{sfig:Far:Inc:p1:d} to  $\theta_i = 38^\circ$, while the radiation patterns for  $\theta_i = 42^\circ$ are shown in  Figs. \ref{sfig:Far:Inc:p2:a} and  \ref{sfig:Far:Inc:p2:b}, and for $\theta_i = 75^\circ$ in  Figs. \ref{sfig:Far:Inc:p2:c} and  \ref{sfig:Far:Inc:p2:d}. In both Fig. \ref{fig:Far:Inc:p1} and Fig. \ref{fig:Far:Inc:p2}, the radiation patterns are evaluated at \textbf{a,c)}  a scattering plane perpendicular to the incidence plane (vertical gray dotted line) and  at \textbf{b,d)}  a scattering plane overlapping the incidence plane.

\begin{figure}[b!]
    \centering
    \def\svgwidth{.8\textwidth}
    \includeinkscape[pretex = \footnotesize]{4-Inc-Obl/5-FarXY-P/4-5-Far-XY-P-15}\\[-16.5em]
    \hspace*{-.2\textwidth}%
        \begin{subfigure}{.375\textwidth}\caption{%
                    \footnotesize$\dfrac{\norm{\vb{E}^\text{sca}_\text{far}}}{\norm{\vb{E}^\text{i}}} \; [10^{-9}]$  }\label{sfig:Far:Inc:p1:a}\end{subfigure}%
        \begin{subfigure}{.4\textwidth}\caption{%
                    \footnotesize$\dfrac{\norm{\vb{E}^\text{sca}_\text{far}}}{\norm{\vb{E}^\text{i}}} \; [10^{-9}]$  }\label{sfig:Far:Inc:p1:b}\end{subfigure}\\[13.75em]
    %
    \def\svgwidth{.8\textwidth}
    \hspace*{-.21\textwidth}%
    \vspace*{-.7em}%
        \begin{subfigure}{.4\textwidth}\caption{%
                    \footnotesize$\dfrac{\norm{\vb{E}^\text{sca}_\text{far}}}{\norm{\vb{E}^\text{i}}} \; [10^{-9}]$  }\label{sfig:Far:Inc:p1:c}\end{subfigure}%
        \begin{subfigure}{.4\textwidth}\caption{%
                    \footnotesize$\dfrac{\norm{\vb{E}^\text{sca}_\text{far}}}{\norm{\vb{E}^\text{i}}} \; [10^{-9}]$  }\label{sfig:Far:Inc:p1:d}\end{subfigure}\\
    \includeinkscape[pretex = \footnotesize]{4-Inc-Obl/5-FarXY-P/4-5-Far-XY-P-38}%
    \caption[  Radiation pattern of a AuNP supported on a substrate illuminated at oblique incidence ]{
    Radiation patterns of a AuNP (light yellow) of radius $a = 12.5$ nm partially embedded in a glass substrate (light blue, $n_\text{s} = 1.5$) with an air matrix ($n_\text{m} = 1$), illuminated by an \textit{p} polarized incident electromagnetic plane wave $\vb{E}^\text{i}$, with a wavelength $\lambda_\text{abs}^\text{res}$ (see Table \ref{tab:Resonances}) and traveling in the $\vb{k}^\text{i}$ direction at an angle of incidence $\theta_i$ of \textbf{a,b)} $15^\circ$ and \textbf{c,d)} $38^\circ$ relative to the normal direction to the glass-air interface. The radiation patterns consider various values of the incrustation parameter $h/a$, with $a$ the AuNP's radius and $h$ the distance between its center and the interface, and an  incident electric field \textbf{a,c)} perpendicular to the incidence plane (vertical gray dotted lines) and \textbf{b,d)} equal to the incidence plane. In all cases, the incident wave vector $\vb{k}^\text{i}$, the perpendicular $\vb{E}_\perp^\text{i}$ and the  parallel $\vb{E}_\parallel^\text{i}$ projections of the incident electric field relative to the scattering plane are schematized.%
    }
    \label{fig:Far:Inc:p1}
\end{figure}

The radiation patterns for a partially embedded AuNP when it is illuminated at $\theta_i < \theta_c = 41.8^\circ$ [Fig. \ref{fig:Far:Inc:p1}], so that the transmitted electromagnetic field is a plane wave,  resembles  that of a supported AuNP ---discussed in Section \ref{s:TIR}---: an asymmetrical one-lobe shape and a two-lobe shape characteristic of a point dipole oriented perpendicularly to the propagating direction of the transmitted electromagnetic field. One effect of the AuNP's embedding is the enhancement of the average amplitude of the radiation patterns according to the absorption and scattering efficiencies, as already discussed for the $s$ polarization case shown in Fig. \ref{fig:Far:Inc:s} and also observed  for $p$ polarization at $\theta_i = 15^\circ$ in Figs. \ref{sfig:Far:Inc:p1:a} and \ref{sfig:Far:Inc:p1:b} nevertheless, a deformation of the radiation patterns as the AuNP is buried into the substrate can be observed, particularly for $\theta_i = 38^\circ$ in Figs. \ref{sfig:Far:Inc:p1:c} and \ref{sfig:Far:Inc:p1:d}. In the first, the one-lobe shape for $h/a\leq 0$ is deformed into a two lobe-shape if $h/a>0$, and in the later there is a rotation of $\sim 10^\circ$ clockwise of the two-lobe radiation pattern for $h/a>0$ relative to the patterns for $h/a\leq0$, which are aligned around the  direction of the transmitted electric field.

\begin{figure}[b!]
    \centering
    \def\svgwidth{.8\textwidth}
    \includeinkscape[pretex = \footnotesize]{4-Inc-Obl/5-FarXY-P/4-5-Far-XY-P-42}\\[-16.7em]
    \hspace*{-.2\textwidth}%
        \begin{subfigure}{.4\textwidth}\caption{%
                    \footnotesize$\dfrac{\norm{\vb{E}^\text{sca}_\text{far}}}{\norm{\vb{E}^\text{i}}} \; [10^{-9}]$  }\label{sfig:Far:Inc:p2:a}\end{subfigure}%
        \begin{subfigure}{.4\textwidth}\caption{%
                    \footnotesize$\dfrac{\norm{\vb{E}^\text{sca}_\text{far}}}{\norm{\vb{E}^\text{i}}} \; [10^{-9}]$  }\label{sfig:Far:Inc:p2:b}\end{subfigure}\\[13.75em]
    %
    \def\svgwidth{.8\textwidth}
    \hspace*{-.21\textwidth}%
    \vspace*{-.7em}%
        \begin{subfigure}{.4\textwidth}\caption{%
                    \footnotesize$\dfrac{\norm{\vb{E}^\text{sca}_\text{far}}}{\norm{\vb{E}^\text{i}}} \; [10^{-9}]$  }\label{sfig:Far:Inc:p2:c}\end{subfigure}%
        \begin{subfigure}{.4\textwidth}\caption{%
                    \footnotesize$\dfrac{\norm{\vb{E}^\text{sca}_\text{far}}}{\norm{\vb{E}^\text{i}}} \; [10^{-9}]$  }\label{sfig:Far:Inc:p2:d}\end{subfigure}\\
    \includeinkscape[pretex = \footnotesize]{4-Inc-Obl/5-FarXY-P/4-5-Far-XY-P-75}%
    \caption[  Radiation pattern of a AuNP supported on a substrate illuminated at oblique incidence ]{
    Radiation patterns of a AuNP (light yellow) of radius $a = 12.5$ nm partially embedded in a glass substrate (light blue, $n_\text{s} = 1.5$) with an air matrix ($n_\text{m} = 1$), illuminated by an \textit{p} polarized incident electromagnetic plane wave $\vb{E}^\text{i}$, with a wavelength $\lambda_\text{abs}^\text{res}$ (see Table \ref{tab:Resonances}) and traveling in the $\vb{k}^\text{i}$ direction at an angle of incidence $\theta_i$ of \textbf{a,b)} $42^\circ$ and \textbf{c,d)} $75^\circ$ relative to the normal direction to the glass-air interface. The radiation patterns consider various values of the incrustation parameter $h/a$,with $a$ the AuNP's radius and $h$ the distance between its center and the interface, and an  incident electric field \textbf{a,c)} perpendicular to the incidence plane (vertical gray dotted lines) and \textbf{b,d)} equal to the incidence plane. In all cases, the incident wave vector $\vb{k}^\text{i}$, the perpendicular $\vb{E}_\perp^\text{i}$ and the  parallel $\vb{E}_\parallel^\text{i}$ projections of the incident electric field relative to the scattering plane are schematized.%
    }
    \label{fig:Far:Inc:p2}
\end{figure}

Another difference that arises between  the radiation patterns of a partially embedded AuNP and a supported AuNP (Section \ref{s:TIR}) when illuminated at an oblique incidence, is that the amplitude of the radiation patterns, at is maximum, is different depending on the polarization state of the incident electric field. While the radiation patterns for supported AuNPs show a greater magnitude for  $p$ polarization than for $s$ polarization ---see Fig. \ref{fig:Far:SuppObl}--- when $\theta_i \gtrsim  \theta_c$, the contrary is observed for the partially embedded AuNP, as it can be verified by comparing the axis scales in Figs. \ref{fig:Far:Inc:s} (up to $4$ for $s$ polarization) and $\ref{fig:Far:Inc:p2}$ (up to $2.5$ for $p$ polarization). Such difference in magnitude of the radiation patterns may arise due to the anisotropy of the electric field for each polarization; particularly the influence of the transmitted electric field described by an evanescent wave, whose effect is easily identified even for slightly embedded AuNPs ---$0.75\leq\abs{h/a}$--- at $\theta_i=42^\circ$ where the radiation patterns correspond to that of an electric point dipole perpendicular to the glass-substrate interface ---see the two-lobe shapes aligned along the interface in Figs. \ref{sfig:Far:Inc:p2:a} and \ref{sfig:Far:Inc:p2:b}---. Lastly, the radiation patterns observed for $\theta_i = 75^\circ$ in Figs. \ref{sfig:Far:Inc:p2:c} and \ref{sfig:Far:Inc:p2:d} show, respectively, a less defined two-lobe shape aligned to the substrate and an asymmetrical shape that deforms from  a one to a two-lobe shape perpendicular to the interface as $h/a$ decreases. Both of these effects  are the result of the contribution of the electric field below the substrate overcoming the contribution of the transmitted field since the penetration depth of the evanescent wave decreases for larger angles of incidence, as well as the embedding  decreases the coupling of the evanescent wave and the AuNP.

The analysis of the scattered electric field in the far-field regime, was in agreement with the discussion of the absorption and scattering efficiencies in that the optical properties of a partially embedded AuNP changes uniformly as the NP is buried into the substrate when it is illuminated by an $s$ polarized electric field, while these changes are also dependent on the angle of incidence for a $p$ polarization, yielding a more sensitive optical response for an incidence at an angle above and near the critical angle. When the system is illuminated by a $p$ polarized electromagnetic field, it was noted that, in general, the average optical properties of the system ($Q_\text{abs}$ and $Q_\text{sca}$) are closer to that of the Mie-limiting case of the AuNP in the substrate (matrix) if more (less) than one eight of the AuNP is in such medium. On the other hand, the radiation patterns  for the $p$ polarized illumination of a partially embedded AuNP  are uniformly transformed (only in shape) as the incrustation of the AuNP changes for all angles of incidence, except values of $\theta_i$ near and above the critical angle where the contribution to the optical properties in the spatial distribution of the induced electric field are dominated by the evanescent wave. To observe such distribution, and evaluate if there is a configuration of the system of interest suited for biosensing, the induced electric field is analyzed in the near-field regime.

In Fig. \ref{fig:Near:Inc:Obl} the spatial distribution of the induced electric field is shown for a partially embedded 12.5 nm AuNP when it is illuminated at an angle of incidence of $\theta_i = 42^\circ>\theta_c$, in an internal configuration and an incrustation parameter $h/a = 0$, by an $s$ [Figs. \ref{sfig:Near:Inc0:42:s1} and \ref{sfig:Near:Inc0:42:s2}] and a $p$ [Figs. \ref{sfig:Near:Inc0:42:p1} and   \ref{sfig:Near:Inc0:42:p2}] polarized incident electric field $\vb{E}^\text{i}$ with a wavelength of $\lambda  =525$ nm; additionally, the near-field distribution of a partially embedded AuNP considering an incrustation parameter $h/a = -0.75$ is shown only for the $p$ polarized incident electric field with  $\lambda  =525$ nm in Figs. \ref{sfig:Near:Inc-75:42:p1},  \ref{sfig:Near:Inc-75:42:p2}.  The magnitude of $\vb{E}^\text{ind}$ is shown for a scattering plane parallel to the incident electric field [Figs. \ref{sfig:Near:Inc0:42:s1}, \ref{sfig:Near:Inc0:42:p1}, and  \ref{sfig:Near:Inc-75:42:p1}] and perpendicular to it [Figs.\ref{sfig:Near:Inc0:42:s2}, \ref{sfig:Near:Inc0:42:p2}, and \ref{sfig:Near:Inc-75:42:p2}].

\begin{figure}[t!]\centering
   \def\svgwidth{.75\textwidth}
   \footnotesize
   \captionsetup[subfigure]{labelfont ={normal,bf,color = white}}
   \includeinkscape{4-Inc-Obl/2-Near-SP/2-NearYX-SP-42-Ver2}\\[-47.25em]
   \hspace*{-.25\textwidth}
       \begin{subfigure}{.25\textwidth}\caption{ } \label{sfig:Near:Inc0:42:s1}\end{subfigure}%
       \begin{subfigure}{.34\textwidth}\caption{ }\label{sfig:Near:Inc0:42:s2}\end{subfigure}\\[13em]
   \hspace*{-.25\textwidth}
       \begin{subfigure}{.25\textwidth}\caption{ } \label{sfig:Near:Inc0:42:p1}\end{subfigure}%
       \begin{subfigure}{.34\textwidth}\caption{ }\label{sfig:Near:Inc0:42:p2}\end{subfigure}\\[13em]
    \hspace*{-.25\textwidth}
       \begin{subfigure}{.25\textwidth}\caption{ } \label{sfig:Near:Inc-75:42:p1}\end{subfigure}%
       \begin{subfigure}{.34\textwidth}\caption{ }\label{sfig:Near:Inc-75:42:p2}\end{subfigure}\\[15em]
   \caption[Induced Electric Field of a 12.5 nm Au NP on substrate illuminated at oblique incidence with a $s$ polarized electric field]{%
   Magnitude of the electric field $\vb{E}^\text{ind}$ induced by a partially embedded 12.5 nm AuNP (dashed black lines)  illuminated by an incident electromagnetic plane wave $\vb{E}^\text{i}$ traveling in the $\vb{k}^\text{i}$ direction, in an internal configuration, at an angle of incidence of  $42^\circ$ relative to the normal direction to the interface ---white dashed lines--- between an air matrix ($n_\text{m} = 1$) and a glass substrate ($n_\text{s} = 1.5$), considering \textbf{a,b)} $s$ polarization of $\vb{E}^\text{i}$ and an incrustation parameter $h/a=0$, \textbf{c,d)} \textit{p} polarization and $h/a=0$, and \textbf{e,f)}  \textit{p} polarization and $h/a=-0.75$. The incident electromagnetic plane wave is evaluated at $\lambda = 525$ nm and  $\norm{\vb{E}^\text{ind}}$ is evaluated at  \textbf{a,c,e)} a scattering plane perpendicular to the incidence plane (vertical gray dotted lines) and at \textbf{b,d,f)} a scattering plane overlapping the incidence plane. }
   \label{fig:Near:Inc:Obl}
 \end{figure}

The choice of the values of $\theta_i$, $h/a$ and polarization for the distribution of the induced electric field shown in Fig. \ref{fig:Near:Inc:Obl} exemplifies the general trends found in the optical properties in the far-field of the partially  embedded AuNP. For the $s$ polarized $\vb{E}^\text{i}$ traveling at $\theta_i = 42^\circ$, it is observed that the stronger enhancement of the near field is localized on the surface of the AuNP inside the substrate and that the hotspots are aligned parallel to the substrate and perpendicular to the incidence plane [Fig. \ref{sfig:Near:Inc0:42:s1}], such as when the AuNP was normally illuminated in Section \ref{s:Emb:Normal}. Nevertheless, the strength of the induced electric field on the overall AuNP's surface is $\sim 4$ times stronger than the incident electric field due to the effect of the transmitted evanescent wave.  For the case with $\theta_i = 42^\circ $ and considering $p$ polarization for $\vb{E}^\text{i}$, the hotspots of the induced electric field are located  perpendicularly to the glass-air interface for both $h/a= 0$ [Figs. \ref{sfig:Near:Inc0:42:p1} and \ref{sfig:Near:Inc0:42:p2}] and $h/a = -0.75$ [Figs. \ref{sfig:Near:Inc-75:42:p1} and \ref{sfig:Near:Inc-75:42:p2}], and the spatial distribution of the induced electric field presents a cylindrical symmetry for both incrustation parameters, meaning that even if the average optical properties of the partially embedded AuNP, with $h/a = -0.75$,  are dominated by their response in the substrate ---as discussed above--- the spatial distribution in the near-field is dominated by the evanescent wave, as it can be seen in Figs. \ref{sfig:Near:Inc0:42:p1},  \ref{sfig:Near:Inc0:42:p2}, \ref{sfig:Near:Inc-75:42:p1} and \ref{sfig:Near:Inc-75:42:p2}, since the largest enhancement of the electric field is located on the AuNP's surface above the substrate.

In this Chapter, the optical properties of a AuNP  of radius $a=12.5$ nm in the presence of an homogeneous surrounding consisting of a glass substrate and an air matrix were calculated numerically by means of the Finite Element Method. The obtained results were compared with the Mie-limiting cases, which correspond to the analytical solution of the light absorption and scattering of a single spherical particle in a homogeneous medium due to its interaction with an electromagnetic plane wave.  The presence of the substrate allows for different illumination schemes based on the polarization of the incident electric field and on the side from which the system is illuminated, as well as an additional degree of freedom: the embedding of the AuNP in the substrate. From the obtained results, it was concluded that a 12.5 nm AuNP can be studied in the small particle approximation even if it is partially embedded and illuminated by an evanescent wave, rather than with a plane wave, in the case of total internal reflection. Additionally, it was observed that the Localized Surface Plasmon Resonance (LSPR) of the partially embedded 12.5 AuNP particle is spectrally localized in between the LSPR of the two Mie-limiting cases but that the values of the absorption and scattering efficiencies can be larger than for the Mie-limiting cases, specifically for angles of incidence in the neighborhood of the critical angle. Such changes, for an $s$ polarized illumination of the system were found to be uniform relative to the incrustation parameter ---introduced in Section \ref{s:Emb}--- while for a $p$ polarized illumination the change of  direction of the transmitted electric above the substrate yield an optical response dependent on both the angle of incidence and on the incrustation parameter. In particular, the spatial distribution of the near-field is strongly determined by the evanescent wave for a $p$ polarized illuminated system, which forced the greater enhancement of the electric field to be on the AuNP's surface in contact with the substrate. In summary, the presence of a substrate enhances the optical properties of a 12.5 AuNP when illuminated by an evanescent wave. Depending on the polarization of the incident electromagnetic field, the enhancement of the induced electric field can be localized in the matrix side, rather than in the substrate, even when the AuNP is partially embedded in between both media.

In this thesis the optical properties of a partially embedded AuNP were calculated by the FEM and characteristics such as tuning of the LSPR and control over the spatial distribution of the electric field were identified. Due to these characteristics, which can be exploited alongside the intrinsic feature of  non-washability  of partial embbeding, bidimensional arrays of partially embedded AuNPs are suited for boisensing. Lastly, the methodology developed in this thesis can be employed for nanospheres of different materials than Au, and different matrices and substrates considering a relatively small spectral window in between the Mie-limiting cases of the system of interest.

























