% !TeX root = ../tesis.tex

On the past section, the AuNP of radius $a = 12.5$ nm was illuminated at normal incidence in four different spatial configurations considering the  presence of a substrate: the AuNP either embedded in the substrate (with a refractive index $n_\text{s}$) or supported on it embedded in an air matrix (with refractive index $n_\text{s}$), and the incident electric field  illuminating the system from the substrate (internal configuration) or from the matrix (external configuration). By considering that the incident electric plane wave $\vb{E}^\text{i}$  propagates from the substrate to the matrix at an angle $\theta_i$, relative to the normal direction to the interface between the two media, the electric field interacting with the AuNP is the transmitted field, that propagates at a transmission angle $\theta_t = \asin(n_\text{m}\sin\theta_i/n_\text{s})$ \cite{born_max_principle_1999} and two differences arises compared to the normal incidence cases: there are two different polarization states for $\vb{E}^\text{i}$ ---$s$ and $p$ polarization\footnote{%
    The $s$ and $p$ polarization states of the electric field are defined by considering its oscillations perpendicular and parallel, respectively, to the incidence plane, defined by the the propagating direction of the incident electric plane wave and the normal direction to the interface between the substrate and the matrix.}%
--- and if $\theta_i$ is greater than the critical angle $\theta_c = \asin(n_\text{m}/n_\text{s})$ the transmitted electric field ceases to be a plane wave and it is now described by an evanescent wave propagating along the interface  \cite{born_max_principle_1999}. Similarly to the past section, the optical properties of the supported AuNP illuminated at an oblique incidence in an internal configuration are studied by analyzing the absorption and scattering efficiencies, the radiation pattern and, lastly, the induced electric field on the AuNP.

\begin{figure}[b!]
    \def\svgwidth{.95\textwidth}
    \centering
    \hspace*{-28.5em}%
    \vspace*{-1.25em}%
        \begin{subfigure}{.71\textwidth}\caption{ }\label{sfig:SuppObl:Eff:Abs}\end{subfigure}%
        \begin{subfigure}{.25\textwidth}\caption{ }\label{sfig:SuppObl:Eff:Sca}\end{subfigure} \\
    \includeinkscape[pretex = \small]{2-SuppObl/1-Efficiencies/1-Oblique-Supp-Eff}%
    \vspace*{-.5em}
    \caption[Absorption and Scattering Efficiencies of a 12.5 nm AuNP on a Interface Illuminated in an internal configuration at oblique incidence]{\textbf{a)} Absorption and \textbf{b)} scattering efficiencies of a $12.5$ nm AuNP in an air matrix ($n_\text{m} = 1$) and supported on a glass substrate ($n_\text{s} = 1.5$) as function of the wavelength $\lambda$ of an  $s$ (filled circle/solid lines) and a $p$ (empty circle/dashed lines) polarized incident electric plane wave propagating in the direction of the wave vector $\vb{k}^\text{i}$, in an internal configuration, at an angle of incidence $\theta_i$ of $15^\circ$ (black),  $38^\circ$ (orange),  $42^\circ$ (blue) and  $75^\circ$ (light orange) relative to the normal direction to the glass-air interface. The green shaded region shows the two Mie-limiting cases of a AuNP embedded in air and in glass; the magenta (supported AuNP) and red (Mie-limiting) markers corresponds to the efficiencies evaluated at the wavelength of resonance for each case.}
\label{fig:SuppObl:Eff}
\end{figure}

In Fig. \ref{fig:SuppObl:Eff} the absorption $Q_\text{abs}$ [Fig.\ref{sfig:SuppObl:Eff:Abs}] and scattering $Q_\text{sca}$ [Fig. \ref{sfig:SuppObl:Eff:Sca}] efficiencies of a 12.5 nm AuNP in air ($n_\text{m} = 1$) supported on a glass substrate ($n_\text{s} = 1.5$) are shown as function of the wavelength $\lambda$ of the incident electric field $\vb{E}^\text{i}$ illuminating the AuNP from the substrate at an incidence angle $\theta_i$ of $15^\circ$ (black),  $38^\circ$ (orange),   $42^\circ$ (blue) and  $75^\circ$ (light orange) considering an $s$ (filled circle/solid lines) and a $p$ (empty circle/dashed lines) polarization for $\vb{E}^\text{i}$. Since the critical angle for a glass-air interface is $\theta_c = 41.8^\circ$, the blue and light orange curves corresponds to the interaction between an evanescent wave and the AuNP. The magenta markers correspond to the values of $Q_\text{abs}$ and $Q_\text{sca}$ evaluated at the wavelength of resonance; the Mie-limiting cases (AuNP embedded in air and in glass)  are signalized by the boundaries of the green shaded region  and the red markers correspond to the resonances of their efficiencies.

\textcolor{red}{\textbf{Falta discutir las graficas de las eficiencias en este espacio}
A general behavior on the absorption efficiency $Q_\text{abs}$ for both polarizations is that its nominal value increases from $\theta_i = 15^\circ$
}

The radiation pattern of the 12.5 nm AuNP embedded in air and supported on a glass substrate is shown in Fig. \ref{fig:Far:SuppObl} when an $s$ polarized [Figs. \ref{sfig:Far:SuppObl:s:a} and \ref{sfig:Far:SuppObl:s:b}] and a  $p$ polarized [Figs. \ref{sfig:Far:SuppObl:p:c} and \ref{sfig:Far:SuppObl:p:d}] incident electric field $\vb{E}^\text{i}$ illuminates the AuNP at an incidence angle of $\theta_i = 15^\circ$ (black) and $\theta_i = 38^\circ$ ---below the critical angle $\theta_c = 41.8^\circ$---, and $\theta_i = 42^\circ$ (black) and $\theta_i = 75^\circ$ ---above  $\theta_c$--- at the wavelengths of resonance of the absorption efficiency ---magenta markers in Fig. \ref{sfig:SuppObl:Eff:Abs}--- for each case. The scattering plane where the radiation pattern is shown in Figs. \ref{sfig:Far:SuppObl:s:a} and \ref{sfig:Far:SuppObl:p:c}  is perpendicular to the incidence plane (vertical gray dotted lines) while the scattering plane equals the incidence plane in Figs. \ref{sfig:Far:SuppObl:s:b} and \ref{sfig:Far:SuppObl:p:d}; the incident wave vector $\vb{k}^\text{i}$ and the components of the incident electric field parallel $\vb{E}^\text{i}_\parallel$ and perpendicular $\vb{E}^\text{i}_\perp$ to the scattering plane are schematized in all cases.

\begin{figure}[h!]
    \centering
    \def\svgwidth{.8\textwidth}
    \hspace*{-.215\textwidth}%
    \vspace*{-.5em}%
        \begin{subfigure}{.32\textwidth}\caption{\footnotesize$\dfrac{\norm{\vb{E}^\text{sca}_\text{far}}}{\norm{\vb{E}^\text{i}}} \times 10^{-9}$  }\label{sfig:Far:SuppObl:s:a}\end{subfigure}%
        \begin{subfigure}{.4\textwidth}\caption{\footnotesize$\dfrac{\norm{\vb{E}^\text{sca}_\text{far}}}{\norm{\vb{E}^\text{i}}} \times 10^{-9}$  }\label{sfig:Far:SuppObl:s:b}\end{subfigure}\\
    \includeinkscape[pretex = \footnotesize]{2-SuppObl/4-5-FarXY/4-5-Far-XY-S}\\
    %
    \def\svgwidth{.8\textwidth}
    \hspace*{-.215\textwidth}%
    \vspace*{-.5em}%
        \begin{subfigure}{.32\textwidth}\caption{\footnotesize$\dfrac{\norm{\vb{E}^\text{sca}_\text{far}}}{\norm{\vb{E}^\text{i}}} \times 10^{-9}$  }\label{sfig:Far:SuppObl:p:c}\end{subfigure}%
        \begin{subfigure}{.4\textwidth}\caption{\footnotesize$\dfrac{\norm{\vb{E}^\text{sca}_\text{far}}}{\norm{\vb{E}^\text{i}}} \times 10^{-9}$  }\label{sfig:Far:SuppObl:p:d}\end{subfigure}\\
    \includeinkscape[pretex = \footnotesize]{2-SuppObl/4-5-FarXY/4-5-Far-XY-P}%
    \caption[  Radiation pattern of a AuNP supported on a substrate illuminated at oblique incidence]{Radiation pattern of a AuNP (light yellow) of radius $a = 12.5$ nm supported on a glass substrate (light blue, $n_\text{s} = 1.5$) with an air matrix ($n_\text{m} = 1$) illuminated by an incident electric plane wave $\vb{E}^\text{i}$, with a wavelength $\lambda$, traveling in the $\vb{k}^\text{i}$ direction at an angle of incidence $\theta_i$ of $15^\circ$ (black),  $38^\circ$ (orange),  $42^\circ$ (blue) and  $75^\circ$ (light orange) relative to the normal direction to the glass-air interface. The radiation patterns consider an \textbf{a,b)} $s$ polarized and  a \textbf{c,d}) $p$ polarized incident electric field and the scattering plane \textbf{a,c)} perpendicular to the incidence plane (vertical gray dotted lines) and \textbf{b,d)} equal to the incidence plane. In all cases the incident wave vector $\vb{k}^\text{i}$, the perpendicular $\vb{E}_\perp^\text{i}$ and the  parallel $\vb{E}_\parallel^\text{i}$ projection of the incident electric field relative to the scattering plane are schematized.%
    }
    \label{fig:Far:SuppObl}
\end{figure}

Since the radiation patterns observed when an $s$ polarized incident electric field illuminates the AuNP follow the same one and a two-lobe shapes for all the considered combinations of $\theta_i$ and $\lambda$ only differing on the magnitud, the induced electric field $\vb{E}^\text{ind}$ ---the internal and the scattered electric field in the near-field regime--- have the same qualitative behavior for all incident angles at this polarization state. Therefore, the norm of the induced electric field $\norm{\vb{E}^\text{ind}}$, evaluated at a scattering plane perpendicular to the incidence plane (vertical gray dashed lines) is shown in Fig. \ref{fig:Near:SuppObl:s} for a 12.5 nm AuNP (black dashed lines) supported on the interface between a glass substrate and an air matrix (white dashed lines) when the incident electric field $\vb{E}^\text{i}$ illuminates the system at an incident angle of $15^\circ$ [Fig. \ref{sfig:Near:SuppObl:s:15}], $38^\circ$ [Fig. \ref{sfig:Near:SuppObl:s:38}], $42^\circ$ [Fig. \ref{sfig:Near:SuppObl:s:42}]  and $75^\circ$ [Fig. \ref{sfig:Near:SuppObl:s:75}].

\begin{figure}[h!]\centering
   \def\svgwidth{.75\textwidth}
   \footnotesize
   \captionsetup[subfigure]{labelfont ={normal,bf,color = white}}
   \includeinkscape{2-SuppObl/2-Near-sP/2-NearYX-sPol}\\[-32.6em]
   \hspace*{-.25\textwidth}
       \begin{subfigure}{.25\textwidth}\textcolor{red}{\caption{ } \label{sfig:Near:SuppObl:s:15}}\end{subfigure}%
       \begin{subfigure}{.34\textwidth}\caption{ }\label{sfig:Near:SuppObl:s:38}\end{subfigure}\\[13em]
    \hspace*{-.25\textwidth}
       \begin{subfigure}{.25\textwidth}\textcolor{red}{\caption{ } \label{sfig:Near:SuppObl:s:42}}\end{subfigure}%
       \begin{subfigure}{.34\textwidth}\caption{ }\label{sfig:Near:SuppObl:s:75}\end{subfigure}\\[15em]
\caption[Induced Electric Field of a 12.5 nm Au NP on substrate illuminated at oblique incidence with a $s$ polarized electric field]{%
Electric field $\vb{E}^\text{ind}$ induced by a supported 12.5 nm AuNP (dashed black lines) illuminated by an $s$ polarized incident electric plane wave $\vb{E}^\text{i}$ traveling in the $\vb{k}^\text{i}$ direction, in an internal configuration, at an angle of incidence of \textbf{a)} $15^\circ$, \textbf{b)} $38^\circ$, \textbf{c)} $42^\circ$ and \textbf{d)} $75^\circ$, relative to the normal direction to the interface ---white dashed lines--- between an air matrix ($n_\text{m} = 1$) and a glass substrate ($n_\text{s} = 1.5$). The incident electric plane wave is evaluated at $\lambda = 510$ nm ---resonance wavelength of the absorption efficiency--- and in all shown cases the scattering plane is perpendicular to the incidence plane (vertical gray dotted lines).
}
 \label{fig:Near:SuppObl:s}
 \end{figure}


Contrastingly to the $s$ polarized incident electric field case, the radiation patterns of a supported 12.5 nm AuNP illuminated with a $p$ polarized electric field have a different qualitative behavior depending on the angle of incidence $\theta_i$ as shown in Figs. \ref{sfig:Far:SuppObl:p:c} and  \ref{sfig:Far:SuppObl:p:d}. Thus, the norm of $\vb{E}^\text{ind}$ is shown in Fig. \ref{fig:Near:SuppObl:p} and it is evaluated at scattering plane perpendicular to the incidence plane (vertical gray dotted line) [Figs. \ref{sfig:Near:SuppObl:p:15:perp}, \ref{sfig:Near:SuppObl:p:38:perp}, \ref{sfig:Near:SuppObl:p:42:perp} and \ref{sfig:Near:SuppObl:p:75:perp}] and at the incidence  plane [Figs. \ref{sfig:Near:SuppObl:p:15:par}, \ref{sfig:Near:SuppObl:p:38:par}, \ref{sfig:Near:SuppObl:p:42:par} and \ref{sfig:Near:SuppObl:p:75:par}] for an incidence angle $\theta_i = 15^\circ$ [Figs. \ref{sfig:Near:SuppObl:p:15:perp}  and \ref{sfig:Near:SuppObl:p:15:par}] and  $\theta_i = 38^\circ$ [Figs. \ref{sfig:Near:SuppObl:p:38:perp}  and \ref{sfig:Near:SuppObl:p:38:par}], both below the critical angle $\theta_c = 41.8^\circ$, and $\theta_i = 42^\circ$ [Figs. \ref{sfig:Near:SuppObl:p:42:perp}  and \ref{sfig:Near:SuppObl:p:42:par}] and  $\theta_i = 75^\circ$ [Figs. \ref{sfig:Near:SuppObl:p:75:perp}  and \ref{sfig:Near:SuppObl:p:75:par}] above $\theta_c$.


 \begin{figure}[h!]\centering
    \def\svgwidth{.7\textwidth}
    \scriptsize
    \captionsetup[subfigure]{labelfont ={small,bf,color = white}}
    \includeinkscape{2-SuppObl/3-Near-pP/2-NearYX-pPol}\\[-62.75em]
    \hspace*{-.255\textwidth}
        \begin{subfigure}{.25\textwidth}\textcolor{red}{\caption{ } \label{sfig:Near:SuppObl:p:15:perp}}\end{subfigure}%
        \begin{subfigure}{.3\textwidth}\caption{ }\label{sfig:Near:SuppObl:p:15:par}\end{subfigure}\\[12.8em]
    \hspace*{-.225\textwidth}
        \begin{subfigure}{.225\textwidth}\textcolor{red}{\caption{ } \label{sfig:Near:SuppObl:p:38:perp}}\end{subfigure}%
        \begin{subfigure}{.34\textwidth}\caption{ }\label{sfig:Near:SuppObl:p:38:par}\end{subfigure}\\[12.8em]
    \hspace*{-.225\textwidth}
        \begin{subfigure}{.225\textwidth}\textcolor{red}{\caption{ } \label{sfig:Near:SuppObl:p:42:perp}}\end{subfigure}%
        \begin{subfigure}{.34\textwidth}\caption{ }\label{sfig:Near:SuppObl:p:42:par}\end{subfigure}\\[12.8em]
    \hspace*{-.225\textwidth}
        \begin{subfigure}{.225\textwidth}\textcolor{red}{\caption{ } \label{sfig:Near:SuppObl:p:75:perp}}\end{subfigure}%
        \begin{subfigure}{.34\textwidth}\caption{ }\label{sfig:Near:SuppObl:p:75:par}\end{subfigure}\\[15.75em]
    \caption[Induced Electric Field of a 12.5 nm Au NP on substrate illuminated at oblique incidence with a $p$ polarized electric field]{\footnotesize%
    Electric field $\vb{E}^\text{ind}$ induced by a supported 12.5 nm AuNP (dashed black lines) illuminated by a $p$ polarized incident electric plane wave $\vb{E}^\text{i}$ traveling in the $\vb{k}^\text{i}$ direction, in an internal configuration, at an angle of incidence of \textbf{a,b)} $15^\circ$, \textbf{c,d)} $38^\circ$, \textbf{e,f)} $42^\circ$ and \textbf{g,h)} $75^\circ$, relative to the normal direction to the interface ---white dashed lines--- between an air matrix ($n_\text{m} = 1$) and a glass substrate ($n_\text{s} = 1.5$). The incident electric plane wave is evaluated at the resonance wavelength of the absorption efficiency ---see Fig. \ref{sfig:SuppObl:Eff:Abs}--- and in all shown cases the norm $\norm{\vb{E}^\text{ind}}$ is evaluated at  \textbf{a,c,e,g)} a scattering plane perpendicular to the incidence plane (vertical gray dotted lines) and at \textbf{b,d,f,h)} a scattering plane equal to the incidence plane.
    }
    \label{fig:Near:SuppObl:p}
  \end{figure}
