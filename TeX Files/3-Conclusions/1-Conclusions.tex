% !TeX root = ../tesis.tex

In this thesis, it was determined the  optical properties of a spherical gold nanoparticle (AuNP) of radius $a = 12.5$~nm partially embedded in an air matrix and a glass substrate  as a function of its embedding degree, characterized by the incrustation parameter $h/a$ ---with $h$ the position of the center of the AuNP relative to the planar air-glass interface---. By means of the Finite Element Method (FEM) ---implemented in the commercial software COMSOL Multiphysics\texttrademark{} Ver. 5.4--- the absorption $Q_\text{abs}$ and scattering $Q_\text{sca}$ efficiencies, the radiation pattern and the spatial distribution of the induced electric field of the partially embedded 12.5~nm~AuNP were calculated when this was illuminated by a monochromatic plane wave traveling at an oblique direction with a defined polarization state; all numerical results were compared with the Mie-limiting cases calculated analyticalally, which considered a 12.5~nm~AuNP embedded in an infinite matrix of air, and an infinite matrix of glass. From the preformed calculations presented  and discussed in  Chapter \ref{ch:Results}, it was observed that the 12.5~nm~AuNP with partial embedding can be described by a mainly dipolar contribution, that its optical response can be spectrally tune and even resemble, with a fixed embedding, the two Mie-limiting cases depending on illumination schema and the distribution of the electric field  on the surface of the AuNP can be chosen at will based on the illumination and embedding of the system, and that the optical response is maximized if the system is illuminated with an evanescent wave at an angle of incidence near the critical angle.

The optical response of a partially embedded 12.5~nm~AuNP can be described in the dipolar approximation when illuminated in the visible light regime, modulated by the incrustation parameter, the polarization of the incident light and the angle of incidence. The dipolar behavior of the partially embedded 12.5~nm~AuNP was observed in the spectral response of  $Q_\text{abs}$ and  $Q_\text{sca}$  showing one maximum value at the wavelength $\lambda_\text{res}$ associated to the Localized Surface Plasmon Resonance (LSPR). Additionally,  the radiation pattern and the spatial distribution of the induced electric field of the partially embedded  2.5~nm~AuNP at  $\lambda_\text{res}$ presented the characteristic shape for a point dipole. While the dipolar response of the  partially embedded 12.5~nm~AuNP was identified for all considered cases, its embedding and illumination schema determined its spectral and spatial response.

The average optical response of the partially embedded  12.5~nm~AuNP, given by $Q_\text{abs}$ and  $Q_\text{sca}$, resembles that of the Mie-limiting for the incrustation parameters $h/a = 1$ (supported AuNP) and $h/a= -1$ (totally embedded AuNP) but the transition between this cases is different for each polarization state of the incident electric plan wave. For example, the LSPR of the partially embedded AuNP is redshifted ---in between the excitation wavelengths of the LSPR for the Mie-limiting cases--- as $h/a$ decreases but its rate of growth is different for $s$ and for $p$ polarization  at a fixed angle of incidence $\theta_i$. On the one hand, in the $s$ polarization scenario considering internal illumination (from the glass substrate to the air matrix), the redshift of the LSPR is a uniform  function of $\theta_i$ , that is, that the redshift of $\lambda_\text{res}$ is the same for all values of $\theta_i$. On the other,  for the $p$ polarization case, the LSPR redshift showed different growth rates as function of $h/a$  for each angle: the closer $\theta_i$ was to the critical angle $\theta_c =  41.8 ^\circ$, the smaller the value of $h/a$ ---greater embedding--- at which the LSPR redshift growth rate was appreciable, for example, for $\theta_i = 38^\circ$ the redshift started to grow continuously at $h/a= 0.25$, while it  did at $h/a= 0$ for $\theta_i = 42^\circ$. Similarly, both  $Q_\text{abs}$ and  $Q_\text{sca}$ at $\lambda_\text{res}$ are enhanced as $h/a$ decreases  and it depends on how the system is illuminated. The enhancement is monotonous when considered $s$ polarization, for all $\theta_i$,  and  $p$ polarization and values of $\theta_i$ not in the neighborhood  of  $\theta_c$, while for $p$ polarization and angles of incidences near $\theta_c$ there is a diminishment of $Q_\text{abs}$ and  $Q_\text{sca}$ for $0.75 < \abs{h/a}$. All of the above described is attributed by the direction and strength of the transmitted electric field transmitted from the substrate to the matrix, as well as if the illumination is given by a plane wave ($\theta_i>\theta_c$) or by an evanescent wave ($\theta_i<\theta_c$). In particular, it was  observed that for all illumination schemes, the partially embedded AuNP optically behaved more like a supported (totally embedded)  AuNP if only on eight of it was inside the glass substrate (air matrix), but a AuNP with a fixed incrustation parameter, such that  $0.75 > \abs{h/a}$, can present optical properties similar to those of a supported AuNP when illuminated at an angle of incidence $\theta_i \gtrapprox \theta_c$ with one polarization and optical properties similar to those of a totally embedded AuNP with the other polarization state.

The spatial distribution of the induced electric field, evaluated at the LSPR excitation wavelength, in the near-field regime, which determines the radiation pattern, can be chosen at will by setting the polarization and direction of the incident plane wave, and for $p$ polarization also by setting the embedding of the AuNP. The electric field on the surface can be spatially confined in regions known as hotspots, where there is an enhancement of the electric field; such enhancement is up to $2.5$ for the Mie-limiting cases but up to $6$ in the presence of the substrate and  its spatial distribution is modulated by the embedding of the AuNP. When the AuNP is illuminated by an $s$ polarized incident electric field, the hotspots are parallel to the glass-air interface due to the continuity of the parallel component to the interface of electric field independently of the angle of incidence $\theta_i$ and of the incrustation parameter, nevertheless such hotspots are not radially located on the surface of the AuNP but rather they are induced on the surface in touch with the substrate. meaning that the hotspots are spreader in the surface as $h/a$ decreases, and such distribution of the induced electric field is attributed to the optical density of the glass which is greater than that of the air. Contrastingly, for $p$ polarization, the hotspots distribution on the surface can be smoothly rotated from its alignment parallel to the substrate at normal incidence  either by fixing the incrustation parameter and increasing the angle of incidence up to the critical angle or by fixing the angle of incidence ($ \theta_i >0^\circ$) and embedding the AuNP into the substrate (with a more limited rotation angle of the hotspost relative to the pervious method); the hotspots can be aligned perpendicular to the glass-air interface if the system is illuminated by an evanescent wave ($\theta_i>\theta_c$) with $p$  polarization. Lastly, the strength of the enhancement of the electric field on the AuNP is, for the $p$ polarization case, greater on the surface in contact with the air matrix, than in the glass substrate unlike the $s$ polarization case, which is a result of the electric evanescent wave traveling along the interface but with the electric field perpendicular to the surface. 


From the above characteristics, it can be concluded that the optical properties of a partially embedded spherical AuNP of radius 12.5 nm, with at least one eight of its volume in the air matrix, is suited for interactions with elements in the air matrix under internal illumination. If the system is illuminated with a $p$ polarized incident electric plane wave traveling at an angle  $\theta_i \gtrapprox \theta_c$, the system is optimized to interact with its surroundings above the substrate since the optical response is maximized in the matrix. Therefore, partially embedded spherical AuNPs are strong candidates for meta-atoms conforming a biosensing-aimed-metasurface.


































