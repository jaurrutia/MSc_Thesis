% !TeX root = ../tesis.tex

\chapter*{Background and Motivation}
\addcontentsline{toc}{chapter}{\protect\numberline{}Background and Motivation}	  		% Comment if you don't want the introduction to appear on the table of content. It will not have a number
\label{chapter:intro}

% this file is called up by thesis.tex
% content in this file will be fed into the main document


Optical metasurfaces are bidimensional arrays of metal/dielectric nanostructures ---known as meta-atoms--- specifically tailored to behave in a way no found in nature when illuminated at a specific wavelength range in the visible light regime \cite{khan_optical_2022,gonzalez-alcalde_large_2020}. Depending on the physical properties of the meta-atoms, that is, their size, shape, orientation and spatial distribution within the bidimensional array \cite{kim_plasmonic_2019,khan_optical_2022}, a metasurface  allows to shape at will the  spatial optical response of the system \cite{chen_review_2016}, thus suiting them for a variety of applications in fields as Spectroscopy \cite{khan_optical_2022}, Sensing\cite{estevez_trends_2014,jain_noble_2008,khan_optical_2022,chen_review_2016,kim_plasmonic_2019}, Color Structuration \cite{gonzalez-alcalde_large_2020} and Communications \cite{chen_review_2016}. In particular, Nanoplasmonics is the field  that studies the electromagnetic properties of metallic meta-atoms, which have been highly employed in Biosensing and Bioimaging due to their strong optical response compared to that of organic dyes typically used \cite{kim_plasmonic_2019}.




. Biosensing-aimed metasurfaces are supported onto a substrate and immersed in an aqueous medium, thus its performance is limited by its washability. One alternative to decrease a metasurface's washability is to partially embed it within the substrate while allowing the metasurface to still interact with the aqueous medium. In this work, we study the optical response of a single partially embedded metallic nanosphere  and extend its behavior analytically into a disordered metasurface by employing an effective medium approach.

The incrustation of the metasurface is studied by calculating the spectral behavior of a single partially embedded nanosphere with the Finite Element Method (FEM) and proposing an effective polarizability  in the elements of the metasurface.

\begin{itemize}
	\item Your field: Context about the field your are working \\
	\textbf{Plasmonics -> Metameterials -> Biosensing}
	\item Motivation: Backgroung about your thesis work and why did you choose this project and why is it important.\\
	\textbf{Fabrication -> Partially embedded NPs -> No analytical (approximated) method physically introduces the incrustation degree. There are numerical solutions and Effective Medium Theories approaching the problem but the later only as a fitting method. }
	\item Objectives: What question are you answering with your work.\\
	\textbf{Can optical non invasive tests (IR-Vis) retrieve the average incrustation degree for monolayers of small spherical particles?}
	\item Methology: What are your secondary goals so you achieve your objective. Also, how are you answering yout question: which method or model.\\
	\textbf{Bruggeman homogenization theories on bidimensional systems?\\
	Is the dipolar approximation is enough or do we need more multipolar terms?\\
	Do we need the depolarization factors?}
	\item Structure: How is this thesis divides and what is the content of each chapter.
\end{itemize}
